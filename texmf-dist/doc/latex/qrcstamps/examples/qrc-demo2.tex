\documentclass{article}
\usepackage{amsmath}
\usepackage[forcolorpaper]{web}
\usepackage{eforms}
\usepackage[scandoc]{qrcstamps}


\title{Creating a QR Code Symbol Stamp Interactively}
\author{D. P. Story}
\subject{Creating QR Code symbols using Acrobat annotation stamps and LaTeX interactively}
\email{dpstory@acrotex.net}
\keywords{stamp, QRCode, stamps}

\useFullWidthForPaper

\parindent0pt\parskip6pt

\begin{document}

\makeinlinetitle

This file demonstrates that it is possible to create QR Code symbols
interactively. Enter some text into the field below then click on the white
space outside the field (or press the \texttt{TAB} key) to commit the data.
The QR Code symbol stamp will be updated appropriately.

\qrCode[size=medium]{http://www.acrotex.net}\space\textField[
    \TU{Maximum of 250 characters allowed, enter a URL, a mailto, or simply text.}
    \MaxLen{250}\Ff{\FfMultiline}\AA{\AAKeystroke{%
if (event.willCommit) {\r\t
    if (!/^\\s*$/.test(event.value)) {\r
        var annot=this.getAnnot({nPage:this.pageNum,cName:"\currentAnnotName"});\r\t
        var props=annot.getProps();\r\t
        props.author="QRC";\r\t
        props.subject=event.value;\r\t
        annot.setProps(props);\r\t
        delete props.APCosObj;\r\t
        annot.setProps(props);\r
    }\r
}}}]{txtFld}{3in}{11bp*6}

Ignore the JavaScript code, it is somewhat obscure, but it works.

\paragraph*{Requirements for this file} For this QR Code symbol stamp to
update as designed, the user must (1) be viewing this document in
\textsf{Acrobat} and (2) the stamp file \texttt{qrc-stamps.pdf}, which comes
with this distribution, must be installed in the user \texttt{Stamps} folder
of \textsf{Acrobat}. This file and these techniques work only for you, my dear
AeB Pro users.

By the way, if you also have Acrobat Reader, try this file in that
application to see what happens.

\end{document}

% \iffalse meta-comment
%
% Copyright (C) 2013 by Scott Pakin <scott+csan@pakin.org>
% --------------------------------------------------------
%
% This file may be distributed and/or modified under the
% conditions of the LaTeX Project Public License, either version 1.3c
% of this license or (at your option) any later version.
% The latest version of this license is in:
%
%    http://www.latex-project.org/lppl.txt
%
% and version 1.3c or later is part of all distributions of LaTeX
% version 2006/05/20 or later.
%
% \fi
%
% \iffalse
%<*driver>
\ProvidesFile{comicsans.dtx}
%</driver>
%<package>\NeedsTeXFormat{LaTeX2e}[1999/12/01]
%<package>\ProvidesPackage{comicsans}
%<*package>
    [2013/12/19 v1.0g Support for Microsoft's Comic Sans font]
%</package>
%
%<*driver>
\documentclass{ltxdoc}
\usepackage{wasysym}
\usepackage{xspace}
\usepackage[T2A,T1]{fontenc}
\usepackage{soul}
\setul{1.5pt}{1pt} % Better match for Comic Sans
\IfFileExists{hyperref.sty}{%
  \usepackage{hyperref}
  \hypersetup{%
    bookmarksopen,
    hyperindex=false,
    pdftitle={The comicsans package},
    pdfauthor={Scott Pakin <scott+csan@pakin.org>},
    pdfsubject={Using Microsoft's Comic Sans font in a LaTeX document},
    pdfkeywords={Comic Sans; font; typeface; casual script; TeX; LaTeX;
                 Microsoft Windows; Vincent Connare}
  }
  \def\UrlBigBreaks{\do/}
}{%
}
\usepackage[largesymbols,plusminus]{comicsans}
\setcounter{IndexColumns}{2}
\EnableCrossrefs
\CodelineIndex
\RecordChanges

% We define only the Cyrillic characters we need.  Later, we'll use
% these explicitly rather than change our input encoding, hyphenation
% patterns, etc.
\DeclareTextSymbol{\cyra}{T2A}{224}
\DeclareTextSymbol{\cyrch}{T2A}{247}
\DeclareTextSymbol{\cyre}{T2A}{229}
\DeclareTextSymbol{\cyrery}{T2A}{251}
\DeclareTextSymbol{\cyrf}{T2A}{244}
\DeclareTextSymbol{\cyrg}{T2A}{227}
\DeclareTextSymbol{\cyri}{T2A}{232}
\DeclareTextSymbol{\cyrishrt}{T2A}{233}
\DeclareTextSymbol{\cyrk}{T2A}{234}
\DeclareTextSymbol{\cyrl}{T2A}{235}
\DeclareTextSymbol{\cyrm}{T2A}{236}
\DeclareTextSymbol{\cyrn}{T2A}{237}
\DeclareTextSymbol{\cyrr}{T2A}{240}
\DeclareTextSymbol{\cyrs}{T2A}{241}
\DeclareTextSymbol{\cyrsh}{T2A}{248}
\DeclareTextSymbol{\cyrt}{T2A}{242}
\DeclareTextSymbol{\cyrzh}{T2A}{230}

% We'd like to be able to use \sum in both Computer Modern and Comic
% Sans in this document.  We therefore declare \cmsum as the Computer
% Modern version of \sum.
\DeclareSymbolFont{cmlargesymbols}{OMX}{cmex}{m}{n}%
\DeclareMathSymbol{\cmsum}{\mathop}{cmlargesymbols}{"50}

% Load this document recursively and format all of the text and code.
\begin{document}
  \DocInput{comicsans.dtx}
  \PrintChanges
  \PrintIndex
\end{document}
%</driver>
% \fi
%
% \CheckSum{345}
%
% \CharacterTable
%  {Upper-case    \A\B\C\D\E\F\G\H\I\J\K\L\M\N\O\P\Q\R\S\T\U\V\W\X\Y\Z
%   Lower-case    \a\b\c\d\e\f\g\h\i\j\k\l\m\n\o\p\q\r\s\t\u\v\w\x\y\z
%   Digits        \0\1\2\3\4\5\6\7\8\9
%   Exclamation   \!     Double quote  \"     Hash (number) \#
%   Dollar        \$     Percent       \%     Ampersand     \&
%   Acute accent  \'     Left paren    \(     Right paren   \)
%   Asterisk      \*     Plus          \+     Comma         \,
%   Minus         \-     Point         \.     Solidus       \/
%   Colon         \:     Semicolon     \;     Less than     \<
%   Equals        \=     Greater than  \>     Question mark \?
%   Commercial at \@     Left bracket  \[     Backslash     \\
%   Right bracket \]     Circumflex    \^     Underscore    \_
%   Grave accent  \`     Left brace    \{     Vertical bar  \|
%   Right brace   \}     Tilde         \~}
%
%
% \changes{v1.0}{2002/09/10}{Initial version}
%
% \GetFileInfo{comicsans.dtx}
%
% \DoNotIndex{\.,\ ,\begingroup,\bye,\d,\fi,\font,\endencoding,\endgroup}
% \DoNotIndex{\endinstallfonts,\endmetrics,\endsetglyph}
% \DoNotIndex{\endsetslot,\input,\let,\n,\newcommand}
% \DoNotIndex{\newenvironment,\newif,\relax,\renewcommand}
% \DoNotIndex{\skewchar,,\undefined,\w}
%
% \title{The \textsf{comicsans} package\thanks{This document
%   corresponds to \textsf{comicsans}~\fileversion, dated \filedate.}}
% \author{Scott Pakin \\ \texttt{scott+csan@pakin.org}}
%
% ^^A  Make up for hyperref not being loaded.
% \ifx\href\undefined
%   \newcommand{\href}[2]{#2}
%   \newcommand{\url}[1]{\texttt{#1}}
%   \newcommand{\localfile}[1]{\texttt{#1}}   ^^A  See alternate definition below.
% \else
%   \newcommand{\localfile}[1]{\begingroup\Url{#1}}  ^^A  Non-hyperlinked \url for local files
% \fi
% \ifx\phantomsection\undefined
%   \newcommand{\phantomsection}{}
% \fi
%
% ^^A  Change the way the index looks.
% \renewcommand{\usage}[1]{\textbf{#1}}
% \makeatletter
% \IndexPrologue{^^A
%   \phantomsection
%   \section*{Index}^^A
%   \addcontentsline{toc}{section}{Index}^^A
%   \markboth{Index}{Index}^^A
%   Numbers written in bold refer to the page
%   where the corresponding entry is described or referenced, the ones
%   underlined to the
%   \ifcodeline@index
%     code line of the
%   \fi
%   definition, the rest to the
%   \ifcodeline@index
%     code lines
%   \else
%     pages
%   \fi
%   where the entry is used.
% }
%
% ^^A  Index and categorize a special variable.  The optional argument
% ^^A  (#1), of the form "singular|plural" categorizes the environment in
% ^^A  the index.  The first required argument (#2) is the index term.  The
% ^^A  next argument (#3) is the index command, usually \special@index or
% ^^A  \index.  The final argument (#4) is the page format, probably "main"
% ^^A  or "usage".
% \newcommand{\specialindexcat}[4][]{^^A
%   \def\oc@category{#1}^^A
%   \def\oc@singular##1|##2|{##1}^^A
%   \def\oc@plural##1|##2|{##2}^^A
%   \ifx\oc@category\empty
%     #3{^^A
%       #2^^A
%       \actualchar{\string\ttfamily\space#2}^^A
%       \encapchar #4}^^A
%   \else
%     #3{^^A
%       #2^^A
%       \actualchar{\string\ttfamily\space#2} (\oc@singular#1|)^^A
%       \encapchar #4}^^A
%     #3{^^A
%       \oc@plural#1|^^A
%       \levelchar#2^^A
%       \actualchar{\string\ttfamily\space#2}^^A
%       \encapchar #4}^^A
%   \fi
% }
%
% ^^A  Define an environment similar to "environment" but designed for
% ^^A  definitions of things other than macros or environments.  The optional
% ^^A  argument (#1), of the form "singular|plural" categorizes the
% ^^A  environment in the index.
% \makeatletter
% \newenvironment{othercode}[2][]{^^A
%   \def\SpecialMainEnvIndex##1{^^A
%     \@bsphack
%     \specialindexcat[#1]{##1}{\special@index}{main}
%     \@esphack}^^A
%   \begin{environment}{#2}^^A
% }{^^A
%   \end{environment}^^A
% }
% \makeatother
%
% ^^A  Define a macro similar to "DescribeEnv" but designed for
% ^^A  descriptions of things other than macros or environments.  The
% ^^A  optional argument (#1), of the form "singular|plural" categorizes
% ^^A  the environment in the index.
% \makeatletter
% \def\DescribeOther{^^A
%   \leavevmode\@bsphack\begingroup\MakePrivateLetters
%   \@ifnextchar[{\Describe@Other}{\Describe@Other[|]}}
% \def\Describe@Other[#1]#2{\endgroup
%   \marginpar{\raggedleft\PrintDescribeEnv{#2}}^^A
%   \specialindexcat[#1]{#2}{\index}{usage}^^A
%   \@esphack\ignorespaces}
% \makeatother
%
% ^^A  Define commands for packages and for the string "Comic Sans".
% \newcommand{\pkgname}[1]{^^A
%   \textsf{#1}^^A
%   \index{#1\actualchar\textsf{#1} (package)\encapchar usage}^^A
%   \index{packages\levelchar#1\actualchar\textsf{#1}\encapchar usage}^^A
% }
% \newcommand{\fname}[2][usage]{^^A
%   \texttt{#2}\specialindexcat[file|files]{#2}{\index}{#1}^^A
% }
% \newcommand{\comsan}{Comic~Sans\xspace}
%
% ^^A  Define a hyphenless language to use for typesetting code listings.
% \newlanguage\hyphenlesslang\relax
%
% ^^A  %%%%%%%%%%%%%%%%%%%%%%%%%%%%%%%%%%%%%%%%%%%%%%%%%%%%%%%%%%%%%%%%%%%%%%
%
% \maketitle
% \sloppy
%
% \section{Introduction}
%
% The \pkgname{comicsans} package makes Microsoft's \comsan font
% available to \LaTeXe.  \pkgname{comicsans} supports all of the
% following:
%
% \begin{itemize}
%   \item Roman text, \textbf{boldface text}, \textsc{small-caps text},
%   and---with a little extra effort---\textit{italic text}
%
%   \item {\usefont{T2A}{comic}{m}{n}
%   \CYRK\cyri\cyrr\cyri\cyrl\cyri\cyrc\cyra{}
%   (\cyrr\cyri\cyrm\cyrs\cyrk\cyri\cyrishrt{}
%   \cyrsh\cyrr\cyri\cyrf\cyrt,
%   \textbf{\cyrzh\cyri\cyrr\cyrn\cyrery\cyrishrt{}
%   \cyrsh\cyrr\cyri\cyrf\cyrt},
%   \textit{\cyrk\cyra\cyrl\cyrl\cyri\cyrg\cyrr\cyra\cyrf\cyri\cyrch\cyre\cyrs\cyrk\cyri\cyrishrt{}
%   \cyrsh\cyrr\cyri\cyrf\cyrt})}
%
%   \item Mathematics using \comsan wherever possible:
%   \[
%     \textstyle
%     y'(x) \approx 3 \times 10^{\log_3 2\hat{\varepsilon}} +
%     \sum_{k=x}^\infty \frac{\xi_k}{p_{k-1}}
%   \]
% \end{itemize}
%
% \comsan is a TrueType~(TTF) font.  As such, it works particularly well
% with pdf\LaTeX{}, which natively supports TrueType fonts.  Some \TeX{}
% distributions also support dynamic conversion of TTF to PK (a
% bitmapped font format long used by \TeX) so \TeX{} backends other than
% pdf\TeX{} can (indirectly) utilize TrueType fonts, as well.
%
%
% \section{Installation}
% \label{sec:installation}
%
% The following is a brief summary of the \pkgname{comicsans} installation
% procedure:
%
% \begin{enumerate}
%   \item Acquire and install the \comsan TrueType (\texttt{.ttf}) files.
%   \item{} [Optional] Generate the italic and/or Cyrillic variants
%     of \comsan
%   \item Install the \pkgname{comicsans} font files and refresh the \TeX{}
%     filename database.
%   \item Point the \TeX\ backends to the \pkgname{comicsans} files.
% \end{enumerate}
%
% \noindent
% Details are presented in
% Sections~\ref{sec:acquire-ttf}--\ref{sec:install-map}.
%
% \subsection{Acquire and install the TrueType files}
% \label{sec:acquire-ttf}
% \pkgname{comicsans} requires the \comsan and \comsan Bold TrueType files
% (\fname{comic.ttf} and \fname{comicbd.ttf}).  You may already have these
% installed.  (On Windows, look in |C:\WINDOWS\Fonts| for \texttt{Comic
% Sans MS (TrueType)} and \texttt{Comic Sans MS Bold (TrueType)}.)  If
% not, see if a package called
% \textsf{msttcorefonts}\index{msttcorefonts=\textsf{msttcorefonts}|usage}
% is available for your operating system or operating-system
% distribution.  If not, then download \fname{comic32.exe} from
% \url{http://corefonts.sourceforge.net/} and use the freely available
% \href{http://www.kyz.uklinux.net/cabextract.php3}{\texttt{cabextract}}
% utility to extract \fname{comic.ttf} and \fname{comicbd.ttf} from
% \fname{comic32.exe}.
%
% Install \fname{comic.ttf} and \fname{comicbd.ttf} in an appropriate,
% \TeX-accessible location such as
% \localfile{/usr/local/share/texmf/fonts/ttf/microsoft/comicsans/}.
% (\TeX{} distributions for Microsoft Windows may automatically search
% the system font directory but I haven't yet tested this hypothesis.)
%
% \subsection{Generate the italic and/or Cyrillic variants (optional)}
% \label{sec:italic-cyrillic}
% To use the T2A-encoded Cyrillic versions of \comsan you'll need to
% install the \pkgname{cyrfinst} package, which is available from
% \href{http://www.ctan.org}{CTAN}.\footnote{In practice only
% \href{http://www.ctan.org/tex-archive/macros/latex/contrib/supported/t2/enc-maps/encfiles/t2a.enc}{\texttt{t2a.enc}}
% need be installed.}
%
% Because Microsoft doesn't make a \comsan Italic, and because TTF
% fonts don't accept the |SlantFont| modification, we need some way of
% handling italicized text.  The best alternative is to convert the TTF
% fonts to PostScript Type~1 format and use |SlantFont| to dynamically
% create oblique variants.  It may be possible to use
% \href{http://ttf2pt1.sourceforge.net}{\texttt{ttf2pt1}} to do the
% conversion but I don't know how to specify the various \TeX{} font
% encodings.  Instead, I use a (free) program called
% \href{http://fontforge.sourceforge.net}{FontForge}\index{FontForge|usage}
% to convert TTF to Type~1:
%
% \begin{description}
%   \item[\TeX{} base~1 (8r) encoding] Open \fname{comic.ttf} in
%   FontForge\index{FontForge|usage}.  Select
%   \textsf{Element}$\rightarrow$\textsf{Font Info\dots}, click on the
%   \textsf{Encoding} tab, and select ``\textsf{T$\varepsilon$X
%   Base~(8r)}'' for the encoding.  Click \textsf{OK}.  Go to
%   \textsf{File}$\rightarrow$\textsf{Generate Fonts\dots}\ and create
%   \fname{rcomic8r.pfb}.  Follow an analogous procedure to generate
%   \fname{rcomicbd8r.pfb} from \fname{comicbd.ttf}.
%
%   \item[T2A Adobe encoding (Cyrillic)] Follow the same steps as above,
%   but for \textsf{Encoding}, click on \textsf{Load}, select the
%   \fname{t2a.enc} file, then choose \textsf{T2AAdobeEncoding} for the
%   encoding.  Generate \fname{rcomiccyr.pfb} from \fname{comic.ttf} and
%   \fname{rcomiccyrbd.pfb} from \fname{comicbd.ttf}.
% \end{description}
%
% \noindent
% If you're unable to run FontForge\index{FontForge|usage} on your
% system and you can't find an alternate TTF$\rightarrow$PFB converter,
% don't worry.  Although you won't be able to typeset italics,
% Section~\ref{sec:usage} describes some \pkgname{comicsans} package
% options that make \comsan utilize either underlined or boldfaced text
% for emphasis.
%
% \subsection{Install font files and refresh \TeX's database}
% The \pkgname{comicsans} package consists of a large number of font
% files.  These are organized in a
% \href{http://www.tex.ac.uk/cgi-bin/texfaq2html?label=tds}{TDS-compliant}
% subdirectory rooted at |texmf|.  You should be able to copy
% \pkgname{comicsans}'s |texmf| tree directly onto your \TeX{} tree
% (i.e.,~|/usr/local/share/texmf|, |C:\localtexmf|, or wherever you
% normally install \TeX{} files).  If you generated italic and/or
% Cyrillic \comsan fonts (Section~\ref{sec:italic-cyrillic}), install
% the corresponding \texttt{.pfb} files as well, typically in
% \localfile{texmf/fonts/type1/microsoft/comicsans}.  Don't forget to
% refresh the filename database if necessary.  See
% \url{http://www.tex.ac.uk/cgi-bin/texfaq2html?label=inst-wlcf} for
% details specific to your \TeX\ distribution.
%
% \subsection{Point the \TeX\ backends to the \pkgname{comicsans} files}
% \label{sec:install-map}
% Most \TeX{} backends (pdf\TeX, Dvips, YAP, Xdvi, etc.)\ need
% to incorporate the contents of \fname{comicsans.map} into their
% private font-map files.  The exact procedure varies from one \TeX{}
% distribution to another.  See
% \url{http://www.tex.ac.uk/cgi-bin/texfaq2html?label=instt1font} for
% distribution-specific instructions on how to automatically update all
% of the various font-map files at once.
%
%
% \subsection*{Notes}
%
% \begin{enumerate}
%   \item The \comsan math fonts don't seem to work properly in older
%   versions of pdf\TeX~($\leq\,13x$).  If you have problems you should
%   upgrade to a newer version.
%
%   \item It \emph{is} possible to get Dvips to use a vector
%   (i.e.,~Type~1) version of \comsan.  If you have the patience, the
%   following is the procedure.  First, for \emph{each}
%   non-|SlantFont|ed line of \fname{comicsans.map}, you'll need a
%   separate Type~1~(|.pfb|) file---eight altogether---each with a
%   different encoding and PostScript font name.  I used
%   FontForge\index{FontForge|usage} to produce these.  For example, I
%   created an \fname{rcomic7m.pfb} file with the PostScript name
%   ``|ComicSansMS-7m|'' and with \fname{texmital.enc} as the encoding
%   vector.  Next, store all of these |.pfb| files in a directory that
%   Dvips searches.  Finally, create a modified \fname{comicsans.map}
%   that omits the encodings (as the |.pfb| files are already properly
%   encoded at this point).  It should look something like the
%   following:
%
% \begingroup\fussy
% \begin{verbatim}
%  rcomic8r ComicSansMS <rcomic8r.pfb
%  rcomicbd8r ComicSansMS-Bold <rcomicbd8r.pfb
%  rcomiccyr ComicSansMS-t2a <rcomict2a.pfb
%  rcomiccyrbd ComicSansMS-Bold-t2a <rcomicbdt2a.pfb
%  rcomic7m ComicSansMS-7m <rcomic7m.pfb
%  rcomicbd7m ComicSansMS-Bold-7m <rcomicbd7m.pfb
%  rcomic7y ComicSansMS-7y <rcomic7y.pfb
%  rcomic9z ComicSansMS-9z <rcomic9z.pfb
%  rcomico8r ComicSansMS "0.167 SlantFont" <rcomic8r.pfb
%  rcomicbdo8r ComicSansMS "0.167 SlantFont" <rcomicbd8r.pfb
%  rcomiccyro ComicSansMS-t2a "0.167 SlantFont" <rcomict2a.pfb
%  rcomiccyrbdo ComicSansMS-Bold-t2a "0.167 SlantFont" <rcomicbdt2a.pfb
% \end{verbatim}
% \endgroup
% \end{enumerate}
%
%
% \section{Usage}
% \label{sec:usage}
%
% Load \pkgname{comicsans} like any other \LaTeXe{} package, by putting
% ``|\usepackage{comicsans}|'' in your document's preamble.  This sets
% the default roman, typewriter, and sans-serif typefaces as shown in
% Table~\ref{tbl:cs-fonts}.  Courier Bold is typeset 10\%~larger than
% the requested point size.  This provides a better visual match to
% \comsan.
%
% \begin{table}[htbp]
%   \centering
%   \begin{tabular}{@{}lll@{}}
%     \hline
%     \multicolumn{1}{@{}c}{Style} &
%     \multicolumn{1}{c}{Default} &
%     \multicolumn{1}{c@{}}{With \pkgname{comicsans}} \\
%     \hline
%
%     Roman &
%     \usefont{OT1}{cmr}{m}{n} Computer Modern &
%     \comsan \\
%
%     Typewriter &
%     \usefont{OT1}{cmtt}{m}{n} Computer Modern Typewriter &
%     \texttt{Courier Bold} \\
%
%     Sans-serif &
%     \usefont{OT1}{cmss}{m}{n} Computer Modern Sans Serif &
%     \textsf{Helvetica} \\
%
%     \hline
%   \end{tabular}
%   \caption{\pkgname{comicsans} font-family redefinitions}
%   \label{tbl:cs-fonts}
% \end{table}
%
% \DescribeOther[package option|package options]{ulemph}
% \LaTeX's |\emph| is usually defined to produce italics.
% Unfortunately, \comsan doesn't include an italic variant.  One
% alternative is to generate a slanted PostScript version of \comsan
% as described in Section~\ref{sec:installation}.  If this is too
% inconvenient or impossible an alternative is to use
% \pkgname{comicsans}'s |ulemph| package option.  With |ulemph|,
% \pkgname{comicsans} utilizes the \pkgname{soul} package's underlining
% capabilities to typeset emphasized text \ul{like this}.  The
% drawback---apart from being ugly---is that underlining is limited to
% |\emph|; it doesn't work with |\em| or any of the italic macros
% (|\textit|, |\itshape|, |\it|, etc.), which are redefined as
% do-nothing commands.  Also, underlined emphasis tends to fail when
% used in math mode.
%
% \DescribeOther[package option|package options]{boldemph}
% The |boldemph| package option, like |ulemph|, alters the way that
% emphasized text is rendered in \LaTeX\@.  |boldemph| typesets |\emph|
% and |\em| in boldface \textbf{like this}.  The various italic macros
% are redefined as do-nothing commands.
%
% \DescribeOther[package option|package options]{largesymbols}
% Mathematical typesetting is clearly not a priority to Microsoft.  As a
% result \comsan lacks most of the math characters that \TeX{} requires.
% The \pkgname{comicsans} package utilizes characters from the Computer
% Modern family to make up for this absense.  While many of the
% characters are more-or-less compatible, the large symbols, with their
% thin strokes and serifed ends, particularly stand out to my eye:
%
% \begingroup
%   \[
%     y'(x) \approx 3 \times 10^{\log_3 2\hat{\varepsilon}} +
%     \cmsum_{k=x}^\infty \frac{\xi_k}{p_{k-1}}
%   \]
% \endgroup
%
% \noindent
% The |largesymbols| package option uses \comsan for a number of
% additional large symbols.  The advantage of |largesymbols| is that
% more mathematical characters match the body font.  The
% disadvantage---and the reason that |largesymbols| is off by
% default---is that the large symbols are merely scaled versions of
% their smaller counterparts, which unfortunately implies that their
% thickness scales as well:
%
% \begingroup
%   \[
%     y'(x) \approx 3 \times 10^{\log_3 2\hat{\varepsilon}} +
%     \sum_{k=x}^\infty \frac{\xi_k}{p_{k-1}}
%   \]
% \endgroup
%
% \noindent
% With the |largesymbols| package option \pkgname{comicsans} gives you
% the ability to decide for yourself which is the lesser of the two
% evils.
%
% \DescribeOther[package option|package options]{plusminus}
% \LaTeX{} defines |\pm| as ``{\usefont{OMS}{cmsy}{m}{n}\char6}'' and
% |\mp| as ``{\usefont{OMS}{cmsy}{m}{n}\char7}''---both taken from the
% Computer Modern Symbol font.  Although \comsan provides a
% plus-or-minus glyph it lacks a corresponding minus-or-plus glyph.  For
% consistency between the two glyphs \pkgname{comicsans} draws both
% plus-or-minus and minus-or-plus from the Computer Modern Bold Symbol
% font: ``{\usefont{OMS}{cmsy}{b}{n}\char6}'' and ``$\mp$''.  The
% |plusminus| package option retains |\mp| as ``$\mp$'' but uses
% \comsan's ``$\pm$'' for |\pm|.  This enables |\pm| to blend better
% with other \comsan characters at the expense of looking quite
% different from |\mp|.
%
%
% \StopEventually{^^A
%
% \section{Copyright and license agreement}
% \label{sec:license}
%
% Copyright \textcopyright{}~2013 by Scott Pakin
%
% \bigskip
%
% \noindent
% This file may be distributed and/or modified under the conditions of
% the \LaTeX{} Project Public License, either version~1.3c of this
% license or (at your option) any later version.  The latest version of
% this license is at \url{http://www.latex-project.org/lppl.txt} and
% version~1.3c or later is part of all distributions of \LaTeX{}
% version~2006/05/20 or later.
% }
%
%
% \section{Implementation: Core components}
%
% This section and the subsequent one contain the commented source code
% for the \pkgname{comicsans} package.  They are likely of little
% interest to the average user and can safely be ignored.  Advanced
% users who want to customize or extend \pkgname{comicsans}---please
% read the license agreement (Section~\ref{sec:license}) first---can use
% these sections to gain a detailed understanding of the code.
%
% \subsection{\texttt{comicsans.sty}}
%
% This is the \pkgname{comicsans} package proper.  It's primary purpose
% is to select \comsan as the default font for text and math.
%
%<*package>
%
% \subsubsection{Option processing}
% \label{sec:opt-proc}
%
% \begin{macro}{\if@ulemph}
% \begin{macro}{\@ulemphtrue}
% \begin{macro}{\@ulemphfalse}
% The author can use underlining for emphasis
% (Section~\ref{sec:emphasis}) using the |ulemph| option.
%    \begin{macrocode}
\newif\if@ulemph \DeclareOption{ulemph}{\@ulemphtrue\@boldemphfalse}
%    \end{macrocode}
% \end{macro}
% \end{macro}
% \end{macro}
%
% \begin{macro}{\if@boldemph}
% \begin{macro}{\@boldemphtrue}
% \begin{macro}{\@boldemphfalse}
% The author can use boldface for emphasis (Section~\ref{sec:emphasis})
% using the |boldemph| option.
%    \begin{macrocode}
\newif\if@boldemph
\DeclareOption{boldemph}{\@boldemphtrue\@ulemphfalse}
%    \end{macrocode}
% \end{macro}
% \end{macro}
% \end{macro}
%
% Using large, mathematical symbols in \comsan is still fairly
% experimental (read as:~ugly).  These symbols are disabled by default,
% but the author can enable them with the |largesymbols| option.
%    \begin{macrocode}
\DeclareOption{largesymbols}{%
  \DeclareSymbolFont{largesymbols}{OMX}{comic}{m}{n}%
}
%    \end{macrocode}
%
% \begin{macro}{\if@csplusminus}
% \begin{macro}{\@csplusminustrue}
% \begin{macro}{\@csplusminusfalse}
% \comsan defines a |plusminus| character (``$\pm$'') but not a
% corresponding |minusplus| character.  For consistency we normally draw
% both |plusminus| and |minusplus| from Computer Modern
% (``{\usefont{OMS}{cmsy}{b}{n}\char6}'' and ``$\mp$'').  However, the
% |plusminus| package option makes |\pm| match other \comsan symbols at
% the expense of not matching |\mp|.
%    \begin{macrocode}
\newif\if@csplusminus
\DeclareOption{plusminus}{\@csplusminustrue}
%    \end{macrocode}
% \end{macro}
% \end{macro}
% \end{macro}
%
% Finally, we process the package options.
%    \begin{macrocode}
\ProcessOptions\relax
%    \end{macrocode}
%
%
% \subsubsection{Default font families}
%
% \begin{macro}{\rmdefault}
% \begin{macro}{\ttdefault}
% \begin{macro}{\sfdefault}
% We select \comsan as the default body font, Courier as the default
% fixed-width font, and Helvetica as the default sans-serif font.  (Yes,
% this is a bit odd, given that \comsan is already sans-serif.)
%    \begin{macrocode}
\renewcommand{\rmdefault}{comic}
\renewcommand{\ttdefault}{pcr}
\renewcommand{\sfdefault}{phv}
%    \end{macrocode}
% \end{macro}
% \end{macro}
% \end{macro}
%
% We redefine Courier Medium as Courier Bold and Courier Italic as
% Courier Bold Oblique in the OT1 font encoding.  We also increase the
% size by~10\% to better match \comsan.
%    \begin{macrocode}
\DeclareFontFamily{OT1}{pcr}{}
\DeclareFontShape{OT1}{pcr}{b}{n}{
   <-> s * [1.1] pcrb7t
}{}
\DeclareFontShape{OT1}{pcr}{b}{it}{
   <-> s * [1.1] pcrbo7t
}{}
\DeclareFontShape{OT1}{pcr}{m}{n}{<->ssub * pcr/b/n}{}
\DeclareFontShape{OT1}{pcr}{bx}{n}{<->ssub * pcr/b/n}{}
\DeclareFontShape{OT1}{pcr}{m}{it}{<->ssub * pcr/b/it}{}
\DeclareFontShape{OT1}{pcr}{bx}{it}{<->ssub * pcr/b/it}{}
%    \end{macrocode}
% We now do the same for the T1 font encoding\dots
%    \begin{macrocode}
\DeclareFontFamily{T1}{pcr}{}
\DeclareFontShape{T1}{pcr}{b}{n}{
   <-> s * [1.1] pcrb8t
}{}
\DeclareFontShape{T1}{pcr}{b}{it}{
   <-> s * [1.1] pcrbo8t
}{}
\DeclareFontShape{T1}{pcr}{m}{n}{<->ssub * pcr/b/n}{}
\DeclareFontShape{T1}{pcr}{bx}{n}{<->ssub * pcr/b/n}{}
\DeclareFontShape{T1}{pcr}{m}{it}{<->ssub * pcr/b/it}{}
\DeclareFontShape{T1}{pcr}{bx}{it}{<->ssub * pcr/b/it}{}
%    \end{macrocode}
% \dots and the TS1 font encoding.  We first ensure that the
% \pkgname{textcomp} package is preloaded to avoid getting an
% ``\texttt{Encoding scheme `TS1' unknown}'' error.
%    \begin{macrocode}
\RequirePackage{textcomp}
\DeclareFontFamily{TS1}{pcr}{}
\DeclareFontShape{TS1}{pcr}{b}{n}{
   <-> s * [1.1] pcrb8c
}{}
\DeclareFontShape{TS1}{pcr}{b}{it}{
   <-> s * [1.1] pcrbo8c
}{}
\DeclareFontShape{TS1}{pcr}{m}{n}{<->ssub * pcr/b/n}{}
\DeclareFontShape{TS1}{pcr}{bx}{n}{<->ssub * pcr/b/n}{}
\DeclareFontShape{TS1}{pcr}{m}{it}{<->ssub * pcr/b/it}{}
\DeclareFontShape{TS1}{pcr}{bx}{it}{<->ssub * pcr/b/it}{}
%    \end{macrocode}
%
% If the |plusminus| package option was specified we draw |\textpm| from
% |\comic9z|---the only \comsan font encoding that takes a |plusminus|
% character from \comsan instead of borrowing the one from Computer
% Modern Bold Symbol.
%    \begin{macrocode}
\if@csplusminus
  \DeclareTextSymbolDefault{\textpm}{U}
  \DeclareTextSymbol{\textpm}{U}{4}
\fi
%    \end{macrocode}
%
%
% \subsubsection{Emphasis}
% \label{sec:emphasis}
%
% Because Microsoft doesn't make a \comsan Italic and because TTF
% fonts don't accept the |SlantFont| modification we need some way of
% handling emphasized text.  The best alternative is to use a program
% such as FontForge\index{FontForge|usage} to convert the TTF fonts to
% PostScript Type 1 format (Section~\ref{sec:installation}).  Failing
% that, the author can specify with the |boldemph| package option that
% bold text should be used whenever emphasized text is requested.  An
% alternative, with the |ulemph| package option, is to utilize the
% \pkgname{soul} package to replace emphasis with underlining.
% Unfortunately, \pkgname{soul} doesn't provide a way to enable
% underlining until the end of the current group (as is needed for
% \LaTeX~2.09's |{\em|~\dots|}| construct).  Furthermore, \pkgname{soul}
% tends to choke on underlined mathematics.
%
% \bigskip
%
% If |boldemph| was given as a package option we utilize bold text for
% emphasis.  Because we lack a true italic---or even an oblique variant
% of \comsan---we replace all of the explicit italic commands with
% |\relax|.
%    \begin{macrocode}
\if@boldemph
  \let\emph=\textbf
  \let\em=\bf
  \let\itshape=\relax
  \let\it=\relax
\fi
%    \end{macrocode}
%
% If |ulemph| was given as a package option we utilize underlined text
% for emphasis.  This requires the \pkgname{soul} package.  Because we
% lack a true italic---or even an oblique variant of \comsan---we
% replace all of the explicit italic commands with |\relax|.
%    \begin{macrocode}
\if@ulemph
  \RequirePackage{soul}
  \setul{1.5pt}{1pt}
  \let\emph=\ul
  \let\itshape=\relax
  \let\it=\relax
%    \end{macrocode}
% Out of necessity, we unfortunately also have to make |\em| a
% do-nothing command.
%    \begin{macrocode}
  \let\em=\relax
\fi
%    \end{macrocode}
%
%
% \subsubsection{Mathematics}
% \label{sec:cs-math}
%
% \begin{othercode}[math font|math fonts]{operators}
% \begin{othercode}[math font|math fonts]{letters}
% \begin{othercode}[math font|math fonts]{symbols}
% For mathematical expressions, we draw operators, letters, and symbols
% from \comsan.  Large symbols normally come from Computer Modern, but
% the |largesymbols| package option (Section~\ref{sec:opt-proc})
% specifies that they should come from \comsan, as well.
%    \begin{macrocode}
\DeclareSymbolFont{operators}{OT1}{comic}{m}{n}
\DeclareSymbolFont{letters}{OML}{comic}{m}{n}
\DeclareSymbolFont{symbols}{OMS}{comic}{m}{n}
%    \end{macrocode}
% \end{othercode}
% \end{othercode}
% \end{othercode}
%
% \begin{macro}{\neq}
% \begin{macro}{\pm}
% We define one additional symbol font, ``|othercomics|'', from which we
% define |\neq| as the glyph ``$\neq$'' and---if the |plusminus| package
% option was specified---|\pm| as the glyph ``$\pm$''.
%    \begin{macrocode}
\let\neq=\undefined
\DeclareSymbolFont{othercomics}{U}{comic}{m}{n}
\DeclareMathSymbol{\neq}{\mathrel}{othercomics}{3}
\if@csplusminus
  \DeclareMathSymbol{\pm}{\mathbin}{othercomics}{4}
\fi
%    \end{macrocode}
% \end{macro}
% \end{macro}
%
% \begin{macro}{\frac}
% \TeX's default fraction bar is much too thin for \comsan.  We
% therefore redefine |\frac| to use a fraction bar with a more
% compatible thickness.
%    \begin{macrocode}
\def\frac#1#2{{%
  \begingroup#1\endgroup\abovewithdelims..0.75pt#2}}
%    \end{macrocode}
% \end{macro}
%
%</package>
%
%
% \subsection{\texttt{comicsans.map}}
% \label{sec:mapfile}
%
% This is a map file for pdf\LaTeX{} that provides the association
% between TFM names (e.g.,~\fname{rcomic8r}) and PostScript names
% (e.g.,~|ComicSansMS|).  It also specifies how fonts should be
% re-encoded so that characters appear at the expected offsets in each
% font.
%
%<*comicsans.map>
% {\language\hyphenlesslang
%    \begin{macrocode}
rcomic8r ComicSansMS "TeXBase1Encoding ReEncodeFont" <8r.enc <comic.ttf
rcomicbd8r ComicSansMS-Bold "TeXBase1Encoding ReEncodeFont" <8r.enc <comicbd.ttf
rcomiccyr ComicSansMS "T2AAdobeEncoding ReEncodeFont" <t2a.enc <comic.ttf
rcomiccyrbd ComicSansMS-Bold "T2AAdobeEncoding ReEncodeFont" <t2a.enc <comicbd.ttf
rcomic7m ComicSansMS "TeXMathItalicEncoding ReEncodeFont" <texmital.enc <comic.ttf
rcomicbd7m ComicSansMS-Bold "TeXMathItalicEncoding ReEncodeFont" <texmital.enc <comicbd.ttf
rcomic7y ComicSansMS "TeXMathSymbolEncoding ReEncodeFont" <texmsym.enc <comic.ttf
rcomic9z ComicSansMS "ComicSansExtraEncoding ReEncodeFont" <csextras.enc <comic.ttf
%    \end{macrocode}
%
% The following four lines assume that you have PostScript Type~1
% versions of the various \comsan fonts.  Although
% Section~\ref{sec:installation} describes a technique for converting
% TrueType to Type~1, my understanding of copyright law is that I am not
% allowed to distribute \fname{rcomico8r.pfb} or \fname{rcomicbdo8r.pfb}
% myself as these are considered derivitive works from \fname{comic.ttf}
% and \fname{comicbd.ttf}.
%    \begin{macrocode}
rcomico8r ComicSansMS "0.167 SlantFont" <rcomic8r.pfb
rcomicbdo8r ComicSansMS-Bold "0.167 SlantFont" <rcomicbd8r.pfb
rcomiccyro ComicSansMS "0.167 SlantFont" <rcomiccyr.pfb
rcomiccyrbdo ComicSansMS-Bold "0.167 SlantFont" <rcomiccyrbd.pfb
%    \end{macrocode}
% }
%
%</comicsans.map>
%
%
% \subsection{\texttt{csextras.enc}}
%
% \fname{csextras.enc} is an encoding file that tells the pdf\LaTeX{}
% backend how to reorder the glyphs in \fname{comic.ttf} to match the
% order expected by \fname{rcomic9z.tfm}.  \fname{csextras.enc}
% specifies only those glyphs that \fname{rcomic9z.tfm} uses (the
% \pkgname{comicsans} ``extra'' glyphs).
%
%<*csextras.enc>
% \begin{othercode}[encoding|encodings]{ComicSansExtraEncoding}
% \begin{othercode}[glyph|glyphs]{integral}
% \begin{othercode}[glyph|glyphs]{Sigma}
% \begin{othercode}[glyph|glyphs]{Pi}
% \begin{othercode}[glyph|glyphs]{notequal}
% \begin{othercode}[glyph|glyphs]{plusminus}
% This encoding defines |integral| (``$\int$''), |summation|
% (``$\sum$''), and |product| (``$\prod$'').  \fname{comic7v.vf} maps \TeX's
% \meta{symbol}|text| and \meta{symbol}|display| symbols onto these.  We
% also define |notequal| (``$\neq$'') because this looks better than the
% composite of |not| and |equal| (``$\not=$''); and we define
% |plusminus| (``$\pm$'') because |comic7y| uses |cmbsy10|'s |plusminus|
% character (``{\usefont{OMS}{cmsy}{m}{n}\char6}''), which better
% matches its |minusplus| (``$\mp$'').
%    \begin{macrocode}
/ComicSansExtraEncoding [
  /integral
%    \end{macrocode}
% The following two symbols are \emph{supposed} to be |/summation| and
% |/product|.  For some reason that I don't yet understand, pdf\LaTeX{}
% is unable to find those symbols in \fname{comic.ttf} even though
% FontForge\index{FontForge|usage} can.  As a workaround we use |/Sigma|
% and |/Pi|, which are sufficiently similar.
%    \begin{macrocode}
  /Sigma
  /Pi
  /notequal
  /plusminus
%    \end{macrocode}
%
% We pad the encoding to exactly 256~characters using |/.notdef|s, as
% some programs (e.g.,~|ttf2pk|) expect to see exactly 256 encoded
% characters.
%    \begin{macrocode}
  /.notdef /.notdef /.notdef /.notdef /.notdef
  /.notdef /.notdef /.notdef /.notdef /.notdef
  /.notdef /.notdef /.notdef /.notdef /.notdef
%    \end{macrocode}
% \centerline{$\vdots$}
% \iffalse
%    \begin{macrocode}
  /.notdef /.notdef /.notdef /.notdef /.notdef /.notdef /.notdef
  /.notdef /.notdef /.notdef /.notdef /.notdef /.notdef /.notdef
  /.notdef /.notdef /.notdef /.notdef /.notdef /.notdef /.notdef
  /.notdef /.notdef /.notdef /.notdef /.notdef /.notdef /.notdef
  /.notdef /.notdef /.notdef /.notdef /.notdef /.notdef /.notdef
  /.notdef /.notdef /.notdef /.notdef /.notdef /.notdef /.notdef
  /.notdef /.notdef /.notdef /.notdef /.notdef /.notdef /.notdef
  /.notdef /.notdef /.notdef /.notdef /.notdef /.notdef /.notdef
  /.notdef /.notdef /.notdef /.notdef /.notdef /.notdef /.notdef
  /.notdef /.notdef /.notdef /.notdef /.notdef /.notdef /.notdef
  /.notdef /.notdef /.notdef /.notdef /.notdef /.notdef /.notdef
  /.notdef /.notdef /.notdef /.notdef /.notdef /.notdef /.notdef
  /.notdef /.notdef /.notdef /.notdef /.notdef /.notdef /.notdef
  /.notdef /.notdef /.notdef /.notdef /.notdef /.notdef /.notdef
  /.notdef /.notdef /.notdef /.notdef /.notdef /.notdef /.notdef
  /.notdef /.notdef /.notdef /.notdef /.notdef /.notdef /.notdef
  /.notdef /.notdef /.notdef /.notdef /.notdef /.notdef /.notdef
  /.notdef /.notdef /.notdef /.notdef /.notdef /.notdef /.notdef
  /.notdef /.notdef /.notdef /.notdef /.notdef /.notdef /.notdef
  /.notdef /.notdef /.notdef /.notdef /.notdef /.notdef /.notdef
  /.notdef /.notdef /.notdef /.notdef /.notdef /.notdef /.notdef
  /.notdef /.notdef /.notdef /.notdef /.notdef /.notdef /.notdef
  /.notdef /.notdef /.notdef /.notdef /.notdef /.notdef /.notdef
  /.notdef /.notdef /.notdef /.notdef /.notdef /.notdef /.notdef
  /.notdef /.notdef /.notdef /.notdef /.notdef /.notdef /.notdef
  /.notdef /.notdef /.notdef /.notdef /.notdef /.notdef /.notdef
  /.notdef /.notdef /.notdef /.notdef /.notdef /.notdef /.notdef
  /.notdef /.notdef /.notdef /.notdef /.notdef /.notdef /.notdef
  /.notdef /.notdef /.notdef /.notdef /.notdef /.notdef /.notdef
  /.notdef /.notdef /.notdef /.notdef /.notdef /.notdef /.notdef
  /.notdef /.notdef /.notdef /.notdef /.notdef /.notdef /.notdef
  /.notdef /.notdef /.notdef /.notdef /.notdef /.notdef /.notdef
  /.notdef /.notdef /.notdef /.notdef /.notdef /.notdef /.notdef
%    \end{macrocode}
% \fi
%    \begin{macrocode}
  /.notdef /.notdef /.notdef /.notdef /.notdef
] def
%    \end{macrocode}
% \end{othercode}
% \end{othercode}
% \end{othercode}
% \end{othercode}
% \end{othercode}
% \end{othercode}
%</csextras.enc>
%
%
% \subsection{\texttt{ttfonts.map}}
%
% Dvips doesn't currently support TrueType fonts.  However, the |ttf2pk|
% utility (included with the FreeType library) can convert a TrueType
% font file (|.ttf|) into a \TeX{} packed-font file (|.pk|) for use with
% Dvips or similar tools.  |ttf2pk| requires a mapping file,
% |ttfonts.map|, which specifies the mapping between \TeX{} font names
% and the corresponding TrueType font file.
%
%<*ttfonts.map>
%
% The first part of \fname{ttfonts.map} contains analogous entries to those in
% \fname{comicsans.map} (Section~\ref{sec:mapfile}).
%
%    \begin{macrocode}
rcomic8r     comic.ttf    Encoding=8r.enc
rcomicbd8r   comicbd.ttf  Encoding=8r.enc
rcomiccyr    comic.ttf    Encoding=t2a.enc
rcomiccyrbd  comicbd.ttf  Encoding=t2a.enc
rcomic7m     comic.ttf    Encoding=texmital.enc
rcomicbd7m   comicbd.ttf  Encoding=texmital.enc
rcomic7y     comic.ttf    Encoding=texmsym.enc
rcomic9z     comic.ttf    Encoding=csextras.enc
%    \end{macrocode}
%
% Although pdf\LaTeX{} can dynamically slant only PostScript files, not
% TrueType files, |ttf2pk| has no such limitation when producing
% |.pk|~bitmaps.
%
%    \begin{macrocode}
rcomico8r     comic.ttf    Encoding=8r.enc  Slant=0.167
rcomicbdo8r   comicbd.ttf  Encoding=8r.enc  Slant=0.167
rcomiccyro    comic.ttf    Encoding=t2a.enc Slant=0.167
rcomiccyrbdo  comicbd.ttf  Encoding=t2a.enc Slant=0.167
%    \end{macrocode}
%
%</ttfonts.map>
%
%
% \section{Implementation: Extras}
%
% The files documented in this section are what I used to automate
% creation of the \TeX/\LaTeX{} bindings for \comsan.  They are needed
% only if you want to modify or extend these bindings.  Please read the
% license agreement (Section~\ref{sec:license}), however, before
% modifying any part of the \pkgname{comicsans} package.
%
%
% \subsection{\texttt{csextras.etx}}
%
% \fname{csextras.etx} is a \pkgname{fontinst} encoding file that is used to
% create \fname{rcomic9z.pl}.  It specifies all of the characters that should
% appear in \fname{rcomic9z.pl}.
%
% We start with some boilerplate initialization.
%
%<*csextras.etx>
%    \begin{macrocode}
\relax
\encoding
\needsfontinstversion{1.800}
%    \end{macrocode}
%
% Next, we specify the symbols that we're interested in.  We begin with
% the large \TeX{} symbols.
%
% \begin{othercode}[glyph|glyphs]{integral}
% ``$\int$''
%    \begin{macrocode}
\setslot{integral}
\endsetslot
%    \end{macrocode}
% \end{othercode}
%
% \begin{othercode}[glyph|glyphs]{summation}
% ``$\sum$''
%    \begin{macrocode}
\setslot{summation}
\endsetslot
%    \end{macrocode}
% \end{othercode}
%
% \begin{othercode}[glyph|glyphs]{product}
% ``$\prod$''
%    \begin{macrocode}
\setslot{product}
\endsetslot
%    \end{macrocode}
% \end{othercode}
%
% The remaining large symbols are all scaled versions of ordinary
% symbols---parentheses, brackets, braces, etc.---and hence don't need
% to appear in this file.  We therefore conclude with |notequal| (a
% nonstandard \TeX{} character) and |plusminus| (which already exists in
% |comic7y| but uses the Computer Modern Bold Symbol version).
%
% \begin{othercode}[glyph|glyphs]{notequal}
% ``$\neq$''
%    \begin{macrocode}
\setslot{notequal}
\endsetslot
%    \end{macrocode}
% \end{othercode}
%
% \begin{othercode}[glyph|glyphs]{plusminus}
% ``$\pm$''
%    \begin{macrocode}
\setslot{plusminus}
\endsetslot
\endencoding
%    \end{macrocode}
% \end{othercode}
%
%</csextras.etx>
%
%
% \subsection{\texttt{csextras.mtx}}
% \label{sec:csextras-mtx}
%
% \fname{csextras.mtx} is a \pkgname{fontinst} metrics file that is used
% to help create \fname{comic7v.vpl}.  \fname{csextras.mtx} maps \TeX{}
% glyph names such as ``|integraltext|'' to \comsan font names such as
% ``|integral|''.
%
% One problem is that \TeX{} defines ``text style'' (small) and
% ``display style'' (large) versions of various symbols, while \comsan
% typically defines only the small size.  We therefore do all that we
% can, which is to scale up the small version to a larger size.  The
% unfortunate result is that display-style symbols tend to be
% excessively thick.  C'est la vie.
%
% We start with some boilerplate initialization.
%
%<*csextras.mtx>
%    \begin{macrocode}
\relax
\metrics
%    \end{macrocode}
%
% \begin{macro}{\bigbiggerbiggest}
% To save typing, we create a macro that defines |\big|, |\Big|,
% |\bigg|, and |\Bigg| versions of a given symbol.
%    \begin{macrocode}
\setcommand\bigbiggerbiggest#1{%
  \setglyph{#1big}
    \glyph{#1}{1000}
  \endsetglyph
  \setglyph{#1Big}
    \glyph{#1}{2500}
  \endsetglyph
  \setglyph{#1bigg}
    \glyph{#1}{4000}
  \endsetglyph
  \setglyph{#1Bigg}
    \glyph{#1}{5500}
  \endsetglyph
}
%    \end{macrocode}
% \end{macro}
%
% \begin{othercode}[glyph|glyphs]{integraltext}
% \begin{othercode}[glyph|glyphs]{integraldisplay}
% Define ``$\textstyle\int$'' and ``$\displaystyle\int$''.
%    \begin{macrocode}
\setglyph{integraltext}
  \glyph{integral}{1000}
\endsetglyph
\setglyph{integraldisplay}
  \glyph{integral}{3000}
\endsetglyph
%    \end{macrocode}
% \end{othercode}
% \end{othercode}
%
% \begin{othercode}[glyph|glyphs]{summationtext}
% \begin{othercode}[glyph|glyphs]{summationdisplay}
% Define ``$\textstyle\sum$'' and ``$\displaystyle\sum$''.
%    \begin{macrocode}
\setglyph{summationtext}
  \glyph{summation}{1000}
\endsetglyph
\setglyph{summationdisplay}
  \glyph{summation}{3000}
\endsetglyph
%    \end{macrocode}
% \end{othercode}
% \end{othercode}
%
% \begin{othercode}[glyph|glyphs]{producttext}
% \begin{othercode}[glyph|glyphs]{productdisplay}
% Define ``$\textstyle\prod$'' and ``$\displaystyle\prod$''.
%    \begin{macrocode}
\setglyph{producttext}
  \glyph{product}{1000}
\endsetglyph
\setglyph{productdisplay}
  \glyph{product}{3000}
\endsetglyph
%    \end{macrocode}
% \end{othercode}
% \end{othercode}
%
% \begin{othercode}[glyph|glyphs]{parenleftbig}
% \begin{othercode}[glyph|glyphs]{parenleftBig}
% \begin{othercode}[glyph|glyphs]{parenleftbigg}
% \begin{othercode}[glyph|glyphs]{parenleftBigg}
% \begin{othercode}[glyph|glyphs]{parenrightbig}
% \begin{othercode}[glyph|glyphs]{parenrightBig}
% \begin{othercode}[glyph|glyphs]{parenrightbigg}
% \begin{othercode}[glyph|glyphs]{parenrightBigg}
% Define a range of sizes for ``(`` and ``)''.
%    \begin{macrocode}
\bigbiggerbiggest{parenleft}
\bigbiggerbiggest{parenright}
%    \end{macrocode}
% \vspace{5\baselineskip}
% \end{othercode}
% \end{othercode}
% \end{othercode}
% \end{othercode}
% \end{othercode}
% \end{othercode}
% \end{othercode}
% \end{othercode}
%
% \begin{othercode}[glyph|glyphs]{bracketleftbig}
% \begin{othercode}[glyph|glyphs]{bracketleftBig}
% \begin{othercode}[glyph|glyphs]{bracketleftbigg}
% \begin{othercode}[glyph|glyphs]{bracketleftBigg}
% \begin{othercode}[glyph|glyphs]{bracketrightbig}
% \begin{othercode}[glyph|glyphs]{bracketrightBig}
% \begin{othercode}[glyph|glyphs]{bracketrightbigg}
% \begin{othercode}[glyph|glyphs]{bracketrightBigg}
% Define a range of sizes for ``[`` and ``]''.
%    \begin{macrocode}
\bigbiggerbiggest{bracketleft}
\bigbiggerbiggest{bracketright}
%    \end{macrocode}
% \vspace{5\baselineskip}
% \end{othercode}
% \end{othercode}
% \end{othercode}
% \end{othercode}
% \end{othercode}
% \end{othercode}
% \end{othercode}
% \end{othercode}
%
% \begin{othercode}[glyph|glyphs]{braceleftbig}
% \begin{othercode}[glyph|glyphs]{braceleftBig}
% \begin{othercode}[glyph|glyphs]{braceleftbigg}
% \begin{othercode}[glyph|glyphs]{braceleftBigg}
% \begin{othercode}[glyph|glyphs]{bracerightbig}
% \begin{othercode}[glyph|glyphs]{bracerightBig}
% \begin{othercode}[glyph|glyphs]{bracerightbigg}
% \begin{othercode}[glyph|glyphs]{bracerightBigg}
% Define a range of sizes for ``\{`` and ``\}''.
%    \begin{macrocode}
\bigbiggerbiggest{braceleft}
\bigbiggerbiggest{braceright}
%    \end{macrocode}
% \vspace{5\baselineskip}
% \end{othercode}
% \end{othercode}
% \end{othercode}
% \end{othercode}
% \end{othercode}
% \end{othercode}
% \end{othercode}
% \end{othercode}
%
% \begin{othercode}[glyph|glyphs]{slashbig}
% \begin{othercode}[glyph|glyphs]{slashBig}
% \begin{othercode}[glyph|glyphs]{slashbigg}
% \begin{othercode}[glyph|glyphs]{slashBigg}
% \begin{othercode}[glyph|glyphs]{backslashbig}
% \begin{othercode}[glyph|glyphs]{backslashBig}
% \begin{othercode}[glyph|glyphs]{backslashbigg}
% \begin{othercode}[glyph|glyphs]{backslashBigg}
% Define a range of sizes for ``/`` and ``\textbackslash''.
%    \begin{macrocode}
\bigbiggerbiggest{slash}
\bigbiggerbiggest{backslash}
%    \end{macrocode}
% \vspace{5\baselineskip}
% \end{othercode}
% \end{othercode}
% \end{othercode}
% \end{othercode}
% \end{othercode}
% \end{othercode}
% \end{othercode}
% \end{othercode}
%
% \begin{othercode}[glyph|glyphs]{angleleftbig}
% \begin{othercode}[glyph|glyphs]{angleleftBig}
% \begin{othercode}[glyph|glyphs]{angleleftbigg}
% \begin{othercode}[glyph|glyphs]{angleleftBigg}
% \begin{othercode}[glyph|glyphs]{anglerightbig}
% \begin{othercode}[glyph|glyphs]{anglerightBig}
% \begin{othercode}[glyph|glyphs]{anglerightbigg}
% \begin{othercode}[glyph|glyphs]{anglerightBigg}
% Define a range of sizes for ``$\langle$`` and ``$\rangle$'' (really
% ``$<$'' and ``$>$'').  Because the naming is inconsistent between
% \comsan and \TeX{} (``|angleleft|'' vs.\ ``|less|'') we can't use our
% |\bigbiggerbiggest| macro.
%    \begin{macrocode}
\setglyph{angleleftbig}
  \glyph{less}{1000}
\endsetglyph
\setglyph{angleleftBig}
  \glyph{less}{2500}
\endsetglyph
\setglyph{angleleftbigg}
  \glyph{less}{4000}
\endsetglyph
\setglyph{angleleftBigg}
  \glyph{less}{5500}
\endsetglyph
%    \end{macrocode}
% \smallskip
%    \begin{macrocode}
\setglyph{anglerightbig}
  \glyph{greater}{1000}
\endsetglyph
\setglyph{anglerightBig}
  \glyph{greater}{2500}
\endsetglyph
\setglyph{anglerightbigg}
  \glyph{greater}{4000}
\endsetglyph
\setglyph{anglerightBigg}
  \glyph{greater}{5500}
\endsetglyph
%    \end{macrocode}
% \end{othercode}
% \end{othercode}
% \end{othercode}
% \end{othercode}
% \end{othercode}
% \end{othercode}
% \end{othercode}
% \end{othercode}
%
% That's all for \fname{csextras.mtx}.
%    \begin{macrocode}
\endmetrics
%    \end{macrocode}
%
%</csextras.mtx>
%
%
% \subsection{\texttt{nompbul.mtx}}
% \label{sec:nompbul-mtx}
%
% \fname{nompbul.mtx} is used by \fname{fontcomic.tex} when producing an
% OMS-encoded version of \comsan.  \comsan's |plusminus| looks fine, but
% the font lacks a matching |minusplus|.  For consistency we discard the
% |plusminus|, too.  The |plusminus| package option
% (Section~\ref{sec:opt-proc}) can re-enable it on a per-document basis.
% \comsan also has puny |bullet| and |openbullet| characters so we
% discard those too.
%
%<*nompbul.mtx>
%
%    \begin{macrocode}
\relax
\metrics
\unsetglyph{plusminus}
\unsetglyph{bullet}
\unsetglyph{openbullet}
\endmetrics
%    \end{macrocode}
%
%</nompbul.mtx>
%
%
% \subsection{\texttt{fontcomic.tex}}
% \label{sec:fontcomic}
%
% \fname{fontcomic.tex} is a \pkgname{fontinst} file that specifies how to
% derive various PL and VPL fonts from the TTF sources.  \fname{fontcomic.tex}
% relies on the \pkgname{cyrfinst} package to produce Cyrillic fonts.
% Due to a restriction of \pkgname{cyrfinst}, \fname{fontcomic.tex} must be
% run through |latex|, not |tex|.
%
% Note that the fonts produced by \fname{fontcomic.tex} do not follow the
% Berry naming scheme except for appending the encoding scheme onto the
% end of the name.  Personally, I find ``|comicbd8r|'' more readable
% than ``|jcsb8r|'' for \comsan Bold in the |8r|~encoding.
%
% We start by inputting \fname{fontinst.sty} and the various |.tex| files
% provided by \pkgname{cyrfinst} for creating Cyrillic fonts.
%
%<*fontcomic.tex>
%    \begin{macrocode}
\input fontinst.sty
\input fnstcorr
\input cyralias
%    \end{macrocode}
%
% I have tested \fname{fontcomic.tex} only with \pkgname{fontinst}
% version~1.800 so we should require that explicitly.
%    \begin{macrocode}
\needsfontinstversion{1.800}
\installfonts
%    \end{macrocode}
%
% \begin{othercode}[file|files]{rcomic8r.pl}
% \begin{othercode}[file|files]{rcomic8r.mtx}
% \begin{othercode}[file|files]{rcomicbd8r.pl}
% \begin{othercode}[file|files]{rcomicbd8r.mtx}
% \begin{othercode}[file|files]{rcomic7m.pl}
% \begin{othercode}[file|files]{rcomic7m.mtx}
% \begin{othercode}[file|files]{rcomicbd7m.pl}
% \begin{othercode}[file|files]{rcomicbd7m.mtx}
% \begin{othercode}[file|files]{rcomic7y.pl}
% \begin{othercode}[file|files]{rcomic7y.mtx}
% \begin{othercode}[file|files]{rcomic9z.pl}
% \begin{othercode}[file|files]{rcomic9z.mtx}
% \begin{othercode}[file|files]{rcomiccyr.pl}
% \begin{othercode}[file|files]{rcomiccyr.mtx}
% \begin{othercode}[file|files]{rcomiccyrbd.pl}
% \begin{othercode}[file|files]{rcomiccyrbd.mtx}
% First, we create some ``raw'' fonts, from which everything else is
% derived.  These are the only fonts that are referenced by
% \fname{comicsans.map} (Section~\ref{sec:mapfile}); all other fonts produced
% by \fname{fontcomic.tex} are defined in terms of the following.
%    \begin{macrocode}
  \transformfont{rcomic8r}%
    {\reencodefont{8r}{\fromafm{rcomic}}}
  \transformfont{rcomicbd8r}%
    {\reencodefont{8r}{\fromafm{rcomicbd}}}
  \transformfont{rcomic7m}%
    {\reencodefont{oml}{\fromafm{rcomic}}}
  \transformfont{rcomicbd7m}%
    {\reencodefont{oml}{\fromafm{rcomicbd}}}
  \transformfont{rcomic7y}%
    {\reencodefont{oms}{\fromafm{rcomic}}}
  \transformfont{rcomic9z}%
    {\reencodefont{csextras}{\fromafm{rcomic}}}
  \transformfont{rcomiccyr}%
    {\reencodefont{t2a}{\fromafm{rcomic}}}
  \transformfont{rcomiccyrbd}%
    {\reencodefont{t2a}{\fromafm{rcomicbd}}}
%    \end{macrocode}
% \end{othercode}
% \end{othercode}
% \end{othercode}
% \end{othercode}
% \end{othercode}
% \end{othercode}
% \end{othercode}
% \end{othercode}
% \end{othercode}
% \end{othercode}
% \end{othercode}
% \end{othercode}
% \end{othercode}
% \end{othercode}
% \end{othercode}
% \end{othercode}
%
% \begin{othercode}[file|files]{rcomico8r.pl}
% \begin{othercode}[file|files]{rcomico8r.mtx}
% \begin{othercode}[file|files]{rcomicbdo8r.pl}
% \begin{othercode}[file|files]{rcomicbdo8r.mtx}
% \begin{othercode}[file|files]{rcomiccyro.pl}
% \begin{othercode}[file|files]{rcomiccyro.mtx}
% \begin{othercode}[file|files]{rcomiccyrbdo.pl}
% \begin{othercode}[file|files]{rcomiccyrbdo.mtx}
% Next, we create ``raw'' oblique versions of \comsan and \comsan Bold
% as Microsoft doesn't provide a true italic.
%    \begin{macrocode}
  \transformfont{rcomico8r}%
    {\slantfont{167}{%
      \reencodefont{8r}{\fromafm{rcomic}}}}
  \transformfont{rcomicbdo8r}%
    {\slantfont{167}{%
      \reencodefont{8r}{\fromafm{rcomicbd}}}}
  \transformfont{rcomiccyro}%
    {\slantfont{167}{%
      \reencodefont{t2a}{\fromafm{rcomic}}}}
  \transformfont{rcomiccyrbdo}%
    {\slantfont{167}{%
      \reencodefont{t2a}{\fromafm{rcomicbd}}}}
%    \end{macrocode}
% \end{othercode}
% \end{othercode}
% \end{othercode}
% \end{othercode}
% \end{othercode}
% \end{othercode}
% \end{othercode}
% \end{othercode}
%
% \begin{othercode}[file|files]{ot1comic.fd}
% \begin{othercode}[file|files]{comic7t.vpl}
% \begin{othercode}[file|files]{comicbd7t.vpl}
% \begin{othercode}[file|files]{comico7t.vpl}
% \begin{othercode}[file|files]{comicbdo7t.vpl}
% \begin{othercode}[file|files]{comicsc7t.vpl}
% We create versions of \comsan and \comsan Bold that are encoded
% with the OT1 encoding (Knuth's original 7-bit encoding scheme).
%    \begin{macrocode}
  \installfamily{OT1}{comic}{}
  \installfont{comic7t}
    {rcomic8r,rcomic7m,latin}
    {OT1}{OT1}{comic}{m}{n}{}
  \installfont{comicbd7t}
    {rcomicbd8r,rcomicbd7m,latin}
    {OT1}{OT1}{comic}{b}{n}{}
  \installfont{comico7t}
    {rcomico8r,rcomic7m,latin}
    {OT1}{OT1}{comic}{m}{sl}{}
  \installfont{comicbdo7t}
    {rcomicbdo8r,rcomicbd7m,latin}
    {OT1}{OT1}{comic}{b}{sl}{}
  \installfont{comicsc7t}
    {rcomic8r,rcomic7m,latin}
    {OT1C}{OT1}{comic}{m}{sc}{}
%    \end{macrocode}
% \end{othercode}
% \end{othercode}
% \end{othercode}
% \end{othercode}
% \end{othercode}
% \end{othercode}
%
% \begin{othercode}[file|files]{t1comic.fd}
% \begin{othercode}[file|files]{comic8t.vpl}
% \begin{othercode}[file|files]{comicbd8t.vpl}
% \begin{othercode}[file|files]{comico8t.vpl}
% \begin{othercode}[file|files]{comicbdo8t.vpl}
% \begin{othercode}[file|files]{comicsc8t.vpl}
% We now do the same thing for the T1 (Cork) 8-bit encoding.
%    \begin{macrocode}
  \installfamily{T1}{comic}{}
  \installfont{comic8t}
    {rcomic8r,latin}
    {T1}{T1}{comic}{m}{n}{}
  \installfont{comicbd8t}
    {rcomicbd8r,latin}
    {T1}{T1}{comic}{b}{n}{}
  \installfont{comico8t}
    {rcomico8r,latin}
    {T1}{T1}{comic}{m}{sl}{}
  \installfont{comicbdo8t}
    {rcomicbdo8r,latin}
    {T1}{T1}{comic}{b}{sl}{}
  \installfont{comicsc8t}
    {rcomic8r,latin}
    {T1C}{T1}{comic}{m}{sc}{}
%    \end{macrocode}
% \end{othercode}
% \end{othercode}
% \end{othercode}
% \end{othercode}
% \end{othercode}
% \end{othercode}
%
% \begin{othercode}[file|files]{ts1comic.fd}
% \begin{othercode}[file|files]{comic8c.vpl}
% \begin{othercode}[file|files]{comicbd8c.vpl}
% \begin{othercode}[file|files]{comico8c.vpl}
% \begin{othercode}[file|files]{comicbdo8c.vpl}
% \comsan provides many of the \pkgname{textcomp} symbols, so we encode
% some fonts for those.  Note that we take the |bullet| and |openbullet|
% characters from Computer Modern Bold Symbol instead of \comsan.  The
% \comsan versions are too small, in my opinion.
%    \begin{macrocode}
  \installfamily{TS1}{comic}{}
  \installfont{comic8c}
    {rcomic8r,nompbul,cmbsy10,textcomp}
    {TS1}{TS1}{comic}{m}{n}{}
  \installfont{comicbd8c}
    {rcomicbd8r,nompbul,cmbsy10,textcomp}
    {TS1}{TS1}{comic}{b}{n}{}
  \installfont{comico8c}
    {rcomico8r,nompbul,cmbsy10,textcomp}
    {TS1}{TS1}{comic}{m}{sl}{}
  \installfont{comicbdo8c}
    {rcomicbdo8r,nompbul,cmbsy10,textcomp}
    {TS1}{TS1}{comic}{b}{sl}{}
%    \end{macrocode}
% \end{othercode}
% \end{othercode}
% \end{othercode}
% \end{othercode}
% \end{othercode}
%
% \begin{othercode}[file|files]{t2acomic.fd}
% \begin{othercode}[file|files]{comiccyr.vpl}
% \begin{othercode}[file|files]{comiccyrbd.vpl}
% \begin{othercode}[file|files]{comiccyro.vpl}
% \begin{othercode}[file|files]{comiccyrbdo.vpl}
% Thanks to the \pkgname{cyrfinst} package, it's fairly straightforward
% to extract the \comsan Cyrillic characters into a \LaTeX-accessible
% font.
%    \begin{macrocode}
  \installfamily{T2A}{comic}{}
  \installfont{comiccyr}
    {rcomiccyr}
    {T2A}{T2A}{comic}{m}{n}{}
  \installfont{comiccyrbd}
    {rcomiccyrbd}
    {T2A}{T2A}{comic}{b}{n}{}
  \installfont{comiccyro}
    {rcomiccyro}
    {T2A}{T2A}{comic}{m}{sl}{}
  \installfont{comiccyrbdo}
    {rcomiccyrbdo}
    {T2A}{T2A}{comic}{b}{sl}{}
%    \end{macrocode}
% \end{othercode}
% \end{othercode}
% \end{othercode}
% \end{othercode}
% \end{othercode}
%
% \begin{othercode}[file|files]{omlcomic.fd}
% \begin{othercode}[file|files]{comic7m.vpl}
% \begin{othercode}[file|files]{comicbd7m.vpl}
% The remaining fonts produced by \fname{fontcomic.tex} are math fonts.  We
% start with math~italic (the OML 7-bit encoding), although we use roman
% \comsan characters.  Missing math~italic characters are taken from
% Computer Modern 10\,pt.\ Math Italic Bold (|cmmib10|).
%    \begin{macrocode}
  \installfamily{OML}{comic}{\skewchar\font=127}
  \installfont{comic7m}
    {rcomic7m,kernoff,cmmib10,kernon,mathit}
    {OML}{OML}{comic}{m}{n}{}
  \installfont{comicbd7m}
    {rcomicbd7m,kernoff,cmmib10,kernon,mathit}
    {OML}{OML}{comic}{b}{n}{}
%    \end{macrocode}
% \end{othercode}
% \end{othercode}
% \end{othercode}
%
% \begin{othercode}[file|files]{omscomic.fd}
% \begin{othercode}[file|files]{comic7y.vpl}
% Next up are the math~symbol characters (OMS 7-bit encoded).  These are
% taken from \comsan when possible, Computer Modern 10\,pt. Bold Symbol
% (|cmbsy10|) when not.  Note that we utilize \fname{nompbul.mtx}
% (Section~\ref{sec:nompbul-mtx}) to exclude the |plusminus| glyph.
%    \begin{macrocode}
  \installfamily{OMS}{comic}{}
  \installfont{comic7y}
    {rcomic7y,rcomic8r,unsetalf,nompbul,cmbsy10,mathsy}
    {OMS}{OMS}{comic}{m}{n}{}
%    \end{macrocode}
% \end{othercode}
% \end{othercode}
%
% \begin{othercode}[file|files]{omxcomic.fd}
% \begin{othercode}[file|files]{comic7v.vpl}
% As our final math font, we produce a 7-bit OMX-encoded (math
% extension) version of \comsan.  \comsan includes \emph{none} of the
% required characters by default.  However, \fname{csextras.mtx}
% (Section~\ref{sec:csextras-mtx}) can rename a few glyphs to improve
% the situation.  Nevertheless, OMX-encoded \comsan is still not a
% particularly pleasing font.  Authors may want to use a different
% OMX-encoded font in its place.
%    \begin{macrocode}
  \installfamily{OMX}{comic}{}
  \installfont{comic7v}
    {rcomic9z,rcomic8r,csextras,cmex10,mathex}
    {OMX}{OMX}{comic}{m}{n}{}
%    \end{macrocode}
% \end{othercode}
% \end{othercode}
%
% \begin{othercode}[file|files]{ucomic.fd}
% \begin{othercode}[file|files]{comic9z.vpl}
% Leftover characters are assigned to a \LaTeX{} ``U''-encoded font,
% |comic9z|.
%    \begin{macrocode}
  \installfamily{U}{comic}{}
  \installfont{comic9z}
    {rcomic9z}
    {CSEXTRAS}{U}{comic}{m}{n}{}
%    \end{macrocode}
% \end{othercode}
% \end{othercode}
%
% Those are all of the \comsan fonts I could think to create.  We can
% finish up now.
%    \begin{macrocode}
\endinstallfonts
\bye
%    \end{macrocode}
%
%</fontcomic.tex>
%
%
% \subsection{\texttt{Makefile}}
%
% The \fname{Makefile} included below automates the generation of the various
% \comsan \LaTeX{} fonts.  I tested this \fname{Makefile} only with GNU~make,
% only on Linux, and only with the \TeX\ Live distribution of \TeX.
%
% Note that the various ``|verbatim|'' lines are present for
% \pkgname{DocStrip}'s sake and do not actually appear in the resulting
% file.\footnote{Without the ``\texttt{verbatim}'' lines,
% \pkgname{DocStrip} would choke on all of the end-of-line
% ``\texttt{\textbackslash}'' characters.}  Also, many \TeX{}
% distributions do not honor tab characters when outputting files,
% although most |make| implementations \emph{require} tabs.  As a
% result, \fname{comicsans.ins} specifies that the following code be written
% to \fname{Makefile.NOTABS} with space- instead of tab-based indentation.  It
% is up to the user to convert spaces to tabs.  (In GNU Emacs, the
% ``\texttt{M-x tabify}'' sequence automates this conversion; entering
% ``\texttt{cat Makefile.NOTABS \string| unexpand > Makefile}'' at the Unix
% prompt---or ``\texttt{cat Makefile.NOTABS \string| perl
% -ne }\verb*|'s/^        /\t/g;|\texttt{ print' > Makefile}'' if you don't
% have \texttt{unexpand}---is even more automatic.)
%
% \bigskip
% \begingroup\fussy
%<*Makefile>
%
% \begin{othercode}[Makefile variable|Makefile variables]{TFMTARGETS}
% \begin{othercode}[Makefile variable|Makefile variables]{VFTARGETS}
% Because we produce so many TFM and VF files, we define
% |TFMTARGETS| and |VFTARGETS| targets for these.
%
%    \begin{macrocode}
%<<verbatim>
TFMTARGETS = comic7m.tfm comic7t.tfm comic7v.tfm        \
             comic7y.tfm comic8c.tfm comic8t.tfm        \
             comicbd7t.tfm comicbd8c.tfm comicbd8t.tfm  \
             comiccyr.tfm comiccyrbd.tfm rcomic.tfm     \
             rcomic7m.tfm rcomic8r.tfm rcomicbd.tfm     \
             rcomicbd8r.tfm rcomiccyr.tfm rcomic7y.tfm  \
             rcomiccyrbd.tfm rcomic9z.tfm comic9z.tfm   \
             rcomicbd7m.tfm comicbd7m.tfm               \
             rcomico8r.tfm rcomicbdo8r.tfm              \
             comico7t.tfm comicbdo7t.tfm                \
             comico8t.tfm comicbdo8t.tfm                \
             comico8c.tfm comicbdo8c.tfm                \
             rcomiccyro.tfm rcomiccyrbdo.tfm            \
             comiccyro.tfm comiccyrbdo.tfm              \
             comicsc7t.tfm comicsc8t.tfm

VFTARGETS =  comic7m.vf comic7t.vf comic7v.vf       \
             comic7y.vf comic8c.vf comic8t.vf       \
             comicbd7t.vf comicbd8c.vf comicbd8t.vf \
             comiccyr.vf comiccyrbd.vf comic9z.vf   \
             comicbd7m.vf                           \
             comico7t.vf comicbdo7t.vf              \
             comico8t.vf comicbdo8t.vf              \
             comico8c.vf comicbdo8c.vf              \
             comiccyro.vf comiccyrbdo.vf            \
             comicsc7t.vf comicsc8t.vf

%verbatim>
%    \end{macrocode}
% \end{othercode}
% \end{othercode}
%
% \begin{othercode}[Makefile variable|Makefile variables]{PACKAGEFILES}
% \begin{othercode}[Makefile target|Makefile targets]{all}
% The primary Makefile targets are the |.tfm|, |.vf|, and |.fd| files.
%    \begin{macrocode}
PACKAGEFILES = $(TFMTARGETS) $(VFTARGETS) $(FDOUTPUTS)

all: $(PACKAGEFILES)
%    \end{macrocode}
% \end{othercode}
% \end{othercode}
%
% We define a rule for converting a VPL file into a VF plus a TFM file
% and a rule for converting a PL file into a TFM file.
%    \begin{macrocode}
%<<verbatim>

.SUFFIXES: .vf .vpl .tfm .pl .ttf .afm

%.vf %.tfm: %.vpl
        vptovf $<

%.tfm: %.pl
        pltotf $<

%verbatim>
%    \end{macrocode}
%
% We would ideally like to define a rule for building a
% |.|\meta{DPI}|pk| file that depends upon a corresponding |.tfm| file.
% Unfortunately, Makefile semantics do not support such usage.  We
% therefore parse out \meta{DPI} and call |make| recursively to ensure
% that the requisite |.tfm| file exists.
%    \begin{macrocode}
%<<verbatim>

%pk: comicsans.map comic.ttf comicbd.ttf
        DPI=`echo $@ | \
          perl -ne '/(\d+)pk$$/ && print $$1'` ; \
        BASE=`echo $@ | \
          perl -ne '/^(.*)\.\d+pk$$/ && print $$1'` ; \
        gsftopk -q --mapfile=comicsans.map $$BASE $$DPI

%verbatim>
%    \end{macrocode}
%
% \begin{othercode}[file|files]{cmmib10.pl}
% \begin{othercode}[file|files]{cmex10.pl}
% \begin{othercode}[file|files]{cmbsy10.pl}
% Kpathsea should find standard .tfm files even if they're not in the
% current directory.  Hence, the following three targets have no
% dependencies.
%    \begin{macrocode}
cmmib10.pl:
        tftopl cmmib10.tfm > cmmib10.pl

cmex10.pl:
        tftopl cmex10.tfm > cmex10.pl

cmbsy10.pl:
        tftopl cmbsy10.tfm > cmbsy10.pl
%    \end{macrocode}
% \end{othercode}
% \end{othercode}
% \end{othercode}
%
% \begin{othercode}[Makefile variable|Makefile variables]{FDOUTPUTS}
% \begin{othercode}[Makefile variable|Makefile variables]{LOGOUTPUTS}
% \begin{othercode}[Makefile variable|Makefile variables]{PLOUTPUTS}
% \begin{othercode}[Makefile variable|Makefile variables]{VPLOUTPUTS}
% \begin{othercode}[Makefile variable|Makefile variables]{MTXOUTPUTS}
% \begin{othercode}[Makefile variable|Makefile variables]{FONTINSTOUTPUTS}
% \pkgname{fontinst} outputs a large number of files.  To make these
% more manageable we define macros to represent various subsets.
%    \begin{macrocode}
%<<verbatim>

FDOUTPUTS  = ts1comic.fd t1comic.fd ot1comic.fd  \
             t2acomic.fd omlcomic.fd omxcomic.fd \
             omscomic.fd ucomic.fd
LOGOUTPUTS = fontcomic.log
PLOUTPUTS =  rcomic.pl rcomicbd.pl rcomiccyrbd.pl      \
             rcomic7m.pl rcomic8r.pl rcomicbd8r.pl     \
             rcomiccyr.pl rcomic9z.pl rcomic7y.pl      \
             rcomicbd7m.pl rcomico8r.pl rcomicbdo8r.pl \
             rcomiccyro.pl rcomiccyrbdo.pl
VPLOUTPUTS = comic8c.vpl comicbd8c.vpl comiccyrbd.vpl \
             comic7m.vpl comiccyr.vpl comic7t.vpl     \
             comicbd7t.vpl comic8t.vpl comicbd8t.vpl  \
             comic7v.vpl comic9z.vpl comic7y.vpl      \
             comicbd7m.vpl                            \
             comico7t.vpl comicbdo7t.vpl              \
             comico8t.vpl comicbdo8t.vpl              \
             comico8c.vpl comicbdo8c.vpl              \
             comiccyro.vpl comiccyrbdo.vpl            \
             comicsc7t.vpl comicsc8t.vpl
MTXOUTPUTS = cmbsy10.mtx cmex10.mtx cmmib10.mtx       \
             rcomic.mtx rcomicbd.mtx rcomiccyrbd.mtx  \
             rcomic7m.mtx rcomic8r.mtx rcomicbd8r.mtx \
             rcomiccyr.mtx rcomic9z.mtx rcomic7y.mtx  \
             rcomicbd7m.mtx                           \
             rcomico8r.mtx rcomicbdo8r.mtx            \
             rcomiccyro.mtx rcomiccyrbdo.mtx

FONTINSTOUTPUTS = $(FDOUTPUTS) $(LOGOUTPUTS) \
                  $(PLOUTPUTS) $(VPLOUTPUTS) \
                  $(MTXOUTPUTS)

%verbatim>
%    \end{macrocode}
% \end{othercode}
% \end{othercode}
% \end{othercode}
% \end{othercode}
% \end{othercode}
% \end{othercode}
%
% \begin{othercode}[Makefile variable|Makefile variables]{AFMINPUTS}
% \begin{othercode}[Makefile variable|Makefile variables]{PLINPUTS}
% \begin{othercode}[Makefile variable|Makefile variables]{CSEXTRAS}
% We now define macros for all of \pkgname{fontinst}'s
% input files, excluding those that need not exist in the
% current directory.
%    \begin{macrocode}
AFMINPUTS = rcomic.afm rcomicbd.afm
PLINPUTS  = cmbsy10.pl cmmib10.pl cmex10.pl
CSEXTRAS  = csextras.etx csextras.mtx
%    \end{macrocode}
% \end{othercode}
% \end{othercode}
% \end{othercode}
%
% The most important part of the Makefile is to run the \fname{fontcomic.tex}
% \pkgname{fontinst} file through \LaTeX.  Normally \pkgname{fontinst}
% files are run through \TeX but the \pkgname{cyrfinst} package, which
% \fname{fontcomic.tex} uses, requires \LaTeX.
%    \begin{macrocode}
%<<verbatim>

$(FONTINSTOUTPUTS): fontcomic.tex \
                    $(AFMINPUTS) $(PLINPUTS) $(CSEXTRAS)
        latex fontcomic.tex

%verbatim>
%    \end{macrocode}
%
% \begin{othercode}[Makefile target|Makefile targets]{doc}
% \begin{othercode}[Makefile variable|Makefile variables]{DOCOUTPUTS}
% To automate building the \pkgname{comicsans} documentation, we define
% a |doc| target, which uses pdf\LaTeX{} and MakeIndex to build a nicely
% formatted PDF document.  For some reason
% ``\verb*|\DoNotIndex{\ }|'' doesn't seem to work.  We therefore
% explicitly |grep| away all of the ``\verb*|\ |'' entries.
% \changes{v1.0e}{2008/07/12}{Modified to run \texttt{pdfopt} on the generated
%   PDF documentation}
% \changes{v1.0d}{2008/07/06}{Modified to use
%   \texttt{\string\string\string\pdfmapfile} to point \texttt{pdflatex} to
%   \texttt{comicsans.map}}
%    \begin{macrocode}
%<<verbatim>

doc: comicsans.pdf

DOCOUTPUTS = comicsans.pdf comicsans.aux comicsans.glo \
             comicsans.out comicsans.log comicsans.idx \
             comicsans.ind comicsans.ilg comicsans.gls

$(DOCOUTPUTS): comicsans.dtx $(PACKAGEFILES) comicsans.sty
        pdflatex '\pdfmapfile{pdftex.map}\pdfmapfile{comicsans.map}\input comicsans.dtx'
        grep -v 'indexentry{! =' comicsans.idx | \
          makeindex -s gind.ist -o comicsans.ind
        makeindex -s gglo.ist comicsans.glo -o comicsans.gls
        pdflatex '\pdfmapfile{pdftex.map}\pdfmapfile{comicsans.map}\input comicsans.dtx'
        pdflatex '\pdfmapfile{pdftex.map}\pdfmapfile{comicsans.map}\input comicsans.dtx'
        pdfopt comicsans.pdf cs.pdf
        mv cs.pdf comicsans.pdf

%verbatim>
%    \end{macrocode}
% \end{othercode}
% \end{othercode}
%
% \begin{othercode}[Makefile variable|Makefile variables]{CSTEXMFDIR}
% \begin{othercode}[Makefile variable|Makefile variables]{CSVFDIR}
% \begin{othercode}[Makefile variable|Makefile variables]{CSTFMDIR}
% \begin{othercode}[Makefile variable|Makefile variables]{CSLTXDIR}
% \begin{othercode}[Makefile variable|Makefile variables]{CSDVIPSMAPDIR}
% \begin{othercode}[Makefile variable|Makefile variables]{CSDVIPSENCDIR}
% \begin{othercode}[Makefile variable|Makefile variables]{CSDOCDIR}
% \begin{othercode}[Makefile variable|Makefile variables]{CSSRCDIR}
% \changes{v1.0g}{2013/12/19}{Specified that the \texttt{Makefile} install
%   \texttt{comicsans.ins} and \texttt{comicsans.dtx} beneath the
%   \texttt{source} directory, as suggested by Norbert Preining}
% \begin{othercode}[Makefile target|Makefile targets]{install}
% \changes{v1.0f}{2008/07/12}{Specified that the \texttt{Makefile} install
%   \texttt{comicsans.pdf} beneath the \texttt{doc} directory}
% \begin{othercode}[Makefile target|Makefile targets]{uninstall}
% Because \pkgname{comicsans} consists of so many files, we provide an
% |install| target to automate installation.  We assume a \TeX{}
% Directory Standard (TDS) distribution although the user can override
% the various directory locations by assigning one or more of
% |CSTEXMFDIR|, |CSVFDIR|, |CSTFMDIR|, |CSLTXDIR|, |CSDVIPSMAPDIR|,
% |CSDVIPSENDDIR|, |CSDOCDIR|, or |CSSRCDIR| on the |make| command line.
% Although we also provide an |uninstall| target, this is not guaranteed
% to remove all of the directories created.  Specifically, if |install|
% creates both a directory and a subdirectory
% (e.g.,~|microsoft/comicsans|), only the subdirectory (|comicsans|)
% will be deleted.
%    \begin{macrocode}
%<<verbatim>

CSTEXMFDIR    = /usr/local/share/texmf
CSVFDIR       = $(CSTEXMFDIR)/fonts/vf/microsoft/comicsans
CSTFMDIR      = $(CSTEXMFDIR)/fonts/tfm/microsoft/comicsans
CSLTXDIR      = $(CSTEXMFDIR)/tex/latex/comicsans
CSDVIPSMAPDIR = $(CSTEXMFDIR)/fonts/map/dvips/comicsans
CSDVIPSENCDIR = $(CSTEXMFDIR)/fonts/enc/dvips/comicsans
CSDOCDIR      = $(CSTEXMFDIR)/doc/latex/comicsans
CSSRCDIR      = $(CSTEXMFDIR)/source/latex/comicsans

install: $(CSTEXMFDIR) $(PACKAGEFILES) comicsans.sty comicsans.pdf
        install -d $(CSVFDIR) $(CSTFMDIR) $(CSLTXDIR) \
          $(CSDVIPSMAPDIR) $(CSDVIPSENCDIR) $(CSDOCDIR) $(CSSRCDIR)
        install -m 664 $(VFTARGETS) $(CSVFDIR)
        install -m 664 $(TFMTARGETS) $(CSTFMDIR)
        install -m 664 $(FDOUTPUTS) comicsans.sty $(CSLTXDIR)
        install -m 664 comicsans.map $(CSDVIPSMAPDIR)
        install -m 664 csextras.enc $(CSDVIPSENCDIR)
        install -m 664 comicsans.pdf README $(CSDOCDIR)
        install -m 664 comicsans.ins comicsans.dtx $(CSSRCDIR)

uninstall:
        $(RM) -rf $(CSVFDIR) $(CSTFMDIR) $(CSLTXDIR) $(CSDOCDIR) $(CSSRCDIR)
        $(RM) -rf $(CSDVIPSMAPDIR) $(CSDVIPSENCDIR)

%verbatim>
%    \end{macrocode}
% \end{othercode}
% \end{othercode}
% \end{othercode}
% \end{othercode}
% \end{othercode}
% \end{othercode}
% \end{othercode}
% \end{othercode}
% \end{othercode}
% \end{othercode}
%
% \begin{othercode}[Makefile variable|Makefile variables]{TARGZFILE}
% \begin{othercode}[Makefile target|Makefile targets]{dist}
% We make it easy to create a |.tar.gz| file containing \fname{comicsans.ins},
% \fname{comicsans.dtx}, and all of the prebuilt \pkgname{comicsans} font
% files.
% \changes{v1.0b}{2006/12/14}{Restructured the distribution tree according to
%   Jim Hefferon's suggestions}
% \changes{v1.0e}{2008/07/12}{Moved the contents of the \texttt{texmf}
%   directory to the top level of \texttt{comicsans.tds.zip} as suggested by
%   Dan Luecking}
% \changes{v1.0f}{2008/07/12}{Restructured the distribution tree according to
%   Jim Hefferon's latest suggestions}
% \changes{v1.0g}{2013/12/18}{Included the prebuilt \texttt{.fd} files in
%   the distribution tree}
%    \begin{macrocode}
TARGZFILE = comicsans.tar.gz

dist: $(TARGZFILE)

$(TARGZFILE): $(PACKAGEFILES) doc
        install -d comicsans/comicsans
        install -m 664 README comicsans.pdf comicsans/comicsans
        install -m 664 comicsans.dtx comicsans.ins comicsans/comicsans
        install -d comicsans/texmf
        $(MAKE) CSTEXMFDIR=comicsans/texmf install
        cp -r comicsans/texmf/fonts/tfm/microsoft/comicsans comicsans/comicsans/tfm
        cp -r comicsans/texmf/fonts/vf/microsoft/comicsans comicsans/comicsans/vf
        install -d comicsans/comicsans/map
        install -m 644 comicsans/texmf/fonts/map/dvips/comicsans/* comicsans/comicsans/map/
        install -d comicsans/comicsans/enc
        install -m 644 comicsans/texmf/fonts/enc/dvips/comicsans/* comicsans/comicsans/enc/
        install -d comicsans/comicsans/latex
        install -m 644 comicsans/texmf/tex/latex/comicsans/* comicsans/comicsans/latex
        cd comicsans/texmf ; \
          zip -r -9 -m ../comicsans.tds.zip *
        $(RM) -r comicsans/texmf
        tar -cf - comicsans | gzip --best > $(TARGZFILE)
        $(RM) -r comicsans
%    \end{macrocode}
% \end{othercode}
% \end{othercode}
%
% \begin{othercode}[Makefile variable|Makefile variables]{DPI}
% \begin{othercode}[Makefile variable|Makefile variables]{PKFILES}
% \begin{othercode}[Makefile target|Makefile targets]{pkfiles}
% My understanding of copyright law is that I am not allowed to
% distribute |.pk| files as these are considered derivitive works from
% \fname{comic.ttf} and \fname{comicbd.ttf}.  However, I believe you \emph{are}
% allowed to generate these files yourself for your own personal use.
% ``\texttt{make pkfiles}'' generates PK files for 600~DPI printers at
% the various standard \LaTeX{} point sizes (taken from \fname{ot1cmr.fd}).
% For printers with a different number of dots per inch, ``\texttt{make
% DPI=}\meta{resolution} |pkfiles|'' should override the 600-DPI
% default.  If you need fonts at additional resolutions you can produce
% them individually with ``|make| \meta{font name}|.|\meta{DPI}|pk|''.
%    \begin{macrocode}
%<<verbatim>

DPI = 600

PKFILES = $(shell perl -ane '                  \
  $$F[0] =~ /^\w/ || next;                     \
  foreach $$size (5..10, 10.95, 12, 14.4,      \
                  17.28, 20.74, 24.88) {       \
    printf "$$F[0].%dpk\n", $(DPI)*$$size/10   \
  }                                            \
' < comicsans.map)

pkfiles: $(TFMTARGETS) $(PKFILES)

%verbatim>
%    \end{macrocode}
% \end{othercode}
% \end{othercode}
% \end{othercode}
%
% \begin{othercode}[Makefile target|Makefile targets]{clean}
% \begin{othercode}[Makefile target|Makefile targets]{cleaner}
% Finally, we define |clean| and |cleaner| target so that ``\texttt{make
% clean}'' will delete the myriad generated files.  ``\texttt{make
% cleaner}'' additionally deletes the files that \fname{comicsans.ins} had
% extracted from \fname{comicsans.dtx}.
%    \begin{macrocode}
clean:
        $(RM) $(PKFILES)
        $(RM) $(TARGZFILE)
        $(RM) $(DOCOUTPUTS)
        $(RM) $(FONTINSTOUTPUTS)
        $(RM) $(PLINPUTS)
        $(RM) $(PACKAGEFILES)

cleaner: clean
        $(RM) comicsans.sty csextras.etx csextras.mtx
        $(RM) nompbul.mtx fontcomic.tex comicsans.map
        $(RM) csextras.enc ttfonts.map
        $(RM) rcomic.afm rcomicbd.afm Makefile.NOTABS
        $(RM) fonttopfb.ff alt-comicsans.map

.PHONY: doc install uninstall dist pkfiles clean cleaner
%    \end{macrocode}
% \end{othercode}
% \end{othercode}
%
%</Makefile>
% \endgroup   ^^A  Matches the \begingroup\fussy
%
%
% \subsection{\texttt{rcomic.afm} and \texttt{rcomicbd.afm}}
%
% \fname{fontcomic.tex} (Section~\ref{sec:fontcomic}) depends
% upon~\fname{rcomic.afm} and \fname{rcomicbd.afm}---the Adobe font
% metric files that specify the widths, heights, and depths of all of
% the characters in \fname{comic.ttf} and \fname{comicbd.ttf}.  Although
% these can be produced automatically by the |ttf2afm| utility,
% |ttf2afm| misses a few characters, most notably |\summation| and
% |\product|.  We therefore include versions of \fname{rcomic.afm} and
% \fname{rcomicbd.afm} that were generated by
% PfaEdit\index{PfaEdit|usage} (FontForge\index{FontForge|usage}'s
% predecessor), which does a better job of finding glyphs than
% |ttf2afm|.  Because these AFM files are long ($\sim$12~pages apiece)
% we omit them from the \pkgname{comicsans} documentation.
%
%<*rcomic.afm>
%
% \begin{center}
%   $\vdots$ \\
%   599 lines of code omitted \\
%   $\vdots$
% \end{center}
%
% \iffalse
%    \begin{macrocode}
StartFontMetrics 2.0
Comment Generated by pfaedit
Comment Creation Date: Wed Jul 17 20:26:04 2002
FontName ComicSansMS
FullName Comic Sans MS
FamilyName ComicSansMS
Weight
Notice (Copyright (c) 1995 Microsoft Corporation. All rights reserved.)
ItalicAngle 0
IsFixedPitch false
UnderlinePosition -272
UnderlineThickness 175
Version Version 2.10
EncodingScheme ISO10646-1
FontBBox -93 -313 1187 1103
CapHeight 729
XHeight 539
Ascender 785
Descender -283
StartCharMetrics 577
C 0 ; WX 500 ; N .notdef ; B 62 0 438 800 ;
C 32 ; WX 298 ; N space ; B 0 0 0 0 ;
C 33 ; WX 237 ; N exclam ; B 58 -33 163 784 ;
C 34 ; WX 424 ; N quotedbl ; B 56 453 336 776 ;
C 35 ; WX 842 ; N numbersign ; B 15 -14 834 770 ;
C 36 ; WX 693 ; N dollar ; B 75 -194 626 841 ;
C 37 ; WX 820 ; N percent ; B 71 -15 786 802 ;
C 38 ; WX 654 ; N ampersand ; B 36 -46 620 764 ;
C 39 ; WX 388 ; N quotesingle ; B 138 556 231 811 ;
C 40 ; WX 366 ; N parenleft ; B 55 -211 340 785 ;
C 41 ; WX 366 ; N parenright ; B 55 -211 340 785 ;
C 42 ; WX 529 ; N asterisk ; B 22 381 471 779 ;
C 43 ; WX 480 ; N plus ; B 23 113 446 511 ;
C 44 ; WX 276 ; N comma ; B 96 -168 244 70 ;
C 45 ; WX 416 ; N hyphen ; B 54 225 371 309 ;
C 46 ; WX 249 ; N period ; B 69 -46 191 76 ;
C 47 ; WX 511 ; N slash ; B 41 -44 475 794 ;
C 48 ; WX 610 ; N zero ; B 29 -20 576 760 ;
C 49 ; WX 450 ; N one ; B 77 -1 392 762 ;
C 50 ; WX 610 ; N two ; B 80 -2 539 750 ;
C 51 ; WX 610 ; N three ; B 71 -23 529 746 ;
C 52 ; WX 610 ; N four ; B 23 -13 581 762 ;
C 53 ; WX 610 ; N five ; B 61 -31 563 754 ;
C 54 ; WX 610 ; N six ; B 54 -36 542 760 ;
C 55 ; WX 610 ; N seven ; B 34 -33 593 737 ;
C 56 ; WX 610 ; N eight ; B 59 -27 550 746 ;
C 57 ; WX 610 ; N nine ; B 54 -47 564 750 ;
C 58 ; WX 298 ; N colon ; B 89 60 203 552 ;
C 59 ; WX 298 ; N semicolon ; B 39 -95 201 549 ;
C 60 ; WX 381 ; N less ; B 9 94 307 519 ;
C 61 ; WX 510 ; N equal ; B 48 134 430 488 ;
C 62 ; WX 381 ; N greater ; B 28 90 360 550 ;
C 63 ; WX 523 ; N question ; B 25 -36 470 722 ;
C 64 ; WX 931 ; N at ; B 53 -69 856 796 ;
C 65 ; WX 731 ; N A ; B 63 -15 661 722 ;
C 66 ; WX 630 ; N B ; B 93 -24 590 768 ;
C 67 ; WX 602 ; N C ; B 43 -12 588 744 ;
C 68 ; WX 721 ; N D ; B 89 -49 672 760 ;
C 69 ; WX 624 ; N E ; B 68 -49 593 784 ;
C 70 ; WX 606 ; N F ; B 84 -52 588 769 ;
C 71 ; WX 679 ; N G ; B 38 -34 662 768 ;
C 72 ; WX 768 ; N H ; B 74 -41 716 759 ;
C 73 ; WX 546 ; N I ; B 36 -18 518 730 ; L J IJ ;
C 74 ; WX 665 ; N J ; B 45 -59 633 740 ;
C 75 ; WX 610 ; N K ; B 104 -54 607 748 ;
C 76 ; WX 550 ; N L ; B 49 -46 532 753 ;
C 77 ; WX 882 ; N M ; B 54 -41 846 750 ;
C 78 ; WX 796 ; N N ; B 60 -39 756 758 ;
C 79 ; WX 798 ; N O ; B 56 -30 756 740 ;
C 80 ; WX 520 ; N P ; B 48 -12 491 768 ;
C 81 ; WX 876 ; N Q ; B 37 -214 855 740 ;
C 82 ; WX 628 ; N R ; B 57 -18 600 751 ;
C 83 ; WX 693 ; N S ; B 65 -28 646 717 ;
C 84 ; WX 679 ; N T ; B 56 -4 715 740 ;
C 85 ; WX 736 ; N U ; B 75 -20 679 734 ;
C 86 ; WX 649 ; N V ; B 70 -40 648 750 ;
C 87 ; WX 1039 ; N W ; B 67 -47 1024 746 ;
C 88 ; WX 723 ; N X ; B 33 -42 687 743 ;
C 89 ; WX 635 ; N Y ; B 14 -35 599 743 ;
C 90 ; WX 693 ; N Z ; B 33 -25 675 737 ;
C 91 ; WX 376 ; N bracketleft ; B 85 -204 343 743 ;
C 92 ; WX 549 ; N backslash ; B 86 -69 496 745 ;
C 93 ; WX 376 ; N bracketright ; B 85 -204 343 743 ;
C 94 ; WX 581 ; N asciicircum ; B 96 547 499 804 ;
C 95 ; WX 626 ; N underscore ; B -18 -169 646 -76 ;
C 96 ; WX 556 ; N grave ; B 72 575 282 812 ;
C 97 ; WX 511 ; N a ; B 24 -33 495 510 ;
C 98 ; WX 593 ; N b ; B 74 -21 547 770 ;
C 99 ; WX 513 ; N c ; B 51 -31 474 520 ;
C 100 ; WX 587 ; N d ; B 50 -23 538 779 ;
C 101 ; WX 547 ; N e ; B 42 -23 529 511 ;
C 102 ; WX 508 ; N f ; B 36 -79 460 781 ;
C 103 ; WX 530 ; N g ; B 28 -276 494 500 ;
C 104 ; WX 577 ; N h ; B 70 -31 527 783 ;
C 105 ; WX 280 ; N i ; B 87 -3 219 732 ; L j ij ;
C 106 ; WX 403 ; N j ; B -9 -292 321 731 ;
C 107 ; WX 540 ; N k ; B 79 -21 526 784 ;
C 108 ; WX 273 ; N l ; B 84 -21 197 786 ;
C 109 ; WX 776 ; N m ; B 59 -61 737 542 ;
C 110 ; WX 523 ; N n ; B 60 -35 492 534 ;
C 111 ; WX 525 ; N o ; B 40 -29 474 507 ;
C 112 ; WX 534 ; N p ; B 58 -284 493 537 ;
C 113 ; WX 520 ; N q ; B 29 -272 461 520 ;
C 114 ; WX 480 ; N r ; B 67 -33 449 515 ;
C 115 ; WX 486 ; N s ; B 20 -30 446 558 ;
C 116 ; WX 471 ; N t ; B 31 -32 443 683 ;
C 117 ; WX 520 ; N u ; B 53 -40 476 521 ;
C 118 ; WX 486 ; N v ; B 30 -20 474 516 ;
C 119 ; WX 684 ; N w ; B 37 -40 658 509 ;
C 120 ; WX 590 ; N x ; B 29 -22 563 540 ;
C 121 ; WX 520 ; N y ; B -2 -283 500 508 ;
C 122 ; WX 538 ; N z ; B 59 -38 508 516 ;
C 123 ; WX 366 ; N braceleft ; B 2 -188 341 794 ;
C 124 ; WX 421 ; N bar ; B 172 -177 260 838 ;
C 125 ; WX 366 ; N braceright ; B 2 -188 341 794 ;
C 126 ; WX 597 ; N asciitilde ; B 48 228 558 457 ;
C 160 ; WX 298 ; N nonbreakingspace ; B 0 0 0 0 ;
C 161 ; WX 237 ; N exclamdown ; B 58 -33 163 784 ;
C 162 ; WX 623 ; N cent ; B 94 15 578 850 ;
C 163 ; WX 793 ; N sterling ; B 18 -74 713 787 ;
C 164 ; WX 611 ; N currency ; B 27 87 602 645 ;
C 165 ; WX 635 ; N yen ; B 64 -26 573 712 ;
C 166 ; WX 404 ; N brokenbar ; B 159 -22 240 828 ;
C 167 ; WX 634 ; N section ; B 57 -38 538 802 ;
C 168 ; WX 556 ; N dieresis ; B 107 578 477 674 ;
C 169 ; WX 795 ; N copyright ; B 43 122 744 792 ;
C 170 ; WX 526 ; N ordfeminine ; B 49 435 445 821 ;
C 171 ; WX 577 ; N guillemotleft ; B 9 83 528 519 ;
C 172 ; WX 480 ; N logicalnot ; B 23 121 447 360 ;
C 173 ; WX 416 ; N hyphenminus ; B 54 225 371 309 ;
C 174 ; WX 795 ; N registered ; B 44 101 752 797 ;
C 175 ; WX 626 ; N overscore ; B -18 831 646 924 ;
C 176 ; WX 409 ; N degree ; B 33 475 375 821 ;
C 177 ; WX 480 ; N plusminus ; B 23 -63 446 511 ;
C 178 ; WX 650 ; N twosuperior ; B 186 440 481 854 ;
C 179 ; WX 650 ; N threesuperior ; B 202 417 463 831 ;
C 180 ; WX 556 ; N acute ; B 104 579 325 813 ;
C 181 ; WX 520 ; N mu1 ; B 10 -199 577 521 ;
C 182 ; WX 693 ; N paragraph ; B 34 -96 648 810 ;
C 183 ; WX 249 ; N periodcentered ; B 69 289 191 411 ;
C 184 ; WX 556 ; N cedilla ; B 178 -218 423 57 ;
C 185 ; WX 650 ; N onesuperior ; B 196 440 468 842 ;
C 186 ; WX 449 ; N ordmasculine ; B 19 435 421 831 ;
C 187 ; WX 577 ; N guillemotright ; B 9 83 528 519 ;
C 188 ; WX 650 ; N onequarter ; B 79 -167 592 842 ;
C 189 ; WX 650 ; N onehalf ; B 79 -175 592 842 ;
C 190 ; WX 650 ; N threequarters ; B 79 -167 592 831 ;
C 191 ; WX 523 ; N questiondown ; B 25 -36 470 722 ;
C 192 ; WX 731 ; N Agrave ; B 63 -15 661 1036 ;
C 193 ; WX 731 ; N Aacute ; B 63 -15 661 1030 ;
C 194 ; WX 731 ; N Acircumflex ; B 63 -15 661 1021 ;
C 195 ; WX 731 ; N Atilde ; B 63 -15 707 980 ;
C 196 ; WX 731 ; N Adieresis ; B 63 -15 661 894 ;
C 197 ; WX 731 ; N Aring ; B 63 -15 661 978 ;
C 198 ; WX 1086 ; N AE ; B 22 -43 1083 792 ;
C 199 ; WX 602 ; N Ccedilla ; B 43 -199 588 744 ;
C 200 ; WX 624 ; N Egrave ; B 68 -49 593 1089 ;
C 201 ; WX 624 ; N Eacute ; B 68 -49 593 1095 ;
C 202 ; WX 624 ; N Ecircumflex ; B 68 -49 593 1103 ;
C 203 ; WX 624 ; N Edieresis ; B 68 -49 593 952 ;
C 204 ; WX 546 ; N Igrave ; B 36 -18 518 1056 ;
C 205 ; WX 546 ; N Iacute ; B 36 -18 518 1043 ;
C 206 ; WX 546 ; N Icircumflex ; B 36 -18 518 1063 ;
C 207 ; WX 546 ; N Idieresis ; B 36 -18 518 904 ;
C 208 ; WX 721 ; N Eth ; B -22 -49 672 760 ;
C 209 ; WX 796 ; N Ntilde ; B 60 -39 756 989 ;
C 210 ; WX 798 ; N Ograve ; B 56 -30 756 1041 ;
C 211 ; WX 798 ; N Oacute ; B 56 -30 756 1033 ;
C 212 ; WX 798 ; N Ocircumflex ; B 56 -30 756 1045 ;
C 213 ; WX 798 ; N Otilde ; B 56 -30 756 980 ;
C 214 ; WX 798 ; N Odieresis ; B 56 -30 756 907 ;
C 215 ; WX 480 ; N multiply ; B 32 109 416 499 ;
C 216 ; WX 798 ; N Oslash ; B 40 -44 797 753 ;
C 217 ; WX 736 ; N Ugrave ; B 75 -20 679 1034 ;
C 218 ; WX 736 ; N Uacute ; B 75 -20 679 1036 ;
C 219 ; WX 736 ; N Ucircumflex ; B 75 -20 679 1048 ;
C 220 ; WX 736 ; N Udieresis ; B 75 -20 679 896 ;
C 221 ; WX 635 ; N Yacute ; B 14 -35 599 1078 ;
C 222 ; WX 520 ; N Thorn ; B 48 -12 494 758 ;
C 223 ; WX 443 ; N germandbls ; B 22 -77 433 767 ;
C 224 ; WX 511 ; N agrave ; B 32 -33 502 808 ;
C 225 ; WX 511 ; N aacute ; B 32 -33 502 813 ;
C 226 ; WX 511 ; N acircumflex ; B 32 -33 502 830 ;
C 227 ; WX 511 ; N atilde ; B 32 -33 521 750 ;
C 228 ; WX 511 ; N adieresis ; B 32 -33 502 674 ;
C 229 ; WX 511 ; N aring ; B 32 -33 502 859 ;
C 230 ; WX 911 ; N ae ; B 32 -24 876 511 ;
C 231 ; WX 513 ; N ccedilla ; B 51 -218 474 520 ;
C 232 ; WX 547 ; N egrave ; B 42 -23 529 812 ;
C 233 ; WX 547 ; N eacute ; B 42 -23 529 813 ;
C 234 ; WX 547 ; N ecircumflex ; B 42 -23 529 826 ;
C 235 ; WX 547 ; N edieresis ; B 42 -23 529 722 ;
C 236 ; WX 280 ; N igrave ; B 12 -3 222 812 ;
C 237 ; WX 280 ; N iacute ; B 46 -3 268 813 ;
C 238 ; WX 280 ; N icircumflex ; B -70 -3 333 801 ;
C 239 ; WX 280 ; N idieresis ; B -60 -3 309 674 ;
C 240 ; WX 508 ; N eth ; B 90 -21 484 751 ;
C 241 ; WX 523 ; N ntilde ; B 60 -35 525 759 ;
C 242 ; WX 525 ; N ograve ; B 43 -29 477 812 ;
C 243 ; WX 525 ; N oacute ; B 43 -29 477 813 ;
C 244 ; WX 525 ; N ocircumflex ; B 43 -29 486 826 ;
C 245 ; WX 525 ; N otilde ; B 43 -29 534 759 ;
C 246 ; WX 525 ; N odieresis ; B 43 -29 498 674 ;
C 247 ; WX 480 ; N divide ; B 23 91 446 557 ;
C 248 ; WX 525 ; N oslash ; B 26 -29 503 511 ;
C 249 ; WX 520 ; N ugrave ; B 53 -40 476 815 ;
C 250 ; WX 520 ; N uacute ; B 53 -40 476 813 ;
C 251 ; WX 520 ; N ucircumflex ; B 53 -40 476 826 ;
C 252 ; WX 520 ; N udieresis ; B 53 -40 476 719 ;
C 253 ; WX 520 ; N yacute ; B -2 -283 500 813 ;
C 254 ; WX 534 ; N thorn ; B 54 -284 493 730 ;
C 255 ; WX 431 ; N ydieresis ; B -2 -283 500 674 ;
C -1 ; WX 731 ; N Amacron ; B 63 -15 661 876 ;
C -1 ; WX 511 ; N amacron ; B 24 -33 495 657 ;
C -1 ; WX 731 ; N Abreve ; B 63 -15 663 996 ;
C -1 ; WX 511 ; N abreve ; B 24 -33 495 776 ;
C -1 ; WX 731 ; N Aogonek ; B 63 -168 768 722 ;
C -1 ; WX 511 ; N aogonek ; B 24 -168 640 510 ;
C -1 ; WX 602 ; N Cacute ; B 43 -12 588 1033 ;
C -1 ; WX 513 ; N cacute ; B 51 -31 474 813 ;
C -1 ; WX 602 ; N Ccircumflex ; B 43 -12 589 1045 ;
C -1 ; WX 513 ; N ccircumflex ; B 51 -31 497 777 ;
C -1 ; WX 602 ; N Cdot ; B 43 -12 588 906 ;
C -1 ; WX 513 ; N cdot ; B 51 -31 474 686 ;
C -1 ; WX 602 ; N Ccaron ; B 43 -12 588 1015 ;
C -1 ; WX 602 ; N ccaron ; B 51 -31 474 795 ;
C -1 ; WX 721 ; N Dcaron ; B 89 -49 672 1015 ;
C -1 ; WX 829 ; N dcaron ; B 50 -23 736 783 ;
C -1 ; WX 721 ; N Dslash ; B -22 -49 672 760 ;
C -1 ; WX 602 ; N dcroat ; B 50 -23 597 779 ;
C -1 ; WX 624 ; N Emacron ; B 68 -49 593 916 ;
C -1 ; WX 547 ; N emacron ; B 42 -23 529 657 ;
C -1 ; WX 624 ; N Ebreve ; B 68 -49 593 1021 ;
C -1 ; WX 547 ; N ebreve ; B 42 -23 529 776 ;
C -1 ; WX 624 ; N Edot ; B 68 -49 593 930 ;
C -1 ; WX 547 ; N edot ; B 42 -23 529 686 ;
C -1 ; WX 624 ; N Eogonek ; B 97 -203 622 784 ;
C -1 ; WX 547 ; N eogonek ; B 42 -188 529 511 ;
C -1 ; WX 624 ; N Ecaron ; B 68 -49 593 1029 ;
C -1 ; WX 547 ; N ecaron ; B 42 -23 529 795 ;
C -1 ; WX 679 ; N Gcircumflex ; B 38 -34 662 1045 ;
C -1 ; WX 530 ; N gcircumflex ; B 28 -276 506 826 ;
C -1 ; WX 679 ; N Gbreve ; B 38 -34 662 996 ;
C -1 ; WX 530 ; N gbreve ; B 28 -276 517 776 ;
C -1 ; WX 679 ; N Gdot ; B 38 -34 662 906 ;
C -1 ; WX 530 ; N gdot ; B 28 -276 494 686 ;
C -1 ; WX 679 ; N Gcedilla ; B 38 -218 662 768 ;
C -1 ; WX 530 ; N gcedilla ; B 28 -276 494 827 ;
C -1 ; WX 768 ; N Hcircumflex ; B 74 -41 716 1045 ;
C -1 ; WX 577 ; N hcircumflex ; B 70 -31 555 1045 ;
C -1 ; WX 768 ; N Hbar ; B -18 -41 785 759 ;
C -1 ; WX 577 ; N hbar ; B -6 -31 527 783 ;
C -1 ; WX 546 ; N Itilde ; B 36 -18 518 979 ;
C -1 ; WX 280 ; N itilde ; B -93 -3 360 759 ;
C -1 ; WX 546 ; N Imacron ; B 36 -18 518 876 ;
C -1 ; WX 280 ; N imacron ; B -41 -3 299 657 ;
C -1 ; WX 546 ; N Ibreve ; B 36 -18 518 996 ;
C -1 ; WX 280 ; N ibreve ; B -63 -3 351 776 ;
C -1 ; WX 546 ; N Iogonek ; B 36 -168 518 730 ;
C -1 ; WX 280 ; N iogonek ; B 23 -168 279 732 ;
C -1 ; WX 546 ; N Idotaccent ; B 36 -18 518 906 ;
C -1 ; WX 280 ; N dotlessi ; B 87 -3 193 500 ;
C -1 ; WX 1126 ; N IJ ; B 36 -59 1094 740 ;
C -1 ; WX 530 ; N ij ; B 87 -292 448 732 ;
C -1 ; WX 665 ; N Jcircumflex ; B 45 -59 633 1045 ;
C -1 ; WX 403 ; N jcircumflex ; B -28 -292 375 826 ;
C -1 ; WX 610 ; N Kcedilla ; B 104 -218 607 748 ;
C -1 ; WX 540 ; N kcedilla ; B 79 -218 526 784 ;
C -1 ; WX 540 ; N kgreenlandic ; B 61 -21 507 569 ;
C -1 ; WX 550 ; N Lacute ; B 49 -46 532 1033 ;
C -1 ; WX 273 ; N lacute ; B 84 -21 325 1058 ;
C -1 ; WX 550 ; N Lcedilla ; B 49 -218 532 753 ;
C -1 ; WX 273 ; N lcedilla ; B 84 -218 330 786 ;
C -1 ; WX 550 ; N Lcaron ; B 49 -46 532 753 ;
C -1 ; WX 464 ; N lcaron ; B 84 -21 388 786 ;
C -1 ; WX 550 ; N Ldot ; B 49 -46 532 753 ;
C -1 ; WX 395 ; N ldot ; B 84 -21 396 786 ;
C -1 ; WX 550 ; N Lslash ; B -63 -46 532 753 ;
C -1 ; WX 227 ; N lslash ; B -43 -18 274 742 ;
C -1 ; WX 796 ; N Nacute ; B 60 -39 756 1033 ;
C -1 ; WX 523 ; N nacute ; B 60 -35 492 813 ;
C -1 ; WX 796 ; N Ncedilla ; B 60 -218 756 758 ;
C -1 ; WX 523 ; N ncedilla ; B 60 -218 492 534 ;
C -1 ; WX 796 ; N Ncaron ; B 60 -39 756 1015 ;
C -1 ; WX 523 ; N ncaron ; B 60 -35 492 795 ;
C -1 ; WX 621 ; N napostrophe ; B 28 -35 589 774 ;
C -1 ; WX 796 ; N Eng ; B 60 -257 756 758 ;
C -1 ; WX 523 ; N eng ; B 60 -292 497 534 ;
C -1 ; WX 798 ; N Omacron ; B 56 -30 756 876 ;
C -1 ; WX 525 ; N omacron ; B 40 -29 474 657 ;
C -1 ; WX 798 ; N Obreve ; B 56 -30 756 996 ;
C -1 ; WX 525 ; N obreve ; B 40 -29 507 776 ;
C -1 ; WX 798 ; N Odblacute ; B 56 -30 756 999 ;
C -1 ; WX 525 ; N odblacute ; B 40 -29 488 779 ;
C -1 ; WX 1193 ; N OE ; B 63 -38 1187 795 ;
C -1 ; WX 896 ; N oe ; B 40 -29 878 511 ;
C -1 ; WX 628 ; N Racute ; B 57 -18 600 1058 ;
C -1 ; WX 480 ; N racute ; B 67 -33 449 813 ;
C -1 ; WX 628 ; N Rcedilla ; B 57 -218 600 751 ;
C -1 ; WX 480 ; N rcedilla ; B 28 -218 449 515 ;
C -1 ; WX 628 ; N Rcaron ; B 57 -18 600 1015 ;
C -1 ; WX 480 ; N rcaron ; B 67 -33 449 795 ;
C -1 ; WX 693 ; N Sacute ; B 65 -28 646 1033 ;
C -1 ; WX 486 ; N sacute ; B 20 -30 446 813 ;
C -1 ; WX 693 ; N Scircumflex ; B 65 -28 646 1045 ;
C -1 ; WX 486 ; N scircumflex ; B 20 -30 446 826 ;
C -1 ; WX 693 ; N Scedilla ; B 65 -218 646 717 ;
C -1 ; WX 486 ; N scedilla ; B 20 -218 446 558 ;
C -1 ; WX 693 ; N Scaron ; B 65 -28 646 1015 ;
C -1 ; WX 403 ; N scaron ; B 20 -30 447 795 ;
C -1 ; WX 679 ; N Tcedilla ; B 56 -313 715 740 ;
C -1 ; WX 471 ; N tcedilla ; B 31 -313 443 683 ;
C -1 ; WX 679 ; N Tcaron ; B 56 -4 715 1039 ;
C -1 ; WX 639 ; N tcaron ; B 31 -32 607 683 ;
C -1 ; WX 679 ; N Tbar ; B 20 -4 679 740 ;
C -1 ; WX 471 ; N tbar ; B 31 -32 443 683 ;
C -1 ; WX 736 ; N Utilde ; B 75 -20 679 979 ;
C -1 ; WX 520 ; N utilde ; B 43 -40 497 759 ;
C -1 ; WX 736 ; N Umacron ; B 75 -20 679 876 ;
C -1 ; WX 520 ; N umacron ; B 53 -40 476 657 ;
C -1 ; WX 736 ; N Ubreve ; B 75 -20 679 996 ;
C -1 ; WX 520 ; N ubreve ; B 53 -40 478 776 ;
C -1 ; WX 736 ; N Uring ; B 75 -20 679 1079 ;
C -1 ; WX 520 ; N uring ; B 53 -40 476 859 ;
C -1 ; WX 736 ; N Udblacute ; B 75 -20 679 999 ;
C -1 ; WX 520 ; N udblacute ; B 53 -40 503 779 ;
C -1 ; WX 736 ; N Uogonek ; B 75 -217 679 734 ;
C -1 ; WX 520 ; N uogonek ; B 53 -217 548 521 ;
C -1 ; WX 1039 ; N Wcircumflex ; B 67 -47 1024 1045 ;
C -1 ; WX 684 ; N wcircumflex ; B 37 -40 658 826 ;
C -1 ; WX 635 ; N Ycircumflex ; B 14 -35 599 1045 ;
C -1 ; WX 520 ; N ycircumflex ; B -2 -283 500 826 ;
C -1 ; WX 635 ; N Ydieresis ; B 14 -35 599 896 ;
C -1 ; WX 693 ; N Zacute ; B 33 -25 675 1058 ;
C -1 ; WX 538 ; N zacute ; B 59 -38 508 813 ;
C -1 ; WX 693 ; N Zdot ; B 33 -25 675 906 ;
C -1 ; WX 538 ; N zdot ; B 59 -38 508 686 ;
C -1 ; WX 693 ; N Zcaron ; B 33 -25 675 1063 ;
C -1 ; WX 538 ; N zcaron ; B 59 -38 508 795 ;
C -1 ; WX 508 ; N longs ; B 153 -79 460 781 ;
C -1 ; WX 426 ; N florin ; B -51 -177 561 763 ;
C -1 ; WX 731 ; N Aringacute ; B 63 -15 661 1102 ;
C -1 ; WX 511 ; N aringacute ; B 32 -33 502 980 ;
C -1 ; WX 1086 ; N AEacute ; B 22 -43 1083 1058 ;
C -1 ; WX 911 ; N aeacute ; B 32 -24 876 813 ;
C -1 ; WX 798 ; N Oslashacute ; B 40 -44 797 1033 ;
C -1 ; WX 525 ; N oslashacute ; B 26 -29 503 813 ;
C -1 ; WX 556 ; N circumflex ; B 94 569 497 826 ;
C -1 ; WX 556 ; N caron ; B 138 574 466 795 ;
C -1 ; WX 556 ; N macron ; B 135 579 475 657 ;
C -1 ; WX 556 ; N breve ; B 104 575 517 776 ;
C -1 ; WX 556 ; N dotaccent ; B 154 577 274 686 ;
C -1 ; WX 556 ; N ring ; B 128 579 458 859 ;
C -1 ; WX 556 ; N ogonek ; B 145 -168 401 84 ;
C -1 ; WX 556 ; N tilde ; B 83 577 536 759 ;
C -1 ; WX 556 ; N hungarumlaut ; B 102 577 464 779 ;
C -1 ; WX 298 ; N uni037E ; B 39 -95 201 549 ;
C -1 ; WX 556 ; N tonos ; B 227 561 360 735 ;
C -1 ; WX 556 ; N dieresistonos ; B 59 561 516 735 ;
C -1 ; WX 731 ; N Alphatonos ; B 63 -15 661 760 ;
C -1 ; WX 249 ; N anoteleia ; B 69 289 191 411 ;
C -1 ; WX 789 ; N Epsilontonos ; B 69 -49 757 784 ;
C -1 ; WX 982 ; N Etatonos ; B 95 -41 931 760 ;
C -1 ; WX 756 ; N Iotatonos ; B 21 -18 728 760 ;
C -1 ; WX 876 ; N Omicrontonos ; B 46 -30 834 760 ;
C -1 ; WX 850 ; N Upsilontonos ; B 21 -35 813 760 ;
C -1 ; WX 969 ; N Omegatonos ; B 31 -69 938 796 ;
C -1 ; WX 280 ; N iotadieresistonos ; B -54 -3 403 735 ;
C -1 ; WX 731 ; N Alpha ; B 63 -15 661 722 ;
C -1 ; WX 630 ; N Beta ; B 93 -24 590 768 ;
C -1 ; WX 616 ; N Gamma ; B 88 -4 581 740 ;
C -1 ; WX 794 ; N Delta ; B 47 -36 782 813 ;
C -1 ; WX 624 ; N Epsilon ; B 68 -49 593 784 ;
C -1 ; WX 693 ; N Zeta ; B 33 -25 675 737 ;
C -1 ; WX 768 ; N Eta ; B 74 -41 716 759 ;
C -1 ; WX 798 ; N Theta ; B 56 -30 756 740 ;
C -1 ; WX 546 ; N Iota ; B 36 -18 518 730 ;
C -1 ; WX 610 ; N Kappa ; B 104 -54 607 748 ;
C -1 ; WX 689 ; N Lambda ; B 7 -40 693 760 ;
C -1 ; WX 882 ; N Mu ; B 54 -41 846 750 ;
C -1 ; WX 796 ; N Nu ; B 60 -39 756 758 ;
C -1 ; WX 720 ; N Xi ; B 41 -16 693 736 ;
C -1 ; WX 798 ; N Omicron ; B 56 -30 756 740 ;
C -1 ; WX 908 ; N Pi ; B 51 -19 858 766 ;
C -1 ; WX 520 ; N Rho ; B 48 -12 491 768 ;
C -1 ; WX 747 ; N Sigma ; B 60 -14 679 743 ;
C -1 ; WX 679 ; N Tau ; B 56 -4 715 740 ;
C -1 ; WX 635 ; N Upsilon ; B 14 -35 599 743 ;
C -1 ; WX 638 ; N Phi ; B 44 -18 622 730 ;
C -1 ; WX 723 ; N Chi ; B 33 -42 687 743 ;
C -1 ; WX 673 ; N Psi ; B 34 -13 650 756 ;
C -1 ; WX 959 ; N Omega ; B 31 -69 929 796 ;
C -1 ; WX 546 ; N Iotadieresis ; B 36 -18 518 904 ;
C -1 ; WX 635 ; N Upsilondieresis ; B 14 -35 599 896 ;
C -1 ; WX 532 ; N alphatonos ; B 24 -33 505 735 ;
C -1 ; WX 491 ; N epsilontonos ; B 76 -26 445 735 ;
C -1 ; WX 523 ; N etatonos ; B 60 -157 501 735 ;
C -1 ; WX 280 ; N iotatonos ; B 85 -3 218 735 ;
C -1 ; WX 520 ; N upsilondieresistonos ; B 36 -31 516 735 ;
C -1 ; WX 532 ; N alpha ; B 24 -33 505 533 ;
C -1 ; WX 586 ; N beta ; B 54 -197 535 723 ;
C -1 ; WX 486 ; N gamma ; B 30 -164 474 516 ;
C -1 ; WX 525 ; N delta ; B 40 -29 491 742 ;
C -1 ; WX 491 ; N epsilon ; B 76 -26 445 514 ;
C -1 ; WX 513 ; N zeta ; B 51 -203 497 757 ;
C -1 ; WX 523 ; N eta ; B 60 -157 501 534 ;
C -1 ; WX 610 ; N theta ; B 29 -20 576 760 ;
C -1 ; WX 280 ; N iota ; B 87 -3 193 500 ;
C -1 ; WX 540 ; N kappa ; B 60 -21 506 522 ;
C -1 ; WX 486 ; N lambda ; B 30 -12 474 667 ;
C -1 ; WX 520 ; N mu ; B 53 -196 476 521 ;
C -1 ; WX 486 ; N nu ; B 30 -20 474 516 ;
C -1 ; WX 513 ; N xi ; B 51 -203 446 778 ;
C -1 ; WX 525 ; N omicron ; B 40 -29 474 507 ;
C -1 ; WX 627 ; N pi ; B 18 -21 613 512 ;
C -1 ; WX 525 ; N rho ; B 15 -203 474 507 ;
C -1 ; WX 513 ; N sigma1 ; B 51 -203 464 520 ;
C -1 ; WX 525 ; N sigma ; B 40 -29 546 507 ;
C -1 ; WX 471 ; N tau ; B 31 -32 443 517 ;
C -1 ; WX 520 ; N upsilon ; B 36 -31 487 527 ;
C -1 ; WX 615 ; N phi ; B 55 -196 560 504 ;
C -1 ; WX 590 ; N chi ; B 15 -182 531 524 ;
C -1 ; WX 585 ; N psi ; B 31 -200 555 533 ;
C -1 ; WX 760 ; N omega ; B 27 -21 710 522 ;
C -1 ; WX 280 ; N iotadieresis ; B -60 -3 309 674 ;
C -1 ; WX 520 ; N upsilondieresis ; B 36 -31 487 674 ;
C -1 ; WX 525 ; N omicrontonos ; B 40 -29 474 735 ;
C -1 ; WX 520 ; N upsilontonos ; B 36 -31 487 735 ;
C -1 ; WX 760 ; N omegatonos ; B 27 -21 710 735 ;
C -1 ; WX 624 ; N afii10023 ; B 68 -49 593 952 ;
C -1 ; WX 850 ; N afii10051 ; B 20 -137 802 740 ;
C -1 ; WX 616 ; N afii10052 ; B 88 -4 581 1033 ;
C -1 ; WX 602 ; N afii10053 ; B 43 -12 588 744 ;
C -1 ; WX 693 ; N afii10054 ; B 65 -28 646 717 ;
C -1 ; WX 546 ; N afii10055 ; B 36 -18 518 730 ;
C -1 ; WX 546 ; N afii10056 ; B 36 -18 518 904 ;
C -1 ; WX 665 ; N afii10057 ; B 45 -59 633 740 ;
C -1 ; WX 1000 ; N afii10058 ; B 12 -10 967 745 ;
C -1 ; WX 977 ; N afii10059 ; B 23 -13 938 759 ;
C -1 ; WX 846 ; N afii10060 ; B 20 -20 773 740 ;
C -1 ; WX 633 ; N afii10061 ; B 78 -8 592 1033 ;
C -1 ; WX 635 ; N afii10062 ; B 14 -19 599 996 ;
C -1 ; WX 729 ; N afii10145 ; B 60 -136 677 750 ;
C -1 ; WX 731 ; N afii10017 ; B 63 -15 661 722 ;
C -1 ; WX 604 ; N afii10018 ; B 75 -7 552 744 ;
C -1 ; WX 630 ; N afii10019 ; B 93 -24 590 768 ;
C -1 ; WX 616 ; N afii10020 ; B 88 -4 581 740 ;
C -1 ; WX 768 ; N afii10021 ; B 43 -164 728 754 ;
C -1 ; WX 624 ; N afii10022 ; B 68 -49 593 784 ;
C -1 ; WX 859 ; N afii10024 ; B 14 -8 839 751 ;
C -1 ; WX 601 ; N afii10025 ; B 32 -21 548 737 ;
C -1 ; WX 747 ; N afii10026 ; B 63 -26 678 756 ;
C -1 ; WX 747 ; N afii10027 ; B 63 -26 678 996 ;
C -1 ; WX 633 ; N afii10028 ; B 78 -8 592 751 ;
C -1 ; WX 738 ; N afii10029 ; B 12 -11 652 745 ;
C -1 ; WX 882 ; N afii10030 ; B 54 -41 846 750 ;
C -1 ; WX 768 ; N afii10031 ; B 74 -41 716 759 ;
C -1 ; WX 798 ; N afii10032 ; B 56 -30 756 740 ;
C -1 ; WX 908 ; N afii10033 ; B 51 -19 858 766 ;
C -1 ; WX 520 ; N afii10034 ; B 48 -12 491 768 ;
C -1 ; WX 602 ; N afii10035 ; B 43 -12 588 744 ;
C -1 ; WX 679 ; N afii10036 ; B 56 -4 715 740 ;
C -1 ; WX 635 ; N afii10037 ; B 14 -19 599 743 ;
C -1 ; WX 638 ; N afii10038 ; B 44 -18 622 730 ;
C -1 ; WX 723 ; N afii10039 ; B 33 -42 687 743 ;
C -1 ; WX 773 ; N afii10040 ; B 60 -145 750 750 ;
C -1 ; WX 617 ; N afii10041 ; B 51 -15 548 754 ;
C -1 ; WX 906 ; N afii10042 ; B 48 -8 848 750 ;
C -1 ; WX 957 ; N afii10043 ; B 48 -145 934 750 ;
C -1 ; WX 759 ; N afii10044 ; B 18 -7 708 741 ;
C -1 ; WX 1075 ; N afii10045 ; B 75 -18 1030 740 ;
C -1 ; WX 604 ; N afii10046 ; B 75 -7 552 740 ;
C -1 ; WX 654 ; N afii10047 ; B 42 -7 604 733 ;
C -1 ; WX 1130 ; N afii10048 ; B 41 -30 1088 751 ;
C -1 ; WX 628 ; N afii10049 ; B 12 -19 574 751 ;
C -1 ; WX 511 ; N afii10065 ; B 24 -33 495 510 ;
C -1 ; WX 525 ; N afii10066 ; B 37 -29 504 772 ;
C -1 ; WX 503 ; N afii10067 ; B 63 -23 468 512 ;
C -1 ; WX 470 ; N afii10068 ; B 65 -23 458 518 ;
C -1 ; WX 592 ; N afii10069 ; B 16 -128 570 520 ;
C -1 ; WX 547 ; N afii10070 ; B 42 -23 529 511 ;
C -1 ; WX 682 ; N afii10072 ; B 32 -13 635 519 ;
C -1 ; WX 439 ; N afii10073 ; B 51 -23 388 523 ;
C -1 ; WX 636 ; N afii10074 ; B 48 -29 559 513 ;
C -1 ; WX 636 ; N afii10075 ; B 48 -29 559 776 ;
C -1 ; WX 492 ; N afii10076 ; B 62 -13 474 519 ;
C -1 ; WX 646 ; N afii10077 ; B 14 -20 578 520 ;
C -1 ; WX 734 ; N afii10078 ; B 21 -29 706 520 ;
C -1 ; WX 559 ; N afii10079 ; B 36 -26 501 541 ;
C -1 ; WX 591 ; N afii10080 ; B 33 -31 559 531 ;
C -1 ; WX 646 ; N afii10081 ; B 103 -22 580 511 ;
C -1 ; WX 534 ; N afii10082 ; B 58 -284 493 537 ;
C -1 ; WX 513 ; N afii10083 ; B 51 -31 474 520 ;
C -1 ; WX 471 ; N afii10084 ; B 31 -32 443 517 ;
C -1 ; WX 520 ; N afii10085 ; B -2 -283 500 508 ;
C -1 ; WX 810 ; N afii10086 ; B 41 -284 783 537 ;
C -1 ; WX 590 ; N afii10087 ; B 29 -22 563 540 ;
C -1 ; WX 618 ; N afii10088 ; B 86 -123 591 530 ;
C -1 ; WX 533 ; N afii10089 ; B 36 -26 480 533 ;
C -1 ; WX 771 ; N afii10090 ; B 52 -25 721 530 ;
C -1 ; WX 792 ; N afii10091 ; B 52 -163 784 530 ;
C -1 ; WX 621 ; N afii10092 ; B 8 -7 596 532 ;
C -1 ; WX 790 ; N afii10093 ; B 62 -22 765 516 ;
C -1 ; WX 491 ; N afii10094 ; B 62 -7 449 516 ;
C -1 ; WX 513 ; N afii10095 ; B 52 -21 469 515 ;
C -1 ; WX 902 ; N afii10096 ; B 41 -31 860 548 ;
C -1 ; WX 505 ; N afii10097 ; B 16 -19 447 512 ;
C -1 ; WX 547 ; N afii10071 ; B 42 -23 529 722 ;
C -1 ; WX 560 ; N afii10099 ; B -33 -200 500 783 ;
C -1 ; WX 470 ; N afii10100 ; B 65 -23 458 813 ;
C -1 ; WX 513 ; N afii10101 ; B 38 -25 455 511 ;
C -1 ; WX 486 ; N afii10102 ; B 20 -30 446 558 ;
C -1 ; WX 280 ; N afii10103 ; B 87 -3 219 732 ;
C -1 ; WX 280 ; N afii10104 ; B -60 -3 309 674 ;
C -1 ; WX 403 ; N afii10105 ; B -9 -292 321 731 ;
C -1 ; WX 889 ; N afii10106 ; B 14 -19 866 529 ;
C -1 ; WX 773 ; N afii10107 ; B 43 -11 734 517 ;
C -1 ; WX 577 ; N afii10108 ; B -33 -31 527 783 ;
C -1 ; WX 492 ; N afii10109 ; B 62 -13 474 813 ;
C -1 ; WX 520 ; N afii10110 ; B -2 -283 500 776 ;
C -1 ; WX 618 ; N afii10193 ; B 66 -168 535 530 ;
C -1 ; WX 616 ; N afii10050 ; B 88 -4 581 868 ;
C -1 ; WX 470 ; N afii10098 ; B 65 -23 461 656 ;
C -1 ; WX 1039 ; N Wgrave ; B 67 -47 1024 1031 ;
C -1 ; WX 684 ; N wgrave ; B 37 -40 658 812 ;
C -1 ; WX 1039 ; N Wacute ; B 67 -47 1024 1033 ;
C -1 ; WX 684 ; N wacute ; B 37 -40 658 813 ;
C -1 ; WX 1039 ; N Wdieresis ; B 67 -47 1024 894 ;
C -1 ; WX 684 ; N wdieresis ; B 37 -40 658 674 ;
C -1 ; WX 635 ; N Ygrave ; B 14 -35 599 1031 ;
C -1 ; WX 520 ; N ygrave ; B -2 -283 500 812 ;
C -1 ; WX 441 ; N endash ; B 21 265 405 371 ;
C -1 ; WX 882 ; N emdash ; B 32 266 834 366 ;
C -1 ; WX 882 ; N afii00208 ; B 32 266 834 366 ;
C -1 ; WX 626 ; N underscoredbl ; B -18 -267 646 -27 ;
C -1 ; WX 180 ; N quoteleft ; B 37 534 139 775 ;
C -1 ; WX 180 ; N quoteright ; B 28 534 135 774 ;
C -1 ; WX 298 ; N quotesinglbase ; B 39 -95 199 110 ;
C -1 ; WX 180 ; N quotereversed ; B 28 534 135 774 ;
C -1 ; WX 393 ; N quotedblleft ; B 34 531 327 789 ;
C -1 ; WX 393 ; N quotedblright ; B 50 487 355 772 ;
C -1 ; WX 411 ; N quotedblbase ; B 39 -95 372 110 ;
C -1 ; WX 596 ; N dagger ; B 24 -102 565 829 ;
C -1 ; WX 596 ; N daggerdbl ; B 28 -102 569 829 ;
C -1 ; WX 387 ; N bullet ; B 89 272 285 469 ;
C -1 ; WX 675 ; N ellipsis ; B 86 -33 607 79 ;
C -1 ; WX 1241 ; N perthousand ; B 71 -15 1158 802 ;
C -1 ; WX 180 ; N minute ; B 28 534 135 774 ;
C -1 ; WX 331 ; N second ; B 28 534 301 774 ;
C -1 ; WX 381 ; N guilsinglleft ; B 51 94 349 519 ;
C -1 ; WX 381 ; N guilsinglright ; B 51 94 349 519 ;
C -1 ; WX 449 ; N exclamdbl ; B 58 -33 377 784 ;
C -1 ; WX 610 ; N radicalex ; B -76 696 644 794 ;
C -1 ; WX 811 ; N fraction ; B 12 -27 783 831 ;
C -1 ; WX 526 ; N nsuperior ; B 95 391 425 781 ;
C -1 ; WX 989 ; N franc ; B 84 -52 958 769 ;
C -1 ; WX 793 ; N afii08941 ; B 41 -74 711 787 ;
C -1 ; WX 1188 ; N peseta ; B 48 -32 1143 768 ;
C -1 ; WX 610 ; N Euro ; B -22 -17 609 773 ;
C -1 ; WX 820 ; N afii61248 ; B 38 -15 786 783 ;
C -1 ; WX 488 ; N afii61289 ; B 9 -21 407 801 ;
C -1 ; WX 1157 ; N afii61352 ; B 60 -39 1100 758 ;
C -1 ; WX 882 ; N trademark ; B 37 413 853 786 ;
C -1 ; WX 959 ; N Ohm ; B 31 -69 929 796 ;
C -1 ; WX 549 ; N estimated ; B 29 -10 520 533 ;
C -1 ; WX 650 ; N oneeighth ; B 79 -191 592 842 ;
C -1 ; WX 650 ; N threeeighths ; B 79 -191 592 831 ;
C -1 ; WX 650 ; N fiveeighths ; B 79 -191 592 838 ;
C -1 ; WX 650 ; N seveneighths ; B 79 -191 592 836 ;
C -1 ; WX 508 ; N partialdiff ; B 90 -21 484 751 ;
C -1 ; WX 794 ; N Delta ; B 47 -36 782 813 ;
C -1 ; WX 908 ; N product ; B 51 -136 858 766 ;
C -1 ; WX 747 ; N summation ; B 21 -153 711 726 ;
C -1 ; WX 480 ; N minus ; B 23 266 446 360 ;
C -1 ; WX 249 ; N uni2219 ; B 69 289 191 411 ;
C -1 ; WX 607 ; N radical ; B 24 -34 588 796 ;
C -1 ; WX 836 ; N infinity ; B 71 149 765 470 ;
C -1 ; WX 699 ; N integral ; B 39 -198 642 840 ;
C -1 ; WX 619 ; N approxequal ; B 60 120 589 586 ;
C -1 ; WX 510 ; N notequal ; B 48 40 430 598 ;
C -1 ; WX 381 ; N lessequal ; B 9 -63 369 519 ;
C -1 ; WX 381 ; N greaterequal ; B 17 -63 369 550 ;
C -1 ; WX 734 ; N H22073 ; B 177 0 559 381 ;
C -1 ; WX 642 ; N H18543 ; B 188 60 449 321 ;
C -1 ; WX 642 ; N H18551 ; B 180 51 458 330 ;
C -1 ; WX 722 ; N lozenge ; B 29 -41 699 799 ;
C -1 ; WX 642 ; N H18533 ; B 183 59 449 324 ;
C -1 ; WX 642 ; N openbullet ; B 174 50 458 333 ;
C -1 ; WX 658 ; N fi ; B 36 -79 597 781 ;
C -1 ; WX 655 ; N fl ; B 34 -79 605 790 ;
C -1 ; WX 388 ; N uF003 ; B 136 582 276 829 ;
C -1 ; WX 388 ; N uF004 ; B 167 582 295 827 ;
C -1 ; WX 388 ; N uF005 ; B 167 582 295 827 ;
C -1 ; WX 0 ; N .null ; B 0 0 0 0 ;
C -1 ; WX 476 ; N nonmarkingreturn ; B 0 0 0 0 ;
C -1 ; WX 416 ; N sfthyphen ; B 54 225 371 309 ;
C -1 ; WX 249 ; N middot ; B 69 289 191 411 ;
EndCharMetrics
EndFontMetrics
%    \end{macrocode}
% \fi
%
%</rcomic.afm>
%
% \bigskip
%
%<*rcomicbd.afm>
%
% \begin{center}
%   $\vdots$ \\
%   598 lines of code omitted \\
%   $\vdots$
% \end{center}
%
% \iffalse
%    \begin{macrocode}
StartFontMetrics 2.0
Comment Generated by pfaedit
Comment Creation Date: Wed Jul 17 20:26:52 2002
FontName ComicSansMS-Bold
FullName Comic Sans MS Bold
FamilyName ComicSansMS
Weight
Notice (Copyright (c) 1995 Microsoft Corporation. All rights reserved.)
ItalicAngle 0
IsFixedPitch false
UnderlinePosition -272
UnderlineThickness 175
Version Version 2.10
EncodingScheme ISO10646-1
FontBBox -112 -292 1230 1103
CapHeight 729
XHeight 539
Ascender 783
Descender -283
StartCharMetrics 576
C 0 ; WX 500 ; N .notdef ; B 62 0 438 800 ;
C 32 ; WX 433 ; N space ; B 0 0 0 0 ;
C 33 ; WX 237 ; N exclam ; B 39 -33 182 784 ;
C 34 ; WX 437 ; N quotedbl ; B 37 453 395 776 ;
C 35 ; WX 842 ; N numbersign ; B -14 -12 831 787 ;
C 36 ; WX 693 ; N dollar ; B 47 -189 644 838 ;
C 37 ; WX 820 ; N percent ; B 52 -15 806 802 ;
C 38 ; WX 654 ; N ampersand ; B 17 -46 639 764 ;
C 39 ; WX 433 ; N quotesingle ; B 147 453 281 774 ;
C 40 ; WX 366 ; N parenleft ; B 34 -211 361 785 ;
C 41 ; WX 366 ; N parenright ; B 34 -211 361 785 ;
C 42 ; WX 529 ; N asterisk ; B 3 381 490 779 ;
C 43 ; WX 610 ; N plus ; B 81 113 543 511 ;
C 44 ; WX 433 ; N comma ; B 105 -168 292 70 ;
C 45 ; WX 610 ; N hyphen ; B 106 205 501 328 ;
C 46 ; WX 433 ; N period ; B 134 -51 295 83 ;
C 47 ; WX 511 ; N slash ; B 21 -44 494 794 ;
C 48 ; WX 610 ; N zero ; B 10 -20 596 760 ;
C 49 ; WX 610 ; N one ; B 115 -1 469 762 ;
C 50 ; WX 610 ; N two ; B 60 -2 558 750 ;
C 51 ; WX 610 ; N three ; B 51 -23 549 746 ;
C 52 ; WX 610 ; N four ; B 3 -13 601 762 ;
C 53 ; WX 610 ; N five ; B 41 -31 583 751 ;
C 54 ; WX 610 ; N six ; B 43 -36 571 760 ;
C 55 ; WX 610 ; N seven ; B 14 -33 613 737 ;
C 56 ; WX 610 ; N eight ; B 39 -27 570 746 ;
C 57 ; WX 610 ; N nine ; B 25 -47 575 750 ;
C 58 ; WX 433 ; N colon ; B 147 60 300 552 ;
C 59 ; WX 433 ; N semicolon ; B 104 -95 305 549 ;
C 60 ; WX 610 ; N less ; B 117 94 453 519 ;
C 61 ; WX 610 ; N equal ; B 92 134 513 489 ;
C 62 ; WX 610 ; N greater ; B 143 90 514 550 ;
C 63 ; WX 565 ; N question ; B 41 -36 524 751 ;
C 64 ; WX 931 ; N at ; B 34 -69 876 796 ;
C 65 ; WX 731 ; N A ; B 44 -15 680 722 ;
C 66 ; WX 630 ; N B ; B 65 -24 602 768 ;
C 67 ; WX 618 ; N C ; B 24 -12 607 744 ;
C 68 ; WX 721 ; N D ; B 69 -49 692 760 ;
C 69 ; WX 624 ; N E ; B 45 -49 609 784 ;
C 70 ; WX 606 ; N F ; B 45 -52 584 769 ;
C 71 ; WX 679 ; N G ; B 19 -34 682 768 ;
C 72 ; WX 768 ; N H ; B 54 -41 735 759 ;
C 73 ; WX 546 ; N I ; B 17 -18 538 730 ; L J IJ ;
C 74 ; WX 665 ; N J ; B 26 -59 653 740 ;
C 75 ; WX 610 ; N K ; B 52 -54 595 748 ;
C 76 ; WX 550 ; N L ; B 29 -46 552 753 ;
C 77 ; WX 882 ; N M ; B 35 -41 865 750 ;
C 78 ; WX 812 ; N N ; B 40 -39 776 758 ;
C 79 ; WX 798 ; N O ; B 29 -30 768 740 ;
C 80 ; WX 532 ; N P ; B 29 -12 511 768 ;
C 81 ; WX 876 ; N Q ; B 18 -214 875 740 ;
C 82 ; WX 640 ; N R ; B 44 -18 620 751 ;
C 83 ; WX 693 ; N S ; B 23 -30 673 717 ;
C 84 ; WX 695 ; N T ; B 4 -4 702 740 ;
C 85 ; WX 736 ; N U ; B 38 -36 690 740 ;
C 86 ; WX 674 ; N V ; B 35 -40 651 750 ;
C 87 ; WX 1039 ; N W ; B 24 -47 1020 746 ;
C 88 ; WX 723 ; N X ; B 5 -42 699 743 ;
C 89 ; WX 635 ; N Y ; B -6 -35 618 743 ;
C 90 ; WX 693 ; N Z ; B 5 -25 687 737 ;
C 91 ; WX 376 ; N bracketleft ; B 65 -204 363 743 ;
C 92 ; WX 549 ; N backslash ; B 66 -69 516 745 ;
C 93 ; WX 376 ; N bracketright ; B 30 -204 327 743 ;
C 94 ; WX 610 ; N asciicircum ; B 84 547 525 804 ;
C 95 ; WX 626 ; N underscore ; B -37 -169 666 -77 ;
C 96 ; WX 556 ; N grave ; B 53 575 302 812 ;
C 97 ; WX 555 ; N a ; B 27 -33 537 510 ;
C 98 ; WX 593 ; N b ; B 43 -21 552 770 ;
C 99 ; WX 513 ; N c ; B 17 -31 479 520 ;
C 100 ; WX 587 ; N d ; B 23 -23 550 779 ;
C 101 ; WX 559 ; N e ; B 22 -23 523 511 ;
C 102 ; WX 508 ; N f ; B 17 -79 480 781 ;
C 103 ; WX 530 ; N g ; B 8 -276 513 500 ;
C 104 ; WX 577 ; N h ; B 51 -31 546 783 ;
C 105 ; WX 280 ; N i ; B 67 -15 238 732 ; L j ij ;
C 106 ; WX 403 ; N j ; B -29 -292 340 731 ;
C 107 ; WX 540 ; N k ; B 45 -21 531 783 ;
C 108 ; WX 273 ; N l ; B 65 -21 201 784 ;
C 109 ; WX 776 ; N m ; B 41 -61 757 542 ;
C 110 ; WX 523 ; N n ; B 33 -35 503 534 ;
C 111 ; WX 525 ; N o ; B 20 -29 494 507 ;
C 112 ; WX 534 ; N p ; B 38 -284 512 537 ;
C 113 ; WX 520 ; N q ; B 9 -272 480 520 ;
C 114 ; WX 480 ; N r ; B 40 -33 460 515 ;
C 115 ; WX 486 ; N s ; B 25 -30 451 521 ;
C 116 ; WX 471 ; N t ; B 12 -32 462 683 ;
C 117 ; WX 520 ; N u ; B 25 -40 487 521 ;
C 118 ; WX 486 ; N v ; B 10 -20 494 516 ;
C 119 ; WX 684 ; N w ; B 17 -40 677 509 ;
C 120 ; WX 590 ; N x ; B 10 -22 582 540 ;
C 121 ; WX 552 ; N y ; B 2 -283 544 508 ;
C 122 ; WX 538 ; N z ; B 23 -38 512 516 ;
C 123 ; WX 366 ; N braceleft ; B -17 -188 360 794 ;
C 124 ; WX 421 ; N bar ; B 151 -177 279 838 ;
C 125 ; WX 366 ; N braceright ; B -17 -188 360 794 ;
C 126 ; WX 610 ; N asciitilde ; B 22 228 571 457 ;
C 160 ; WX 476 ; N nonbreakingspace ; B 0 0 0 0 ;
C 161 ; WX 237 ; N exclamdown ; B 60 -33 204 784 ;
C 162 ; WX 610 ; N cent ; B 38 16 562 849 ;
C 163 ; WX 793 ; N sterling ; B -2 -74 732 787 ;
C 164 ; WX 610 ; N currency ; B 0 87 614 645 ;
C 165 ; WX 610 ; N yen ; B 23 -26 571 712 ;
C 166 ; WX 421 ; N brokenbar ; B 154 -177 280 836 ;
C 167 ; WX 634 ; N section ; B 37 -38 563 802 ;
C 168 ; WX 556 ; N dieresis ; B 93 568 495 701 ;
C 169 ; WX 795 ; N copyright ; B 24 122 763 792 ;
C 170 ; WX 610 ; N ordfeminine ; B 64 435 500 821 ;
C 171 ; WX 638 ; N guillemotleft ; B 3 83 635 519 ;
C 172 ; WX 610 ; N logicalnot ; B 74 91 536 395 ;
C 173 ; WX 610 ; N hyphenminus ; B 106 205 501 328 ;
C 174 ; WX 795 ; N registered ; B 24 101 771 797 ;
C 175 ; WX 626 ; N overscore ; B -37 831 666 923 ;
C 176 ; WX 610 ; N degree ; B 107 475 488 821 ;
C 177 ; WX 610 ; N plusminus ; B 74 -63 537 511 ;
C 178 ; WX 610 ; N twosuperior ; B 145 440 480 854 ;
C 179 ; WX 610 ; N threesuperior ; B 154 405 454 841 ;
C 180 ; WX 556 ; N acute ; B 84 579 345 813 ;
C 181 ; WX 610 ; N mu1 ; B -2 -199 604 521 ;
C 182 ; WX 760 ; N paragraph ; B 15 -96 702 810 ;
C 183 ; WX 610 ; N periodcentered ; B 221 283 382 417 ;
C 184 ; WX 556 ; N cedilla ; B 159 -218 442 57 ;
C 185 ; WX 610 ; N onesuperior ; B 155 418 466 842 ;
C 186 ; WX 610 ; N ordmasculine ; B 84 435 525 831 ;
C 187 ; WX 638 ; N guillemotright ; B 3 83 635 519 ;
C 188 ; WX 610 ; N onequarter ; B 38 -191 590 842 ;
C 189 ; WX 610 ; N onehalf ; B 31 -200 583 842 ;
C 190 ; WX 610 ; N threequarters ; B 31 -191 583 841 ;
C 191 ; WX 565 ; N questiondown ; B 41 -36 524 781 ;
C 192 ; WX 731 ; N Agrave ; B 63 -15 700 1036 ;
C 193 ; WX 731 ; N Aacute ; B 63 -15 700 1030 ;
C 194 ; WX 731 ; N Acircumflex ; B 63 -15 700 1054 ;
C 195 ; WX 731 ; N Atilde ; B 63 -15 746 980 ;
C 196 ; WX 731 ; N Adieresis ; B 63 -15 700 921 ;
C 197 ; WX 731 ; N Aring ; B 63 -15 700 978 ;
C 198 ; WX 1086 ; N AE ; B 2 -43 1102 792 ;
C 199 ; WX 618 ; N Ccedilla ; B 24 -199 607 745 ;
C 200 ; WX 624 ; N Egrave ; B 45 -49 609 1089 ;
C 201 ; WX 624 ; N Eacute ; B 45 -49 609 1095 ;
C 202 ; WX 624 ; N Ecircumflex ; B 45 -49 609 1103 ;
C 203 ; WX 624 ; N Edieresis ; B 45 -49 609 979 ;
C 204 ; WX 546 ; N Igrave ; B 36 -18 557 1027 ;
C 205 ; WX 546 ; N Iacute ; B 17 -18 538 1030 ;
C 206 ; WX 546 ; N Icircumflex ; B 36 -18 557 1063 ;
C 207 ; WX 546 ; N Idieresis ; B 17 -18 538 931 ;
C 208 ; WX 721 ; N Eth ; B -42 -49 692 760 ;
C 209 ; WX 812 ; N Ntilde ; B 40 -39 776 989 ;
C 210 ; WX 798 ; N Ograve ; B 56 -30 795 1037 ;
C 211 ; WX 798 ; N Oacute ; B 29 -30 768 1033 ;
C 212 ; WX 798 ; N Ocircumflex ; B 56 -30 795 1042 ;
C 213 ; WX 798 ; N Otilde ; B 56 -30 795 980 ;
C 214 ; WX 798 ; N Odieresis ; B 29 -30 768 934 ;
C 215 ; WX 610 ; N multiply ; B 97 109 520 499 ;
C 216 ; WX 798 ; N Oslash ; B 20 -44 816 753 ;
C 217 ; WX 736 ; N Ugrave ; B 38 -36 690 1034 ;
C 218 ; WX 736 ; N Uacute ; B 38 -36 690 1036 ;
C 219 ; WX 736 ; N Ucircumflex ; B 38 -36 690 1048 ;
C 220 ; WX 736 ; N Udieresis ; B 38 -36 690 922 ;
C 221 ; WX 635 ; N Yacute ; B 14 -35 638 1078 ;
C 222 ; WX 520 ; N Thorn ; B 29 -12 514 758 ;
C 223 ; WX 533 ; N germandbls ; B 41 -77 491 767 ;
C 224 ; WX 555 ; N agrave ; B 27 -33 537 808 ;
C 225 ; WX 555 ; N aacute ; B 27 -33 537 813 ;
C 226 ; WX 555 ; N acircumflex ; B 27 -33 537 830 ;
C 227 ; WX 555 ; N atilde ; B 27 -33 537 750 ;
C 228 ; WX 555 ; N adieresis ; B 27 -33 537 701 ;
C 229 ; WX 555 ; N aring ; B 27 -33 537 859 ;
C 230 ; WX 911 ; N ae ; B 13 -24 896 511 ;
C 231 ; WX 513 ; N ccedilla ; B 31 -218 493 520 ;
C 232 ; WX 559 ; N egrave ; B 22 -23 523 812 ;
C 233 ; WX 559 ; N eacute ; B 22 -23 523 813 ;
C 234 ; WX 559 ; N ecircumflex ; B 22 -23 523 826 ;
C 235 ; WX 559 ; N edieresis ; B 22 -23 523 749 ;
C 236 ; WX 280 ; N igrave ; B 12 -15 261 812 ;
C 237 ; WX 280 ; N iacute ; B 46 -15 307 813 ;
C 238 ; WX 280 ; N icircumflex ; B -70 -15 372 826 ;
C 239 ; WX 280 ; N idieresis ; B -54 -15 348 701 ;
C 240 ; WX 508 ; N eth ; B 42 -21 476 751 ;
C 241 ; WX 523 ; N ntilde ; B 33 -35 544 759 ;
C 242 ; WX 525 ; N ograve ; B 43 -29 517 812 ;
C 243 ; WX 525 ; N oacute ; B 24 -29 498 813 ;
C 244 ; WX 525 ; N ocircumflex ; B 43 -29 525 826 ;
C 245 ; WX 525 ; N otilde ; B 43 -29 574 759 ;
C 246 ; WX 525 ; N odieresis ; B 24 -29 507 701 ;
C 247 ; WX 610 ; N divide ; B 74 91 537 557 ;
C 248 ; WX 533 ; N oslash ; B 15 -34 522 507 ;
C 249 ; WX 520 ; N ugrave ; B 53 -40 515 815 ;
C 250 ; WX 520 ; N uacute ; B 53 -40 515 813 ;
C 251 ; WX 520 ; N ucircumflex ; B 53 -40 535 826 ;
C 252 ; WX 520 ; N udieresis ; B 53 -40 521 701 ;
C 253 ; WX 552 ; N yacute ; B 2 -283 544 813 ;
C 254 ; WX 534 ; N thorn ; B 35 -284 512 730 ;
C 255 ; WX 552 ; N ydieresis ; B 2 -283 544 701 ;
C -1 ; WX 731 ; N Amacron ; B 44 -15 680 896 ;
C -1 ; WX 555 ; N amacron ; B 27 -33 537 676 ;
C -1 ; WX 731 ; N Abreve ; B 44 -15 682 996 ;
C -1 ; WX 555 ; N abreve ; B 27 -33 537 776 ;
C -1 ; WX 731 ; N Aogonek ; B 44 -168 753 722 ;
C -1 ; WX 555 ; N aogonek ; B 27 -168 646 510 ;
C -1 ; WX 618 ; N Cacute ; B 24 -12 607 1033 ;
C -1 ; WX 513 ; N cacute ; B 17 -31 479 813 ;
C -1 ; WX 618 ; N Ccircumflex ; B 24 -12 624 1070 ;
C -1 ; WX 513 ; N ccircumflex ; B 17 -31 516 826 ;
C -1 ; WX 618 ; N Cdot ; B 24 -12 607 925 ;
C -1 ; WX 513 ; N cdot ; B 17 -31 479 706 ;
C -1 ; WX 618 ; N Ccaron ; B 24 -12 607 1015 ;
C -1 ; WX 513 ; N ccaron ; B 17 -31 486 795 ;
C -1 ; WX 721 ; N Dcaron ; B 69 -49 692 1015 ;
C -1 ; WX 764 ; N dcaron ; B 23 -23 741 783 ;
C -1 ; WX 721 ; N Dslash ; B -42 -49 692 760 ;
C -1 ; WX 587 ; N dcroat ; B 23 -23 633 779 ;
C -1 ; WX 624 ; N Emacron ; B 45 -49 609 935 ;
C -1 ; WX 559 ; N emacron ; B 22 -23 523 676 ;
C -1 ; WX 624 ; N Ebreve ; B 45 -49 609 1021 ;
C -1 ; WX 559 ; N ebreve ; B 22 -23 535 776 ;
C -1 ; WX 624 ; N Edot ; B 45 -49 609 964 ;
C -1 ; WX 559 ; N edot ; B 22 -23 523 706 ;
C -1 ; WX 624 ; N Eogonek ; B 45 -198 609 784 ;
C -1 ; WX 559 ; N eogonek ; B 22 -188 523 511 ;
C -1 ; WX 624 ; N Ecaron ; B 45 -49 609 1029 ;
C -1 ; WX 559 ; N ecaron ; B 22 -23 523 795 ;
C -1 ; WX 679 ; N Gcircumflex ; B 19 -34 682 1045 ;
C -1 ; WX 530 ; N gcircumflex ; B 8 -276 526 826 ;
C -1 ; WX 679 ; N Gbreve ; B 19 -34 682 996 ;
C -1 ; WX 530 ; N gbreve ; B 8 -276 535 776 ;
C -1 ; WX 679 ; N Gdot ; B 19 -34 682 925 ;
C -1 ; WX 530 ; N gdot ; B 8 -276 513 706 ;
C -1 ; WX 679 ; N Gcedilla ; B 19 -218 682 768 ;
C -1 ; WX 530 ; N gcedilla ; B 8 -276 513 827 ;
C -1 ; WX 768 ; N Hcircumflex ; B 54 -41 735 1045 ;
C -1 ; WX 577 ; N hcircumflex ; B 51 -31 628 1045 ;
C -1 ; WX 874 ; N Hbar ; B 21 -41 867 759 ;
C -1 ; WX 577 ; N hbar ; B -46 -31 546 783 ;
C -1 ; WX 546 ; N Itilde ; B 17 -18 538 979 ;
C -1 ; WX 280 ; N itilde ; B -93 -15 399 759 ;
C -1 ; WX 546 ; N Imacron ; B 17 -18 538 896 ;
C -1 ; WX 280 ; N imacron ; B -60 -15 357 676 ;
C -1 ; WX 546 ; N Ibreve ; B 17 -18 538 996 ;
C -1 ; WX 280 ; N ibreve ; B -72 -15 379 776 ;
C -1 ; WX 546 ; N Iogonek ; B 17 -205 538 730 ;
C -1 ; WX 280 ; N iogonek ; B 8 -168 304 732 ;
C -1 ; WX 546 ; N Idotaccent ; B 17 -18 538 925 ;
C -1 ; WX 280 ; N dotlessi ; B 67 -15 213 512 ;
C -1 ; WX 1175 ; N IJ ; B 17 -59 1148 740 ;
C -1 ; WX 559 ; N ij ; B 67 -292 492 732 ;
C -1 ; WX 665 ; N Jcircumflex ; B 26 -59 653 1045 ;
C -1 ; WX 403 ; N jcircumflex ; B -29 -292 443 826 ;
C -1 ; WX 610 ; N Kcedilla ; B 52 -228 595 748 ;
C -1 ; WX 540 ; N kcedilla ; B 45 -228 531 783 ;
C -1 ; WX 540 ; N kgreenlandic ; B 60 -21 545 568 ;
C -1 ; WX 550 ; N Lacute ; B 29 -46 552 1033 ;
C -1 ; WX 273 ; N lacute ; B 65 -21 345 1058 ;
C -1 ; WX 550 ; N Lcedilla ; B 29 -218 552 753 ;
C -1 ; WX 273 ; N lcedilla ; B 41 -203 325 784 ;
C -1 ; WX 550 ; N Lcaron ; B 29 -46 552 753 ;
C -1 ; WX 460 ; N lcaron ; B 65 -21 412 784 ;
C -1 ; WX 550 ; N Ldot ; B 29 -46 552 753 ;
C -1 ; WX 414 ; N ldot ; B 65 -21 396 784 ;
C -1 ; WX 628 ; N Lslash ; B 1 -46 637 753 ;
C -1 ; WX 354 ; N lslash ; B 6 -14 357 742 ;
C -1 ; WX 812 ; N Nacute ; B 40 -39 776 1033 ;
C -1 ; WX 523 ; N nacute ; B 33 -35 503 813 ;
C -1 ; WX 812 ; N Ncedilla ; B 40 -218 776 758 ;
C -1 ; WX 523 ; N ncedilla ; B 33 -267 503 534 ;
C -1 ; WX 812 ; N Ncaron ; B 40 -39 776 1015 ;
C -1 ; WX 523 ; N ncaron ; B 33 -35 503 795 ;
C -1 ; WX 640 ; N napostrophe ; B 37 -35 620 824 ;
C -1 ; WX 812 ; N Eng ; B 40 -262 776 758 ;
C -1 ; WX 523 ; N eng ; B 33 -217 503 534 ;
C -1 ; WX 798 ; N Omacron ; B 29 -30 768 896 ;
C -1 ; WX 525 ; N omacron ; B 20 -29 494 676 ;
C -1 ; WX 798 ; N Obreve ; B 29 -30 768 996 ;
C -1 ; WX 525 ; N obreve ; B 20 -29 526 776 ;
C -1 ; WX 798 ; N Odblacute ; B 29 -30 768 999 ;
C -1 ; WX 525 ; N odblacute ; B 20 -29 564 779 ;
C -1 ; WX 1193 ; N OE ; B 43 -38 1207 795 ;
C -1 ; WX 896 ; N oe ; B 20 -29 898 511 ;
C -1 ; WX 640 ; N Racute ; B 44 -18 620 1033 ;
C -1 ; WX 480 ; N racute ; B 40 -33 460 813 ;
C -1 ; WX 640 ; N Rcedilla ; B 44 -218 620 751 ;
C -1 ; WX 480 ; N rcedilla ; B 12 -218 460 515 ;
C -1 ; WX 640 ; N Rcaron ; B 44 -18 620 1015 ;
C -1 ; WX 480 ; N rcaron ; B 40 -33 460 795 ;
C -1 ; WX 693 ; N Sacute ; B 23 -30 673 1033 ;
C -1 ; WX 486 ; N sacute ; B 25 -30 451 813 ;
C -1 ; WX 693 ; N Scircumflex ; B 23 -30 673 1045 ;
C -1 ; WX 486 ; N scircumflex ; B 11 -30 453 826 ;
C -1 ; WX 693 ; N Scedilla ; B 23 -218 673 717 ;
C -1 ; WX 486 ; N scedilla ; B 25 -218 451 521 ;
C -1 ; WX 693 ; N Scaron ; B 50 -30 700 1015 ;
C -1 ; WX 507 ; N scaron ; B 0 -30 426 795 ;
C -1 ; WX 695 ; N Tcedilla ; B 4 -292 702 740 ;
C -1 ; WX 471 ; N tcedilla ; B 12 -292 462 683 ;
C -1 ; WX 695 ; N Tcaron ; B 4 -4 702 1015 ;
C -1 ; WX 639 ; N tcaron ; B 12 -32 622 683 ;
C -1 ; WX 695 ; N Tbar ; B 4 -4 702 740 ;
C -1 ; WX 471 ; N tbar ; B 12 -32 462 683 ;
C -1 ; WX 736 ; N Utilde ; B 38 -36 690 979 ;
C -1 ; WX 520 ; N utilde ; B 24 -40 516 759 ;
C -1 ; WX 736 ; N Umacron ; B 38 -36 690 896 ;
C -1 ; WX 520 ; N umacron ; B 25 -40 492 676 ;
C -1 ; WX 736 ; N Ubreve ; B 38 -36 690 996 ;
C -1 ; WX 520 ; N ubreve ; B 25 -40 496 776 ;
C -1 ; WX 736 ; N Uring ; B 38 -36 690 1079 ;
C -1 ; WX 520 ; N uring ; B 25 -40 487 859 ;
C -1 ; WX 736 ; N Udblacute ; B 38 -36 690 999 ;
C -1 ; WX 520 ; N udblacute ; B 25 -40 544 779 ;
C -1 ; WX 736 ; N Uogonek ; B 38 -208 690 740 ;
C -1 ; WX 520 ; N uogonek ; B 25 -208 592 521 ;
C -1 ; WX 1039 ; N Wcircumflex ; B 24 -47 1020 1045 ;
C -1 ; WX 684 ; N wcircumflex ; B 17 -40 677 826 ;
C -1 ; WX 635 ; N Ycircumflex ; B -6 -35 618 1045 ;
C -1 ; WX 552 ; N ycircumflex ; B 2 -283 544 826 ;
C -1 ; WX 635 ; N Ydieresis ; B 14 -35 638 923 ;
C -1 ; WX 693 ; N Zacute ; B 5 -25 687 1033 ;
C -1 ; WX 538 ; N zacute ; B 23 -38 512 813 ;
C -1 ; WX 693 ; N Zdot ; B 5 -25 687 925 ;
C -1 ; WX 538 ; N zdot ; B 23 -38 512 706 ;
C -1 ; WX 693 ; N Zcaron ; B 5 -25 687 1063 ;
C -1 ; WX 538 ; N zcaron ; B 23 -38 512 795 ;
C -1 ; WX 508 ; N longs ; B 133 -79 480 781 ;
C -1 ; WX 588 ; N florin ; B -35 -177 615 763 ;
C -1 ; WX 731 ; N Aringacute ; B 44 -15 680 1091 ;
C -1 ; WX 555 ; N aringacute ; B 27 -33 537 980 ;
C -1 ; WX 1086 ; N AEacute ; B 2 -43 1102 1058 ;
C -1 ; WX 911 ; N aeacute ; B 13 -24 896 813 ;
C -1 ; WX 798 ; N Oslashacute ; B 20 -44 816 1033 ;
C -1 ; WX 525 ; N oslashacute ; B 15 -34 522 813 ;
C -1 ; WX 556 ; N circumflex ; B 75 569 516 826 ;
C -1 ; WX 556 ; N caron ; B 119 574 486 795 ;
C -1 ; WX 556 ; N macron ; B 96 559 514 676 ;
C -1 ; WX 556 ; N breve ; B 84 575 535 776 ;
C -1 ; WX 556 ; N dotaccent ; B 134 558 293 706 ;
C -1 ; WX 556 ; N ring ; B 108 579 478 859 ;
C -1 ; WX 556 ; N ogonek ; B 125 -168 421 84 ;
C -1 ; WX 556 ; N tilde ; B 63 577 555 759 ;
C -1 ; WX 556 ; N hungarumlaut ; B 83 577 540 779 ;
C -1 ; WX 433 ; N uni037E ; B 104 -95 305 549 ;
C -1 ; WX 556 ; N tonos ; B 207 561 388 763 ;
C -1 ; WX 617 ; N dieresistonos ; B 44 561 579 735 ;
C -1 ; WX 731 ; N Alphatonos ; B 44 -15 680 763 ;
C -1 ; WX 433 ; N anoteleia ; B 134 283 295 417 ;
C -1 ; WX 847 ; N Epsilontonos ; B 49 -49 833 784 ;
C -1 ; WX 982 ; N Etatonos ; B 65 -41 950 763 ;
C -1 ; WX 775 ; N Iotatonos ; B 41 -18 767 763 ;
C -1 ; WX 876 ; N Omicrontonos ; B 31 -30 846 763 ;
C -1 ; WX 850 ; N Upsilontonos ; B 7 -35 833 763 ;
C -1 ; WX 959 ; N Omegatonos ; B 12 -69 948 796 ;
C -1 ; WX 280 ; N iotadieresistonos ; B -112 -15 423 735 ;
C -1 ; WX 731 ; N Alpha ; B 44 -15 680 722 ;
C -1 ; WX 630 ; N Beta ; B 65 -24 602 768 ;
C -1 ; WX 550 ; N Gamma ; B 29 -46 552 755 ;
C -1 ; WX 794 ; N Delta ; B 7 -36 781 813 ;
C -1 ; WX 624 ; N Epsilon ; B 45 -49 609 784 ;
C -1 ; WX 693 ; N Zeta ; B 5 -25 687 737 ;
C -1 ; WX 768 ; N Eta ; B 54 -41 735 759 ;
C -1 ; WX 798 ; N Theta ; B 29 -30 768 740 ;
C -1 ; WX 546 ; N Iota ; B 17 -18 538 730 ;
C -1 ; WX 610 ; N Kappa ; B 52 -54 595 748 ;
C -1 ; WX 689 ; N Lambda ; B -3 -51 701 772 ;
C -1 ; WX 882 ; N Mu ; B 35 -41 865 750 ;
C -1 ; WX 812 ; N Nu ; B 40 -39 776 758 ;
C -1 ; WX 720 ; N Xi ; B 13 -32 667 756 ;
C -1 ; WX 798 ; N Omicron ; B 29 -30 768 740 ;
C -1 ; WX 908 ; N Pi ; B 31 -30 877 766 ;
C -1 ; WX 532 ; N Rho ; B 29 -12 511 768 ;
C -1 ; WX 725 ; N Sigma ; B 11 -17 686 746 ;
C -1 ; WX 695 ; N Tau ; B 4 -4 702 740 ;
C -1 ; WX 635 ; N Upsilon ; B -6 -35 618 743 ;
C -1 ; WX 615 ; N Phi ; B 24 -18 581 730 ;
C -1 ; WX 723 ; N Chi ; B 5 -42 699 743 ;
C -1 ; WX 722 ; N Psi ; B 35 -13 690 756 ;
C -1 ; WX 959 ; N Omega ; B 12 -69 948 796 ;
C -1 ; WX 546 ; N Iotadieresis ; B 17 -18 538 931 ;
C -1 ; WX 635 ; N Upsilondieresis ; B -6 -35 618 923 ;
C -1 ; WX 555 ; N alphatonos ; B 27 -33 537 763 ;
C -1 ; WX 491 ; N epsilontonos ; B 56 -26 464 763 ;
C -1 ; WX 523 ; N etatonos ; B 41 -157 521 763 ;
C -1 ; WX 280 ; N iotatonos ; B 65 -15 246 763 ;
C -1 ; WX 520 ; N upsilondieresistonos ; B 16 -31 555 735 ;
C -1 ; WX 555 ; N alpha ; B 27 -33 537 513 ;
C -1 ; WX 586 ; N beta ; B 34 -197 555 723 ;
C -1 ; WX 523 ; N gamma ; B 10 -198 513 516 ;
C -1 ; WX 525 ; N delta ; B 20 -29 499 749 ;
C -1 ; WX 491 ; N epsilon ; B 56 -26 464 514 ;
C -1 ; WX 513 ; N zeta ; B 31 -203 516 757 ;
C -1 ; WX 523 ; N eta ; B 41 -157 521 534 ;
C -1 ; WX 610 ; N theta ; B 10 -20 596 760 ;
C -1 ; WX 280 ; N iota ; B 67 -15 213 512 ;
C -1 ; WX 540 ; N kappa ; B 60 -21 545 521 ;
C -1 ; WX 486 ; N lambda ; B 10 -12 494 667 ;
C -1 ; WX 520 ; N mu ; B 33 -205 495 521 ;
C -1 ; WX 486 ; N nu ; B 10 -20 494 516 ;
C -1 ; WX 513 ; N xi ; B 31 -203 465 778 ;
C -1 ; WX 525 ; N omicron ; B 20 -29 494 507 ;
C -1 ; WX 627 ; N pi ; B 9 -36 609 415 ;
C -1 ; WX 525 ; N rho ; B -5 -203 493 507 ;
C -1 ; WX 513 ; N sigma1 ; B 31 -203 484 520 ;
C -1 ; WX 559 ; N sigma ; B 20 -29 585 507 ;
C -1 ; WX 471 ; N tau ; B 12 -32 462 517 ;
C -1 ; WX 520 ; N upsilon ; B 16 -31 507 527 ;
C -1 ; WX 639 ; N phi ; B 24 -201 617 504 ;
C -1 ; WX 590 ; N chi ; B -4 -182 551 524 ;
C -1 ; WX 622 ; N psi ; B 24 -200 588 533 ;
C -1 ; WX 760 ; N omega ; B 18 -21 740 522 ;
C -1 ; WX 280 ; N iotadieresis ; B -54 -15 348 701 ;
C -1 ; WX 520 ; N upsilondieresis ; B 16 -31 507 701 ;
C -1 ; WX 525 ; N omicrontonos ; B 20 -29 494 763 ;
C -1 ; WX 520 ; N upsilontonos ; B 16 -31 507 763 ;
C -1 ; WX 760 ; N omegatonos ; B 18 -21 740 763 ;
C -1 ; WX 624 ; N afii10023 ; B 45 -49 609 979 ;
C -1 ; WX 850 ; N afii10051 ; B 0 -137 822 740 ;
C -1 ; WX 550 ; N afii10052 ; B 29 -46 552 1033 ;
C -1 ; WX 602 ; N afii10053 ; B 24 -12 607 744 ;
C -1 ; WX 693 ; N afii10054 ; B 23 -30 673 717 ;
C -1 ; WX 546 ; N afii10055 ; B 17 -18 538 730 ;
C -1 ; WX 546 ; N afii10056 ; B 17 -18 538 931 ;
C -1 ; WX 665 ; N afii10057 ; B 26 -59 653 740 ;
C -1 ; WX 1000 ; N afii10058 ; B -8 -10 986 745 ;
C -1 ; WX 977 ; N afii10059 ; B 3 -13 958 759 ;
C -1 ; WX 846 ; N afii10060 ; B 0 -20 793 740 ;
C -1 ; WX 633 ; N afii10061 ; B 59 -8 612 1033 ;
C -1 ; WX 644 ; N afii10062 ; B 9 -19 633 996 ;
C -1 ; WX 729 ; N afii10145 ; B 40 -136 696 750 ;
C -1 ; WX 731 ; N afii10017 ; B 44 -15 680 722 ;
C -1 ; WX 604 ; N afii10018 ; B 56 -7 572 744 ;
C -1 ; WX 630 ; N afii10019 ; B 65 -24 602 768 ;
C -1 ; WX 550 ; N afii10020 ; B 29 -46 552 755 ;
C -1 ; WX 768 ; N afii10021 ; B 24 -164 748 754 ;
C -1 ; WX 624 ; N afii10022 ; B 45 -49 609 784 ;
C -1 ; WX 859 ; N afii10024 ; B -5 -8 859 751 ;
C -1 ; WX 601 ; N afii10025 ; B 12 -21 567 737 ;
C -1 ; WX 747 ; N afii10026 ; B 44 -26 698 756 ;
C -1 ; WX 747 ; N afii10027 ; B 44 -26 698 996 ;
C -1 ; WX 633 ; N afii10028 ; B 59 -8 612 751 ;
C -1 ; WX 738 ; N afii10029 ; B 2 -11 682 745 ;
C -1 ; WX 882 ; N afii10030 ; B 35 -41 865 750 ;
C -1 ; WX 768 ; N afii10031 ; B 54 -41 735 759 ;
C -1 ; WX 798 ; N afii10032 ; B 29 -30 768 740 ;
C -1 ; WX 908 ; N afii10033 ; B 31 -30 877 766 ;
C -1 ; WX 532 ; N afii10034 ; B 29 -12 511 768 ;
C -1 ; WX 618 ; N afii10035 ; B 24 -12 607 744 ;
C -1 ; WX 695 ; N afii10036 ; B 4 -4 702 740 ;
C -1 ; WX 644 ; N afii10037 ; B 9 -19 633 743 ;
C -1 ; WX 615 ; N afii10038 ; B 24 -18 581 730 ;
C -1 ; WX 723 ; N afii10039 ; B 5 -42 699 743 ;
C -1 ; WX 773 ; N afii10040 ; B 40 -145 770 750 ;
C -1 ; WX 617 ; N afii10041 ; B 31 -15 568 754 ;
C -1 ; WX 906 ; N afii10042 ; B 28 -8 868 750 ;
C -1 ; WX 957 ; N afii10043 ; B 28 -145 953 750 ;
C -1 ; WX 759 ; N afii10044 ; B -1 -7 727 741 ;
C -1 ; WX 1075 ; N afii10045 ; B 56 -18 1049 740 ;
C -1 ; WX 604 ; N afii10046 ; B 56 -7 572 740 ;
C -1 ; WX 654 ; N afii10047 ; B 22 -7 623 733 ;
C -1 ; WX 1130 ; N afii10048 ; B 21 -30 1108 751 ;
C -1 ; WX 628 ; N afii10049 ; B -7 -19 593 751 ;
C -1 ; WX 555 ; N afii10065 ; B 27 -33 537 510 ;
C -1 ; WX 525 ; N afii10066 ; B 17 -29 524 772 ;
C -1 ; WX 503 ; N afii10067 ; B 44 -23 488 512 ;
C -1 ; WX 470 ; N afii10068 ; B 45 -23 478 518 ;
C -1 ; WX 592 ; N afii10069 ; B -4 -128 590 520 ;
C -1 ; WX 559 ; N afii10070 ; B 22 -23 523 511 ;
C -1 ; WX 682 ; N afii10072 ; B 12 -13 655 519 ;
C -1 ; WX 439 ; N afii10073 ; B 32 -23 408 523 ;
C -1 ; WX 636 ; N afii10074 ; B 34 -29 584 513 ;
C -1 ; WX 636 ; N afii10075 ; B 34 -29 584 776 ;
C -1 ; WX 492 ; N afii10076 ; B 43 -13 493 519 ;
C -1 ; WX 646 ; N afii10077 ; B -5 -20 598 520 ;
C -1 ; WX 734 ; N afii10078 ; B 1 -29 725 520 ;
C -1 ; WX 559 ; N afii10079 ; B 17 -26 521 541 ;
C -1 ; WX 591 ; N afii10080 ; B 14 -31 579 531 ;
C -1 ; WX 646 ; N afii10081 ; B 83 -22 600 511 ;
C -1 ; WX 534 ; N afii10082 ; B 38 -284 512 537 ;
C -1 ; WX 513 ; N afii10083 ; B 17 -31 479 520 ;
C -1 ; WX 471 ; N afii10084 ; B 12 -32 462 517 ;
C -1 ; WX 552 ; N afii10085 ; B 2 -283 544 508 ;
C -1 ; WX 810 ; N afii10086 ; B 21 -284 803 537 ;
C -1 ; WX 590 ; N afii10087 ; B 10 -22 582 540 ;
C -1 ; WX 618 ; N afii10088 ; B 66 -123 610 530 ;
C -1 ; WX 533 ; N afii10089 ; B 17 -26 500 533 ;
C -1 ; WX 771 ; N afii10090 ; B 32 -25 740 530 ;
C -1 ; WX 792 ; N afii10091 ; B 32 -163 804 530 ;
C -1 ; WX 621 ; N afii10092 ; B -12 -7 615 532 ;
C -1 ; WX 790 ; N afii10093 ; B 42 -22 784 516 ;
C -1 ; WX 491 ; N afii10094 ; B 42 -7 469 516 ;
C -1 ; WX 513 ; N afii10095 ; B 33 -21 489 515 ;
C -1 ; WX 902 ; N afii10096 ; B 21 -31 880 548 ;
C -1 ; WX 505 ; N afii10097 ; B -3 -19 466 512 ;
C -1 ; WX 559 ; N afii10071 ; B 22 -23 523 749 ;
C -1 ; WX 560 ; N afii10099 ; B -53 -200 519 783 ;
C -1 ; WX 470 ; N afii10100 ; B 45 -23 478 813 ;
C -1 ; WX 513 ; N afii10101 ; B 19 -25 475 511 ;
C -1 ; WX 486 ; N afii10102 ; B 25 -30 451 521 ;
C -1 ; WX 280 ; N afii10103 ; B 67 -15 238 732 ;
C -1 ; WX 280 ; N afii10104 ; B -54 -15 348 701 ;
C -1 ; WX 403 ; N afii10105 ; B -29 -292 340 731 ;
C -1 ; WX 889 ; N afii10106 ; B -5 -19 886 529 ;
C -1 ; WX 773 ; N afii10107 ; B 23 -11 753 517 ;
C -1 ; WX 577 ; N afii10108 ; B -44 -31 546 783 ;
C -1 ; WX 492 ; N afii10109 ; B 43 -13 493 813 ;
C -1 ; WX 552 ; N afii10110 ; B 2 -283 544 776 ;
C -1 ; WX 618 ; N afii10193 ; B 47 -168 555 530 ;
C -1 ; WX 616 ; N afii10050 ; B 69 -4 601 868 ;
C -1 ; WX 470 ; N afii10098 ; B 45 -23 480 656 ;
C -1 ; WX 1039 ; N Wgrave ; B 24 -47 1020 1031 ;
C -1 ; WX 684 ; N wgrave ; B 17 -40 677 812 ;
C -1 ; WX 1039 ; N Wacute ; B 24 -47 1020 1033 ;
C -1 ; WX 684 ; N wacute ; B 17 -40 677 813 ;
C -1 ; WX 1039 ; N Wdieresis ; B 24 -47 1020 921 ;
C -1 ; WX 684 ; N wdieresis ; B 17 -40 677 701 ;
C -1 ; WX 635 ; N Ygrave ; B -6 -35 618 1031 ;
C -1 ; WX 552 ; N ygrave ; B 2 -283 544 812 ;
C -1 ; WX 441 ; N endash ; B 1 265 424 371 ;
C -1 ; WX 882 ; N emdash ; B 13 266 854 365 ;
C -1 ; WX 882 ; N afii00208 ; B 13 266 854 365 ;
C -1 ; WX 626 ; N underscoredbl ; B -18 -291 646 -62 ;
C -1 ; WX 226 ; N quoteleft ; B 37 534 178 775 ;
C -1 ; WX 226 ; N quoteright ; B 37 534 178 775 ;
C -1 ; WX 433 ; N quotesinglbase ; B 105 -168 292 70 ;
C -1 ; WX 226 ; N quotereversed ; B 37 534 178 775 ;
C -1 ; WX 433 ; N quotedblleft ; B 15 531 400 789 ;
C -1 ; WX 433 ; N quotedblright ; B 24 531 410 789 ;
C -1 ; WX 429 ; N quotedblbase ; B 27 -168 409 70 ;
C -1 ; WX 610 ; N dagger ; B 12 -102 592 829 ;
C -1 ; WX 596 ; N daggerdbl ; B 8 -102 589 829 ;
C -1 ; WX 610 ; N bullet ; B 175 272 411 469 ;
C -1 ; WX 675 ; N ellipsis ; B 67 -33 626 79 ;
C -1 ; WX 1241 ; N perthousand ; B 52 -15 1177 802 ;
C -1 ; WX 226 ; N minute ; B 37 534 178 775 ;
C -1 ; WX 423 ; N second ; B 37 534 368 775 ;
C -1 ; WX 610 ; N guilsinglleft ; B 117 94 453 519 ;
C -1 ; WX 610 ; N guilsinglright ; B 143 90 514 550 ;
C -1 ; WX 449 ; N exclamdbl ; B 39 -33 397 784 ;
C -1 ; WX 610 ; N radicalex ; B -76 696 644 794 ;
C -1 ; WX 811 ; N fraction ; B -7 -27 802 831 ;
C -1 ; WX 526 ; N nsuperior ; B 75 391 445 781 ;
C -1 ; WX 989 ; N franc ; B 45 -52 969 769 ;
C -1 ; WX 793 ; N afii08941 ; B 35 -74 732 787 ;
C -1 ; WX 1268 ; N peseta ; B 29 -32 1230 768 ;
C -1 ; WX 618 ; N Euro ; B -57 -12 607 744 ;
C -1 ; WX 853 ; N afii61248 ; B 39 -15 827 783 ;
C -1 ; WX 539 ; N afii61289 ; B 53 -47 504 813 ;
C -1 ; WX 1206 ; N afii61352 ; B 50 -39 1178 758 ;
C -1 ; WX 882 ; N trademark ; B 18 413 872 786 ;
C -1 ; WX 959 ; N Ohm ; B 12 -69 948 796 ;
C -1 ; WX 549 ; N estimated ; B 29 -10 520 533 ;
C -1 ; WX 650 ; N oneeighth ; B 31 -191 583 842 ;
C -1 ; WX 650 ; N threeeighths ; B 31 -191 583 841 ;
C -1 ; WX 650 ; N fiveeighths ; B 31 -191 583 838 ;
C -1 ; WX 650 ; N seveneighths ; B 31 -191 583 836 ;
C -1 ; WX 508 ; N partialdiff ; B 42 -21 476 751 ;
C -1 ; WX 794 ; N Delta ; B 7 -36 781 813 ;
C -1 ; WX 908 ; N product ; B 31 -136 877 766 ;
C -1 ; WX 747 ; N summation ; B 2 -153 731 726 ;
C -1 ; WX 610 ; N minus ; B 50 247 551 382 ;
C -1 ; WX 607 ; N radical ; B 4 -34 607 796 ;
C -1 ; WX 836 ; N infinity ; B 52 149 784 470 ;
C -1 ; WX 699 ; N integral ; B 19 -198 661 840 ;
C -1 ; WX 610 ; N approxequal ; B 19 120 587 586 ;
C -1 ; WX 610 ; N notequal ; B 92 40 513 598 ;
C -1 ; WX 610 ; N lessequal ; B 88 -63 487 519 ;
C -1 ; WX 610 ; N greaterequal ; B 96 -63 487 550 ;
C -1 ; WX 734 ; N H22073 ; B 177 0 559 381 ;
C -1 ; WX 642 ; N H18543 ; B 188 60 449 321 ;
C -1 ; WX 642 ; N H18551 ; B 180 51 458 330 ;
C -1 ; WX 722 ; N lozenge ; B 9 -41 718 799 ;
C -1 ; WX 642 ; N H18533 ; B 183 59 449 324 ;
C -1 ; WX 642 ; N openbullet ; B 174 50 458 333 ;
C -1 ; WX 658 ; N fi ; B 17 -79 626 779 ;
C -1 ; WX 655 ; N fl ; B 14 -79 625 790 ;
C -1 ; WX 388 ; N uF003 ; B 131 582 281 829 ;
C -1 ; WX 388 ; N uF004 ; B 162 582 300 827 ;
C -1 ; WX 388 ; N uF005 ; B 162 582 300 827 ;
C -1 ; WX 0 ; N .null ; B 0 0 0 0 ;
C -1 ; WX 433 ; N nonmarkingreturn ; B 0 0 0 0 ;
C -1 ; WX 610 ; N sfthyphen ; B 106 205 501 328 ;
C -1 ; WX 249 ; N middot ; B 69 289 191 411 ;
EndCharMetrics
EndFontMetrics
%    \end{macrocode}
% \fi
%
%</rcomicbd.afm>
%
%
% \section{Implementation: Vietnamese typesetting support}
% \changes{v1.0a}{2002/12/02}{Included H\`an Th\'{\^e} Th\`anh's modifications
%   for Vietnamese typesetting}
%
% In October 2006, H\`an Th\'{\^e} Th\`anh requested a few changes to
% the \pkgname{comicsans} package to support Vietnamese typesetting.
% Unfortunately, these changes require converting the \comsan fonts
% from TTF to Type~1 format using
% \href{http://fontforge.sourceforge.net}{FontForge}\index{FontForge|usage},
% which doesn't run natively under Windows.  (Also, there is always some
% quality loss when converting font formats.)  Furthermore, Microsoft's
% license prohibits distributing the generated Type~1 files directly.
%
% This section presents Th\`anh's instructions (reformatted but
% otherwise verbatim from his e-mail) and supplemental files needed to
% use the \comsan fonts in a Vietnamese-language context.
%
% \bigskip\itshape
%
% Hi,
%
% I am working vietnamese support for the math font survey and encounter
% a problem with the \pkgname{comicsans} package.  The explanation is
% rather lengthy and dry, however the solution consists of 2 changes:
%
% \begin{enumerate}
%   \item replace the pfb's for each encoding by a single pfb, ie~replace
%   \fname{rcomic8r.pfb} and \fname{rcomiccyr.pfb} by
%   \fname{ComicSansMS.pfb}.  \fname{ComicSansMS.pfb} is just a pfb
%   converted by
%   \href{http://fontforge.sourceforge.net}{fontforge}\index{FontForge|usage}
%   from \fname{comic.ttf} by running
%
% \begin{verbatim}
%  fontforge fonttopfb.ff comic.ttf comicbd.ttf
% \end{verbatim}
%
%   \fname{fonttopfb.ff} is a script to convert ttf to pfb using
%   \href{http://fontforge.sourceforge.net}{fontforge}\index{FontForge|usage},
%   attached with this mail.
%
%   \bigskip
%
%<*fonttopfb.ff>
% \begingroup
% \language\hyphenlesslang
%    \begin{macrocode}
#! /usr/bin/env fontforge

i = 1;
while (i < $argc)
  Print("converting ", $argv[i], "...");
  Open($argv[i]);

  SetFontOrder(3); # convert from quadratic to cubic curves
  ScaleToEm(1000);  # to standard Postscript sizes, also scales underline value

  # clear TT hints and generate T1 hints
  SelectAll();
  ClearInstrs();
  ClearHints();
  AutoHint();

  Generate($fontname+".pfb", "", -1);
  i++;
endloop
%    \end{macrocode}
% \endgroup
%</fonttopfb.ff>
%
%   \item reencode the fonts explicitly by changing the map file
%   \fname{comicsans.map} so that the following lines:
% \begin{verbatim}
%  rcomico8r ComicSansMS "0.167 SlantFont" <rcomic8r.pfb
%  rcomicbdo8r ComicSansMS "0.167 SlantFont" <rcomicbd8r.pfb
%  rcomiccyro ComicSansMS "0.167 SlantFont" <rcomiccyr.pfb
%  rcomiccyrbdo ComicSansMS "0.167 SlantFont" <rcomiccyrbd.pfb
% \end{verbatim}
%   become
% \begin{verbatim}
%  rcomico8r ComicSansMS "0.167 SlantFont TeXBase1Encoding ReEncodeFont" <ComicSansMS.pfb <8r.enc
%  rcomicbdo8r ComicSansMS-Bold "0.167 SlantFont" <ComicSansMS-Bold.pfb <8r.enc
%  rcomiccyro ComicSansMS "0.167 SlantFont T2AAdobeEncoding ReEncodeFont" <ComicSansMS.pfb <t2a.enc
%  rcomiccyrbdo ComicSansMS-Bold "0.167 SlantFont T2AAdobeEncoding ReEncodeFont" <ComicSansMS-Bold.pfb <t2a.enc
% \end{verbatim}
%
%<*alt-comicsans.map>
% {\language\hyphenlesslang
%    \begin{macrocode}
rcomic8r ComicSansMS "TeXBase1Encoding ReEncodeFont" <8r.enc <comic.ttf
rcomicbd8r ComicSansMS-Bold "TeXBase1Encoding ReEncodeFont" <8r.enc <comicbd.ttf
rcomiccyr ComicSansMS "T2AAdobeEncoding ReEncodeFont" <t2a.enc <comic.ttf
rcomiccyrbd ComicSansMS-Bold "T2AAdobeEncoding ReEncodeFont" <t2a.enc <comicbd.ttf
rcomic7m ComicSansMS "TeXMathItalicEncoding ReEncodeFont" <texmital.enc <comic.ttf
rcomicbd7m ComicSansMS-Bold "TeXMathItalicEncoding ReEncodeFont" <texmital.enc <comicbd.ttf
rcomic7y ComicSansMS "TeXMathSymbolEncoding ReEncodeFont" <texmsym.enc <comic.ttf
rcomic9z ComicSansMS "ComicSansExtraEncoding ReEncodeFont" <csextras.enc <comic.ttf
%    \end{macrocode}
%    \begin{macrocode}
rcomico8r ComicSansMS "0.167 SlantFont TeXBase1Encoding ReEncodeFont" <ComicSansMS.pfb <8r.enc
rcomicbdo8r ComicSansMS-Bold "0.167 SlantFont" <ComicSansMS-Bold.pfb <8r.enc
rcomiccyro ComicSansMS "0.167 SlantFont T2AAdobeEncoding ReEncodeFont" <ComicSansMS.pfb <t2a.enc
rcomiccyrbdo ComicSansMS-Bold "0.167 SlantFont T2AAdobeEncoding ReEncodeFont" <ComicSansMS-Bold.pfb <t2a.enc
%    \end{macrocode}
% }
%</alt-comicsans.map>
% \end{enumerate}
%
% Do you think it is possible to adapt these changes to your package?
% It would simplify my life a lot~~{\normalfont\smiley}
%
% \begin{flushright}
%   Thanks for your consideration, \\
%   Th\`anh
% \end{flushright}
% \normalfont
%
% \Finale
\endinput

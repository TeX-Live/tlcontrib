\documentclass{article}
\usepackage[fleqn]{amsmath}
\usepackage[
    web={centertitlepage,designv,tight*,usetemplates,
        forcolorpaper,latextoc,pro},%usesf,
        eforms,aebxmp
]{aeb_pro}
%\usepackage{multicol}
%\usepackage{aeb_mlink}
\usepackage[tracking]{popupmenu}
\usepackage{array}
\usepackage[altbullet]{lucidbry}

%\previewOn

\usepackage{makeidx}
\makeindex
\usepackage{acroman}

\leftmargini\parindent


\usepackage[active]{srcltx}

\urlstyle{rm}
\let\tops\texorpdfstring
\let\pkg\textsf
\let\app\textsf
\let\opt\texttt
\let\uif\textsf
\let\env\texttt
\def\meta#1{\textit{\texttt{#1}}}
\def\ameta#1{$\langle\textit{\texttt{#1}}\rangle$}
\newdimen\aebdimen \aebdimen6pt %\partopsep  \advance\aebdimen\partopsep
\newcommand\bVerb[1][]{\begingroup#1\vskip\aebdimen\parindent0pt}%
\def\eVerb{\vskip\aebdimen\endgroup\noindent}
\def\SUB#1{\ensuremath{{}_{\text{#1}}}}
\def\darg#1{\texttt{\{#1\}}}
\def\takeMeasure{\bgroup\obeyspaces\takeMeasurei}
\def\takeMeasurei#1{\global\setbox\webtempboxi\hbox{\ttfamily#1}\egroup}
\def\bxSize{\wd\webtempboxi+2\fboxsep+2\fboxrule}
\let\amtIndent\leftmargini
\makeatletter
\def\setDisplayNumber#1#2{\kern0pt
    \setlength\abovedisplayshortskip{0pt}%
    \setlength\belowdisplayshortskip{0pt}%
    \setlength\abovedisplayskip{0pt}%
    \setlength\belowdisplayskip{0pt}%
    \begin{equation}\label{#2}\end{equation}\kern0pt
}
\renewenvironment{quote}[1][]
   {\def\@rgi{#1}\ifx\@rgi\@empty
    \let\rghtm\@empty\else\def\rghtm{\rightmargin\leftmargin}\fi
    \list{}{\rghtm} %{\rightmargin\leftmargin}%
    \item\relax}
   {\endlist}
\renewcommand*\descriptionlabel[1]{\hspace\labelsep
    \normalfont #1}
\setcounter{secnumdepth}{4}
\setcounter{tocdepth}{5}
\renewcommand*{\theparagraph}{\texorpdfstring{\protect\P\protect\ }{\textparagraph}}
\renewcommand{\paragraph}
    {\renewcommand{\@seccntformat}[1]{\theparagraph}%
    \@startsection{paragraph}{4}{0pt}{6pt}{-3pt}{\bfseries}} %\color{\aeb@subsubsectioncolor}
\renewcommand*\l@paragraph{\@dottedtocline{4}{5.0em}{1em}} %{7.0em}{4.1em}}
\def\chgCurrLblName#1{\def\@currentlabelname{#1}}
\def\echgCurrLblName#1{\edef\@currentlabelname{#1}}
\makeatother

\reversemarginpar
\def\FmtMP#1{\marginpar{\slshape\raggedleft\small#1}}


\newcommand{\YT}{\textsf{YouTube}}


%\def\tutpath{doc/tutorial}
%\def\tutpathi{tutorial}

\DeclareDocInfo
{
    university={\AcroTeX.Net},
    title={The manual for the \tops{\protect\pkg}{}{popupmenu} package},
%    title={The \texorpdfstring{\textsf{popupmenu} Package\\[1em]}{: }
%        Create popup-menus to perform actions},
    author={D. P. Story},
    email={dpstory@acrotex.net},
    subject={Creating popup-menues},
    talksite={\url{www.acrotex.net}},
    version={1.2, 2020/07/26},
    keywords={AcroTeX, PDF, popup-menus, JavaScript},
    copyrightStatus=True,
    copyrightNotice={Copyright (C) \the\year, D. P. Story},
    copyrightInfoURL={http://www.acrotex.net}
}

\def\dps{$\hbox{$\mathfrak D$\kern-.3em\hbox{$\mathfrak P$}%
   \kern-.6em \hbox{$\mathcal S$}}$}

\universityLayout{fontsize=Large}
\titleLayout{fontsize=LARGE}
\authorLayout{fontsize=Large}
\tocLayout{fontsize=Large,color=aeb}
\sectionLayout{indent=-40pt,fontsize=large,color=aeb} %,afterskip=1sp}
\subsectionLayout{indent=-20pt,color=aeb} %,afterskip=1sp}
\subsubsectionLayout{indent=0pt,color=aeb} %,afterskip=1sp}
\subsubDefaultDing{\texorpdfstring{$\bullet$}{\textrm\textbullet}}

\def\thispkg{\pkg{popupmenu}}

\let\tops\texorpdfstring
\newenvironment{eqComments}[1][\strut]{\smallskip\leftskip-\labelwidth
\item[]\textbf{\textcolor{blue}{#1}}}{\par\smallskip}

\begin{popupmenu}{Intro}
  \item{title=title,return=Title: The manual for the popupmenu package}
  \item{title=author,return=Author: D. P. Story}
  \item{title=package name,return=Package: popupmenu}
\end{popupmenu}

\urlPath{\homeAtUA}{http://www.math.uakron.edu/~dpstory}
\urlPath{\homeAeB}{http://www.acrotex.net}
\urlPath{\blogAeB}{http://blog.acrotex.net}
\urlPath{\urlCTAN}{https://www.ctan.org}
\urlPath{\embedYT}{http://www.youtube.com/embed}
\urlPath{\watchYT}{http://www.youtube.com/watch?v}
\begin{popupmenu}{Various}
    \item{title=Various AcroTeX Links,return=none}
    \item{title=-}
    \begin{submenu}{title=AeB at U of Akron}
        \item{title=Home page,return=\homeAtUA/acrotex.html}
        \item{title=Tutorials,return=\homeAtUA/acrotex.html\#educational}
    \end{submenu}
    \begin{submenu}{title=Commercial AcroTeX}
        \item{title=AcroTeX main page,return=\homeAeB}
        \item{title=AcroTeX blog,return=\blogAeB}
    \end{submenu}
    \begin{submenu}{title=AcroTeX on CTAN}
        \item{title=Contributions: AcroTeX ,return=\urlCTAN/author/story}
        \item{title=The popupmenu Package,return=\urlCTAN/pkg/popupmenu}
    \end{submenu}
    \begin{submenu}{title=YouTube Videos}
        \begin{submenu}{title=Action Videos}
            \item{title=Kung-Fu fighting (Bruce Lee version),return=\embedYT/GZ9e3Dy7obA}
            \item{title=Rocket Jump,return=\embedYT/7XzdZ4KcI8Y}
        \end{submenu}
        \begin{submenu}{title=Miscellaneous}
            \item{title=J\"{u}rgen's favorite song,return={\watchYT=mLDF5MBMWHE}}
            \item{title=\Esc"Sea Hunt\Esc" US TV series (1958-61) lead-in,return=\embedYT/Lz0aMoWh8Q4}
            \item{title=Learn \cs{LaTeX} in one video,return=\embedYT/VhmkLrOjLsw}
        \end{submenu}
    \end{submenu}
\end{popupmenu}

\puUseMenus{Intro,Various}

\chngDocObjectTo{\newDO}{doc}
\begin{docassembly}
var titleOfManual="The popupmenu MANUAL";
var manualfilename="Manual_BG_Print_popupmenu.pdf";
var manualtemplate="Manual_BG_Brown.pdf"; // Blue, Green, Brown
var _pathToBlank="C:/Users/Public/Documents/ManualBGs/"+manualtemplate;
var doc;
var buildIt=false;
if ( buildIt ) {
    console.println("Creating new " + manualfilename + " file.");
    doc = \appopenDoc({cPath: _pathToBlank, bHidden: true});
    var _path=this.path;
    var pos=_path.lastIndexOf("/");
    _path=_path.substring(0,pos)+"/"+manualfilename;
    \docSaveAs\newDO ({ cPath: _path });
    doc.closeDoc();
    doc = \appopenDoc({cPath: manualfilename, oDoc:this, bHidden: true});
    f=doc.getField("ManualTitle");
    f.value=titleOfManual;
    doc.flattenPages();
    \docSaveAs\newDO({ cPath: manualfilename });
    doc.closeDoc();
} else {
    console.println("Using the current "+manualfilename+" file.");
}
var _path=this.path;
var pos=_path.lastIndexOf("/");
_path=_path.substring(0,pos)+"/"+manualfilename;
\addWatermarkFromFile({
    bOnTop:false,
    bOnPrint:false,
    cDIPath:_path
});
\executeSave();

\end{docassembly}

\begin{document}

\maketitle

\selectColors{linkColor=black}
\tableofcontents
\selectColors{linkColor=webgreen}

\section{Introduction}\label{s:intro}

Acrobat JavaScript has an interesting function, \texttt{app.popUpMenuEx()},
that we exploit in this package. The function takes as its argument an array
of structured menu items and displays these items as a pop-up menu. When an
item is selected, a value is returned, which can then be acted upon in some
way. Here is a simple example, pass your mouse pointer over the button:
\pushButton[\CA{Info}\BC{}\S{S}\H{N}\AAmouseenter{%
  var cChoice = \popUpMenu(Intro);\r
  if ( cChoice != null ) app.alert(cChoice); }]{intro}{}{11bp}\olBdry. This
documentation describes the environments, commands, and JavaScript required
to create such pop-up menus.

\subsection{Sample files}

The file \texttt{pu-exmpls.tex} is the only demo file; it incorporates the examples of this
documentation, as well as few other bits and pieces.

\subsection{Options, requirements, and workflows}

\paragraph*{Options.} There are two options for this package
\opt{tracking}\FmtMP{\opt{tracking}} and \opt{!tracking} (the default). The
push button \pushButton[\CA{Info}\BC{}\S{S}\H{N}\AAmouseenter{var cChoice =
  \popUpMenu(Intro);\r if ( cChoice != null ) app.alert(cChoice);
  }]{intro}{}{11bp} introduced earlier is an example of a menu system with
  \emph{no tracking} (\opt{!tracking}). As you selected menu items appearance
  of the item does not change. When the \opt{tracking} option is taken, the
  menu keeps track of which menu item is selected by placing a check mark to
  the left of the menu item; for example,
  \pushButton[\CA{Info}\BC{}\S{S}\H{N}\AAmouseenter{trackingPU=true;\r var
  cChoice = \puProcessMenu(Intro);\r if ( cChoice != null )
  app.alert(cChoice); }]{intro}{}{11bp}\olBdry, notice as you select items, a
  check mark appears; select another item, the previous check mark is removed,
  and the latest item selected is now checked. More on tracking in \hyperref[s:tracking]{Section~\ref*{s:tracking}}.

\paragraph*{Requirements.}
This package requires the \pkg{eforms} package, which is part of \pkg{acrotex}.\footnote{\url{http://www.ctan/pkg/acrotex}}

\paragraph*{Workflows.} This is a general {\LaTeX} package, any workflows can be used to build
a \thispkg{} document: \app{pdflatex}, \app{lualatex}, \app{xelatex}, or
\app{dvips\,->\,\ameta{\upshape{distiller\,\string|\,ps2pdf}}}.\footnote{\pkg{popupmenu}
uses document JavaScript when \cs{puUseMenus} is expanded in the preamble;
\app{ps2pdf} does not support the creation of document JavaScript. If all
menu structures are defined as field scripts, \app{ps2pdf} should work. See
Section~\ref*{ps2pdf} for more information.}

\section{The \tops{\protect\pkg}{}{popupmenu} environment}

%\typeout{!! \Intro }

To generate a pop-up menu using \texttt{app.popUpMenuEx()} you need to pass to it through its argument
a menu-array. The most convenient way of creating this ``menu-array'' is with the \env{popupmenu} environment. Before
discussing the full syntax of \env{popupmenu} we reproduce the \env{popupmenu} environment that produced the \textsf{Intro}
button in first paragraph of this section:
\begin{Verbatim}[fontsize=\small,commandchars=!()]
\begin{popupmenu}{Intro}
  \item{title=title,return=Title: The manual for the popupmenu package}
  \item{title=author,return=Author: D. P. Story}
  \item{title=package name,return=Package: popupmenu}
\end{popupmenu}
\end{Verbatim}
This environment defines a JavaScript variable \texttt{Intro} and a {\LaTeX} command \cs{Intro}
that expands to what you see below.
\begin{Verbatim}[fontsize=\small,commandchars=!()]
var Intro = [
  {cName: "title", cReturn: "Title: The manual for the popupmenu package"},
  {cName: "author", cReturn: "Author: D. P. Story"},
  {cName: "package name", cReturn: "Package: popupmenu"}
];
\end{Verbatim}
The above is a properly formed ``menu-array''.

\newtopic\noindent
The rollover push button was created with the following code:
\begin{Verbatim}[fontsize=\small,commandchars={|~@}]
\pushButton[\CA{Info}\BC{}\S{S}\H{N}\AAmouseenter{%
  var cChoice = \popUpMenu(Intro);\r
  if ( cChoice != null ) app.alert(cChoice); }]{intro}{}{11bp}
\end{Verbatim}
The mouse-enter JavaScript is \texttt{var\:cChoice\:=\:\string\popUpMenu(Intro)}. The
array name (\texttt{Intro}) is passed to the convenience command
\cs{popUpMenU} to get the return value \texttt{cChoice}.\footnote{The name of the return variable is your choice,
you can say \texttt{var retn=\cs{popUpMenu(Intro)}}, for example.} Finally, if
nonempty, an alert dialog box is emitted with the return value displayed in
it. BTY,\footnote{By the way} \cs{popUpMenu} expands to the JavaScript method
\texttt{app.popUpMenuEx()}.

\section{The \tops{\env}{}{popupmenu} environment}

This general syntax for the \env{popupmenu} environment is,
\bVerb\takeMeasure{\small\string\item\darg{title=\ameta{str},marked=\ameta{\upshape{true|false}},enabled=\ameta{\upshape{true|false}},return=\ameta{str}}}%
\setlength\eflength{\bxSize}%
\def\1{\rlap{\hskip\eflength\ding{182}}}
\def\2{\rlap{\hskip\eflength\ding{183}}}
\def\3{\rlap{\hskip\eflength\ding{184}}}
\def\4{\rlap{\hskip\eflength\ding{185}}}
\begin{minipage}{\linewidth}
\begin{minipage}{\bxSize}
\xdef\panelWidth{\the\linewidth}% *[\ameta{opts}]\darg{\ameta{ytvID}}\darg{\ameta{text}}
\begin{Verbatim}[frame=single,commandchars=!(),rulecolor=\color{red},fontsize=\small]
!1\begin{popupmenu}{!ameta(name)}
!2\item{title=!ameta(str),marked=!ameta(!upshape(true|false)),enabled=!ameta(!upshape(true|false)),return=!ameta(str)}
!3\item{title=-}
...
!4\begin{submenu}{title=!ameta(str),marked=!ameta(!upshape(true|false))}
\item{title=!ameta(str),marked=!ameta(!upshape(true|false)),enabled=!ameta(!upshape(true|false)),return=!ameta(str)}
\item{title=-}
...
\end{submenu}
...
\end{popupmenu}
\end{Verbatim}
\end{minipage}\hfill
\begin{minipage}{\linewidth-\panelWidth}
\setDisplayNumber\label{display:puenv}
\end{minipage}\end{minipage}\eVerb
At the top-level \ding{182}, the \env{popupmenu} environment takes one
argument. The \ameta{name} argument plays two roles: (1) it becomes the name
of the JavaScript \emph{menu-array}; (2) it becomes a command name
\cs{\ameta{name}} that expands to the JavaScript menu-array. For this reason,
\ameta{name} must consist of ASCII letters only. The body of \env{popupmenu}
consists of one or more \cs{item} commands (\ding{183} and \ding{184}). The
body may contain zero or more \env{submenu} environments (\ding{185});
\env{submenu} may then contain one or more \cs{item} commands. You can have
sub-menus inside other sub-menus.

Generally, some underlying JavaScript, such as when the \opt{tracing} option
is in force, sets a menu item to \texttt{marked=true} or
\texttt{marked=false}; so as a rule, it is not recommended to initially specify
the \texttt{marked} key.

\paragraph*{Discussion of \cs{item\darg{\ameta{KV-pairs}}}.} There are four key-value pairs.
\begin{quote}
\begin{description}
  \item[\texttt{title=\ameta{str\upshape{\string|-}}}] (required) The
      value of \texttt{title} (\ameta{str}) is the menu item title; a
      value of `\texttt{-}' is reserved to signal that a separator line
      should be drawn.
  \item[\texttt{marked=\ameta{\upshape{true\string|false}}}] (optional)
      If \texttt{true}, the menu item is marked with a check. The default
      is \texttt{false} (not marked). Leave this key to JavaScript.
  \item[\texttt{enabled=\ameta{\upshape{true\string|false}}}] (optional)
      If \texttt{true} the item is to appear enabled; otherwise the menu
      item is grayed out. The default is \texttt{true} (enabled).
  \item[\texttt{return=\ameta{str}}] (optional) A string to be returned
      when the menu item is selected. If \texttt{return} is not specified
      or has no value, the value of the \texttt{title} key is returned.
      There is a special return value; if \ameta{str} is the word
      `\texttt{none}', the return value will be \texttt{null} (no
      action). In this case, the value of the \texttt{title} key can be
      used as a heading. (There is an example of the `\texttt{none}' value
      in Section~\ref*{Exepu}.)
\end{description}
\end{quote}

\paragraph*{Discussion of \env{submenu\darg{\ameta{KV-pairs}}}.} The argument of \env{submenu}
takes only two key-value pairs: \texttt{title} and \texttt{marked}, see descriptions above.

\paragraph*{Placement of the \env{popupmenu} environment.} The environment may appear in the
preamble or in the body of the document.
\begin{itemize}
    \item\textbf{In the preamble.} When one or more \env{popupmenu}
        environments are declared in the preamble, their corresponding
        menu-array can be placed as document JavaScript. To place one or
        more menu-arrays in the document JavaScript section of the PDF,
        insert the \cs{puUseMenus} command following the last
        \env{popupmenu} environment in the preamble.
\bVerb\takeMeasure{\string\puUseMenus\darg{\ameta{menu-array-names}}}\par\noindent
\begin{minipage}{\linewidth}
\begin{minipage}{\bxSize}
\xdef\panelWidth{\the\linewidth}%
\begin{Verbatim}[frame=single,commandchars=!(),rulecolor=\color{red}]
\puUseMenus{!ameta(menu-array-names)}
\end{Verbatim}
\end{minipage}\hfill
\begin{minipage}{\linewidth-\panelWidth}
\setDisplayNumber\label{display:puUseMenus}
\end{minipage}\end{minipage}\eVerb
where \ameta{menu-array-names} is a comma-delimited list of
menu-array names; for example, \cs{puUseMenus\darg{myMenu,yourMenu}}.

Tracing can only occur when (1) the \env{popupmenu} environment appears in
the preamble; (2) and the \ameta{name} of the pop-up menu appears as one of
the arguments of \cs{puUseMenus}; and (3) the command \cs{puProcessMenu} is
used at the field level to open the pop-up, more on this in \hyperref[s:tracking]{Section~\ref*{s:tracking}}.

\item\textbf{In the body of the document.} For \env{popupmenus}
    environments declared in the body of the document, the corresponding
    menu-arrays can still be used anywhere after the declaration. Refer to
    \hyperref[s:notracking]{Section~\ref*{s:notracking}} for details.
\end{itemize}

\section{Executing a pop-up menu}\label{Exepu}

Two commands used to open a pop-up menu, \cs{popUpMenu} and \cs{puProcessMenu}:
\bVerb\takeMeasure{\string\puProcessMenu\darg{\ameta{name}}}%
\setlength\eflength{\bxSize}\def\1{\sffamily\rlap{\hskip\eflength(for no-tracking menus)}}%
\def\2{\sffamily\rlap{\hskip\eflength(for tracking menus)}}\par\noindent
\begin{minipage}{\linewidth}
\begin{minipage}{\bxSize}
\xdef\panelWidth{\the\linewidth}%
\begin{Verbatim}[frame=single,commandchars={!~@},rulecolor=\color{red}]
!1\popUpMenu(!ameta~name@)
!2\puProcessMenu(!ameta~name@)
\end{Verbatim}
\end{minipage}\hfill
\begin{minipage}{\linewidth-\panelWidth}
\setDisplayNumber\label{display:puCMDS}
\end{minipage}\end{minipage}\eVerb where \ameta{name} is the name given to some
\env{popupmenu} environment, refer to
\hyperref[display:puenv]{display~(\ref*{display:puenv})}, defined earlier in
the document. The first one is designed for no-tracking menus, the second is
for tracking menus.

\paragraph*{The \texttt{Various} menu.}\label{para:Various}
In subsequent sections, we'll use the following pop-up menu, which is defined
in the preamble of this document.
\begin{Verbatim}[fontsize=\small,numbers=left]
\urlPath{\homeAtUA}{http://www.math.uakron.edu/~dpstory}
\urlPath{\homeAeB}{http://www.acrotex.net}
\urlPath{\blogAeB}{http://blog.acrotex.net}
\urlPath{\urlCTAN}{https://www.ctan.org}
\urlPath{\embedYT}{http://www.youtube.com/embed}
\urlPath{\watchYT}{http://www.youtube.com/watch?v}
\begin{popupmenu}{Various}
  \item{title=Various AcroTeX Links,return=none} % return value of 'none'
  \item{title=-}
  \begin{submenu}{title=AeB at U of Akron}
    \item{title=Home page,return=\homeAtUA/acrotex.html}
    \item{title=Tutorials,return=\homeAtUA/acrotex.html\#educational}
  \end{submenu}
  \begin{submenu}{title=Commercial AcroTeX}
    \item{title=AcroTeX main page,return=\homeAeB}
    \item{title=AcroTeX blog,return=\blogAeB}
  \end{submenu}
  \begin{submenu}{title=AcroTeX on CTAN}
    \item{title=Contributions: AcroTeX,
      return=\urlCTAN/author/story}
    \item{title=The popupmenu Package,
      return=\urlCTAN/pkg/popupmenu}
  \end{submenu}
  \begin{submenu}{title=YouTube Videos}
    \begin{submenu}{title=Action Videos}
      \item{title=Kung-Fu fighting (Bruce Lee version),
        return=\embedYT/GZ9e3Dy7obA}
      \item{title=Rocket Jump,return=\embedYT/7XzdZ4KcI8Y}
    \end{submenu}
    \begin{submenu}{title=Miscellaneous}
      \item{title=J\"{u}rgen's favorite song,
        return={\watchYT=mLDF5MBMWHE}}
      \item{title=\Esc"Sea Hunt\Esc" US TV series (1958-61) lead-in,
        return=\embedYT/Lz0aMoWh8Q4}
      \item{title=Learn \cs{LaTeX} in one video,
        return=\embedYT/VhmkLrOjLsw}
    \end{submenu}
  \end{submenu}
\end{popupmenu}
\end{Verbatim}
\textbf{Notes:}
\begin{itemize}
\item In lines~(1)--(6), several URLs are declared using
    \cs{urlPath}\FmtMP{\cs{urlPath}}, which is defined in \pkg{popupmenu}
    package.
\item Line~(8) The `\texttt{none}; return value is used.
\item Line~(12) The fragment (\texttt{\#}) is escaped (\verb~\#~).
\item Line~(31) The value of the \texttt{title} key is passed through the
    \pkg{hyperref} command \cs{pdfstringdef}, consequently, you can use
    standard {\LaTeX} markup for Latin-1 characters.
\item Line~(32) The return value has an equal sign (\texttt{=}), the return value
  is enclosed in braces to avoid a \pkg{xkeyval} parsing error.
\item Line~(33) The double quote needs to be escaped (because ultimately,
    the value will appear within double quotes. We use a special
    \cs{Esc}\FmtMP{\cs{Esc}} command of \pkg{popupmenu}.
\item Line~(35) To place a backslash(`\texttt\eqbs'), use the \cs{cs}\FmtMP{\cs{cs}} command.
\end{itemize}

\newtopic\noindent
The \env{popupmenu} can be placed in the preamble or in the body of the document. Let's begin with the
one declared in the body.

\subsection{Declaring \tops{\protect\env}{}{popupmenu} in the body}\label{s:notracking}

%\previewOn

\begin{popupmenu}{LocalMenu}
  \item {title=First Item}
  \item {title=Second Item}
\end{popupmenu}

Here, in the body, we declare a (simple) menu: \pushButton[\CA{My Menu}\AAmouseenter{%
\LocalMenu\r
var cChoice = \popUpMenu(LocalMenu);\r
if ( cChoice != null ) app.alert("You chose the \\""+cChoice+"\\"");
}]{LocalMenuBtn}{}{11bp}\begingroup
\def\1{\llap{\rlap{\ding{182}}\hskip\amtIndent\relax}}%
\def\2{\llap{\rlap{\ding{183}}\hskip\amtIndent\relax}}%
\def\3{\llap{\rlap{\ding{184}}\hskip\amtIndent\relax}}%
\begin{Verbatim}[xleftmargin=\amtIndent,fontsize=\small,commandchars={|~@}]
|1\begin{popupmenu}{LocalMenu}
  \item {title=First Item}
  \item {title=Second Item}
\end{popupmenu}
\pushButton[\CA{My Menu}\AAmouseenter{%
|2\LocalMenu\r   //|sffamily Expand the command version of the menu-array
|3var cChoice = \popUpMenu(LocalMenu);\r  //|sffamily use \popUpMenu
if ( cChoice != null )
  app.alert("You chose the \\""+cChoice+"\\"");
}]{LocalMenuBtn}{}{11bp}
\end{Verbatim}
\endgroup\noindent
In line~\ding{182} we decare our simple menu. In line~\ding{183} we expand
the command version of the \emph{menu-array}. (Refer to comment \ding{182} of
\hyperref[display:puenv]{display~(\ref*{display:puenv})} on
page~\pageref*{display:puenv}.) Finally, in line~\ding{184}, we execute
\cs{popUpMenu(LocalMenu)} (See
\hyperref[display:puCMDS]{display~(\ref*{display:puCMDS})} on
page~\pageref*{display:puCMDS}, and the comments that follow). Using this
technique, there is \emph{no tracking}; that is, the menu item chosen is not
checked.

The \env{popupmenu} can be declared in the preamble to obtain the same
results, but still no tracking. To obtain tracking of the menu items, you
must (1) declare \env{popupmenu} in the preamble; (2) include its name
(\texttt{LocalMenu}) in the list of \ameta{menu-array-names} of the
\cs{puUseMenus} command of
\hyperref[display:puUseMenus]{display~(\ref*{display:puUseMenus})}; (3)
delete line~\ding{183}; (4) replace \cs{popUpMenu} in line~\ding{184} with
\cs{puProcessMenu}; and (5) the \opt{tracking} option must be specified.
Details of setting up tracking are found in the
\hyperref[{s:tracking}]{Section~\ref*{s:tracking}}.

\subsection{Declaring \tops{\protect\env}{}{popupmenu} in the preamble}\label{s:tracking}

A \env{popupmenu} environment can be declared anywhere \emph{before} its first use in
field Java\-Script to actually display the menu to the user; however, to obtain
tracking of the menu items chosen you must (1) specify
\cs{usepackage[tracking]\darg{popupmenu}}, the \opt{tracking} option; (2)
declare the menu (that is, the \env{popupmenu} environment) in the preamble; (3) list the menu name
amongst the arguments of the \cs{puUseMenus} command; (4) use
\cs{puProcessMenu} in lieu of \cs{popUpMenu} in the field JavaScript.

\newtopic\noindent A bare-bones push button is as follows:
\begin{Verbatim}[xleftmargin=\amtIndent,fontsize=\small,commandchars={|~@}]
\pushButton[|ameta~KV-pairs@\AAmouseenter{%
  var cChoice = \puProcessMenu(|ameta~name@);\r
  if ( cChoice != null ) |ameta~some-action@
}}]{|ameta~btnName@}{|ameta~wd@}{|ameta~ht@}
\end{Verbatim}


\newtopic\noindent
For example, make a selection: \pushButton[\CA{My Menu}\BC{}\W0\S{S}\H{N}\AAmouseenter{%
var cChoice = \puProcessMenu(Various);\r
if ( cChoice != null ) {\r\t
  if (PUdebug)\r\t\t
    app.alert("URL: \\""+cChoice+"\\"");\r\t
  else app.launchURL(cChoice);\r
}}]{mymenu}{}{11bp}\olBdry\,
\pushButton[\TU{Toggle Debug}\BC{}\W0\S{S}\H{N}\CA{Action}\AAmouseup{if(PUdebug){\r\t
  PUdebug=false;event.target.buttonSetCaption("Action");}\r
  else {PUdebug=true;event.target.buttonSetCaption("Debug");}
}]{DebugBtn}{.5in}{11bp}

\newtopic\noindent The verbatim listing of the first push button follows:
\begin{Verbatim}[xleftmargin=\amtIndent,fontsize=\small,commandchars={|~@}]
\pushButton[\CA{My Menu}\BC{}\W0\S{S}\H{N}\AAmouseenter{%
var cChoice = \puProcessMenu(Various);\r
if ( cChoice != null ) {\r\t
  if (PUdebug)\r\t\t
    app.alert("URL: \\""+cChoice+"\\"");\r\t
  else app.launchURL(cChoice);\r
}}]{mymenu}{}{11bp}
\end{Verbatim}
This push button references the \texttt{Various} menu, tediously listed on
page~\pageref*{para:Various}. The second button, labeled \textsf{Action},
allows you to play around with the pop-up menu without going to the web
sites. Click on it, and the caption now says \textsf{Debug}. Now, instead of
going to the web site, an alert box appears announcing your choice. Good for
testing things.

\paragraph*{Multiple action types.} In all the examples of this document, as well as the demo files,
all actions are the same, either the return is a URL and the action is
\texttt{app.launchURL(URL)} or the return is text and results are reported in
an alert box. You can have multiple action types, as is illustrated in the
following local declaration.

%\previewOff

\newtopic\noindent Pick your choice:
\begin{popupmenu}{puMulti}
  \item {title=Item 1,return=0}
  \item {title=Item 2,return=1}
  \begin{submenu}{title=Other items}
    \item{title=Item 3,return=2}
  \end{submenu}
\end{popupmenu}
\begin{defineJS}[\makeesc|]{\puMultiActn}
|puMulti
var cChoice = |popUpMenu(puMulti);
if ( cChoice != null ) {
  switch (cChoice) {
    case "0":
      app.alert("You chose Item 1 from the menu");
      break;
    case "1":
      app.launchURL("|homeAeB");
      break;
    default:
      console.show()
      console.println("Menu returned a value of \\""+cChoice+"\\"");
      break;
  }
}
\end{defineJS}
\pushButton[\CA{Multi}\AAmouseenter{\puMultiActn}]{MultiBtn}{}{11bp}\olBdry.
The verbatim list is,
\begin{Verbatim}[xleftmargin=\amtIndent,fontsize=\small,commandchars={*~@}]
%*textsf~ Use the *texttt~defineJS@ environment to define the action@
\begin{defineJS}[\makeesc|]{\puMultiActn}
|puMulti
var cChoice = |popUpMenu(puMulti);
if ( cChoice != null ) {
  switch (cChoice) {
    case "0":
      app.alert("You chose Item 1 from the menu");break;
    case "1":
      app.launchURL("|homeAeB");break;
    default: console.show();
      console.println("Menu returned a value of \\""+cChoice+"\\"");
      break;
  }
}
\end{defineJS}
*goodbreak
%*textsf~ Declare *texttt~popupmenu@ environment, return values are integers@
\begin{popupmenu}{puMulti}
  \item {title=Item 1,return=0}
  \item {title=Item 2,return=1}
  \begin{submenu}{title=Other items}
    \item{title=Item 3,return=2}
  \end{submenu}
\end{popupmenu}
%*textsf~ Now execute the pop-up menu as a mouse enter event@
\pushButton[\CA{Multi}\AAmouseenter{\puMultiActn}]{MultiBtn}{}{11bp}
\end{Verbatim}

\section{Remarks on \tops{\protect\app}{}{ps2pdf}}\label{ps2pdf}

To use the \app{dvips\,->\,ps2pdf} workflow, document JavaScript needs to be avoided.
Do not use the \opt{tracking} option, do not use \cs{puUseMenus}, and use only the
\cs{popUpMenu} command in field JavaScript. A simple outline of an document is found below,
it is a working example.
\begin{Verbatim}[fontsize=\small,commandchars={*~@}]
\documentclass{article}
\usepackage{popupmenu}
\parindent0pt\parskip6pt
\begin{document}
%*textsf~ Declare in the body of the text, can use anywhere after this declaration@
\begin{popupmenu}{LocalMenu}
  \item {title=First Item}
  \item {title=Second Item}
\end{popupmenu}

Pop-up menu using \verb|\pushButton| of \textsf{eforms}:
\pushButton[\CA{My Menu}\AAmouseenter{\LocalMenu\r
var cChoice = \popUpMenu(LocalMenu);\r
if ( cChoice != null )\r\t
  app.alert("You chose the \\""+cChoice+"\\" menu item");
}]{LocalMenuBtn}{}{11bp}.

Pop-up menu using \verb|\PushButton| of \textsf{hyperref}:
\PushButton[name=hyperbutton1,onenter={\LocalMenu
  var cChoice = \popUpMenu(LocalMenu);
  if ( cChoice != null )
    app.alert("You chose the \\""+cChoice+"\\" menu item");}
]{My Menu}
\end{document}
\end{Verbatim}


\paragraph*{Back to my retirement.} \dps


\end{document}

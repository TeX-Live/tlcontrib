\documentclass{article}
\usepackage[fleqn]{amsmath}
\usepackage[pdf,cfg=quiz,forpaper,pointsonleft,
% compile with exactly one of the following three
    nosolutions
%   answerkey
%   vspacewithsolns
]{eqexam}

\examNum{1}\numVersions{2}\forVersion{a}
\longTitleText
    {Quiz~\nExam--003}
    {Quiz~\nExam--007}
\endlongTitleText
\shortTitleText
    {Q{\nExam}s3}
    {Q{\nExam}s7}
\endshortTitleText

\title[\sExam]{\bfseries\Exam}
\author{D. P. Story}
\subject[C1]{Calculus I}
\date{Spring \the\year}
\keywords{Test~\nExam, Section \ifAB{003}{007}}
\email{dpstory@uakron.edu}

\vspacewithkeyOn
\solAtEndFormatting{\eqequesitemsep{3pt}}
\everymath{\displaystyle}

\begin{document}

\maketitle

\begin{exam}{Part1}

\begin{instructions}[Instructions:]
Solve each of the following problems without error. \textit{Show all
details.} Box in your $\boxed{\text{answers}}$. Use good notation, you
\emph{will} be marked off for bad notation. \textbf{Note:} The value of a
limit can be a number, the symbol $+\infty$, the symbol $-\infty$, or may
be labeled DNE (for ``does not exist'').
\end{instructions}

\begin{problem}[4]
Compute $ \vA{\lim_{x\to-1}\frac{4x^2+x}{x}}\vB{\lim_{x\to2}\frac{1-3x}{x+1}}$
\begin{solution}[2in]
As discussed in class, this is a ``Skill Level 0'' limit problem:
\[
\begin{verA}
\lim_{x\to-1}\frac{4x^2+x}{x}
    = \frac{4(-1)^2+(-1)}{-1}
    = \boxed{-3}
\end{verA}
\begin{verB}
    \lim_{x\to2}\frac{1-3x}{x+1}
    \lim_{x\to2}\frac{1-3(2)}{2+1}
    = \boxed{-\frac{5}{3}}
\end{verB}
\]
\ifkeyalt\adjDisplayBelow\fi
\end{solution}
\end{problem}

\begin{problem}[3]
Define the function $ f(x) = \begin{cases} 2x^3 - 1 & x < -2\\ 2- x^2 & x
\ge -2\end{cases}$. Compute $\lim_{x\to\vA{-2^-}\vB{-2^+}} f(x) $, show the
details of your reasoning.

\begin{solution}[2in]
We use standard techniques:
\begin{verA}
\begin{alignat*}{2}
    \lim_{x\to-2^-} f(x) &
        = \lim_{x\to-2^-} (2x^3-1) &&\qquad\text{since $ x < -2$}\\&
        = 2(-2)^3 - 1&&\qquad\text{now a skill level 0 problem}\\&
        = \boxed{-17}
\end{alignat*}
\end{verA}
\begin{verB}
\begin{alignat*}{2}
    \lim_{x\to-2^+} f(x) &
        = \lim_{x\to-2^+} (2- x^2) &&\qquad\text{since $ x < -2$}\\&
        = 2 - (-2)^2&&\qquad\text{now a skill level 0 problem}\\&
        = \boxed{-2}
\end{alignat*}
\end{verB}
\ifkeyalt\adjDisplayBelow\fi
\end{solution}
\end{problem}

\begin{problem}[3]
Compute $\vA{\lim_{x\to2} \frac{1-x}{(x-2)^2}}
    \vB{\lim_{x\to3} \frac{x-2}{(3-x)^2}}$

\begin{solution}[1in]
\begin{verA}
Notice the denominator goes to zero, but the numerator does not;
this indicates a vertical asymptote usually. Because the
denominator is squared, it's always positive. When $x$ is
``close'' to $2$, $1 - x < 0$, that is, when $x$ is ``close'' to
$2$ the numerator is \emph{negative}. The ratio of the numerator and
denominator is \emph{negative} when $x$ is ``close'' to $2$. Thus, we
conclude,
\[
        \boxed{\lim_{x\to2} \frac{1-x}{(x-2)^2} = -\infty}
\]
\end{verA}
\begin{verB}
Notice the denominator goes to zero, but the numerator does not;
this indicates a vertical asymptote usually. Because the
denominator is squared, it's always positive. When $x$ is
``close'' to $3$, $x - 2 > 0$, that is, when $x$ is ``close'' to
$3$ the numerator is \emph{positive}. The ratio of the numerator and
denominator is \emph{positive} when $x$ is ``close'' to $3$. Thus, we
conclude,
\[
        \boxed{\lim_{x\to3} \frac{x-2}{(3-x)^2} = +\infty}
\]
\end{verB}
\ifkeyalt\adjDisplayBelow\fi
\end{solution}
\end{problem}

\end{exam}
\end{document}

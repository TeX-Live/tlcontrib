\documentclass{article}
\usepackage[forcolorpaper]{web}
\usepackage[scandoc]{qrcstamps}

\title{Create QR Code Symbols using Stamps}
\author{D. P. Story}
\email{dpstory@acrotex.net}
\subject{Creating QR Code symbols using Acrobat annotation stamps and LaTeX}
\keywords{stamp annotations, QR Code, LaTeX, PDF, AcroTeX}

\useFullWidthForPaper
\previewOff

\def\cs#1{\texttt{\char`\\#1}}

\parindent0pt\parskip6pt

\begin{document}

\makeinlinetitle

\textbf{Small}

\defineAPath{\UADPStory}{http://www.math.uakron.edu/~dpstory}
\qrCode{http://www.acrotex.net} \url{http://www.acrotex.net}\qquad
\qrCode{mailto:dpstory@acrotex.net} \url{mailto:dpstory@acrotex.net}

\qrCode{Home of AcroTeX.Net Blog\n
http://www.acrotex.net/blog} \parbox{2in}{Home of Acro\negthinspace\TeX.Net Blog\\
\url{http://www.acrotex.net/blog}}\hfill\parbox{3.25in}{You can format the
line of a text message, as is done in the QR symbol to the left, by using
\cs{n} (new line) within the required argument of \cs{qrCode}.}


\textbf{Medium}

%https://theqrplace.wordpress.com/2011/05/02/qr-code-tech-info-mecard-format/
\qrCode[size=medium]{MECARD:N:Story,Don;NICKNAME:Dr. D. P. Story;
ADR:1475 AcroTeX Drive, Niceville, FL 32578;TEL:9999999999;EMAIL:dpstory@acrotex.net;
MEMO:Founder of AcroTeX.Net;
URL:http://www.acrotex.net;;} (MECARD)

\textbf{Large}

\def\TEXT{%
Now is the time for all good men to come to the aid of their country.
Now is the time for all good men to come to the aid of their country.
Now is the time for all good men to come to the aid of their country.
Now is the time for all good men to come to the aid of their country.
Now is the time for all good men to come to the aid of their country.
Now is the time for all good men to come to the aid of their country.
%Now is the time for all good men to come to the aid of their country.
Now is the time for all good men to come to the aid of their country.}

\qrCode[size=large,widthTo=2in,allowresize]{\TEXT} (Now is the time\dots)\hfill
\parbox{3in}{This is an example of a QR symbol containing text. This paragraph
is near the limit of 500 characters. If you recompile after un-commenting the
one line that is commented out, you'll be over the 500 character limit. See
what happens when you compile. This example as well as others were used in
the development of the package to check whether there truly is a 500
character limit.}

\end{document}

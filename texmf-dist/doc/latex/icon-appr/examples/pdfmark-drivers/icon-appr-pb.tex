\documentclass{article}
\usepackage[web={designv,tight},eforms
]{aeb_pro}
\usepackage{icon-appr}\previewOff
\usepackage{fancyvrb}

\title{Creating icon appearances for push buttons\texorpdfstring{\\[1ex]}{}
  using \textsf{aeb\_pro}}
\author{D. P. Story}
\university{Acro\TeX.Net}
\email{dpstory@acrotex.net}
\subject{Techniques for creating button appearances}
\keywords{icon appearances of push button,JavaScript}
\version{1.0}
\norevisionLabel

\let\pkg\textsf
\let\app\textsf
\def\hardspace{{\fontfamily{cmtt}\selectfont\symbol{32}}}
\newcommand{\cs}[1]{\texttt{\char`\\#1}}

\begin{embedding}
\embedIcon[placement={Avatar1,[2]Avatar2}]{../graphics/man1.pdf}
\embedIcon[placement={[1]Avatar1,[1]Avatar2}]{../graphics/scot.gif}
\embedIcon[placement={[2]Avatar1,[0]Avatar2}]{../graphics/girl.png}
\end{embedding}

\begin{document}

\maketitle

\section{Introduction}

Here we use JavaScript methods for creating icon appearances for push buttons
(but not for check box or radio button fields). The technique requires
\app{aeb\_pro}. \app{Adobe Distiller} is needed to create the PDF (after
apply \app{dvips}), then open the newly created PDF in the full \app{Acrobat}
application and save the document.

You can use purely EPS files as well that do not use JavaScript methods and
does not require \pkg{aeb\_pro}; however, \pkg{graphicxsp} is required
instead. Refer to \texttt{icons-appr-eps.tex}.\medskip

\noindent\textbf{The premise.} We have two fields we want to supply icon
appearances to; these are \texttt{Avatar1} and \texttt{Avatar2}. In the
preamble of this document, we have,
\begin{Verbatim}[xleftmargin=\parindent,fontsize=\small]
\begin{embedding}
\embedIcon[placement={Avatar1,[2]Avatar2}]{../graphics/man1.pdf}
\embedIcon[placement={[1]Avatar1,[1]Avatar2}]{../graphics/scot.gif}
\embedIcon[placement={[2]Avatar1,[0]Avatar2}]{../graphics/girl.png}
\begin{embedding}
\end{Verbatim}
The required argument is the path to the image file. For this situation, for
the optional argument, there are two keys of interest, \texttt{placement} and
\texttt{page}, the latter applies to a PDF image file. The value of
\texttt{page} is a base~0 page number (the first page is page~0).

The \texttt{placement} key ``places'' the image on the button faces of the
field names listed (\texttt{Avatar1} and \texttt{Avatar2}). There is an
optional argument that precedes the field name that determines the face of
the button the icon is to appear on; the values are \texttt{[0]} (default,
normal icon); \texttt{[1]} (down icon); and \texttt{[2]} (rollover icon). The
optional argument precedes the field name, and is shown in the example above.
There must be no space between the optional argument and the field name; if
you type `\texttt{[2]\hardspace Avatar1}', for example, the field name is interpreted
as `\texttt{\hardspace Avatar1}', which is incorrect. The example is shown below.\medskip

\def\myPresets{\BC{}\BG{}\S{S}\autoCenter{n}\importIcons{y}
  \FB{true}\TP{1}}

\hspace*{20pt}\pushButton[\presets{\myPresets}\A{\JS{app.alert("Avatar 1")}}]{Avatar1}{.5in}{.5in}\qquad
\pushButton[\presets{\myPresets}\A{\JS{app.alert("Avatar 2")}}]{Avatar2}{.5in}{.5in}\medskip

\noindent The verbatim listing of the above push buttons with custom appearance follows:
\begin{Verbatim}[xleftmargin=\parindent,fontsize=\small]
\def\myPresets{\BC{}\BG{}\S{S}\autoCenter{n}
    \FB{true}\importIcons{y}\TP{1}}
\pushButton[\presets{\myPresets}
    \A{\JS{app.alert("Avatar 1")}}]{Avatar1}{.5in}{.5in}\qquad
\pushButton[\presets{\myPresets}
    \A{\JS{app.alert("Avatar 2")}}]{Avatar2}{.5in}{.5in}
\end{Verbatim}
Note that I've inserted a \cs{presets} key with value of \cs{myPresets}, to
conveniently enter standard optional parameters. The meaning of \cs{FB} and
\cs{TP} are explained in the \pkg{icon-appr} package documentation. An
important key-value is \cs{importIcons}, we set its value as
\verb!\importIcons{y}!; this declares that the push button field is the
target of some JavaScript that will supply the icon appearances.

\section{Check box and radio buttons fields}

We use JavaScript methods to populate the appearance faces of push buttons;
however, there are no JavaScript methods for populating appearance faces for
check boxes and radio button fields. Other methods need to be used; these are
methods demonstrated in \texttt{icons-appr-eps.tex}.

\end{document}

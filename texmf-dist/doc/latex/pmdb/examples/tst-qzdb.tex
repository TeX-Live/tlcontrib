\documentclass{article}
\usepackage{amstext}
\usepackage[forcolorpaper]{web}
\usepackage{exerquiz}
\usepackage[dbmode,!tight]{pmdb}

\reversemarginpar

\pmCBPresets{\textColor{red}}

%\previewOn\pmpvOn

\title{Poor man's DB demo}
\author{D. P. Story}
\email{dpstory@acrotex.net}
\version{1.0}
\norevisionLabel


\PTsHook{($\eqPTs^{\text{pts}}$)}

\def\cs#1{\texttt{\eqbs#1}}
\rfooter{\dirTOCItem}
\lfooter{\displayChoices{}{11bp}\cgBdry[1em]\clrChoices{}{11bp}}
\useBeginQuizButton[\CA{Begin}]
\useEndQuizButton[\CA{End}]

\useMCCircles

% Declares input for quiz items
\InputQuizItems


\begin{document}

\maketitle
\tableofcontents

\section{Section 1}

Expand the \cs{InputQuizItems} here.

\begin{quiz*}{myquiz1}
Solve each of these problems, passing is 100\%.
\begin{questions}

\pmInput{probs/prob1.tex}

\pmInput{probs/prob2.tex}

\pmInput{probs/prob3.tex}

\end{questions}
\end{quiz*}

\section{Section 2}

%\InputQuizItems


\begin{quiz*}{myquiz2}
Solve each of these problems, passing is 100\%.
\begin{questions}

\pmInput{probs/prob4.tex}

\pmInput{probs/prob6.tex}

\pmInput{probs/prob5.tex}

\end{questions}
\end{quiz*} %\ScoreField\currQuiz\CorrButton\currQuiz

\section{Section 3}

Here, we input an \textsf{exercise} environment, we do so in `paragraph mode' by expanding
\cs{InputParas} first.

\InputParas

\pmInput{exrs/ex1.tex}

\end{document} 
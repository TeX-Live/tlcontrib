\documentclass{article}
\usepackage{amstext}
\usepackage{web}
\usepackage[usesumrytbls,allowrandomize]{exerquiz}
\usepackage[usebatch]{thorshammer}

\DeclareQuiz{q1}

% It is important to freeze the seed so that (1) you can reproduce the exact
% same quiz at a later time; (2) allow content written to the AUX file to
% come up to date. This is important when using summary tables.
\useRandomSeed{380105620}

\setInitMag{fitwidth}
\hypersetup{pdfpagemode=UseNone} % don't need to see bookmarks
\hypersetup{pdfpagelayout=OneColumn} %,pdfencoding=auto
\reversemarginpar

\showCreditMarkup

%\previewOn\pmpvOn
\useBeginQuizButton[\CA{Begin}]
\useEndQuizButton[\CA{End}]

\PTsHook{($\eqPTs^{\text{pts}}$)}

\useMCCircles

% reset the paths for \instrPath and \classPath for your system
\instrPath{/C/Users/dpstory/Desktop/Test Folder/target/_Thor}
\classPath{/C/Users/dpstory/Desktop/Test Folder/target/myClass}
\classMember{Peter}{Pan}{A/_Thor}
\classMember{J\oct374rgen}{Gilg}{B/_Thor}
\classMember{Thors}{Hammer}{C/_Thor}

\autoCopyOn
\DeclareCoverPage{0}

\begin{makeClassFiles}
\sadQuizzes
\end{makeClassFiles}

\begin{document}

\thispagestyle{empty}

\noindent In this file, we attempt to duplicate and extend the
work of \texttt{theexuc.tex} by developing a scheme to randomize
the questions. We begin with the simple case, of creating two
versions of the same quiz. We modify the preamble to include the
option \texttt{allowrandomize} of \textsf{exerquiz}. This file
randomizes the multiple choice and multiple selection question
that are marked up using \verb~\bChoices/\eChoices~. The `quiz
body' concept is used to introduced multiple equivalent quizzes.

\newpage

\declareQuizBody{qzbody}

\begin{qzbody}

\thQuizHeader

\noindent\textbf{Instructions:} (For the student) Press
`\textsf{Begin}' to begin the quiz; after completing the quiz,
press `\textsf{End}'. Use the `\textsf{Save}' button to save the
document.

\begin{quiz*}{\currQuiz}
Solve each of these problems, passing is 100\%.
\begin{questions}
  \item\PTs{3} Which of these are true ?
\begin{answers}{4}
\Ans1 True & \Ans0 False
\end{answers}

  \item \PTs{4} Select which of the following is true.
\begin{answers}{4}
\Ans1 True & \Ans0 False & \Ans0 Maybe & \Ans0 Sometimes
\end{answers}

  \item\PTs{2} $9+8=\RespBoxMath{17}{1}{.0001}{[0,1]}$

\essayQ{5} % num points assigned
\item\PTs{5} Write a short history of Acro\negthinspace\TeX.\par
\RespBoxEssay{\linewidth}{1in}

\item\PTs{3} Which of the following are numbers?
\begin{manswers}{6}
\bChoices[random=true]
  \Ans[-1]{0}d\eAns
  \Ans[1]{1}17\eAns
  \Ans[-1]{0}p\eAns
  \Ans[1]{1}88\eAns
  \Ans[-1]{0}s\eAns
  \Ans[1]{1}105\eAns
\eChoices
\end{manswers}

\multipartquestion

    \item\PTs{20} Answer each of the following multiple selection problems. Each correct answer
    is worth $3$ points, and each incorrect answer is worth $-2$ points.
    \begin{questions}

\rowsep{3pt}

        \item\PTs{9} Select which people who served as a President
                     of the United States. (Select all correct choices.)

        \begin{manswers}*{2}%
            \bChoices[random=true]
                \Ans[-2]{0} Henry Clay\eAns
                \Ans[-2]{0} Ben Franklin\eAns
                \Ans[3]{1}  Andrew Jackson\eAns
                \Ans[3]{1}  Ronald Reagan\eAns
                \Ans[-2]{0} George Meade\eAns
                \Ans[3]{1}  Grover Cleveland\eAns
                \Ans[-2]{0} John Jay\eAns
                \Ans[-2]{0} Paul Revere\eAns
            \eChoices
        \end{manswers}

\rowsep{3pt}

        \item\PTs{9} Select which people who served as a Chancellor of the
            German Republic. (Select all correct choices.)
        \begin{manswers}*{2}%
            \bChoices[nCols=2,random=true]
                \Ans[-2]{0} Gustav Heinemann\eAns
                \Ans[-2]{0} Theodor Heu{\ss}\eAns
                \Ans[3]{1}  Konrad Adenauer\eAns
                \Ans[-2]{0} Richard von Weizs\"{a}cker\eAns
                \Ans[3]{1}  Willy Brandt\eAns
                \Ans[-2]{0} Heinrich L\"{u}bke\eAns
                \Ans[-2]{0} Roman Herzog\eAns
                \Ans[3]{1} Ludwig Erhard\eAns
            \eChoices
        \end{manswers}

\rowsep{3pt}

      \item\PTs{2} If you select all choices in part~(a), you will
          receive -1 points as a penalty for bad guessing. \textbf{Question:}
          Determine the \emph{number of correct choices} in part~(a)?
          \begin{answers}{4}
          \bChoices[random=true]
            \Ans0 1\eAns
            \Ans0 2\eAns
            \Ans1 3\eAns
            \Ans0 4\eAns
            \Ans0 5\eAns
            \Ans0 6\eAns
            \Ans0 7\eAns
            \Ans0 8\eAns
          \eChoices
          \end{answers}
    \end{questions}

%\essayQ{3} % num points assigned
%\item\PTs{3} Comment on the experience of taking a quiz the `Thorsten way.'\par
%\RespBoxEssay{\linewidth}{1in}

\end{questions}
\writeProListAux
\end{quiz*}\quad\thQuizTrailer
\end{qzbody}

\typeout{!! inputting qzbody first time}
\InputQuizBody{qzbody}

\typeout{!! inputting qzbody second time}
\InputQuizBody{qzbody}

\end{document}

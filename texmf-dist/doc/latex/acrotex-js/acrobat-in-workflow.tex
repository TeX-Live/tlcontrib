\RequirePackage[!use=preview,!use=efpmca,!use=usebw,!use=finalbuild]{spdef}
\documentclass{article}
\usepackage[fleqn]{amsmath}
\usepackage[web={designv,forcolorpaper},useacrobat]{aeb_pro}
\usepackage{eforms}
\usepackage{graphicx}
\usepackage{fancyvrb}

\ifpdf\def\x{%
  \usepackage[T1]{fontenc}
  \usepackage{lucidabr}}
\else\def\x{%
  \usepackage[altbullet]{lucidbry}
  \usepackage[active]{srcltx}
}\fi\x

\usepackage{pifont}

%\usepackage{fancyvrb}

\reversemarginpar

\let\uif\textsf
\let\env\texttt
\def\cs#1{\texttt{\eqbs#1}}
\let\tops\texorpdfstring

\makeatletter
\renewcommand{\paragraph}
    {\@startsection{paragraph}{4}{0pt}{6pt}{-3pt}
    {\normalfont\normalsize\bfseries}}
\renewcommand{\subparagraph}
    {\@startsection{subparagraph}{5}{\parindent}{6pt}{-3pt}%
    {\normalfont\normalsize\bfseries}}
\def\parboxValign{t}
\newcommand{\FmtMP}[2][0pt]{\mbox{}\marginpar{%
    \raisebox{.5\baselineskip+#1}{%
    \expandafter\parbox\expandafter[\parboxValign]%
        {\marginparwidth}{\aebbkFmtMp#2}}}}
\def\aebbkFmtMp{\kern0pt\itshape\small
    \ifusebw\color{gray}\else\color{blue}\fi
    \raggedleft\hspace{0pt}}
\newcommand{\BlogArticle}{%
    \makebox[0pt][l]{\hspace{-1pt}\color{blue}\Pisymbol{webd}{254}%
    }\raisebox{.5pt}{\ifusebw\color{black}\else
    \color{red}\fi\ding{045}}}
\def\dps{$\mbox{$\mathfrak D$\kern-.3em\mbox{$\mathfrak P$}%
   \kern-.6em \hbox{$\mathcal S$}}$}

\makeatother

\setlength{\mathindent}{\the\parindent}

\def\AEBP{\app{AeB Pro}}
\def\AEB{\app{AeB}}
\let\app\textsf
\let\pkg\textsf
\def\amtIndent{\parindent}
\def\meta#1{$\langle\textit{\texttt{#1}}\rangle$}
\def\SC#1{{\small#1}}
\def\PDF{\SC{PDF}}
\let\opt\texttt

%\reversemarginpar
\def\jsSupInstr{\tops{{\AEB} and {\AEBP}\\[1ex]}{AeB and AeB Pro: }Including
  \tops{\app}{}{Acrobat} in the \tops{\LaTeX}{LaTeX} workflow}
\title{\jsSupInstr}
\author{D. P. Story}
\version{1.0}

\chngDocObjectTo{\newDO}{doc}
\begin{docassembly}
var titleOfManual="Acrobat workflow";
var manualfilename="Manual_BG_Print_acrobat.pdf";
var manualtemplate="Manual_BG_Brown.pdf"; // Blue, Green, Brown
var _pathToBlank="C:/Users/Public/Documents/ManualBGs/"+manualtemplate;
var doc;
var buildIt=false;
if ( buildIt ) {
    console.println("Creating new " + manualfilename + " file.");
    doc = \appopenDoc({cPath: _pathToBlank, bHidden: true});
    var _path=this.path;
    var pos=_path.lastIndexOf("/");
    _path=_path.substring(0,pos)+"/"+manualfilename;
    \docSaveAs\newDO ({ cPath: _path });
    doc.closeDoc();
    doc = \appopenDoc({cPath: manualfilename, oDoc:this, bHidden: true});
    f=doc.getField("ManualTitle");
    f.value=titleOfManual;
    doc.flattenPages();
    \docSaveAs\newDO({ cPath: manualfilename });
    doc.closeDoc();
} else {
    console.println("Using the current "+manualfilename+" file.");
}
var _path=this.path;
var pos=_path.lastIndexOf("/");
_path=_path.substring(0,pos)+"/"+manualfilename;
\addWatermarkFromFile({
    bOnTop:false,
    bOnPrint:false,
    cDIPath:_path
});
\executeSave();
\end{docassembly}

\begin{document}

\maketitle

\section{Introduction}

\app{Adobe Acrobat} (\app{AA}) is a desktop application that has many
powerful features beyond the functionality of a PDF viewer. In this article
the focus is a {\LaTeX} workflow that includes \app{AA} in its workflow.

{\AEB} and {\AEBP}, as well as other packages written under the {\AEB}
banner, use FDF files in various situations (as described in
Section~4).\footnote{An FDF file is a file type that was created by
\app{Adobe}. The \app{Acrobat} and \app{Adobe Reader} applications can use
FDF files in various ways.} These packages use \app{Acrobat} to import the
FDF files to perform designated JavaScript actions as required by the package
or document author. The importing of FDF files is only needed during the PDF
creation stage of document development; once the PDF file is created, the
document can be viewed by \app{Adobe Reader} or any other PDF viewer.

For FDF files to be imported, the \app{Acrobat} application may need to be
configured.

\section{Configuration of \tops{\app}{}{Acrobat}}\label{s:config}

The current version of \app{AA} is \app{Adobe Acrobat DC}. In this and
previous versions of \app{AA} no special configuration is needed; however if
you purchased \app{AA} after December 2020, or you updated your \app{AA}
after December 2020, you need to make one change in the \uif{Preferences}.

\begin{figure}[htb]\fboxsep0pt\relax\centering
  \fbox{\includegraphics[width=.5\linewidth]{figures/SecurityPrefs}}
  \caption{\tops{\protect\uif}{}{Security (Enhanced) Preferences}}\label{fig:SP}
\end{figure}

Start \app{Acrobat} and open the menu \uif{Edit \ifpdf\texttt{>} \else> \fi
Preferences} (\uif{Ctrl+K}). In the \uif{Preferences} dialog box, select
\uif{Security (Enhanced)} category from the left panel; the top most line of
the right panel is a checkbox labeled \uif{Enable Protected Mode at startup}.
\emph{Clear this checkbox.}{\def\parboxValign{c}\FmtMP{\app{Acrobat DC}
users, action required!}}\footnote{When the box is checked, importing FDF
files, as described in the next two sections, is \emph{prohibited}. We want
FDF files imported, so we clear the checkbox.} Refer to
\hyperref[fig:SP]{Figure~\ref*{fig:SP}} for a visual. If this checkbox is not
present in your \app{Acrobat}, there is nothing to do until this checkbox
appears in a future updated version of your \uif{AA} application.

\section{When FDF files are imported}

It is important to be aware of when and how FDF files are imported, and to
understand the actions required of the document author in response to this
``import FDF event.''

For a document that uses FDF files to perform tasks, the files are imported
\emph{after} PDF creation when the newly created PDF document is \emph{first
opened} in \app{Acrobat}, and \emph{before the PDF file is moved to another
location and before any auxiliary files (the FDF files) are deleted}. Once
the FDF files are imported, it is important \emph{to save}\FmtMP{Save the
PDF} the PDF file, this incorporates the results of the actions taken by the
imported FDF files into the document, actions, such as the embedding of
document JavaScript.

Another important point to realize is that the use of FDF files is not
restricted to the
\app{dvips\ifpdf\texttt{->}\,\else->\,\fi\app{Distiller}\,\ifpdf\texttt{->}\,\ignorespaces
\else->\,\fi\app{Acrobat}} method of creating a PDF; indeed, any PDF creator
used by the {\LaTeX} community may be used, for example,
\app{pdftex\ifpdf\texttt{->}\allowbreak\,\else->\,\fi\app{Acrobat}}.  The
only prerequisite is to use {\LaTeX} markup that initiates the ``import FDF
event.'' This markup is available through packages that support it; this
includes, but not restricted to, {\AEB} and {\AEBP}.

\section{Applications of FDF files to a \tops{\LaTeX}{LaTeX} workflow}

With regard to producing a PDF file through a {\LaTeX} workflow, there are
two important applications to importing FDF files: importing document
JavaScript and post-PDF creation actions. These two applications to FDF files
are covered in the next two subsections.

\subsection{Document JavaScript}

In this case, FDF files are required only for the \app{dvips\,\ignorespaces
\ifpdf\texttt{->}\,\else->\,\fi\app{Distiller}\,\ifpdf\texttt{->}\,\else->\,\ignorespaces
\fi\app{Acrobat}} workflow; the other PDF creators (\app{pdflatex},
\app{lualatex}, and \app{xelatex}) have primitives that facilitate the
creation of document JavaScript without FDF files or \app{Acrobat}.

For the \pkg{exerquiz} package, part of the {Acro\!\TeX} eDucation Bundle
(\AEB), creation of Document JavaScript is automatic and invisible for all
PDF creators; it is only in the
\texttt{tex\,->\,dvi\,->\,\app{dvips}\,->\,\ignorespaces
\app{Distiller}\,->\,\app{Acrobat}} workflow that FDF files are used.

Document JavaScript, by the way, is code that is embedded in the PDF and is
available to be called throughout the document. Usually, Document JavaScript
is general code that is repeatedly used.


\subsection{Post PDF Creation actions}

The {\AEBP} Bundle has many ``post PDF creation'' actions, such as auto
saving the document, importing sounds, attaching documents, inserting pages,
extracting pages, and many more. With proper {\LaTeX} markup, when the
document is compiled, FDF files are written to the hard drive, when the
document is then first opened in \app{Acrobat}, the FDF files are imported
and the actions are executed. Again, for the FDF files to be imported, they
must not have already been deleted, and \app{Acrobat} must be correctly
configured (\hyperref[s:config]{Section~\ref*{s:config}}). One very simple
example, is automatic saving of the document when \app{Acrobat} is first
opened. For example, an \pkg{aeb\_pro} document has a \env{docassembly}
environment that is placed in the preamble. Within this environment, place
any post PDF creation JavaScript actions:
\begin{Verbatim}[xleftmargin=\parindent]
\begin{docassembly}
console.show();
console.println("Saving document");
\executeSave();
\end{docassembly}
\end{Verbatim}
The \verb~\executeSave~ command is defined by \pkg{aeb\_pro} and its action
is to save the document. It is the \env{docassembly} environment that writes
the FDF file that contains its contents. \app{Acrobat} inputs this command on
opening and the PDF is saved.\footnote{\cs{executeSave} must be the last post PDF
creation action; otherwise, as later actions may not be executed.}

\paragraph*{The \pkg{docassembly} package.} This package uses the post PDF creation actions
described above, but is a stand alone package and is available to all {\LaTeX} workflows who own
the \app{Acrobat} application. The \pkg{docassembly} package defines the
\env{docassembly} environment and supports the same functionality of as {\AEBP}. The \pkg{docassembly}
package may be downloaded from CTAN at \url{https://ctan.org/pkg/docassembly}.

\end{document}

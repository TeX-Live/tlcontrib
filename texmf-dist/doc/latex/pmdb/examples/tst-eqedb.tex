\documentclass{article}
\usepackage[%nosolutions,
    forcolorpaper,
    pointsonright,totalsonright,
    links,
    vspacewithsolns
    ]{eqexam} %online,
\usepackage[dbmode,tight]{pmdb}


\email{dpstory@uakron.edu}

\university
{%
      NORTHWEST FLORIDA STATE COLLEGE\\
         Department of Mathematics
}
\email{storyd@nwfsc.edu}
\subject[MAC1105]{College Algebra}
\examNum{1}\numVersions{1}\forVersion{a}
\longTitleText
    {Test~\nExam}
\endlongTitleText
\shortTitleText
    {T\nExam}
\endshortTitleText
\altTitle{\vA{11:00 am, C-204}\vB{12:30 pm, L-138}}
\title[\sExam]{\Exam}
\author{Dr.\ D. P. Story}
\date{\thisterm, \the\year}
\duedate{10/09/14}
\keywords{MAC 1105, Exam \nExam, {\thisterm} semester, \theduedate, at NWFSC}

%\previewOn\pmpvOn
\editSourceOn
%\useEditBtn   % the default
\useEditLnk  % try it with links
\InputProbs
\reversemarginpar

\let\env\texttt
\def\cs#1{\texttt{\eqbs#1}}

\cfooteqe{\displayChoices{}{11bp}\cgBdry[1em]\clrChoices{}{11bp}}

%\NoTotals
\TotalsOnRight
%\advance\marginparwidth12pt


\begin{document}

\maketitle

\section{My first exam}

Welcome to this section. Now that you've learned quite a lot, we can have
a little quiz.


\begin{exam}{dps1}
\begin{instructions}
Solve each without error.
\end{instructions}

\pmInput{eqexam/prob1.tex}

\pmInput{eqexam/prob2.tex}

\begin{eqComments}
There are two ways of inputting content for a \env{problem*}: (1) in the source file, input items
of the \env{parts} environment; (2) and input the whole \env{problem*} environment. The next problem
is input using method (1); a \env{problem*} shell is placed in this source file and the individual items
within the \env{parts} environment are input with \cs{pmInput}.
\end{eqComments}

\begin{problem*}[5ea]
Solve each.
\begin{parts}
    \pmInput{eqexam/item1.tex}
    \pmInput{eqexam/item2.tex}
\end{parts}
\end{problem*}

\begin{eqComments}
The whole of the next \env{problem*} environment is input with \cs{pmInput}. Interestingly, the items
of the \env{parts} environment are still itemized by \cs{pmInput}.
\end{eqComments}

\pmInput{eqexam/prob3.tex}

\end{exam}

\end{document}

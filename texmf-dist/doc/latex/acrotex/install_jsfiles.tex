\documentclass{article}
\usepackage[fleqn]{amsmath}
%
% the driver line is not necessary if you
% have aebpro.cfg configured to your driver.
%
\usepackage[web]{aeb_pro}
\usepackage{eforms}
\usepackage[altbullet]{lucidbry}
\usepackage{pifont}

\usepackage{fancyvrb}

\makeatletter
\renewcommand{\paragraph}
    {\@startsection{paragraph}{4}{0pt}{6pt}{-3pt}
    {\normalfont\normalsize\bfseries}}
\renewcommand{\subparagraph}
    {\@startsection{subparagraph}{5}{\parindent}{6pt}{-3pt}%
    {\normalfont\normalsize\bfseries}}
\makeatother

\mathindent\parindent

\def\AEBP{\app{AeB Pro}}
\def\AEB{\app{AeB}}
\def\app#1{\textsf{#1}}
\def\amtIndent{\parindent}
\def\meta#1{$\langle\textit{\texttt{#1}}\rangle$}
\def\SC#1{{\small#1}}
\def\PDF{\SC{PDF}}
\let\opt\texttt

\def\copyGetPath{\qquad\pushButton[\TU{Copy app.getPath to console}
\CA{getPath}\A{\JS{console.show();\r
console.println('\\napp.getPath("user");');
}}]{cpyGtPth}{}{11bp}}

\def\copyGetPathJS{\qquad\pushButton[\TU{Copy app.getPath to console}
\CA{getPath}\A{\JS{console.show();\r
console.println('\\napp.getPath("user", "javascript");');
}}]{cpyGtPthJS}{}{11bp}}

%\reversemarginpar
\def\jsSupInstr{How to install JavaScript support files}
\title{\jsSupInstr}
\author{D. P. Story}
\version{1.0}

\chngDocObjectTo{\newDO}{doc}
\begin{docassembly}
var titleOfManual="How to install JavaScript files";
var manualfilename="Manual_BG_Print_jsfiles.pdf";
var manualtemplate="Manual_BG_Brown.pdf"; // Blue, Green, Brown
var _pathToBlank="C:/Users/Public/Documents/ManualBGs/"+manualtemplate;
var doc;
var buildIt=true;
if ( buildIt ) {
    console.println("Creating new " + manualfilename + " file.");
    doc = \appopenDoc({cPath: _pathToBlank, bHidden: true});
    var _path=this.path;
    var pos=_path.lastIndexOf("/");
    _path=_path.substring(0,pos)+"/"+manualfilename;
    \docSaveAs\newDO ({ cPath: _path });
    doc.closeDoc();
    doc = \appopenDoc({cPath: manualfilename, oDoc:this, bHidden: true});
    f=doc.getField("ManualTitle");
    f.value=titleOfManual;
    doc.flattenPages();
    \docSaveAs\newDO({ cPath: manualfilename });
    doc.closeDoc();
} else {
    console.println("Using the current "+manualfilename+" file.");
}
var _path=this.path;
var pos=_path.lastIndexOf("/");
_path=_path.substring(0,pos)+"/"+manualfilename;
\addWatermarkFromFile({
    bOnTop:false,
    bOnPrint:false,
    cDIPath:_path
});
\executeSave();
\end{docassembly}

\begin{document}

\maketitle

\section{Introduction}

Because some of the \app{Acrobat} JavaScript methods have security
restrictions, they can be only executed in the console window, or as part
of a batch sequence, or from the JavaScript folder reserved by
\app{Acrobat} for that use. The developer (of {\LaTeX} packages or {\PDF}
form documents) may want to use these restricted JavaScript methods as
part of document assembly. To use the restricted methods, they must be
called from the user JavaScript folder. The scripts are contained in a
\SC{JS} file that \app{Acrobat} reads when it first starts up.

In this article, we provide detailed steps for locating the user
JavaScript folder. For {\AEB} and {\AEBP}, the files \texttt{aeb.js} and
\texttt{aeb\_pro.js} are placed in the folder. Other JavaScript files may
be placed in the folder as well, for example, the \texttt{config.js} file
is one that is commonly used for general purposes.

Before we get started, if you are using a version of \app{Acrobat} prior
to version~8.1, you need not read any further. The security restrictions
described below were not in effect for those versions.

\section{Installing the JavaScript support files}

The functionality of {\AEB} and {\AEBP} depends, to some extent, on the
proper installation of their JavaScript support files.
\begin{itemize}
    \item \AEB: For authors using the $\app{dvips}+\app{Acrobat Distiller}$
        workflow, \texttt{aeb.js} is required; this file is not used by
        \app{pdflatex} or \app{xelatex} users.

    \item \AEBP: For authors using the $\app{dvips}+\app{Acrobat
        Distiller}$ workflow, \texttt{aeb\_pro.js} is required; for authors
        that prefer \app{pdflatex}, \app{lualatex}, or \app{xelatex} as the
        {\PDF} creator \emph{and who own} the \app{Acrobat} application,
        the \texttt{aeb\_pro.js} is required with the \opt{useacrobat}
        option of \AEBP. When using the \opt{nopro} option of \AEBP, the
        JavaScript file \texttt{aeb\_pro.js} is not used.

\end{itemize}
The placement of the JavaScript files has changed over the years and is a
function of the version number \meta{ver}. Start \app{Acrobat} and open the
console debugger window by pressing \texttt{Ctrl+J} (for \app{Windows OS})
or \texttt{Cmd+J} (for \app{Mac OS}). Now execute the code
\texttt{app.getPath("user")} in the console window. To do this, press the
\textsf{getPath} button provided below, which copies the necessary code to
the console window.
\begin{equation}
\begin{minipage}{.8\linewidth}
\begin{Verbatim}[commandchars={!~@}]
app.getPath("user");!copyGetPath
\end{Verbatim}
\end{minipage}\label{display:getPath}
\end{equation}
Now, with the cursor on the line containing the script, press the
\texttt{Ctrl+Enter} (for \app{Windows OS}) or \texttt{Cmd+Enter} (for
\app{Mac~OS}) key. This will execute the JavaScript and return the `root'
path to the user's application data for \app{Acrobat}.

The rest of the instructions depend on your operating system,
\app{Windows} or \app{Mac}.

\subsection{For users of \app{Windows OS}}

There are several cases, jump to the numbered paragraph that describes your
version (\meta{ver}) of \app{Acrobat}.

%/C/Users/D. P. Story/AppData/Roaming/Adobe/Acrobat/11.0
%/C/Users/D. P. Story/AppData/Roaming/Adobe/Acrobat/Privileged/11.0/JavaScripts

\def\Stepi{\ding{182}\enspace}
\def\Stepii{\ding{183}\enspace}
\def\Stepiii{\ding{184}\enspace}

\paragraph*{{\Stepi}For $\boldmath\text{\meta{ver}}$, $\boldmath 8.1 \le
\text{\meta{ver}} < 10.1.1$.} For the \app{Windows OS}, executing the script
in display~\eqref{display:getPath} returns a string of the
form:\footnote{This string is for \app{Windows OS} beyond \app{WinXP}.}
\begin{equation}
\begin{minipage}{.8\linewidth}
\begin{Verbatim}[xleftmargin=\amtIndent,fontsize=\small,commandchars=!()]
/C/Users/!meta(username)/AppData/Roaming/Adobe/Acrobat/!meta(ver)
\end{Verbatim}
\end{minipage}\label{display:path}
\end{equation}
Navigate to this folder location. The correct location for the \SC{JS}
files is in the \texttt{JavaScripts} subfolder. If the folder does not
exist, create it. The final path for the JavaScript support files is then
\begin{Verbatim}[xleftmargin=\amtIndent,fontsize=\small,commandchars=!()]
/C/Users/!meta(username)/AppData/Roaming/Adobe/
    Acrobat/!meta(ver)/JavaScripts
\end{Verbatim}
Finally, copy \texttt{aeb.js} (and \texttt{aeb\_pro.js}, if applicable)
into this folder. Close \app{Acrobat}, the next time \app{Acrobat}
is started, it will read in the \SC{JS} files.

\paragraph*{{\Stepii}For $\boldmath\text{\meta{ver}}$, $\boldmath10.1.1 \le
\text{\meta{ver}} < 12$.} For these versions of \app{Acrobat} security has
have tightened up even more; the \texttt{JavaScripts} folder is now a
subfolder of a special \texttt{Privileged} folder. Execute
\texttt{app.getPath("user")} in the console window, as in
display~\eqref{display:getPath}. The method brings back a string similar to
the one in display~\eqref{display:path}. The correct path is obtained by
simply inserting \texttt{Privileged} into the path in
display~\eqref{display:path}, between \texttt{Acrobat} and \meta{ver}, as
displayed below.
\begin{Verbatim}[xleftmargin=\amtIndent,fontsize=\small,commandchars=!()]
/C/Users/!meta(username)/AppData/Roaming/Adobe/
    Acrobat/!textbf(Privileged)/!meta(ver)/JavaScripts
\end{Verbatim}
But this is a mere description of the correct location, the path must
exist! The path must be created, if it does not already exist.

Navigate to the folder, one folder above the path shown in
display~\eqref{display:path}, that is, navigate to,
\begin{Verbatim}[xleftmargin=\amtIndent,fontsize=\small,commandchars=!()]
/C/Users/!meta(username)/AppData/Roaming/Adobe/Acrobat/
\end{Verbatim}
If a \texttt{Privileged} folder exists, enter into it; otherwise, create
it, then enter it.
\begin{Verbatim}[xleftmargin=\amtIndent,fontsize=\small,commandchars=!()]
/C/Users/!meta(username)/AppData/Roaming/Adobe/Acrobat/Privileged
\end{Verbatim}
Now create a folder named \meta{ver}, unless it already exists, and enter
it. The version is always the major release \texttt{8.0}, \texttt{9.0},
\texttt{10.0}, or \texttt{11.0}.
\begin{Verbatim}[xleftmargin=\amtIndent,fontsize=\small,commandchars=!()]
/C/Users/!meta(username)/AppData/Roaming/Adobe/
    Acrobat/Privileged/!meta(ver)
\end{Verbatim}
Finally, create the \texttt{JavaScript} folder, unless it already exists,
to obtain the final path to the user folder JavaScripts:
\begin{Verbatim}[xleftmargin=\amtIndent,fontsize=\small,commandchars=!()]
/C/Users/!meta(username)/AppData/Roaming/Adobe/
    Acrobat/Privileged/!meta(ver)/JavaScripts
\end{Verbatim}
Copy \texttt{aeb.js} (and \texttt{aeb\_pro.js}, if applicable) into this
folder. Close \textsf{Acrobat}, the next time \textsf{Acrobat} is started,
it will read in the \SC{JS} files.

\paragraph*{{\Stepiii}For $\boldmath\text{\meta{ver}}$, $\boldmath
\text{\meta{ver}} > 12 $.} Beginning with the next major version following
version~11, Adobe changed its versioning scheme. There are now two tracks,
\emph{subscription} and \emph{classic} (also called \emph{perpetual}). For
subscription, the version, as I understand it, will always be \texttt{DC}
(Document Cloud), and for the classic, for which you've purchased a
\emph{perpetual license}, the \meta{ver} is a calendar year. The first
classic version is \texttt{2015}, the next one is most likely \texttt{2017},
since major releases occur about every two years.

Execute the code of display~\eqref{display:getPath} by pressing the button
provided to obtain a path of the form,
\begin{Verbatim}[xleftmargin=\amtIndent,fontsize=\small,commandchars=!()]
/C/Users/!meta(username)/AppData/Roaming/Adobe/Acrobat/DC
\end{Verbatim}
for the subscription license or,
\begin{Verbatim}[xleftmargin=\amtIndent,fontsize=\small,commandchars=!()]
/C/Users/!meta(username)/AppData/Roaming/Adobe/Acrobat/2015
\end{Verbatim}
for the classic (or perpetual) license. Of course, this last path is for
the first classic version.

Now, to obtain the path to the \texttt{JavaScripts} folder, follow the
instructions for numbered paragraph \ding{183}, but with \meta{ver} equal
to either \texttt{DC} or \texttt{2015} (a calendar year).

%/C/Users/D. P. Story/AppData/Roaming/Adobe/Acrobat/Privileged/11.0/JavaScripts

\subsection{For users of \app{Mac OS}}

\app{Mac OS} does not have the security restrictions that \app{Windows OS}
does, as described in paragraphs \ding{183} and \ding{184}.

Follow the instructions of paragraph \ding{182}, by first pressing the button
provided in display~\eqref{display:getPath}, then, in the console window,
execute this script as described in the paragraph following
display~\eqref{display:getPath}. After working through \ding{182}, one
obtains any of the following paths depending on the version (\meta{ver}) you
are using:\medskip

\noindent
For $\text{\meta{ver}} = \texttt{8.0}, \texttt{9.0}, \texttt{10.0},
\texttt{11.0}$:
\begin{Verbatim}[xleftmargin=\amtIndent,fontsize=\small,commandchars=!()]
/Users/!meta(user)/Library/Application\ Support/Adobe/
    Acrobat/!meta(ver)/JavaScripts
\end{Verbatim}
\medskip

\noindent
For $\text{\meta{ver}} > 12$, the path is,
\begin{Verbatim}[xleftmargin=\amtIndent,fontsize=\small,commandchars=!()]
/Users/!meta(user)/Library/Application\ Support/Adobe/
    Acrobat/DC/JavaScripts
\end{Verbatim}
for a subscription license, and,
\begin{Verbatim}[xleftmargin=\amtIndent,fontsize=\small,commandchars=!()]
/Users/!meta(user)/Library/Application\ Support/Adobe/
    Acrobat/2015/JavaScripts
\end{Verbatim}
or more generally,
\begin{Verbatim}[xleftmargin=\amtIndent,fontsize=\small,commandchars=!()]
/Users/!meta(user)/Library/Application\ Support/Adobe/
    Acrobat/!meta(year)/JavaScripts
\end{Verbatim}
for the perpetual license.

In all cases, if the \texttt{JavaScripts} folder does not exist, it must
be created. Copy any \SC{JS} support files into the \texttt{JavaScripts}
folder. The script files are read the next time \app{Acrobat} is started.

\section{Testing the placement of the JavaScript support files}

Assuming you have successfully followed the instructions and created the
\texttt{JavaScripts} folder, test the validity of what you have done by
closing \app{Acrobat} then opening it again. Open the console window
(\texttt{Ctrl+J} or \texttt{Cmd+J}, for \app{Mac OS}) and execute this
command in the console window:
\begin{Verbatim}[xleftmargin=\amtIndent,commandchars={!~@}]
app.getPath("user","javascript");!quad!copyGetPathJS
\end{Verbatim}
If you have followed the instructions, the return string should read the
complete path to the \texttt{JavaScripts} folder.
\end{document}

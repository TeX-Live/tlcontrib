\documentclass{article}

\usepackage[deffolder=exmpldefs,ignorecfg,
    useselverses,useverses=verses]{fetchbibpes}

\title{The \textsf{fetchbibpes} package\\[3pt]Inputting selected verses only}
\author{D. P. Story}
\date{\today}


\selectedVersesFrom
{
    {onlyverses={Col 1:6}}
    {alt=Alt,onlyverses={Col 1:6}}
    {from=ISV,alt=Alt,onlyverses={Gen 1:1}}
    {onlyverses=1Ki}                    % exclude all verses from 1Ki KJV
%   {from=ISV,onlyverses=1Ki}           % uncomment to exclude 1Ki ISV
%   {from=ISV,onlyverses={Gen 1:1}}     % uncomment to fetch Gen 1:1 ISV
}

\useBookStyle{abbr=long,roman=false}

\addtolength{\marginparwidth}{36pt}
\parindent0pt \parskip6pt

\begin{document}

\maketitle

In the preamble, we declare:
\small
\begin{verbatim}
\selectedVersesFrom
{
    {onlyverses={Col 1:6}}
    {alt=Alt,onlyverses={Col 1:6}}
    {from=ISV,alt=Alt,onlyverses={Gen 1:1}}
    {onlyverses=1Ki}                  % exclude all verses from 1Ki KJV
%   {from=ISV,onlyverses=1Ki}         % uncomment to exclude 1Ki ISV
%   {from=ISV,onlyverses={Gen 1:1}}   % uncomment to fetch Gen 1:1 ISV
}
\end{verbatim}
\normalsize
The way the system works is that any combination of
\texttt{book\symbol{32}bible[\symbol{32}alt]} specified by the
\texttt{onlyverses}, \texttt{from} and \texttt{alt} keys are
`registered'. (The \texttt{alt} key is optional so the third
component may not appear.) What does registered mean? Well, using the first
grouping above as an example, among all verses in the \textsf{DEF} files
input that use a combination of \texttt{Col KJV} (\texttt{KJV}, not specified
by the \texttt{from} key, it the default in this case), only \texttt{Col 1:6
KJV} is defined. From the third grouping, \texttt{Gen ISV Alt} is registered,
so only \texttt{Gen 1:1 ISV Alt} is defined.

Now we fetch some verses with comments in the margin.\marginpar{\bfseries Comments}

\verb|\fetchverse[from*=KJV]{Col 1:6}|\marginpar{\footnotesize\texttt{Col KJV} is
registered with valid `only verse'}\\[3pt]
\fetchverse[from*=KJV]{Col 1:6}

\verb|\fetchverse[alt=Alt,transl=KJV Alt]{Col 1:6}|\marginpar{\footnotesize\texttt{Col KJV Alt}
is registered, valid `only verse'}\\[2pt]
\fetchverse[alt=Alt,transl=KJV Alt]{Col 1:6}

\verb|\fetchverse[transl=KJV]{Col 1:8}|\marginpar{\footnotesize\texttt{Col KJV} is registered, but
the verse is not one of the `only verses'}\\[3pt]
\fetchverse[transl=KJV]{Col 1:8}

\bigskip
\verb|\fetchverse[transl=KJV]{Mat 2:1}|\marginpar{\footnotesize\texttt{Mat KJV} not registered so any passage
from \texttt{Mat KJV} is permitted}\\[3pt]
\fetchverse[transl=KJV]{Mat 2:1}

\verb|\fetchverse[from=ISV,alt=Alt,transl=ISV Alt]{Gen 1:1}|\marginpar{\footnotesize\texttt{Gen ISV Alt}
is registered, valid `only verse'}\\[3pt]
\fetchverse[from=ISV,alt=Alt,transl=ISV Alt]{Gen 1:1}

\bigskip
\verb|\fetchverse[from*=ISV]{Gen 1:1}|\marginpar{\footnotesize\texttt{Gen ISV} is registered, but
the verse is not an `only verse' }\\[3pt]
\fetchverse[from*=ISV]{Gen 1:1}

Additional comments on the last two are needed. We \emph{explicitly} registered
\texttt{Gen~ISV~Alt}; however, to make this system work in the way desired,
\texttt{Gen~ISV} is \emph{implicitly} registered too.

\verb|\fetchverse{1Ki 1:1}|\marginpar{\footnotesize\texttt{1Ki KJV} is registered, but none of the
verses are `only verses' }\\[3pt]
\fetchverse{1Ki 1:1}

\bigskip

\verb|\fetchverse[from*=ISV]{1Ki 1:1}|\marginpar{\footnotesize\texttt{1Ki ISV} is not register so
any passage from \texttt{1Ki ISV} is permitted}\\[3pt]
\fetchverse[from*=ISV]{1Ki 1:1}

\end{document}

\documentclass{article}
\usepackage[fleqn]{amsmath}
\usepackage[pdf,forpaper,cfg=hw,nopoints,useforms,
% Try compiling this file under one of these options:
%     nosolutions,
%    answerkey,
    vspacewithsolns,
]{eqexam}

\subject[CA]{College Algebra}
\title[HW2]{HW \#2}
\author{Dr.\ D. P. Story}
\date{Spring 2011}
\duedate{03/05/11}
\keywords{Homework due \theduedate}

% Make the end of solution label blank
\def\exrtnlabelformat{}
\def\exrtnlabelformatwp{}
\solAtEndFormatting{\eqequesitemsep{3pt}}


\begin{document}

\maketitle

\begin{exam}{HW}

\begin{instructions}[]
In preparation for the quiz on Thursday, solve each of these short
problems in the space provided before looking at their solutions at the
end of the document.
\[
   \text{\url{http://faculty.nwfsc.edu/web/math/storyd}}
\]
All class assignments and other announcements will be posted on
this web site.
\end{instructions}

\begin{problem*}[4ea]
Let $P(-4,2)$ and $Q(2,-3)$ be two points in the plane.
\begin{parts}
\item Find the distance $d(P,Q)$ between $P$ and $Q$.
\begin{solution}[1in]
We use the distance formula
\[
    d(P,Q)=\sqrt{(2+4)^2+(-3-2)^2}=\sqrt{61}
\]
to obtained the required answer.
\end{solution}

\item Find the midpoint $M$ between $P$ and $Q$.
\begin{solution}[\sameVspace]
We use the midpoint formula
\[
    M=\left(\frac{-4+2}{2},\frac{2+(-3)}{2}\right)=\left(-1,-\dfrac{1}{2}\right)
\]
to obtained the required answer.
\end{solution}
\end{parts}
\end{problem*}

\begin{problem*}[3ea]
Complete each of the two sentences below with correct entries.
\begin{parts}
  \item The function $ g(x) = | x+2 | $ can be graphed from the library
  function $ f(x) = |x| $ by shifting it \fillin[u]{.25in}{2} units
  \fillin[u]{1.25in}{horizontally} (horizontally/vertically) \fillin[u]{.75in}{left} (left/right/up/down).
% Make the solution label blank.
\ifkeyalt
\begin{solution}[]
% We want this to be shown at the end of the file, but not if the author changes the option
% to answerkey
    The function $ g(x) = | x+2 | $ can be graphed from the library
    function $ f(x) = |x| $ by shifting it \fillin[u]{.25in}{2} units
  \fillin[u]{1.25in}{horizontally} (horizontally/vertically) \fillin[u]{.75in}{left}
  (left/\penalty0right/\penalty0up/\penalty0down). % Note: \penalty0 is inserted to help TeX break the line after the forward slash /
\end{solution}
\fi
  \item The function $ g(x) = 5 - x^2 $ can be graphed from the library
  function $ f(x) = x^2 $ by first reflecting it with respect to the
  \fillin[u]{.25in}{$x$} axis, then shifting it \fillin[u]{.25in}{5} units
  \fillin[u]{1.25in}{vertically} (horizontally/vertically) \fillin[u]{.75in}{upward} (left/right/up/down).
\ifkeyalt
\begin{solution}[]
% We want this to be shown at the end of the file, but not if the author changes the option
% to answerkey.
  The function $ g(x) = 5 - x^2 $ can be graphed from the library function
  $ f(x) = x^2 $ by first reflecting it with respect to the
  \fillin[u]{.25in}{$x$} axis, then shifting it \fillin[u]{.25in}{5} units
  \fillin[u]{1.25in}{vertically} (horizontally/vertically)
  \fillin[u]{.75in}{upward} (left/right/up/down).
\end{solution}
\fi
\end{parts}
\end{problem*}

\begin{problem}[5]
The circle $ x^2 + y^2 = 25 $ passes through
the point $P(3,4)$.  Let $\ell$ be the line passing though the origin and
the point $P$.  Find the equation of the line perpendicular to line $\ell$ and passing
through point $P$.
\begin{solution}[1in]
The slope of the line perpendicular to $\ell$ is $ m = -\tfrac{3}{4} $, the line
must pass through $(3,4)$; thus, the line is $ y-4 = -\tfrac{3}{4}(x-3) \implies
y = -\tfrac{3}{4} x + \tfrac{25}{4} $. Thus,
\[
    \text{Ans:}\quad\boxed{ y = -\tfrac{3}{4} x + \tfrac{25}{4} }
\]
This is the equation of the line tangent to the circle at $P(3,4)$.
\end{solution}
\end{problem}

\begin{problem}[3]
If the slope the a line is negative, then the line is
    \begin{answers}{4}
    \bChoices
        \Ans0 increasing\eAns
        \Ans1 decreasing\eAns
        \Ans0 constant\eAns
        \Ans0 none of these\eAns
    \eChoices
    \end{answers}

% The \texttt{answers} and \texttt{manswers} environments can now
% be copied and pasted into the solutions environment as well.
%
\ifkeyalt
\begin{solution}[.25in]
If the slope the a line is negative, then the line is
    \begin{answers}{4}
    \bChoices
        \Ans0 increasing\eAns
        \Ans1 decreasing\eAns
        \Ans0 constant\eAns
        \Ans0 none of these\eAns
    \eChoices
    \end{answers}
\end{solution}
\fi
\end{problem}
\end{exam}
\end{document}

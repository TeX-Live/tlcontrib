\documentclass{article}
\usepackage{amsmath}
\usepackage{graphicx}
\usepackage{array,longtable}

\begin{document}

\noindent The following list of blend modes  is taken verbatim from section
7.2.4 of the \textsl{PDF Reference}, Sixth Edition (Version 1.7).

\noindent\hfill\begin{longtable}{>{\ttfamily}lp{3.5in}}
Normal    & Selects the source color, ignoring the backdrop.
%            $$B(c_b,c_s)=c_s$$
\\[3pt]
Multiply  &  Multiplies the backdrop and source color values: The
            result color is always at least as dark as either of the two
            constituent colors. Multiplying any color with black produces black;
            multiplying with white leaves the original color unchanged. Painting
            successive overlapping objects with a color other than black or
            white produces progressively darker colors.
%            $$B(c_b,c_s)=c_b\times c_s$$
\\[3pt]

Screen    & Multiplies the complements of the backdrop and source
            color values, then complements the result: The result color is
            always at least as light as either of the two constituent colors.
            Screening any color with white produces white; screening with black
            leaves the original color unchanged. The effect is similar to
            projecting multiple photographic slides simultaneously onto a single
            screen.
%            $$
%                B(c_b,c_s) = 1 - [ (1 - c_b) \times (1 - c_s) ]
%                    = c_b + c_s - (c_b\times c_s)
%            $$
\\[3pt]
Overlay   & Multiplies or screens the colors, depending on the
            backdrop color value. Source colors overlay the backdrop while
            preserving its highlights and shadows. The backdrop color is not
            replaced but is mixed with the source color to reflect the lightness
            or darkness of the backdrop.
%            $$B(c_b,c_s)=\operatorname{HardLight}(c_b,c_s)$$
\\[3pt]

Darken    & Selects the darker of the backdrop and source colors:
            The backdrop is replaced with the source where the
            source is darker; otherwise, it is left unchanged.
%            $$B(c_b,c_s)=\min(c_b,c_s)$$
\\[3pt]

Lighten   & Selects the lighter of the backdrop and source colors:
            The backdrop is replaced with the source where the
            source is lighter; otherwise, it is left unchanged.
%            $$B(c_b,c_s)=\max(c_b,c_s)$$
\\[3pt]

ColorDodge &Brightens the backdrop color to reflect the source
            color. Painting with black produces no changes.
%            $$
%                B(c_b,c_s)=\begin{cases}
%                \min(1,c_b/(1-c_s)) & \text{if $ c_s < 1$}\\[3pt]
%                1             & c_s = 1
%            \end{cases}
%            $$
\\[3pt]

ColorBurn & Darkens the backdrop color to reflect the source color.
            Painting with white produces no change.
%            $$
%                B(c_b,c_s)=\begin{cases}
%                \min(1,(1-c_b)/c_s) & \text{if $ c_s > 0$}\\[3pt]
%                0             & c_s = 0
%            \end{cases}
%            $$
\\[3pt]


HardLight & Multiplies or screens the colors, depending on the
            source color value. The effect is similar to shining a harsh
            spotlight on the backdrop.
%            $$
%                B(c_b,c_s)=\begin{cases}
%                \operatorname{Multiply}(c_b,2\times c_s) & \text{if $ c_s \le 0.5$}\\[3pt]
%                \operatorname{Screen}(c_b,2\times c_s -1)             & c_s > 0.5
%            \end{cases}
%            $$
\\[3pt]

SoftLight & Darkens or lightens the colors, depending on the source
            color value. The effect is similar to shining a diffused spotlight
            on the backdrop.
%            $$
%                B(c_b,c_s)=\begin{cases}
%                c_b - (1 - 2\times c_s)\times c_b\times (1-c_b) & \text{if $ c_s < 0.5$}\\[3pt]
%                c_b + (2 \times c_s - 1) \times (\operatorname{D}(c_b)-c_b)  & c_s > 0.5
%            \end{cases}
%            $$
%            where
%            $$
%                D(x)=\begin{cases}
%                ((16\times x -12) \times x + 4) \times x & \text{if $ x < 0.25$}\\[3pt]
%                \sqrt{x}  & c_s > 0.25
%            \end{cases}
%            $$
\\[3pt]

Difference &Subtracts the darker of the two constituent colors from
            the lighter color: Painting with white inverts the backdrop color;
            painting with black produces no change.
%            $$
%                B(c_b,c_s) = | c_b - c_s |
%            $$
\\[3pt]

Exclusion & Produces an effect similar to that of the \texttt{Difference}
            mode but lower in contrast. Painting with white inverts the backdrop
            color; painting with black produces no change.
%            $$
%                B(c_b,c_s) = c_b + c_s - 2\times c_b \times c_s
%            $$
\\[3pt]
Hue & Creates a color with the hue of the source color and the
            saturation and luminosity of the backdrop color.
%            $$
%                B(C_b,C_s)=\operatorname{SetLum}(\operatorname{SetSat}(C_s,\operatorname{Sat}(C_b)),\operatorname{Lum}(C_b))
%            $$
\\[3pt]

Saturation & Creates a color with the saturation of the source color
            and the hue and luminosity of the backdrop color. Painting with this
            mode in an area of the backdrop that is a pure gray (no saturation)
            produces no change.
%            $$
%                B(C_b,C_s)=\operatorname{SetLum}(\operatorname{SetSat}(C_b,\operatorname{Sat}(C_s)),\operatorname{Lum}(C_b))
%            $$
\\[3pt]

Color & Creates a color with the hue and saturation of the source
            color and the luminosity of the backdrop color. This preserves the
            gray levels of the backdrop and is useful for coloring monochrome
            images or tinting color images.
%            $$
%                B(C_b,C_s)=\operatorname{SetLum}(C_s,\operatorname{Lum}(C_b))
%            $$
\\[3pt]
Luminosity & Creates a color with the luminosity of the source color
            and the hue and saturation of the backdrop color. This produces an
            inverse effect to that of the \texttt{Color} mode.
%            $$
%                B(C_b,C_s)=\operatorname{SetLum}(C_b,\operatorname{Lum}(C_s))
%            $$
\\[3pt]
\end{longtable}
\end{document}

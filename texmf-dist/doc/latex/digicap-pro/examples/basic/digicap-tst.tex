\documentclass{article}
\usepackage[%
    web={tight,pro,usesf,usetemplates},
    uselayers,eforms,aebxmp
]{aeb_pro}
\usepackage{digicap-pro}

\margins{.5in}{.5in}{24pt}{.5in}    % left,right,top, bottom
\screensize{4.8in}{6in}             % height, width
\renewcommand{\titleauthorproportion}{.4}
\DeclareDocInfo
{
    title=The DigiCap Pro Package\texorpdfstring{\\[1ex]}{: }Digital Captioning,
    author=D. P. Story,
    university=Acro\negthinspace\TeX.Net,
    email=dpstory@acrotex.net,
    subject=Test file for the DigiCap Pro package,
    keywords={Captioning, opacity, layers, roll-over},
    talksite=\url{http://www.acrotex.net},
    talkdate={\today},
    copyrightStatus=True,
    copyrightNotice={Copyright (C) \the\year, D. P. Story},
    copyrightInfoURL=http://www.acrotex.net
}
\talkdateLabel{Published:}


\newcommand{\cs}[1]{\texttt{\char`\\#1}}

\pagestyle{empty}

\parskip6pt
\parindent0pt

\begin{document}

\maketitle

\begin{center}
\digiCap[inclgraphicx={scale=.6},vcaption=b,hcaption=c,outerboxsep=0pt]
    {./digis/4CzSojSc}[borderwidth=0bp,fboxsep=10bp,bordercolor=nocolor,bgop=.7]
    {\bfseries\makebox[\linewidth][c]{\textcolor{red}
    {Yankees fire first salvo in rivalry with two-hitter}}\\\relax\footnotesize
    Chien-Ming Wang took a no-hitter into the fifth inning and
    surrendered just two hits in a complete-game gem as the
    Yankees beat the Red Sox, 4-1, on Friday at Fenway Park.}%
\end{center}

Two new commands \cs{graphicCaption} and  \cs{graphicCaptionRollover} are introduced.
These two use \cs{graphicxbox} and another new command called \cs{opcolorbox}.

The above photo has a transparent caption at the bottom. This can be changed to
a center or top positioning.

This style of captioning (using a transparent background with text on top) is used not only
on the \href{http://www.mlb.com}{MLB web site}, but in magazine layouts as well, such as the
paper version of \href{http://www.tvguide.com}{TV GUIDE}.

The background image and text were taken from \href{http://www.mlb.com}{www.mlb.com}.

% use a command to set popular options.
\def\myOpts{inclgraphicx={scale=.6},vcaption=b,hcaption=r,outerboxsep=0pt}

\begin{center}
\digiCap*[\myOpts,rollovername=YankeesRedSox]%
    {./digis/wobwRHNb}[borderwidth=0bp,fboxsep=10bp,bordercolor=nocolor,bgop=.7]
    {\parskip6pt\bfseries\makebox[\linewidth][c]{\textcolor{webblue}{Tribe rally
    falls short after Sabathia struggles}}\\\relax\footnotesize
    C.C. Sabathia had no answers for Oakland on Friday, allowing
    12 hits in a 9-7 loss. The Tribe battled from an eight-run
    deficit but couldn't catch the A's.}
\end{center}

For the next three pages, the caption is hidden in a separate layer. Rollover the photo
to see the caption.

We stayed up late watching this rain-delayed game over the Internet. Late inning
heroics fell short of the mark.

Photos and descriptions of Indians action obtained from \href{http://indians.mlb.com}{indians.mlb.com}.

\begin{center}
\digiCap*[inclgraphicx={scale=.6},vcaption=t,hcaption=r,outerboxsep=0pt,rollovername=IndiansAs]%
    {./digis/xn2cpnQk}[borderwidth=4bp,fboxsep=10bp,width=.85\linewidth,bordercolor=nocolor,bgop=.7]%
    {\parskip6pt\bfseries\makebox[\linewidth][c]{\textcolor{webblue}
    {Free passes, missed chances hurt Tribe}}\\\relax\footnotesize Fausto Carmona
    issued eight walks in 3 1/3 innings, and the Indians scored just one
    run with the bases full and one out in Saturday's 7-3 loss to the
    A's.}
\end{center}

This was a disappointing game. The opposing team has
several of their ``stars'' out of the line-up due to injuries, and
we still could not beat them, thanks to the bad pitching performance
of Fausto.

Here the caption is \texttt{.85}\cs{linewidth}, it is placed at the
top and has a right horizontal alignment. Note the \texttt{4bp}
spacing between the caption and the edge of the photo, there are
controls for this spacing.

\begin{center}
\digiCap*[inclgraphicx={scale=.6},vcaption=c,hcaption=l,outerboxsep=0pt,
    rollovername=IndiansAsThirdGame]{./digis/IiFUCdIQ}[borderwidth=4bp,fboxsep=10bp,
    bordercolor=blue,bgop=.7]{\parskip6pt\bfseries
    \makebox[\linewidth]{\textcolor{webblue}{Lee ices A's win
    streak with an eight-inning gem}}\\\relax\footnotesize
    Lee ices A's win streak with an eight-inning gem Cliff Lee retired
    14 straight in a 7-1 Indians win over the A's on Sunday. Grady
    Sizemore had three RBIs as Cleveland halted its three-game skid.}%
\end{center}

The game on April 13th. My wife watched this while I assembled the new outdoor grill.

You can place a transparent border around a caption, as illustrated above. This one goes
the width of the image, you can have some separation around the border as well (not illustrated).


\newpage

\begin{center}
\digiCap*[inclgraphicx={height=\textheight,width=\linewidth,keepaspectratio},%
     vcaption=c,hcaption=r,outerboxsep=0bp]{./digis/sciencefair}[width=.4\linewidth,%
     borderwidth=0bp,fboxsep=7bp,bordercolor=nocolor,bgcolor=webyellow,bgop=.6]{%
        \begin{minipage}[b][\graphicHeight-2\fboxsep-2\fboxrule]{\linewidth}\color{webbrown}\small\bfseries\parskip6pt
        A photo of the High School Science Fair project of my beloved son.
        The project tested a subject's reaction time while
        talking on a call-phone (hands-on and hands-free) and while
        distracted. A statistical analysis of the data was
        performed on the data. The results were inconclusive.

        This caption demonstrates that it is possible to create a vertical caption, positioned
        on the left, right or in the center.
        \end{minipage}
    }\\[6pt]
My son's science fair project.
\end{center}

\end{document}

%%%%%%%%%%%%%%%%%%%%%%%%%%%%%%%%%%%%%%%%%%%%%%%%%%%%%%%%%%%%
%%%%%%%%%%%%%%%%%%%%%%%%%%%%%%%%%%%%%%%%%%%%%%%%%%%%%%%%%%%%
%%                                                        %%
%% JEOPARDY DEMO FILE                                     %%
%%                                                        %%
%% The AcroTeX Software Development Team (Copyright 2007) %%
%% http://www.acrotex.net
%%                                                        %%
%%%%%%%%%%%%%%%%%%%%%%%%%%%%%%%%%%%%%%%%%%%%%%%%%%%%%%%%%%%%
%%%%%%%%%%%%%%%%%%%%%%%%%%%%%%%%%%%%%%%%%%%%%%%%%%%%%%%%%%%%

\documentclass{jj_game} % or dvips, pdftex, luatex, xetex
\usepackage{amsmath}

% This is an example of how you can design your own unique
% graphical display of the game and how you can use jj_game
% commands to place the jj_game elements on the page.

%
% Raise the text in the banner to fit the graphic better
%
\renewcommand{\bannerTextControl}[1]{\raisebox{3pt}{#1}}
%
% Adjust font used in banner text
%
\renewcommand{\bannerTextFont}{\sffamily\bfseries\large}

%% Possible Definition of a CENT Symbol
\def\cents{\hbox{\rm\rlap/c}}

\titleBanner{Money Sense Jeopardy}

%
% Add a little vertical space under the currency statement
% of the question page.
%
\aboveCurrencySkip{12pt}

%
% Declare the graphics to be used on the first, second, the
% the question pages.
%
%\defineInstructionPageGraphic{Quiz_Jeopardy}
%\defineGameboardPageGraphic{Quiz_Jeopardy_GB}
%\defineQuestionPagesGraphic{Quiz_Jeopardy}

%% You can also define your own Color(s), like so:
%% \definecolor{my_color}{rgb}{0.92,0.67,0.1}
%% So you are very flexible in Color Design of your
%% unique Jeopardy Game
\definecolor{demo_gb}{rgb}{0.58, 0.58, 0}
\definecolor{demo}{rgb}{0.275, 0.275, 0.275}

\DeclareColors
{
    fillCells: transparent,
    fillBanner: transparent,
    textBanner: black,
    textBoard: black,
    fillInstructions: cornsilk,
    fillGameBoard: demo_gb,
    fillQuestions: cornsilk,
    dollarColor: red, % Color of the Value of the Question (in the question environment)
    linkColor: red,   % Color of the Links (the answer possibilities)
}

%% Note, that the measurement of the resulting PDF is given with the
%% Number of Categories, Number of the appropriate Questions, Cell Width
%% and Cell Height. The Geometry of the PDF depends on that next settings in
%% the "\GameDesign"!

\GameDesign
{
    Cat: [\sffamily\bfseries] Fractions,  % Special Font settings in the [...] for the Categories
    Cat: [\sffamily\bfseries] Decimals,
    Cat: [\sffamily\bfseries] Numbers/Rounding,
    Cat: [\sffamily\bfseries] Money,
    NumQuestions: 5,
    CellWidth: 1.5in,
    CellHeight: .5in,
%   Goal: 1,500,                % specify absolute goal
    GoalPercentage: 90,         % specify relative goal
    ExtraHeight: 0pt,
    Champion: Financial Wizardry!,
}

\APDollar
{
    Font: Helv,
    Size: 20,
    TextColor: 1 1 1,               % color of text for the Gameboard cell
    BorderColor: 0 0 0,             % color of border for the Gameboard cell
    FillColor: 0.275 0.275 0.275    % the fill color for the Gameboard cell
}

\APHidden
{
    Font: Helv,
    Champion: You are TeXerrific!,
    Size: 20,
    TextColor: 0 0 0,
    BorderColor: 0 0 0,
    FillColor: 0.92 0.67 0.1
}

\APRight
{
   Font: Helv,
   Size: 20,
   TextColor: 0.62 0.55 0.067,
}

\APWrong
{
   Font: Helv,
   Size: 20,
   TextColor: 0.96 0.38 0.12,
}

\APScore
{
    Font: Helv,
    Size: 20,
    CellWidth: 4in,
    BorderColor: ,
    FillColor: 1 0.8 0,
    AutoPlacement: true,
    Score: "Points: ",
    Currency: "$",
    align: c,
}

\begin{document}

\begin{instructions}

\vspace*{12pt}

\textcolor{red}{\bfseries Method of Scoring:}

If you answer a question correctly, the dollar value of that
question is added to your total.  If you miss a question, the value
is {\it subtracted\/} from your total.  So think carefully before
you answer!

\textcolor{red}{\textbf{Instructions:}}

Solve the problems in
any order you wish.

\textcolor{red}{\textbf{Important:}}

Acrobat Reader 5.0 or later required.

\begin{center}
\Acrobatmenu{NextPage}{\fcolorbox{red}{lightgray}{\sffamily\textbf{Start the Quiz}}}
\end{center}

\end{instructions}

%% Some Local Definitions of Layout
\DeclareColors{fillBanner: BrickRed} % Local defined Background Color for the Gameboard

\begin{Questions}

\begin{Category}{Fractions}

%% Some Local Definitions of Layout
\DeclareColors{fillBanner: transparent}  % Local defined Background Color for the Questions

\begin{Question}

An equivalent way to write a fraction is as a \dots
\Ans0 Product
\Ans1 Decimal
\Ans0 Sum
\Ans0 Factor
\end{Question}

\begin{Question}

The fraction $\dfrac{3}{10}$ written as a decimal is\dots
\Ans0 $0.03$
\Ans0 $3.10$
\Ans1 $0.3$
\Ans0 $1.3$
\end{Question}

\begin{Question}

A number equivalent to the fraction $\dfrac{99}{99}$ is\dots
\Ans0 $100$
\Ans0 $9$
\Ans1 $1$
\Ans0 $198$
\end{Question}

\begin{Question}

The fraction $\dfrac{77}{77}$ is equivalent to\dots
\Ans0 $\dfrac{60}{80}$
\Ans1 $\dfrac{30}{30}$
\Ans0 $154$
\Ans0 $0$
\end{Question}

\begin{Question}

The fraction $\dfrac{3}{2}$ is equal to\dots
\Ans0 $2\dfrac{2}{3}$
\Ans0 $6$
\Ans1 $1\dfrac{1}{2}$  % changed from $1\dfrac{1}{3}$
\Ans0 $1$
\end{Question}

\end{Category}


\begin{Category}{Decimals}

\begin{Question}

Compare the decimals $0.4$ and $0.40$\dots
\Ans0 $0.4>0.40$
\Ans0 $0.4<0.40$
\Ans1 $0.4=0.40$
\Ans0 none of the above
\end{Question}

\begin{Question}

Compare the decimals $0.35$ and $0.75$\dots
\Ans0 $0.35>0.75$
\Ans1 $0.35<0.75$
\Ans0 $0.35=0.75$
\Ans0 none of the above
\end{Question}

\begin{Question}

The fraction $\dfrac{1}{2}$ is equivalent to the
decimal\dots
\Ans0 $0.20$
\Ans0 $1.25$
\Ans0 $0.12$
\Ans1 $0.50$
\end{Question}


\begin{Question}

The largest number among $1.26$, $0.58$, $1.09$, $1.091$ and $0.35$
is\dots
\Ans1 $1.26$
\Ans0 $0.58$
\Ans0 $1.09$
\Ans0 $1.091$
\Ans0 $0.35$
\end{Question}


\begin{Question}

The smallest number among $1.26$, $0.58$, $1.09$, $1.091$ and $0.35$
is\dots
\Ans0 $1.26$
\Ans0 $0.58$
\Ans0 $1.09$
\Ans0 $1.091$
\Ans1 $0.35$
\end{Question}

\end{Category}


\begin{Category}{Numbers \& Rounding}

\begin{Question}

The number one million, seventy-nine thousand five is written
as\dots
\Ans0 $1{,}795{,}000$
\Ans1 $1{,}079{,}005$
\Ans0 $1{,}790{,}500$
\Ans0 $1{,}709{,}050$
\end{Question}

\begin{Question}

The smallest number you can make with the digits $3$, $6$, $4$, $7$,
$2$ is\dots
\Ans0 $42{,}736$
\Ans0 $23{,}647$
\Ans0 $32{,}467$
\Ans1 $23{,}467$
\end{Question}

\begin{Question}

The largest number you can make with the digits 5, 9, 0, 3, 8, 1 is\dots
\Ans0 $590{,}381$
\Ans0 $183{,}095$
\Ans1 $985{,}310$
\Ans0 $958{,}013$
\end{Question}

\begin{Question}

The expanded form of four hundred thirty-two thousand, one
hundred three is\dots
\Ans1 $400{,}000+30{,}000+2{,}000+100+3$
\Ans0 $400+32{,}000+103$
\Ans0 $400{,}000+30,000+2{,}000+100+30$
\Ans0 $4{,}000{,}000+30{,}000+2{,}000+100+3$
\end{Question}

\begin{Question}

In which set would all the numbers round to $60$?
\Ans0 $55$, $52$, $69$, $67$
\Ans1 $56$, $59$, $63$, $64$
\Ans0 $57$, $61$, $56$, $68$
\Ans0 $58$, $62$, $57$, $69$
\end{Question}

\end{Category}

\begin{Category}{Money}

\begin{Question}

The change received back from $\$1.00$ after buying an ice cream
cone consisted of a quarter, a dime and three pennies.  The ice
cream cone cost\dots
\Ans0 $78$\cents
\Ans1 $62$\cents
\Ans0 $53$\cents
\Ans0 $38$\cents
\end{Question}

\begin{Question}

Which of the following equals $\$1.47$?
\Ans0 four quarters, five dimes, four nickels, seven pennies
\Ans0 six quarters, one nickel, two pennies
\Ans1 five quarters, two dimes, two pennies
\Ans0 four quarters, one dime, one nickel, two pennies
\end{Question}

\begin{Question}

John spent a total of $\$3.00$ on baseball cards.  To find out
how much money he has left, we need to know\dots
\Ans0 How many cards John bought
\Ans0 How much a card costs
\Ans0 How many times John bought cards
\Ans1 How much money John had at first
\end{Question}

\begin{Question}

If you pay for a $\$13.97$ toy with a $\$50$ bill, your change
is\dots
\Ans0 $\$35.03$
\Ans1 $\$36.03$
\Ans0 $\$35.97$
\Ans0 $\$36.97$
\end{Question}

\begin{Question}

Kate has three dimes.  Anna has four nickels.  Which number sentence tells
how many cents they have together?
\Ans0 $4+3$
\Ans0 $3+10+4+5$
\Ans1 $(3\times 10)+(4\times 5)$
\Ans0 $7\times(10+5)$
\end{Question}

\end{Category}

\end{Questions}

\end{document}

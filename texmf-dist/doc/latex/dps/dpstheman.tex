\documentclass{article}
\usepackage[%
  web={designv,forcolorpaper,latextoc,extended},
  eforms={execJS}
]{aeb_pro}
%\usepackage[designv,nodirectory,forcolorpaper,latextoc]{web}
%\usepackage[execJS]{eforms}
\usepackage{graphicx}
\reversemarginpar

%\usepackage{myriadpro}
\usepackage[altbullet]{lucidbry}
\usepackage{aeb_mlink}
\usepackage{acroman}

\usepackage[active]{srcltx}

\def\darg#1{\texttt{\{#1\}}}
\let\key\texttt

%\setlength{\marginparsep}{0pt}
\addtolength{\marginparwidth}{15pt}

\def\SUB#1{\ensuremath{{}_{\mbox{\scriptsize\ttfamily#1}}}}

%\margins{.25in}{.25in}{24pt}{.25in} % left,right,top, bottom
%\screensize{6.5in}{5in} % height, width

\university{\AcroTeX.Net}
\title{dps Package\texorpdfstring{\\[1em]}{: }Das Puzzle Spiel}
\author{D. P. Story}
\email{dpstory@acrotex.net}
\subject{Documentation of the dps package}
\keywords{Adobe Acrobat, games, matching}

\versionLabel{Dated:}
\version{2020/06/03}
\copyrightyears{2006-\the\year}
\nocopyright
\prepared{2020/06/03} %2020/06/03 v1.7
\revisionLabel{Distribution Dated:}

\DesignTitlePageTrailer
{%
    ul=Copyright \copyright\ \webcopyrightyears,
    ur=\thewebemail,
    lr=\url{http://www.acrotex.net},
    raise=-25pt
}


\def\dps{$\hbox{$\mathfrak D$\kern-.3em\hbox{$\mathfrak P$}%
   \kern-.6em \hbox{$\mathcal S$}}$}

\let\tops\texorpdfstring
\makeatletter
\let\web@copyright=\@gobble
\def\changelinkcolorto#1{\def\@linkcolor{#1}}
\newdimen\totaltextwidth
\totaltextwidth=\fullscreenwidth
\advance\totaltextwidth\oddsidemargin
\renewcommand{\@oddhead}{\hspace{-\oddsidemargin}%
    \hbox to\totaltextwidth{\normalfont\footnotesize
    \web@headerhook\web@lhead\hfil\web@chead\hfil
    \web@rightheader}\hss}%
\renewcommand{\paragraph}
    {\@startsection{paragraph}{4}{0pt}{6pt}{-3pt}
    {\normalfont\normalsize\bfseries}}
\renewcommand{\subparagraph}
    {\@startsection{subparagraph}{5}{\parindent}{6pt}{-3pt}%
    {\normalfont\normalsize\bfseries}}
\makeatother

\def\AcroTeX{Acro\negthinspace\TeX}
%\newenvironment{sverbatim}
%{\par\footnotesize\verbatim}{\endverbatim}

%\newcommand\redpoint{\par\removelastskip
%\vskip\medskipamount
%\noindent
%  \makebox[0pt][r]{\large\color{red}$\blacktriangleright$\enspace}}
%\newcommand\handpoint{\par\ifdim\lastskip>0pt\relax\vskip-\lastskip\fi
%\vskip\medskipamount\noindent
%  \makebox[\parindent][l]{\large\color{blue}\ding{042}}}
%\newcommand\newtopic{\par\ifdim\lastskip>0pt\relax\vskip-\lastskip\fi
%\vskip\medskipamount\noindent}
%\newcommand{\cs}[1]{\texttt{\char`\\#1}}

\makeatletter
\newcommand{\linkSave}[2][]{\def\argi{#1}\ifx\argi\@empty\def\argi{#2}\fi
    \setLinkText[\Border{0 0 0}\A{\JS{this.exportDataObject({cName: "\argi", nLaunch: 0})}}]
    {\textcolor{webgreen}{#2}}}
\makeatother

\def\FmtMP#1{\marginpar{\slshape\raggedleft\small#1}}

\optionalPageMatter{%
\par\minimumskip\vspace{\stretch{1}}
    \noindent\begin{minipage}{\linewidth}
    \begin{center}
    \fcolorbox{blue}{webyellow}{%
    \begin{minipage}{.67\linewidth}\parskip3pt
    \textcolor{red}{Abstract: }\textcolor{blue}{Das Puzzle Spiel} is a {\LaTeX} package for
    creating a puzzle, a message actually, and a series of
    questions and answers. The document consumer matches the
    questions with the answers. With each match, another letter
    appears in the puzzle. Upon completion of all questions, the
    message hidden in the puzzle is revealed. The game has an application
    as a classroom learning device.
    \end{minipage}}%
%    \par\bigskip
%    \fcolorbox{blue}{webyellow}{%
%    \begin{minipage}{.67\linewidth}\parskip3pt \tightsettings
%    \textcolor{red}{Distribution: } Click on the link below, and
%    save \texttt{dps.txt} to your hard drive.\footnote{If the link does not work for you,
%    save \texttt{dps.txt} using the Adobe Reader user interface.}
%    \begin{itemize}
%    \item[] \texttt{\linkSave{dps.txt}}
%    \end{itemize}
%    Rename \texttt{dps.txt} to \texttt{dps.zip}. Unzip
%    \texttt{dps.zip} somewhere in your {\LaTeX} search path. Unzipping
%    will create a folder named \texttt{dps}. If using a {\TeX} system
%    requires it, refresh your file name database.
%
%    Brief descriptions of
%    these files are given on
%    \hyperref[distribution]{page~\pageref*{distribution}}.
%\par\smallskip
%    \end{minipage}}
    \end{center}
    \end{minipage}
}

\makeatletter
\renewcommand{\section}{%
   \renewcommand{\@seccntformat}[1]{\csname the##1\endcsname.\enspace}%
   \@startsection{section}{1}{-62.5pt}%
   {-3ex\@plus -1ex \@minus-.2ex}%
   {1ex\@plus .2ex}% 6pt
   {\noindent\normalfont\large\bfseries\color{blue}}}%
\renewcommand\subsection{%
   \renewcommand{\@seccntformat}[1]{\csname the##1\endcsname.\enspace}%
   \@startsection{subsection}{2}{-42.5pt}%
   {-2.5ex\@plus -1ex \@minus -.2ex}%
   {1ex \@plus .2ex}%
   {\noindent\normalfont\normalsize\bfseries\color{blue}}}
\makeatother

%\parindent0pt
%\parskip\medskipamount

%
% Attach the document target.pdf to this document
%
%\begin{execJS}{execjs}
%console.println("Importing data object");
%var myPath = /.*\//i.exec(this.path);
%var thisPath = this.path.replace(/\.pdf$/, ".tex");
%aebTrustedFunctions(this, aebImportDataObject, {cName:"dps.txt", cDIPath: myPath+"dps.txt" })
%%aebTrustedFunctions(this, aebImportDataObject, {cName:"dps.dtx", cDIPath: myPath+"dps.dtx" })
%%aebTrustedFunctions(this, aebImportDataObject, {cName:"dps.ins", cDIPath: myPath+"dps.ins" })
%%aebTrustedFunctions(this, aebImportDataObject, {cName:"dps_demo.tex", cDIPath: myPath+"dps_demo.tex" })
%%aebTrustedFunctions(this, aebImportDataObject, {cName:"pmg_d1.tex", cDIPath: myPath+"pmg_d1.tex" })
%%aebTrustedFunctions(this, aebImportDataObject, {cName:"pmg_d1_p.tex", cDIPath: myPath+"pmg_d1_p.tex" })
%%aebTrustedFunctions(this, aebImportDataObject, {cName:"pmg_d2.tex", cDIPath: myPath+"pmg_d2.tex" })
%%aebTrustedFunctions(this, aebImportDataObject, {cName:"pmg_d3.tex", cDIPath: myPath+"pmg_d3.tex" })
%\end{execJS}

\chngDocObjectTo{\newDO}{doc}
\begin{docassembly}
var titleOfManual="The DPS Package";
var manualfilename="Manual_BG_Print_dps.pdf";
var manualtemplate="Manual_BG_Brown.pdf"; // Blue, Green, Brown
var _pathToBlank="C:/Users/Public/Documents/ManualBGs/"+manualtemplate;
var doc;
var buildIt=false;
if ( buildIt ) {
    console.println("Creating new " + manualfilename + " file.");
    doc = \appopenDoc({cPath: _pathToBlank, bHidden: true});
    var _path=this.path;
    var pos=_path.lastIndexOf("/");
    _path=_path.substring(0,pos)+"/"+manualfilename;
    \docSaveAs\newDO ({ cPath: _path });
    doc.closeDoc();
    doc = \appopenDoc({cPath: manualfilename, oDoc:this, bHidden: true});
    f=doc.getField("ManualTitle");
    f.value=titleOfManual;
    doc.flattenPages();
    \docSaveAs\newDO({ cPath: manualfilename });
    doc.closeDoc();
} else {
    console.println("Using the current "+manualfilename+" file.");
}
var _path=this.path;
var pos=_path.lastIndexOf("/");
_path=_path.substring(0,pos)+"/"+manualfilename;
\addWatermarkFromFile({
    bOnTop:false,
    bOnPrint:false,
    cDIPath:_path
});
\executeSave();
\end{docassembly}

\begin{document}


\begingroup

\linewidth=\fullscreenwidth\relax
%\advance\linewidth\oddsidemargin
%\setlength{\oddsidemargin}{0pt}
\maketitle

\endgroup

\changelinkcolorto{black}

\tableofcontents

\changelinkcolorto{webgreen}


\section{Introduction}

The work on this package was inspired by one of my son's worksheets
in pre-algebra. The worksheet consisted of a series of simple
algebraic expressions the student was to simplify. The simplified
form was listed somewhere in the answers column.  The answer had a
letter associated with it which the student then placed in a puzzle.
Upon completion of the worksheet, the puzzle (message) is completely
filled in; if the message makes sense, the student can determine
that he/she did the worksheet correctly.

I set out to duplicate this worksheet for electronic media, but
also to have an option for paper as well.
\begin{figure}[htb]\centering\fboxsep0pt
\fbox{\includegraphics[width=.45\linewidth]{graphics/normal}}\hfil
\fbox{\includegraphics[width=.45\linewidth]{graphics/normal-partial}}
\caption{\texttt{dps\_demo.pdf}: Initial layout (left) and partially worked (right)}
\end{figure}
\section{Requirements}

The following packages are required for \textsf{dps} beyond that of
the standard {\LaTeX} distribution:
\begin{itemize}
    \item \pkg{AeB}:\footnote{\url{https://ctan.org/pkg/acrotex}} The
        \pkg{web} and \pkg{eforms} components of the
        Acro\negthinspace\TeX{} eDucation Bundle are used.
    \item
        \texttt{\href{https://ctan.org/pkg/random}{random.tex}}:\footnote{\url{https://ctan.org/pkg/random}}
        A \TeX/\LaTeX{} macro file to generate random numbers, the package
        was written by Donald Arseneau.
    \item If the \opt{usebtnappr} option is taken, the \pkg{icon-appr}
        package, dated 2020/06/05 or later, is
        required.\footnote{\url{https://ctan.org/pkg/icon-appr}}
\end{itemize}
The package should work for users of \app{dvips\,->\,distiller}, \app{pdflatex},
\app{lualatex}, and \app{xelatex}. A document author who owns the
\app{Acrobat} application and plans to use either the \opt{usebtnappr} or
\opt{uselayers} option should have the \pkg{aeb\_pro} package (completely) installed.


\section{Comments on the demo files}\label{distribution}

The distribution comes with the following demonstration files.

\begin{itemize}
    \item \texttt{examples/basic/dps\_demo.tex}\FmtMP{\texttt{basic} folder}: A demo file that you can use to
        try the package with different options. Build the source file for
        screen or for paper. Two column format for questions, and half the
        answers on the left, and half on the right.

        This demo file has more answers than questions.

    \item \texttt{examples/basic/dps\_d1.tex}\\ The original file constructing during
    the development of this package. This puzzle has the famous
    u-umlaut. Can be compiled with the
    \opt{forpaper}/\allowbreak\opt{forcolorpaper} option of web. The layout
    is designated as ``design 1'' (d1): questions in center in two column format
    and answer on the left and right.

    This demo file illustrates the special name of \texttt{cr}, as well as \texttt{space} and \texttt{punc}.

    \item \texttt{examples/basic/dps\_d1\_p.tex}\\Same as \texttt{dps\_d1.tex}, but set up
        as a paper document (using the \opt{forcolorpaper} option). The font size is
        enlarged to make it easier for filling the puzzle out using a
        pencil.

    \item \texttt{examples/basic/dps\_d2.tex}\\ Same puzzle/questions as
        \texttt{dps\_d1.tex}, but uses the layout designated as
        ``design~2'' (d2): Questions and answer in column format on left
        and right, puzzle in the center. This leaves a lot of white space
        in the middle, perhaps for a graphic. This example also manually
        inserts the answer key command \cs{AnswerKey}, which manifests
        itself when the \opt{showletters} option is taken.

    \item \texttt{examples/basic/dps\_d3.tex}\\ Same puzzle/questions as
        \texttt{dps\_d1.tex}, but uses the layout designated as
        ``design~3'' (d3): Puzzle and questions vertically aligned
        (questions in two column format), answer in single column. This
        file uses many of the formatting commands mentioned in the
        documentation.
    \item \texttt{examples/basic/dps\_signin.tex}\\ Demonstrates how to require
    the player to enter his/her name, useful when puzzle is to be handed
    in for extra credit.
    \item \texttt{examples/basic/dps\_demo.tex}\\ Indicates how to change the appearance
    of some of the form elements of the puzzle.
    \item \texttt{examples/basic/stat\_match1.tex}\\
      An example of a puzzle with extended length questions. Demo can be
      used as a paper assignment or a digital assignment.
    \item\texttt{examples/basic/stat\_match1-print.tex}\\
      A layout with long questions designed for paper.
\end{itemize}
There are several ``advanced examples'' that demonstrate two methods
of posing \emph{extended length questions}; these two methods correspond to the two
options \opt{usebtnappr} and \opt{uselayers}.
%use the full power of
%\app{Adobe Acrobat} and its \app{Adobe Distiller} companion. The four
%examples are found in the \texttt{examples/aeb\_pro} folder. They are compiled
%with a \app{dvips \texttt{->} distiller} workflow.
\begin{itemize}
\item \texttt{examples/advanced/usebtnappr/basic/stat\_match1.tex}\FmtMP{\texttt{usebtnappr} folder}\\
    Uses the \opt{usebtnappr} to create icon pushbutton appearances of the
    extended length questions. This example works for all common workflows:
    \app{pdflatex}, \app{lualatex}, \app{xelatex}, and
    \app{dvips\,->\,distiller}.

\item \texttt{examples/advanced/usebtnappr/sideshow/first\_date.tex}\\
    A file developed from by cyber buddy, who I've never met,
    to offer advice for going on a date \textsl{mit eine
    Fr\"{a}lein}. This example works for all common workflows:
    \app{pdflatex}, \app{lualatex}, \app{xelatex}, and
    \app{dvips\,->\,distiller}. Uses a graphical sideshow.

\item \texttt{examples/advanced/uselayers/basic/stat\_math1-tb.tex}\FmtMP{\texttt{uselayers} folder}\\
    Uses the \opt{uselayers} option and the \pkg{textpos} package.

    An extra credit assignment given to my statistics class in 2006. The
    questions are placed in separate layers and appear when a question
    checkbox is selected; this allows for more wordy questions without
    taking up a lot of space in the question part of the puzzle.

\item \texttt{examples/advanced/uselayers/basic/stat\_math1-ep.tex}\\
    Same as puzzle as \texttt{stat\_math1-tb.tex} but uses the
    \pkg{eso-pic} package.

\item \texttt{examples/advanced/uselayers/basic/stat\_math1\_g.tex}\\ Same as a
    \texttt{stat\_math1-tb.tex}, but more graphical. If memory serves, this
    version was developed my good cyber friend J\"{u}rgen.

\item \texttt{examples/advanced/uselayers/sideshow/first\_data.tex}\\ A file developed
    from by cyber buddy, who I've never met, to offer advice for going on a
    date \textsl{mit eine Fr\"{a}lein}. Uses a graphical sideshow.

\item \texttt{examples/advanced/uselayers/sideshow/first\_data\_g.tex}\\
    More graphical version of \texttt{first\_data.tex}, but with different
    design and graphical sideshow.
\end{itemize}

\section{Package Options}\label{packopts}

Here we list the options of package \textsf{dps}.

\begin{description}
\item[\normalfont{\opt{nonrandomized}}:] The default behavior is to
    randomize the questions and to randomize the answers. With this
    option, the questions and answer are listed in the order that they
    appear in the \env{Composing} environment. This makes it easy for
    the document author to quickly solve the puzzle and to see if the
    check marks and letters appear as they should.

\item[\normalfont{\opt{!nonrandomized}}:] (Convenience option) Cancels the effects of
    \opt{nonrandomized}; in this case, the answers are randomized (the
    default).

\item[\normalfont{\opt{viewmode}}:] Useful when using a dvi previewer or a
    PDF previewer. Here you can see the placement of the puzzle (with the
    letters to the puzzle filled in) and the boxes were the checkboxes go.
    Useful in the designing the layout of your game phase.

%\end{description}
\begin{figure}[htb]\centering\fboxsep0pt
\fbox{\includegraphics[width=.67\linewidth]{graphics/view-mode}}
\caption{Compiled with \opt{viewmode} option}
\end{figure}
%\begin{description}

\item[\normalfont{\opt{!viewmode}}:] (Convenience option) Cancels the effects of
\opt{viewmode}, this same as not specifying \opt{viewmode}.

\item[\normalfont{\opt{showletters}}:] Sets up a visual relation
    between the questions, the answers and the puzzle elements.  Useful
    in design phase, can be used with \texttt{viewmode}. Refer to \hyperref[fig:shwltrs]{Figure~\ref*{fig:shwltrs}}.

    When \opt{showletters}, the command \cs{AnswerKey} is populated with
    the answer key to the puzzle. \cs{AnswerKey}\FmtMP{\cs{AnswerKey}} is automatically inserted
    into the running footer when the \opt{showanswerkey} option is taken;
    otherwise, it can be manually inserted into the document by expanding
    \cs{AnswerKey} in a location somewhere \emph{after} the building of the
    puzzle, the questions and the answers.

\begin{figure}[htb]\centering\fboxsep0pt
\fbox{\includegraphics[width=.67\linewidth]{graphics/show-letters}}
\caption{Compiled with \opt{showletters} option}\label{fig:shwltrs}
\end{figure}

\item[\normalfont{\opt{!showletters}}:] (Convenience option) Cancels the effects of
\opt{showletters}; in this case, the letters are not shown (the default).

\item[\normalfont{\opt{showanswerkey}}:] Shows the answer key in the
    footer of the document. If the \textsf{graphicx} package is loaded,
    the answer key is rotated $180^\circ$. This option is meant to be
    used when the \texttt{forpaper} option is taken in the \textsf{web}
    package. Refer to \hyperref[fig:ansky]{Figure~\ref*{fig:ansky}}.

\begin{figure}[htb]\centering\fboxsep0pt
\fbox{\includegraphics[width=.67\linewidth]{graphics/answer-key}}
\caption{Compiled with \opt{showanswerkey} option}\label{fig:ansky}
\end{figure}


    The positioning of the running footer may be adjust using the
    convenience command
    \cs{setdpsfootskip\darg{\ameta{skip}}}\FmtMP{\cs{setdpsfootskip}}.
    When there is  no paper option for \pkg{web} (neither \opt{forpaper} or
    \opt{forcolorpaper}), \ameta{skip} is the distance up from the bottom
    edge of the screen page. The default is
    \cs{setdpsfootskip\darg{.25in}}. When there is a paper option, this
    command does nothing; location of the running footer is determined by
    the {\LaTeX} \cs{footskip} register.

\item[\normalfont{\opt{!showanswerkey}}:] (Convenience option) Turns off
    the \opt{showanswerkey} option.

\item[\normalfont{\opt{savedata}}:]\label{item:savedata} Saves three pieces of information to
    the local hard drive : (1) the seed value used by the \texttt{random}
    package to randomize the questions and answers; (2) the value of the
    last number generated; and (3) the answer key. This information is
    saved to the file \cs{jobname\_data.sav}.
    \begin{itemize}
    \item The seed value can be used to reproduce the exact
        randomization at a later time. Open the file
        \cs{jobname\_data.sav} in your editor, it might look like this:
\begin{Verbatim}[fontsize=\small]
\randomi=482053344 % Initial seed:
\dpsLastSeed{271256117}
% Answer Key: 1--e; 2--s; 3--i; 4--u; 5--l; 6--d; 7--a; 8--z; 9--p;
\end{Verbatim}
    Copy the number indicated as the \textsf{Initial seed} and paste it
    into the argument \cs{useRandomSeed\darg{482053344}}. Place this
    command in the preamble, below the \verb!\usepackage{dps}! line.
    This should reproduce the same randomization.

    \item The second line is use when the \cs{useLastSeed} command
        appears in the preamble. This is the last number generated by the
        current compile.

    \item The third line in the above verbatim listing is the answer key,
        for the randomization initiated by the seed on the first line.
        You can copy and paste this second line into the {\TeX/\LaTeX}
        document and publish the solution in a separate document, at a
        later time. This is useful when publishing for paper.
    \end{itemize}
\item[\normalfont{!\opt{savedata}}:] (Convenience option) Turns
    off the \opt{savedata} option.


\item[\normalfont\opt{usebtnappr}] Use this option to provide support for
    longer questions without taking up anymore space on the digital page.
    The option uses icon appearances of a push button.
    Refer to Section~\ref{s:usebtnappr}.


\item[\normalfont\opt{uselayers}] Use this option to provide support for
    longer questions without taking up anymore space on the digital page.
    The option places the long question in separate layers (OCGs). Refer to
    Section~\ref{s:uselayers}.

%\item[\normalfont\opt{wrtcontent}] An option that is only effective with the
%\opt{usebtnappr} option. Refer to Section~\ref{s:usebtnappr} for more information.
%\item[\normalfont\opt{!wrtcontent}] (Convenience option) Turns off the
%\opt{usebtnappr} option.

\item[\normalfont{\opt{lang=english|german|custom}}:] Currently, there
    are only two language options. The value of \opt{custom} allows you to
    create your own language strings.

    To use the \texttt{lang=custom}
    option, you must create the file \texttt{dps\_str\_cus.def}. Do this by
    taking the file \texttt{dps\_str\_us.def} (or \texttt{dps\_str\_de.def},
    if you prefer), copying the file and changing its name to
    \texttt{dps\_str\_cus.def}. Open \texttt{dps\_str\_cus.def} in your
    favorite editor and change the text strings to a language of your
    choice, or change the strings that tickle your funny bone more than the
    ones provided.\footnote{Or tickle your funny bone \emph{less}, if you
    are a crusty one.} To represent accented characters, use the octal
    encoding defined in \texttt{pd1enc.def}, as distributed with
    the \pkg{hyperref} bundle. The u-umlaut, for example, should appear
    in the file as \verb!\string\374! and not as \verb!\"{u}!.

    \textbf{\textcolor{red}{Change notification:}} If you have already written
    your own \texttt{dps\_str\_cus.def} for an earlier version of \pkg{dps};
    there is a change your should attend to: In the definition of
    \cs{regretPleased} change \texttt{(\,nMissed\,>\,n\,)} to read \texttt{(\,nMissed\,>\,nPassing\,)}.

\newtopic\noindent\makebox[0pt][r]{\large\color{red}$\blacktriangleright$\enspace}\ignorespaces
For your convenience, the \Nameref{appendix} contains a
listing of the octal codes for accented letters.
Here is an important option of \textsf{web} package.

\item[\normalfont{\opt{forpaper/forcolorpaper}}:] These are options
of the \textsf{web} package, not the \textsf{dps} package. Then this
option is taken, the puzzle created is meant for paper publication.
No \app{Acrobat} forms will be created, the \texttt{showletters} option of
\textsf{dps} is automatically taken. If you originally designed the
game for the screen, you may have to rework the sizing and design of
the game to fit everything into the constraint of a piece of paper.
Landscape is an option to think about, depending on your design and
the number of questions and answers.
\end{description}


\section{Designing your Puzzle}

There are several test files that you can use as basis of
constructing your own puzzle.\footnote{My apologies, this game is
more properly described as a matching game with message, but my
friend insisted that I call it a puzzle so that the package could be
named Das Puzzle Spiel, or dps, which are my initials. \dps} The
files themselves illustrate adequately the structure of the puzzle,
but a few remarks are in order.

\subsection{\texorpdfstring{\protect\cs{DeclarePuzzle}}{\textbackslash DeclarePuzzle}}\label{s:DP}

The first step is to have a message, either funny, serious, or
whatever. For the purpose of illustration, suppose your message is
``Hello, J\"{u}rgen!'' This message has all the elements that need
to be discussed, letters, punctuation, spaces, capitalized letters,
and accented letters.

\newtopic\noindent{\normalfont\cs{DeclarePuzzle\darg{\ameta{puzzle-args}}}:} The
\ameta{puzzle-args} consists of a series of \textit{paired parameters}:\setlength{\leftmargini}{\parindent}
\begin{quote}\ttfamily
   \string\nPuzzleCols\{\ameta{nCols}\}\quad\% \textsf{optional}\\
   \string\DeclarePuzzle\{\%\\
     \null\qquad\darg{\ameta{letter\SUB1}}\darg{\ameta{name\SUB1}}\\
     \null\qquad\darg{\ameta{letter\SUB2}}\darg{\ameta{name\SUB2}}\\
     \null\qquad\dots\\
     \null\qquad\dots\\
     \null\qquad\darg{\ameta{letter\SUB{n}}}\darg{\ameta{name\SUB{n}}}\\
   \}
\end{quote}
\cs{nPuzzleCols\darg{\ameta{nCols}}} is a convenience command for setting
the number of columns in the puzzle; \ameta{nCols} is passed to
\cs{insertPuzzle\darg{\ameta{nCols}}}, which appears in the body of the document.

\newtopic\textbf{Parameter description for \cs{DeclarePuzzle}.}
 The argument \ameta{letter} represents a letter in the puzzle;
 \ameta{letter} plays two roles: (1) it is used to typeset the letters into
 the document when certain options, such as \opt{viewmode}, are used; (2) it
 is used as the default value of a text fields that is created (when the
 puzzle is built to be interactive). This creates a problem for special
 characters, such as \"{u}; it is a typeset letter, and is \verb~\string\374~
 when placed into a text field (\cs{341}) is the (octal) PDFDocEncoding of
 u-umlaut. The way around this conundrum is to use \cs{texorpdfstring}: use
 \ameta{letter} to be,
 \begin{Verbatim}[xleftmargin=\amtIndent,commandchars={!()}]
 \texorpdfstring{\protect\"{u}}{\ifxetex! !"(u)\else\string\374\fi}
 \end{Verbatim}
 Notice the use of \cs{protect} to protect this moving argument. Also
 included above is a special case for \app{xelatex}, which does not support
 the octal notation; in this case, using the capabilities of your editor,
 simply type in an u-umlaut. The above complex expression is normally not
 needed. Here, we try to provide guidance for all PDF creators. If you stick
 to one, the \cs{ifxetex} conditional is not needed.

 The second argument pair is \ameta{name}, this is a unique name that is used
 in the construction of the underlying text field name: the field name
 becomes \texttt{puzzle.\ameta{name}}. As a result, \ameta{name} needs to be
 a JavaScript identifier (or, basically consist of letters and numbers). In
 the case of special characters such as our umlaut problem, we can assign a
 name like so:
\begin{Verbatim}[xleftmargin=\amtIndent,commandchars={!()},
  codes={\def\OR{\llap{\rlap{\textsf{or}}\kern15pt}}}]
\texorpdfstring{\protect\"{u}}{\ifxetex! !"(u)\else\string\374\fi}{uml}
!OR
\tops{\protect\"{u}}{\ifxetex! !"(u)\else\string\374\fi}{uml}
\end{Verbatim}
where \cs{tops}\FmtMP{\cs{tops}} is an alias for \cs{texorpdfstring}. This
argument pair is seen several times in the demonstration files. There are
three special names, these are \texttt{space}, \texttt{punc}, and
\texttt{cr}; as a argument pair, these should appear as follows:
\verb!{}{space}!\FmtMP{\texttt{space}}, \verb~{,}{punc}~\FmtMP{\texttt{punc}}, and
\verb~{}{cr}~\FmtMP{\texttt{cr}}, respectively. Spaces and punctuation are not normally part of
the puzzle to be discovered by answering questions, though they could be.
The special name \texttt{cr} terminates a row.

%The argument consists of a series of \textit{paired parameters}, the first of
%each pair is a letter in the puzzle, a punctuation or a space; and the second
%in the pair is a name, which is used as a field name for the underlying form
%fields.

\newtopic\indent
In the design of the puzzle there are three sets of form fields to
manage: checkboxes for the questions, checkboxes for the answers,
and text fields for the puzzle. The checkboxes and puzzle letter are
all tied together by a common base field name, which is the second
parameter in the parameter pairs.

\newtopic\indent
In our puzzle, ``Hello, J\"{u}rgen!'', we place the \cs{DeclarePuzzle}
in the preamble, as shown in \hyperref[fig:HJ]{Figure~\ref*{fig:HJ}}.
For letters, the second parameter in the pair can simply be the same, as in
\verb!{H}{H}! and \verb!{e}{e}!.

\begin{figure}[htb]\centering
\begin{minipage}{3in}
\begin{Verbatim}[fontsize=\small,commandchars={*()}]
\DeclarePuzzle{%
   {H}{H}
   {e}{e}
   {l}{l}
   {l}{l}
   {o}{o}
   {,}{punc}
   {}{space}
   {J}{J}
   {\tops{*textbf(\protect)\"{u}}%
     {\ifxetex*space*"(u)\else*textbf(\string)\374}\fi}{uml}
   {r}{r}
   {g}{g}
   {e}{e}
   {n}{n}
   {!}{punc}
}
\end{Verbatim}
\end{minipage}
\caption{The ``Hello, J\"{u}rgen'' Puzzle}\label{fig:HJ}
\end{figure}

%For punctuation, the first argument is the punctuation mark, the
%second is a special name \texttt{punc}; consequently, you see in the
%above \verb!{,}{punc}! and \verb!{,}{punc}!. Punctuation is not part
%of the puzzle (though it could be), the macro that ultimately processes
%this list of paired parameters looks for the name \texttt{punc}, and
%handles it appropriately.

%The space character also has a special second parameter
%\texttt{space}. The space will have a place in the puzzle, but has
%no question/answer corresponding to it.

%Finally, there are the special latin-1 letters.  In the puzzle above, I have
%an u-umlaut.  The problem is that this umlaut appears in two contexts, a
%{\TeX} and PDF context; the u-umlaut will appear as typeset content by {\TeX}
%and will appear as a value of an Acrobat form field. The representation is
%different in each context. In this case, use the \textsf{hyperref} command
%\cs{texorpdfstring} (use can use a special alias command \cs{tops} as well).
%\begin{Verbatim}[xleftmargin=\amtIndent,commandchars=!()]
%{\texorpdfstring{\"{u}}{\string\374}}{uml}
%{\tops{\"{u}}{\string\374}}{uml}
%\end{Verbatim}
%The first argument of the u-umlaut is
%\verb!\texorpdfstring{\"{u}}{\string\374}! (or
%\verb!\tops{\"{u}}{\string\374}!). When typeset, the first argument of the
%command \cs{texorpdfstring} (when the option \opt{viewmode} is active, for
%example) is used and in the PDF context, the second argument is used.

\begin{quote}
\textbf{\textcolor{red}{Rule:}} As a general rule, and this rule will be
repeated later, there should be one question for each distinct second
argument (\ameta{name}). There should be \emph{no question} corresponding to the
names \texttt{space}, \texttt{punc}, and \texttt{cr}.
\end{quote}
In the example of \hyperref[fig:HJ]{Figure~\ref*{fig:HJ}}, letters like \texttt{e} and \texttt{l} appear
more than once in the puzzle. Note that the second argument of each
of these two \texttt{e}'s is the same.  There should be only one for
this \texttt{e}, and when the question associated with \texttt{e} is
correctly answered, both \texttt{e}'s in the puzzle will appear. (If
you want a question for each \texttt{e}, then you need to name the
fields differently, \verb!{e}{e1}! and \verb!{e}{e2}!, for example.)

\subsection{Begin composing the questions and answers}\label{composing}

We create questions and answers for the puzzle within the \env{Composing}
environment. Questions are posed within the \env{cQ} environment and answers
are written within the \env{cA} environment.
\begin{Verbatim}[xleftmargin=\amtIndent,commandchars=!()]
\DeclarePuzzle{!ameta(puzzle-args)}
\begin{Composing}
\begin{cQ}{!ameta(name)}
!qquad!ameta(!sffamily(some question))
\end{cQ}
\begin{cA}[!ameta(alt-letter)]{!ameta(name)}
!qquad!ameta(!sffamily(some answer))
\end{cA}
...
\end{Composing}
\end{Verbatim}
\begin{description}
\item[\normalfont\env{Composing}:] You compose the questions and
    answers within the \env{Composing} environment. At the top of
    the environment, certain counters are initialized. All the work
    is done at the end of the environment: The number of questions
    and answers are known, at which point, the order of the
    questions and answers are randomized.

    The \env{Composing} environment follows \cs{DeclarePuzzle}.

\item[\normalfont\env{cQ} and \env{cA}:] Within the
    \texttt{Composing} environment, you compose your questions and
    answers within the \env{cQ} and \env{cA} environments,
    respectively.  Each question must be followed by its answer.
    You can have more answers than questions, but these answers
    \emph{must} be listed last.
%    (The answers will be randomized anyway.)
\end{description}
Each of these environments has one required argument, and \env{cA} has one
optional argument. The required argument, \ameta{name}, is the field name to
which this answer corresponds. (The second argument of the paired arguments
of \cs{DeclarePuzzle}.)

This optional argument, \ameta{alt-letter}, of the \texttt{cA} environment is
only relevant when the document is compiled with the \texttt{showletters}
option. The value of the argument is a letter to appear in the answers column
(and in the answer key). Normally, one of the first entries in the
\cs{DeclarePuzzle} command is used. Cases where you would want to include
this optional argument are,
\begin{quote}
\begin{enumerate}
\item[(1)] when giving an answer that does not correspond to a question.
    For example, in \texttt{dps\_demo.tex}, there are several answers that
    are distractions
\begin{Verbatim}[xleftmargin=\amtIndent]
\begin{cA}[w]{fake1}
$14x+10$
\end{cA}
\end{Verbatim}
In the puzzle this answer appears in the list of answers but does not correspond
to any question. The letter associated with this answer is `w'.

\item[(2)] the letter is capitalized, suggesting a proper name or the
beginning of a sentence, use the optional argument to list the letter in
lower case. For example,
\begin{Verbatim}[xleftmargin=\amtIndent]
\begin{cA}[h]{H}
$-2x-2$
\end{cA}
\end{Verbatim}
\end{enumerate}
\end{quote}

\paragraph*{Easily set up the \env{Composing} environment.} Once you
decide on your puzzle (this is easy), you need to set up the corresponding
environments \env{Composing}, \env{cQ}, and \env{cA}.
\begin{quote}
\textbf{\textcolor{red}{Rule:}} As a general rule, and this rule will be
repeated later, there should be one question for each distinct second
argument, that is, for each distinct \ameta{name}, with the exception of
the names \texttt{space}, \texttt{punc}, and \texttt{cr}.
\end{quote}
Because of human errors, sometimes we have questions/\allowbreak answer pairs
that correspond to a duplicate field name---perhaps this letter occurs
multiple times. No, no, that violates the red rule above. After having made
this same error several times myself, I decided to let {\TeX} do the work for
me.

\newtopic\noindent Just after the end of the
    argument of \cs{DeclarePuzzle}, prior to authoring the questions and
    answers, place the \cs{writeComposingEnv}:
\begin{Verbatim}
\writeComposingEnv
\end{Verbatim}
Remember, \cs{DeclarePuzzle} is in the preamble and the beginning of document
has not been encountered; the command writes an outline for the
puzzle---based on the arguments of \cs{DeclarePuzzle}---to the file
\cs{jobname\_comp.def} and ends the document. Once this file is created, copy
and paste its contents into your source file. It should look like this:
\begin{verbatim}
\begin{Composing}

\begin{cQ}{M}
\end{cQ}
\begin{cA}{M}
\end{cA}

\begin{cQ}{a}
\end{cQ}
\begin{cA}{a}
\end{cA}
...
\end{Composing}
\end{verbatim}
The correct number of environments should be there, with the correct
argument inserted for each environment. Cool! Now, just compose your
questions and answers.

After \cs{jobname\_comp.def} is created and copied into the puzzle document,
\emph{delete the} command \cs{writeComposingEnv}. This command is not part of
the puzzle, but a helper command. As an option, rather than pasting the
contents into your document following the \cs{DeclarePuzzle}, you can fill in
your questions and answers to your puzzle within the \cs{jobname\_comp.def},
input the file \cs{jobname\_comp.def} with \verb!\input{\jobname_comp.def}!.
\textcolor{red}{\textbf{Warning:}} If you leave your questions and answers in
the auxiliary file \cs{jobname\_comp.def} you may overwrite them by
un-commenting \cs{writeComposingEnv} and compiling your puzzle document.

\newtopic\noindent Another rule:
\begin{quote}
\textbf{\textcolor{red}{Rule:}} You must have one correct answer per
    question. Each answer is unique in the list of all answers (no two
    answers can be the same). You can have more answers than questions
    (distractions---answers ``close''  in appearance to the correct
    answer). Each of the distractions must have a unique \ameta{name}.
\end{quote}

\section{Placing the Content}

The content consists of five parts, plus whatever you wish to include
in the document:
\begin{enumerate}

\item The title and instructions.

\item The puzzle

\item The questions

\item The answers

\item The message field

\end{enumerate}
Each of these is discussed the the sections that follow.


\subsection{The title and instructions}

The title and instructions are your bailiwick, see package demo files
for suggestions.

\subsection{The puzzle}

\begin{description}
\item[\normalfont{\cs{insertPuzzle\darg{empty|\ameta{nCols}}}}:] The
    puzzle, which consists of the first of the paired arguments declared in
    the \cs{DeclarePuzzle} command, is laid out in a tabular format. The
    one required argument of \cs{insertPuzzle} is either the empty argument
    (\darg{}) or \ameta{nCols}, the number of columns per row to be used.

    If you say \cs{insertPuzzle\darg{}}, then
    \cs{nPuzzleCols\darg{\ameta{nCols}}} is expected to appear in the
    preamble. If the argument of \cs{insertPuzzle} is empty and
    \cs{nPuzzleCols} does not appear in the preamble, a warning is issued
    and \ameta{nCols} is set to 10.

    In the demo files, \cs{insertPuzzle} is enclosed in a \texttt{minipage}
    environment (and in a \cs{fbox} as well). By placing \cs{insertPuzzle} in
    this way, you can control the width of the table, and it may help fit it with
    the other components of the game.

\item[\normalfont{\cs{PuzzleAppearance}}:] Use this command to change
    the appearance of the \app{Acrobat} text field that comprise the
    interactive puzzle. The command takes one argument, this argument
    consists of one or more commands from the \textsf{eForms} package
    that change the appearance of a field. For example,
\begin{Verbatim}[commandchars=!()]
\PuzzleAppearance{\BC{1 0 0}}
\PuzzleAppearance{\BC{red}}   %!textsf( if !pkg(xcolor) is loaded)
\end{Verbatim}
changes the boundary color to red. See the \pkg{eforms}
documentation.

\item[\normalfont{\cs{rowsep\darg{\ameta{skip}}}}:] By setting \cs{rowsep}, you can adjust
    the vertical space between tabular rows. The commands takes one
    argument, the amount of vertical skip, for example,
\begin{Verbatim}
\rowsep{2ex}
\end{Verbatim}
\item[\normalfont{\cs{wdPuzzleFields\darg{\ameta{length}}}}] Sets the width of a puzzle field
  to \ameta{length}. The default is 1.6em
\item[\normalfont{\cs{htPuzzleFields\darg{\ameta{length}}}}] Sets the height of the puzzle field
to \ameta{length}. The default is 11bp.
\end{description}

\subsection{The questions}

\begin{description}
\item[\normalfont{\cs{displayRandomizedQuestions}}:] The questions are
    inserted by this command. This command must be placed in an
    \texttt{enumerate} environment. This will number the questions,
    so when, for example, the \texttt{showletters}, discussed later,
    is taken, there is a visual mapping between the questions, the
    answers and the puzzle.

If the number of questions is not great, you can list the questions
in a single column; however, in the examples that I have done, I've
determined that a two column format (using the \texttt{multicol}
package) seems to me to be the best layout for the questions.

\item[\normalfont{\cs{QuesAppearance}}:] Use this command to change the
    appearance of the \app{Acrobat} checkboxes  for the questions that
    appear in the label margin. The command takes one argument, this
    argument consists of one or more commands from the
    \textsf{eForms} package that change the appearance of a field.
    For example,
\begin{Verbatim}[commandchars=!()]
\QuesAppearance{\BC{gray}} %!textsf( assumes xcolor pkg or just !cs(BC{.5 .5 .5}))
\end{Verbatim}
changes the boundary color to a gray color. See the \pkg{eforms} documentation.

\item[\normalfont{\cs{widestFmtdQNum\darg{\ameta{str}}}}] Sets the width of the checkbox
for the questions. The argument \ameta{str} is a string whose width determines the width
of the checkboxes. The default is \cs{widestFmtdQNum\darg{00.}}. If the numbers are typeset
in bold, then \cs{widestFmtdQNum\darg{\string\textbf\darg{00.}}} is a little wider to
account for the bold font.
\item[\normalfont{\cs{htOfQ\darg{\ameta{length}}}}] Sets the height of the checkbox for the question.
The default is \cs{htOfQ\darg{13bp}}.
\end{description}

\subsection{The answers}

\begin{description}
\item[\normalfont{\cs{displayRandomizedAnswers}}:] As with the
    questions, the answer are displayed in by a similar command. Again,
    this command should be in a list environment, preferably the
    \texttt{itemize}. The list label is suppressed by placing
    \cs{item[]} before each answer.  When the \texttt{showletters}
    option is taken, the letter this answer corresponds to will be the
    label.

One of the design layouts in the demo files has the questions in a
two column format in the center with two columns of answers, half
the answers to the left of the questions, and the other half to the
right.  The two commands
\begin{Verbatim}
\displayRandomizedAnswersLeftPanel
\displayRandomizedAnswersRightPanel
\end{Verbatim}
are used for that purpose.\footnote{{\TeX} doesn't know his left
from his right, so you can actually place the left panel listing on
the right. {\TeX} will not object!} As with
\cs{displayRandomizedAnswers}, each of these commands should be in
an \texttt{itemize} environment.

\item[\normalfont{\cs{AnsAppearance}}:] Use this command to change the
    appearance of the \app{Acrobat} checkboxes for the answers that appear in
    the label margin. The command takes one argument, this argument
    consists of one or more commands from the \textsf{eForms} package
    that change the appearance of a field. For example,
\begin{Verbatim}[commandchars=!()]
\AnsAppearance{\BC{gray}} %!textsf( assumes xcolor pkg, or !cs(BC{.5 .5 .5}))
\end{Verbatim}
changes the boundary color to a gray color. See the \textsf{eforms} documentation.

\item[\normalfont{\cs{ltrFmtA\darg{\ameta{\string\cmd\darg{\#1}}}}}] Formats the letters in
the list of answers when the option \opt{showletters} is active. In the argument, \texttt{\#1}
references the current letter to be typeset. For example, \verb~\ltrFmtA{\textbf{\textcolor{blue}{#1}}}~
typesets the letters in bold and blue.

\item[\normalfont{\cs{widestFmtdALtr\darg{\ameta{str}}}}] Sets the width of the checkbox
for the answers. The argument \ameta{str} is a string whose width determines the width
of the checkboxes. The default is \cs{widestFmtdALtr\darg{w}}.

\item[\normalfont{\cs{htOfA\darg{\ameta{length}}}}] Sets the height of the checkbox for the answer.
The default is \cs{htOfA\darg{13bp}}.
\end{description}

\subsection{The message field}

\begin{description}
\item[\normalfont{\cs{placeMessageField[\ameta{opts}]\darg{\ameta{wd}}\darg{\ameta{ht}}}}:]
    This \app{Acrobat} text field is used to write messages to the user.  If
    the user tries to choose an answer before selecting an answer, s/he
    gets the message
\[
    \texttt{"You must choose a question to answer before you answer!"}
\]
For the interactive version of this game\footnote{Does that mean
there is also a non-interactive version of this game, sir? Yes, yes
there is. \dps}, there is a language option for \pkg{dps} to
change the messages from English, the default, to German, for
example.

The parameters are \ameta{wd} (the width of the text field),
\ameta{ht} (the height of the text field and \ameta{opt} (the
optional argument for changing the appearance of the field, as
described in the documentation of the \pkg{eforms} package.
\end{description}
The message field is automatically removed when the document is
compiled in the \opt{forpaper} option of the \pkg{web}
package.

\subsection{Auxiliary files}

Developing a puzzle file creates a number of auxiliary files, in addition to
the usual ones of a typical {\LaTeX} file. The \pkg{dps} package creates a
large number of CUT files and a few SAV files. If you create a sideshow (page~\pageref{s:sdshw}),
additional PDF files are created as well. As a general rule, do not delete these files
until you are finished building your puzzle and you have verified that is working correctly.

\section{Commands for controlling randomization}

When randomizing is to be used (that is, the option \opt{!nonrandomized} is in effect, or
the \opt{nonrandomized} option is not specified), there are several commands to
control the initial seed of the random number generator.
\begin{Verbatim}[commandchars=!()]
\useRandomSeed{!ameta(seed)}  %!textsf( eg, )\useRandomSeed{187968637}
\inputRandomSeed
\useLastSeed
\end{Verbatim}
All commands are placed in the preamble. Only one of the three should appear during any compile.
\begin{description}
\item[\normalfont\cs{useRandomSeed\darg{\ameta{seed}}}]\hskip-\labelsep\relax\space initializes the
    random number generator to \ameta{seed}.
\item[\normalfont\cs{inputRandomSeed}]\hskip-\labelsep\relax\space inputs
    the seed value saved by the \opt{savedata} option; as a result, the
    same randomization sequence is obtained each time the file is
    compiled.\footnote{This assumes that randomized items are neither added
    or deleted.} In effect, this command freezes the randomization sequence
    as long as the file \cs{jobname\_data.sav} has not been deleted. To
    continue to use this random sequence after \cs{jobname\_data.sav} is
    deleted, save the value of the initial seed first, as described in the
    description of the \opt{savedata} option on
    \hyperref[item:savedata]{page~\pageref*{item:savedata}}.

\item[\normalfont\cs{useLastSeed}]\hskip-\labelsep\relax\space initializes
    the random number generator with the last random number generated in
    the \emph{previous} compile. In this way, you get a new randomization
    each time you compile the file.
\end{description}
The last two commands assume \cs{jobname\_data.sav} exists; otherwise they take an initial seed based on
the current date and time.

\section{Printing and clearing the puzzle}

The \pkg{dps} package provides the following two commands, which produce push buttons:
\begin{Verbatim}[commandchars=!()]
\printDPS[!ameta(form-options)]{!ameta(wd)}{!ameta(ht)} %!textsf( prints the document)
\resetDPS[!ameta(form-options)]{!ameta(wd)}{!ameta(ht)} %!textsf( clears the document)
\end{Verbatim}
The \ameta{wd} and \ameta{ht} are the width and height of the push button;
the optional argument \ameta{form-options} are key-value pairs to change the
appearance of the buttons. Familiarity with the \pkg{eforms} package is
needed.

The \pkg{dps} provides the command \cs{dpsResetHook} to add JavaScript lines to the
action of \cs{resetDPS}. The command takes one argument to pass the JavaScript lines
to \cs{resetDPS}; for example, \verb|\dpsResetHook{dpsHideFld("btnEmoji");}|\FmtMP{\cs{dpsHideFld}} appears
in several of the example file. This hides the \texttt{btnEmoji} image.

\section{Methods of handling long questions}

All the basic examples (those in the folder \texttt{examples/basic}) feature
\emph{short questions} that conveniently fit into the space provided.\footnote{With the one exception
of \texttt{stat\_match1-print.tex} in the \texttt{basic} folder.} More
complex questions require longer questions.  The problem, then, is to
develop a method of posing long questions, without disturbing the puzzle design,
for an \textit{interactive puzzle}.

In this section, we detail two methods of posing long questions, while saving
space on a one page puzzle document; these are (1) place the long questions
into an appearance of an push button (yes, you can do that); and (2) place
the long questions into a layer (optional content group, OCG). All {\LaTeX}
workflows are supported by method (1), while method~(2) requires the
\app{dvips\,->\,distiller} workflow.

Both methods are designed for an interactive puzzle\FmtMP{interactive puzzles}, not a paper puzzle.

\subsection{The \tops{\protect\opt{usebtnappr}}{usebtnappr} option}\label{s:usebtnappr}

\textbf{Demo file.} The demonstration file for this subsection is \texttt{stat\_match1.tex},
found in the \texttt{examples/advanced/icon-appr} folder.

\newtopic\noindent This solution requires two files: (1) the puzzle file; and the (2) icon file.

\subsubsection{The puzzle file}

Within the puzzle file, the \opt{usebtnappr} option is specified. The
\cs{usepackage} command for \pkg{dps} below specifies the minimal options:
\begin{Verbatim}[xleftmargin=\amtIndent,commandchars=!()]
\usepackage[%
  !textbf(usebtnappr),
  nonrandomized,
  savedata
]{dps}
\end{Verbatim}
%More on the \opt{wrtcontent} option in the section titled \mlNameref{ss:workflow}.

\paragraph*{In the preamble.} We place two environments;
\env{embedding}, to embed the icons; and \env{setContent}, to pose the
long questions.

\subparagraph*{Embedding the icons.} The \opt{usebtnappr} brings in the
\FmtMP{\pkg{icon-appr} package}\pkg{icon-appr} package,\footnote{\url{https://ctan.org/pkg/icon-appr}} which defines the
\env{embedding} environment and the \cs{embedIcon} command. The \pkg{dps}
package defines \cs{dpsEmbedIcons}. This command embeds all the dynamically
created PDF graphics that comprise the long questions. \cs{dpsEmbedIcons} is
followed by other optional \cs{embedIcon} commands.
\begin{Verbatim}[commandchars=!()]
\begin{embedding}
\dpsEmbedIcons
!ameta(other-embeds)
\end{embedding}
\end{Verbatim}
Below is the example from the demo file.
\begin{Verbatim}[xleftmargin=\amtIndent]
\begin{embedding}
\dpsEmbedIcons
\embedIcon[name=Emoji,placement=btnEmoji]{MyEmoji.pdf}
\end{embedding}
\end{Verbatim}
The second line is the optional one; here, we explicitly embed, using the
syntax of \pkg{icon-appr}, an additional graphic for use by the puzzle.

\subparagraph*{Special note for \app{xelatex} users.} The \env{embedding}
environment is placed in the preamble but the indirect references (a PDF
term) of the embedded graphics are not calculated until the first page is
shipped out (a {\LaTeX} term); therefore, the \emph{puzzle should be on the second
page} when the \emph{\pkg{textpos} package is used}\FmtMP{\pkg{textpos} pkg} to position the graphics. This
issue does not arise when using the \pkg{eso-pic} package.

\subparagraph*{Posing long questions.} Following the \env{embedding} environment comes the usual content for
designing the puzzle: the \cs{DeclarePuzzle} data structure, followed by the
\env{Composing} environment. It is within the \env{cQ} environment of the
\env{Composing} environment that there is a change in content.

The\FmtMP{posing long question} \opt{usebtnappr} option
defines the \env{setContent} environment, which is placed in the \env{cQ}
environment.
\begin{Verbatim}[commandchars=!()]
\begin{cQ}{!ameta(name)}
  !ameta(question-prompt)
\begin{setContent}{!ameta(name)}
  !ameta(long-question)
\end{setContent}
\end{cQ}
\end{Verbatim}
The \ameta{name} argument of \env{setContent} is same as the argument of the
enclosing \env{cQ} environment. \env{setContent} is a verbatim write
environment; \cs{begin\darg{setContent}} can follow the end of the
\emph{question-prompt} (one or more words that suggest what the question is
about), but the \cs{end\darg{setContent}} must be in the left-most margin, as
shown above. We illustrate first with an example taken from the demo file:
\begin{Verbatim}[xleftmargin=\amtIndent,commandchars=!()]
\begin{Composing}
\begin{cQ}{R}
Branches of Statistics!textbf(\begin{setContent}{R})
The two branches of statistics are descriptive and
\underbar{\hspace{.5in}}.
!textbf(\end{setContent})
\end{cQ}
\begin{cA}[r]{R}
inferential
\end{cA}
...
\end{Composing}
\end{Verbatim}
A long question is not required, in the demo file most have long questions and
some do not.

The \env{setContent} environment writes its content verbatim to a CUT file
(named \cs{jobname-sc(\ameta{num}).cut}). For example, the CUT file for the
first question of the demo file reads,
\begin{Verbatim}[xleftmargin=\amtIndent]
\textbf{Problem 1}\newline
The two branches of statistics
are descriptive and \underbar{\hspace{.5in}}.
\end{Verbatim}
The first line seen above can be modified using the following two commands.
\begin{Verbatim}
\newcommand{\quesNumTxt}[1]{\protect\textbf{Problem #1}}
\newcommand{\quesNumTxTPost}{\protect\newline}
\end{Verbatim}
These may be redefined.

\paragraph*{On the puzzle page.} Aside from the layout of the puzzle, questions,
and answers commands, above all these, place the following commands:
\begin{Verbatim}[commandchars=!()]
\placeQuesIcon{!ameta(placement-cmds)}
\placeOtherIcon{!ameta(placement-cmds)}
\end{Verbatim}
The nature of the \ameta{placement-cmds} depends on the ``placement'' package used. The demo files
use either the \FmtMP{\pkg{eso-pic}, \pkg{textpos} packages}\pkg{eso-pic} or
\pkg{textpos} package. The \ameta{placement-cmds} place the text field
\cs{dpsQuesIcon\darg{\#1}\darg{\ameta{wd}\darg{\ameta{ht}}}}.
The following is taken from the demo file, it uses the \pkg{eso-pic} package.
\begin{Verbatim}[xleftmargin=\amtIndent,fontsize=\small,commandchars={@~^}]
\placeQuesIcon{\AddToShipoutPictureFG*{\AtTextCenter{\put(-72,0)
  {@textbf~\dpsQuesIcon{#1}{2.25in}{9\baselineskip}^}}}}
\end{Verbatim}
If other icons were embedded in the \env{embedding} environment, those icons can be placed
using the \cs{placeOtherIcon} command.
\begin{Verbatim}[xleftmargin=\amtIndent,fontsize=\small,commandchars={@~^}]
\placeOtherIcon{\AddToShipoutPictureFG*{\AtTextCenter{\put(-72,0)
  {@textbf~\dpsOtherIcon[\I{\csOf{Emoji}}]{btnEmoji}{2.25in}{9\baselineskip}^}}}}
\end{Verbatim}

\subsubsection{The icons file}\label{ss:iconfile}

The \opt{usebtnappr} option requires the use of a second file, named
\texttt{icons.tex}.\footnote{The name of the file is hard-wired into the
package and cannot be changed at this time.} The \texttt{icons.tex} file in
the \texttt{examples/advanced/icon-appr}. The icons file (\texttt{icons.tex})
is a very short file and is placed in the same folder as the puzzle file.
\begin{Verbatim}[commandchars={*@^}]
\documentclass{article}
\usepackage[!useacrobat]{icon-doc}
\margins{3pt}{3pt}{3pt}{3pt}        %*textsf@ left,right,top,bottom (web command)^
\screensize{9\baselineskip}{2.25in} %*textsf@ height,width (web command)^
\begin{document}
\small
\createRequiredIcons{12}{stat_match1}
\end{document}
\end{Verbatim}
The icons file is placed in the came folder as the puzzle file. The \pkg{dps}
distribution has the \FmtMP{\pkg{icon-doc} package}\pkg{icon-doc} package, a
short package designed for the icons file. The package has two options
\opt{useacrobat} and \opt{!useacrobat}, the default is \opt{!useacrobat}.
Document authors who use \app{pdflatex}, \app{lualatex}, or
\app{dvips\,->\,distiller} need not bother with this option; the option is
designed for \app{xelatex} users.

Use the \cs{margins} command to adjust the boundary margins of icons file.
Use the \cs{screensize} to adjust the height of the icons file and the width
of the icons file. The dimensions shown above are the ones used by the demo
file.

The \cs{createRequiredIcons\darg{\ameta{num-ques}}\darg{\ameta{puzzle-file}}}
command has two arguments: \ameta{num-ques} is the number of questions in the
puzzle file; \ameta{puzzle-file} is the base name of the puzzle file.
The command has two behaviors:
\begin{enumerate}
  \item \textbf{For \app{xelatex} authors that do not have \app{Acrobat}.}
      When the icon file is compiled, the result is \ameta{num-ques} PDFs,
      each PDF contains a \emph{single} long question. Since the author
      does not have \app{Acrobat}, the icons file is compiled with the
      \opt{!useacrobat} option. This case requires the
      \FmtMP{\pkg{shellesc} pkg}\pkg{shellesc} package and that your
      {\LaTeX} editor be setup to use the \texttt{shell-escape} switch of your editor.
      Refer to Figure~\ref{fig:shesc} to see how to do this for the \app{WinEdt} editor.
      Other editors/\TeX{} systems may support the \texttt{shell-escape} switch.
\begin{figure}[htb]\centering
  \includegraphics[width=.67\linewidth]{graphics/shell-escape}
  \caption{Setting \texttt{shell-escape} on \app{WinEdt}}\label{fig:shesc}
\end{figure}
  \item \textbf{For all other cases of workflow.} When the icons file is
      compiled, the result is a \emph{single} PDF (\texttt{icons.pdf}) with a page
      for each long question in the puzzle file, in an order determined by
      the randomization option (\opt{!nonrandomized} or
      \opt{nonrandomized}).

      For a \app{xelatex} author who has \app{Acrobat}, the
      \opt{useacrobat} needs to be used.
\end{enumerate}

\paragraph*{Additional comments for \app{xelatex} authors.}\label{para:addCmts} In the case the author
does not own \app{Acrobat}, when the icons file is compiled, each of the
files \cs{jobname-sc(\ameta{num}).cut} is \emph{wrapped in a document
template}, compiled, and saved as \texttt{icon-\ameta{num}.pdf}.
The document wrapper is determined by the contents of the
\FmtMP{\env{icondoc} env}\env{icondoc} environment. The default declaration
of \env{icon-body} is,
\begin{Verbatim}[commandchars=!()]
\begin{icondoc}
\documentclass{article}
\usepackage{web}
\margins{3pt}{3pt}{3pt}{3pt}
\screensize{9\baselineskip}{2.25in}
\begin{document}
\small
\dpsInputContent    %!textsf( required, defined in icon-doc)
\end{document}
\end{icondoc}
\end{Verbatim}
Notice that this markup is similar to the source file for \texttt{icons.tex};
they differ in two respects, however: (1) the \pkg{icon-doc} package is not
used; and (2) in the body, we use \cs{dpsInputContent} rather than
\cs{createRequiredIcons}. If you use multiple workflows, the layout of the
icons file and the declarations of the \env{icondoc} environment should be
the same. Changes to the \env{icondoc} environment are made in the icons file. For example,
\begin{Verbatim}[xleftmargin=\amtIndent,commandchars={*@^}]
%*textsf@ icons.tex^
%*textsf@ modify a design parameters^
\documentclass{article}
\usepackage[!useacrobat]{icon-doc}
\margins{4pt}{4pt}{4pt}{4pt}       %*textsf@ left,right,top,bottom (web command)^
\screensize{10\baselineskip}{3in}  %*textsf@ height,width (web command)^
%*textsf@ change *env@icondoc^ to reflect changes above^
*textbf@\begin{icondoc}^
\documentclass{article}
\usepackage{web}
\margins{4pt}{4pt}{4pt}{4pt}
\screensize{10\baselineskip}{3in}
\begin{document}
\small
\dpsInputContent    %*textsf@ required, defined in icon-doc^
\end{document}
*textbf@\end{icondoc}^
\begin{document}
\small
\createRequiredIcons{12}{stat_match1}
\end{document}
\end{Verbatim}
\textbf{Why is \app{xelatex} so special (such a problem)?} The basic problem when dealing
with \app{xelatex} is that it does not obey the \key{page} key of \cs{includegrapics}. When a graphic is embedded
in a document we say,
\begin{quote}
    \cs{embedIcon[name=Q\string\n,hyopts=\darg{page=\string\n}]\darg{icons.pdf}}
\end{quote}
where, \cs{n} is question number; in this case, \cs{n} is also the page
number where the question is located in \texttt{icons.pdf}. Since, the
\key{key} is not obeyed, we cannot bundle all the questions in a single PDF
and pull out the page we want; no, the questions must be their own PDF, the
embedding command is then,
\begin{quote}
    \cs{embedIcon[name=Q\string\n]\darg{icons-\string\n.pdf}}
\end{quote}
where \cs{n} is the question number.

\newtopic Hey, we're almost done with this option!

\subsubsection{The workflow to build the puzzle file}\label{ss:workflow}

Let's summarize the workflow to build a puzzle file that uses icon appearances
to display the long questions. We assume you have jumped through all the hoops of the previous section,
your puzzle file and icons file are ready to go.
\begin{enumerate}
    \item Compile the puzzle file, using the compiler of your choice. If
        you require a randomized listing of the questions and answers,
        either compile with \opt{!nonrandomized}, or the \opt{nonrandomize}
        option commented out or deleted. If you are randomizing, compile
        with the command \cs{inputRandomSeed} in the preamble to input that
        same initial seed back in when your later compile the puzzle file
        in step~3.\footnote{Or, you can open the SAV file and copy and
        paste the initial seed into the argument of \cs{useRandomSeed}.}
    \item Compile the icons file, using your favorite compiler. If you are
        a \app{xelatex} user, the option \opt{useacrobat} and
        \opt{!useacrobat} as appropriate. (Refer to \textbf{\mlnameref{para:addCmts}}
        on page~\pageref{para:addCmts} for more details.)
    \item Return to the puzzle file. Compile the puzzle file again to
        obtain the final version.

    \item Bring your PDF into \app{Adobe Reader DC} (or \app{Adobe Acrobat DC} if you
        have it) and save the file. After saving, test the puzzle to be sure
        the icons are displayed with the long questions. If it does not work,
        repeat steps~1--3 more carefully.
\end{enumerate}
To familiarize yourself to the procedure, build the demo file
\texttt{stat\_match1.tex} using your favorite PDF creator. Try it with
several PDF creators, just for fun. The process is straight forward once
everything is properly set up.

\subsection{The \tops{\protect\opt{uselayers}}{uselayers} option}\label{s:uselayers}

\textbf{Demo files.} There are several example files in the
\texttt{examples/advanced/ocgs} folder, but in this discussion, we'll
reference, once again, \texttt{stat\_match1} found in the \texttt{ex1}
folder. There are two versions of this file \texttt{stat\_match1-ep.tex} and
\texttt{stat\_match1-tb.tex}. The latter uses the \pkg{textpos} package, the
latter uses the \pkg{eso-pic} package. For variety, we'll reference the
\pkg{textpos} version.

\newtopic\noindent
The \opt{uselayers} option is actually a simpler approach (no icon file
needed), but there has more restrictions on the workflow. This option
requires \FmtMP{\pkg{aeb\_pro} pkg} the \pkg{aeb\_pro} package (with
correctly installed JS files \texttt{aeb.js} and \texttt{aeb\_pro.js}) and
requires a \FmtMP{\app{Acrobat} required}\app{dvips->distiller} PDF creator workflow.

\subsubsection{The puzzle file}

A minimal specification for the \pkg{aeb\_pro} and \pkg{dps} packages is
listed below:
\begin{Verbatim}[xleftmargin=\amtIndent,commandchars=!()]
\usepackage[%
  web={extended,tight},
  eforms,
  !textbf(uselayers)
]{aeb_pro}
\usepackage[!textbf(uselayers),
  nonrandomized,
  savedata
]{dps}
\end{Verbatim}

\paragraph*{Posing long questions.} The \cs{DeclarePuzzle} is as described in
\hyperref[s:DP]{Section~\ref*{s:DP}}. As in the case of \hyperref[s:usebtnappr]{Section~\ref*{s:usebtnappr}},
the \env{setContent} environment is used to pose long questions.
\begin{Verbatim}[commandchars=!()]
\begin{cQ}{!ameta(name)}
  !ameta(question-prompt)
\begin{setContent}{!ameta(name)}
  !ameta(long-question)
\end{setContent}
\end{cQ}
\end{Verbatim}
The definition of \env{setContent} differs from that of \env{setContent} for
the \opt{usebtnappr} option. A long question is not required.

The \env{setContent} environment writes its content verbatim to a CUT file
(named \cs{jobname-sc(\ameta{name}-\ameta{num}).cut}). For example, the CUT file for the
first question of the demo file reads,
\begin{Verbatim}[xleftmargin=\amtIndent]
\textbf{Problem 1}\newline
The two branches of statistics
are descriptive and \underbar{\hspace{.5in}}.
\end{Verbatim}
The first line, which is not part of the \env{setContent} content, seen above
can be modified using the following two commands.
\begin{Verbatim}
\newcommand{\quesNumTxt}[1]{\protect\textbf{Problem #1}}
\newcommand{\quesNumTxTPost}{\protect\newline}
\end{Verbatim}
These may be redefined.

\begin{comment}
\fmtOCGQues{%
  \parbox[t][9\baselineskip][t]{2.25in}{\kern0pt\small\hfuzz11pt
  \psshadowbox[framesep=0pt]{\fcolorbox{red}{cornsilk}{%
  \parbox{\linewidth}{\dpsQuesLayer{#1}\vskip3pt}}}}%
}
% These declaration go on the page where they are to appear
\placeQuesLayer{%
  \begin{textblock*}{2.25in}[0,0](2.5in+.725in,3in) %(.33\paperwidth,.42\paperheight)\hfuzz11pt
  \insertQuesLayer{#1}
  \end{textblock*}%
}

\placeOtherLayer{%
  \begin{textblock*}{2.25in}[0,0](2.5in+.725in,3in)\centering
  \xBld{owclogo}\parbox{2.25in}{\includegraphics[width=2.25in]{owc_self}}\eBld
  \end{textblock*}%
}
\end{comment}

\paragraph*{In the body of the puzzle page.} The following three commands are placed
on the same page as the puzzle is to appear. They precede the layout of the puzzle, questions,
and answers.
\begin{Verbatim}[commandchars=!()]
\fmtOCGQues{!ameta(formatting-comm)}
\placeQuesLayer{!ameta(placement-cmds)}
\placeOtherLayer{!ameta(placement-cmds)}
\end{Verbatim}
The argument of \FmtMP{\cs{fmtOCGQues}}\cs{fmtOCGQues} is a convenient way of designing how your long question
appears on the page. Within the argument, use \cs{dpsQuesLayer{\#1}} to symbolically reference
the question. For example, from the demo file,
\begin{Verbatim}[xleftmargin=\amtIndent,commandchars=!()]
\fmtOCGQues{%
  \parbox[t][9\baselineskip][t]{2.25in}{\kern0pt\small\hfuzz11pt
  \psshadowbox[framesep=0pt]{\fcolorbox{red}{cornsilk}{%
  \parbox{\linewidth}{!textbf(\dpsQuesLayer{#1})\vskip3pt}}}}
}
\end{Verbatim}
Here, we've used \cs{psshadowbox} from \pkg{pstricks-add},

The nature of the \ameta{placement-cmds} depends on the ``placement'' package used. The demo files
use either the \FmtMP{\pkg{eso-pic}, \pkg{textpos} packages}\pkg{eso-pic} or
\pkg{textpos} package. The \ameta{placement-cmds} argument places the question
\cs{insertQuesLayer\darg{\#1}}.

For the command \FmtMP{\cs{placeQuesLayer}}\cs{placeQuesLayer} places the question layer,
symbolically represented by \cs{insertQuesLayer\darg{\#1}}. For example, from the demo file,
\begin{Verbatim}[xleftmargin=\amtIndent,commandchars=!()]
\placeQuesLayer{%
  \begin{textblock*}{2.25in}[0,0](2.5in+.725in,3in)
  !textbf(\insertQuesLayer{#1})
  \end{textblock*}%
}
\end{Verbatim}
\cs{insertQuesLayer} sets the layer and inputs the formatted content.

Use \FmtMP{\hfuzz4pt\cs{placeOtherLayer}}\cs{placeOtherLayer} to place other
non-question content in a layer. For example, from the demo file,
\begin{Verbatim}[xleftmargin=\amtIndent,commandchars=!()]
\placeOtherLayer{%
  \begin{textblock*}{2.25in}[0,0](2.5in+.725in,3in)\centering
  \xBld{owclogo}\parbox{2.25in}
    {\includegraphics[width=2.25in]{owc_self}}\eBld
  \end{textblock*}%
}
\end{Verbatim}
Here, you must use the \pkg{aeb\_pro} commands to create a named layer, using
the \cs{xBld}/\allowbreak\cs{eBld} command pair. The argument of \cs{xBld} is the name
of the layer. Within the \cs{xBld}/\cs{eBld} pair, we create our content of a
graphic.

\subsection{Developing an end of game event}

There are commands and JavaScript hooks to enable a knowledgable author to
create a special end-of-game event. For example, the graphic placed by
\cs{placeOtherIcon} or \cs{placeOtherLayer}, becomes visible when the player
finishes the puzzle. All the demo files in the \texttt{advanced} folder have
end of game events. The \pkg{dps} provides the command \cs{dpsFinishedEvent}
to add JavaScript lines to the action to the end of game event. The command
takes one argument to pass the JavaScript lines to the end of game event; for
example, \verb|\dpsFinishedEvent{dpsShowFld("btnEmoji");}|\allowbreak
\FmtMP{\leavevmode\llap{\cs{dpsFinishedEvent}}} appears in several of the
example file. This shows the \texttt{btnEmoji} image when the player has
finished the puzzle.

\section{Creating a sideshow}\label{s:sdshw}

Some of the advanced examples use a sideshow; as the player progresses through
the puzzle, with each success, a new piece of a graphic is revealed. A partially worked puzzle
is shown in \hyperref[fig:sideshow]{Figure~\ref*{fig:sideshow}}.

\begin{figure}[htb]\centering\fboxsep0pt
\fbox{\includegraphics[width=.67\linewidth]{graphics/sideshow}}
\caption{Puzzle with sideshow, shown on left}\label{fig:sideshow}
\end{figure}

There are two commands, placed in the preamble, that control the behavior of
the sideshow.
\begin{Verbatim}
\randomizePicMappings
\sortPicMappings
\end{Verbatim}
The order the sideshow pictures appears can be in their natural order, or in
a randomized order. To randomize the order, insert \cs{randomizePicMappings}
in the preamble; otherwise, the pictures appear in their natural order.

Second command, \cs{sortPicMappings} implies \cs{randomizePicMappings}, but
with a twist. The pictures are placed in random positions as they appear; at
the end of the puzzle, a bubble sort is applied, and the pictures are sorted
to their natural order. Cool.

In all cases, pressing the \textsf{Clear} button creates a new randomization
of the sideshow.


\paragraph*{Tiled graphics.}\label{para:tiledgraphics} A sideshow appears in pieces, called \emph{tiles}, that appear
one at a time as the  puzzle is solved, see
\hyperref[fig:sideshow]{Figure~\ref*{fig:sideshow}}. The graphic to be used
must have been broken down into a series of \emph{tiled sub-graphics}. These
tiles must be created in a certain way and labeled in a specific manner. Use
the \FmtMP{\pkg{tile-graphic} pkg}\pkg{tile-graphic} package to tile the
graphic into either individual PDF tiled sub-graphics or as a single package
of tiled graphics.\footnote{\url{https://ctan.org/pkg/tile-graphic}}

\subparagraph*{The base name of the graphic.}\label{subpara:basename} The \pkg{tile-graphic} has a naming convention
that this package respects:
\begin{itemize}
  \item When the tiled graphics are individual files, they are named, for
      example, \texttt{mypic\_01}, \texttt{mypic\_02}, \texttt{mypic\_03},
      \dots. The base name of this example is \texttt{mypic}. The graphical
      file format of the tiles is any format the PDF creator supports for
      graphical inclusion. It is usually most convenient for the tiled
      files to be PDF files.
  \item When the tiled graphics are packaged,\footnote{\app{xelatex} does not support packaged files,
  the tiles should be individual PDFs, as describe previously.} the \pkg{tile-graphic}
      package names the package file, for example,
      \texttt{mypic\_package.pdf}. When packaged, the graphical file format
      for the tiles are always PDFs. The base name of this example is \texttt{mypic}.
\end{itemize}


\newtopic
Additional details of how to create a sideshow are dependent on whether the
option \opt{usebtnappr} or \opt{uselayers} is taken.

\subsection{Sideshow with the \tops{\protect\opt{usebtnappr}}{usebtnappr} option}

\textbf{Demo file.} The demo file for this feature is \texttt{first\_date.tex} found
in the folder \texttt{examples/advanced/usebtnappr/sideshow}.

\newtopic\noindent First we embed the pictures of the sideshow in the \env{embedding}
environment.
\begin{Verbatim}[commandchars=!()]
\begin{embedding}
\dpsEmbedIcons
!ameta(other-embedding commands)
!textbf(\sideshowPackaged!quad)%!sffamily( optional)
!bfseries\dpsEmbedSideShow[!ameta(ext)]{!ameta(n-pics)}{!ameta(path)}
\end{embedding}
\end{Verbatim}
There are two ways to present the sideshow pictures, as described in the
paragraph \textbf{\nameref{para:tiledgraphics}} above, to the
\cs{dpsEmbedSideShow} command: (1) as individual tiled graphics; or (2) as a
packaged graphic, the pages of which are the tiled graphics. The form of how
the picture is presented is signaled to the \cs{dpsEmbedSideShow} command by
the command \cs{sideshowPackage}, as indicated above. The first argument is
\ameta{ext}, the file extension of the graphic, this usually not needed; when
there is no \ameta{ext} specified, the extension is assumed to be
\texttt{pdf}. The second argument is \ameta{n-pics}, the number of tile
sub-graphics in the sideshow. The third argument, \ameta{path}, is the path
to the sideshow graphic. At the end of the \ameta{path} is the base name of
the graphic, as described in \textbf{\nameref{subpara:basename}} on
page~\pageref*{subpara:basename}; for example, \texttt{graphics/mypic}
indicates the tile files are in the \texttt{graphics} sub-folder, with base
name of \texttt{mypic}.

\newtopic After embedding the sideshow graphics, insert them into the puzzle board
using,
\begin{Verbatim}[commandchars=!()]
\insertSideshow{!ameta(nRows)}{!ameta(nCols)}{!ameta(width)}{!ameta(height)}
\end{Verbatim}
where, \ameta{nRows} is the number of rows of the tiled graphic;
\ameta{nCols} is the number of columns of the tiled graphic; \ameta{width} is
the width of the tile; and \ameta{height} is the height of the tile. These
latter two are adjusted so the tile fits into the space allotted and the
aspect ratio is preserved.

\newtopic\noindent
\textbf{\textcolor{red}{Remark.}} When compiling a puzzle with the
\opt{usebtnappr} option, the puzzle file compiles in two ``modes,'' depending
on the state of the switch \cs{ifwrtContent}. Each time the document is
compiled, the package looks for the file \texttt{icons-pglst.sav}, if it
exists, the switch \cs{ifwrtContent} is set false; otherwise, it is set to
true. The file \texttt{icons-pglst.sav} is created by \texttt{icons.tex}, the
existence of \texttt{icons-pglst.sav} means \texttt{icons.tex} has been
built. Once the puzzle file knows the \texttt{icons.tex} is built, the
switch \cs{ifwrtContent} is set to false, and its behavior is changed
slightly. If something goes wrong, delete \texttt{icons-pglst.sav}, and rebuild
the puzzle file first, the icons file next, then finally the puzzle file again.

\subsection{Sideshow with the \tops{\protect\opt{uselayers}}{uselayers} option}

\textbf{Demo file.} The two TEX files in the \texttt{examples/advanced/uselayers/sideshow} folder.

\newtopic\noindent
When the \opt{uselayers} option is specified, only EPS files are supported.
\begin{Verbatim}[commandchars=!()]
\insertSideshow{!ameta(nRows)}{!ameta(nCols)}[!ameta(hy-opts)]{!ameta(path)}
\end{Verbatim}

% layers
%\insertSideshow{3}{2}[hiresbb,width=.5\linewidth]{flowers1/DSC_0453}

\section{Some small degree of security}

If a puzzle is create for a class of students to take for credit; then some
security is appropriate. Typically, you post the puzzle with a print button
(see \cs{printDPS} above. After the student completes the puzzle, he/she
prints the results and hands it in for some credit. The \pkg{dps} package also
provides \cs{clearOnCloseOrSave}:
\begin{Verbatim}
\clearOnCloseOrSave
\end{Verbatim}
Place this command in the preamble.  Now, when the puzzle is built, a student
tries to close or save the puzzle, the puzzle is cleared before closing or
saving. The only record is the printed version.\footnote{Perhaps this is a
mere nuisance, the student can make many copies of the printed puzzle, and
hand it around to others in the class. Nothing is foolproof.}

%\section{Advanced hooks}
%There are several built-in hooks that can be exploited to author special effects.

% The two commands \afterQhookA and \OnFocusQhookAA are hooks onto
% the \dpsQ command of the dps package. This allows us to post process
% the user's choice of a questions, and allows us to execute JS on focus.
% We use these two to call to JS function to support these additional

% \afterQhookA
% \OnFocusQhookAA

% afterCorrectChoiceHook()
% dpsFinishedHook()

\section{Let's have some Fun}

In order to make answering the questions ``fun,'' and in addition to
the puzzle (or message), I implemented a point system. Each time the
user checks an incorrect answer, that is recorded as a miss. After
completion of the game, the JavaScript determines if the user has
passed the test. To make it more interesting, a penalty point system
is also used: If the user incorrectly answer the same questions
multiple times (guessing!), penalty points are given.

\newtopic\noindent
The document author can set the various parameters of this aspect of the game.
\begin{description}
\item[\normalfont{\cs{threshold\darg{\ameta{n}}}}:] The number, \ameta{n}, of
    times a person is allowed to miss the same question before being
    ``awarded'' penalty points.  The command \cs{threshold} with its one
    argument defines the command \cs{dsthreshold} which expands to the
    argument, \texttt{n}, of \cs{threshold}. Set \cs{threshold} in the
    preamble, and use \cs{dsthreshold} as part of the instructions or
    description of the game. The default: \verb!\threshold{3}!.

\item[\normalfont{\cs{penaltypoints\darg{\ameta{n}}}}:] The number, \ameta{n},
of penalty points to be added into the final score. Penalty points
are ``awarded'' for missing the same question more than the number
specified by the argument of \cs{threshold}. The command
\cs{penaltypoints} with its one argument defines the command
\cs{dspenaltypoints} which expands to the argument, \texttt{n}, of
\cs{penaltypoints}. Set \cs{penaltypoints} in the preamble, and use
\cs{dspenaltypoints} as part of the instructions or description of
the game. Default: \verb!\penaltypoints{3}!.

\item[\normalfont{\cs{passing\darg{\ameta{n}}}}:] The maximum number,
\ameta{n}, of questions the user needs to miss and still pass the
test. Passing or not does not depend on the number of penalty
points. The command \cs{passing} with its one argument defines the
command \cs{dspassing} which expands to the argument, \ameta{n}, of
\cs{passing}. Set \cs{passing} in the preamble, and use
\cs{dspassing} as part of the instructions or description of the
game. The default is \verb!\passing{4}!.

The number of incorrect answers and the total penalty points are
combined. Based on the combined score, a final evaluation of the
user's knowledge on the subject is displayed.
\end{description}

\section{Checking for validity}

When creating the game, human error can sneak in. The most critical
part is the \cs{Declare\-Puzzle} command and getting the names of your
fields set up the way you want. Letters with the same field name
(the second parameter of the pair) will only need one question, they
will all ``light up'' when the question is answered.

As explained earlier, after you've decided on your puzzle and the
field names, you can then create your \texttt{Composing} environment
using the \cs{writeComposingEnv} helper command. Review the
discussion in \Nameref{composing}.

To help you lay out your design, use the \texttt{viewmode} option,
possibly along with the \texttt{showletter} option. This gives you a
nice preview and you can see where everything is located. You may
have to adjust the parameter for \cs{insertPuzzle} to fit the puzzle
into the allotted space. If enclosing puzzle, questions and answer
in frames, there may have to be some adjustment of the depth of the
controlling \texttt{minipage} environments, and so on, etc., etc.,
etc., and, of course, etc.

Assuming you have successfully posed questions and answers, and
designed a layout for your puzzle, the questions, and the answers, you are
ready to test it. Using the \texttt{nonrandomized} option is nice
for checking your puzzle; the questions and answers are listed in
the same order.

If you have customized the text using the \texttt{lang=custom}
option, you need to check that your text displays correctly; this is
especially important if your new text contains accented characters,
such as our old friend \"{u}. To test your customized string, open
your puzzle document in \app{Acrobat} (\app{Reader} will not do here) and
execute these JavaScript lines in the console:\footnote{To execute
from the console, open the console window by pressing
\texttt{Ctrl+J}, paste in the code, highlight all the lines and
press the \texttt{Enter} key on the keyboard, or \texttt{Ctrl+Enter}
on the keypad.}
\begin{verbatim}
this.resetForm()
nMissed = 0;
nPenaltyPoints = 0;
nPassing = 4;
checkForFinished();
\end{verbatim}
By changing the variables \texttt{nMissed}, \texttt{nPenaltyPoints}
and \texttt{nPassing}, and executing, you get the different messages
appearing in the message text field.

\section{Using the \textsf{web} Package}

The web package has a many features that can be utilized as a part
of your overall puzzle design.

The package has a powerful template management system for inserting
background graphics into a document, and a system for painting the
background a color other than the default white.

Use the \cs{margins} and \cs{screensize} commands to set the
dimensions of your puzzle game page:
\begin{Verbatim}[commandchars=!()]
\margins{!ameta(left)}{!ameta(right)}{!ameta(top)}{!ameta(bottom)}
\screensize{!ameta(height)}{!ameta(width)}
\end{Verbatim}
Enter the title and author's name, as well as other metadata:
\begin{Verbatim}[commandchars=!()]
\title{!ameta(doc-title)}
\author{!ameta(doc-author)}
\end{Verbatim}
See the documentation, \texttt{aeb\_man.pdf}, of the \href{http://www.ctan.org/pkg/acrotex}{\textsf{AeB} distribution}
for details.

\section{Thanks}

My thanks to J\"{u}rgen Gilg, of u-umlaut fame, for his help and
kind suggestions during the development of this game.

That's all for now. Hope you enjoy \textsl{Das Puzzle Spiel} and
find it a useful learning tool.

\newtopic\noindent
Now, I simply must get back to my retirement. \dps

\newpage

\section{Appendix}\label{appendix}

The following is a subset of the PDFDocEncoding character set for PDF, these are
useful for creating your custom localization file \texttt{dps\_str\_cust.def}, as discussed
in \mlNameref{packopts}.

\subsection{German Umlaut (dieresis)}
Here a little tabular how to substitute the German Umlaut
(dieresis) in PD1.\\[6pt]
%
\begin{tabular}{cl|cl|cl}
\"{A}&\verb!\string\304!&\"{O}&\verb!\string\326!&\"{U}&\verb!\string\334!\\
\"{a}&\verb!\string\344!&\"{o}&\verb!\string\366!&\"{u}&\verb!\string\374!\\
{\ss}&\verb!\string\337!
\end{tabular}

\subsection{Accents}
Here a little tabular how to substitute accents in PD1.\\[6pt]
%
\begin{tabular}{cl|cl|cl}
\`{A}&\verb!\string\300!&\`{E}&\verb!\string\310!&\`{I}&\verb!\string\314!\\
\`{a}&\verb!\string\340!&\`{e}&\verb!\string\350!&\`{\i}&\verb!\string\354!\\
\'{A}&\verb!\string\301!&\'{E}&\verb!\string\311!&\'{I}&\verb!\string\315!\\
\'{a}&\verb!\string\341!&\'{e}&\verb!\string\351!&\'{\i}&\verb!\string\355!\\
\^{A}&\verb!\string\302!&\^{E}&\verb!\string\312!&\^{I}&\verb!\string\316!\\
\^{a}&\verb!\string\342!&\^{e}&\verb!\string\352!&\^{\i}&\verb!\string\356!\\
\`{O}&\verb!\string\322!&\`{U}&\verb!\string\331!&\"e&\verb!\string\353!\\
\`{o}&\verb!\string\362!&\`{u}&\verb!\string\371!&\c{C}&\verb!\string\307!\\
\'{O}&\verb!\string\323!&\'{U}&\verb!\string\332!&\c{c}&\verb!\string\347!\\
\'{o}&\verb!\string\363!&\'{u}&\verb!\string\372!\\
\^{O}&\verb!\string\324!&\^{U}&\verb!\string\333!\\
\^{o}&\verb!\string\364!&\^{u}&\verb!\string\373!
\end{tabular}
\end{document}

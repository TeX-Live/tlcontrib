\documentclass{article}
\usepackage[designi,usesf]{web}
\usepackage{eforms}
\usepackage[longcount]{cntdwn}

\title{The \texttt{cntdwn} Package
    \texorpdfstring{\\[6pt]}{: }Long Countdown}
\author{D. P. Story}
\subject{Demo file illustrating the long countdown of the cntdwn package}
\keywords{AeB, AcroTeX, clocks, countdown, cntdwn package}

\university{Acro\negthinspace\TeX.Net}
\email{dpstory@acrotex.net}
\version{1.0}
\copyrightyears{2010}
\revisionLabel{}

\thispagestyle{empty}
\parindent0pt\parskip6pt


\setLongCntDwn{NewYearsLocal}
{%
    date=2015/01/01,
    time=00:01:00,
    endmsg=Happy New Year from wherever I currently am!,
    onfinish=stop
}
\setLongCntDwn{NewYearsCET}
{%
    date=2015/01/01,
    time=00:01:00,
    tzoffset=+0100,
    endmsg=Happy New Year from wherever central Europe!,
    onfinish=stop
}


\begin{document}

\maketitle

The long countdown shows the time remaining to an event in the (usually
distant) future.

\begin{tabular}{rc}
\multicolumn{2}{c}{Time to New Years Day}\\[3bp]
Local:&\lcntdwnDisplay{NewYearsLocal}{2.5in}{11bp}\\[3bp]
CET:&\lcntdwnDisplay{NewYearsCET}{2.5in}{11bp}
\end{tabular}

In central Europe, the two countdowns should read the same, in the U.S.,
they may differ by at least 5 hours.

When the count has reached 0 seconds, a message appears in the text field
wishing you a Happy New Year, from Acro\negthinspace\TeX. I hope you stay
up until 1 second after midnight to see the message.  Then, you can set
the dates to \texttt{2016/01/01} for me, now there's a good fellow.

This code is available through the \texttt{longcount} option of the \texttt{cntdwn} package.

\end{document}

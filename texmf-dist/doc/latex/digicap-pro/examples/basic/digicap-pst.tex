\documentclass{article}
\usepackage[%
    web={tight,pro,usesf,usetemplates},
    uselayers,eforms,aebxmp
]{aeb_pro}
\usepackage{digicap-pro}

\usepackage{pst-blur}

\margins{.5in}{.5in}{24pt}{.5in}    % left,right,top, bottom
\screensize{4.8in}{6in}             % height, width
\renewcommand{\titleauthorproportion}{.4}
\DeclareDocInfo
{
    title=The DigiCap Pro Package\texorpdfstring{\\[1ex]}{: }Using PSTricks,
    author=D. P. Story,
    university=Acro\negthinspace\TeX.Net,
    email=dpstory@acrotex.net,
    subject=Test file for the DigiCap Pro package,
    keywords={Captioning, opacity, layers, roll-over, pstricks},
    talksite=\url{http://www.acrotex.net},
    talkdate={\today},
    copyrightStatus=True,
    copyrightNotice={Copyright (C) \the\year, D. P. Story},
    copyrightInfoURL=http://www.acrotex.net
}
\talkdateLabel{Published:}


\newcommand{\cs}[1]{\texttt{\char`\\#1}}

\pagestyle{empty}

\parskip6pt
\parindent0pt

\begin{document}

\maketitle

\begin{center}
\psframebox[framesep=0pt,linewidth=0pt,shadow=true,blur=true,shadowsize=6pt,blurradius=3pt]{%
\digiCap[vcaption=b,hcaption=c,outerboxsep=0pt]
    {./digis/AdobeDon}[borderwidth=0bp,fboxsep=10bp,bordercolor=nocolor,bgop=.4]
    {\bfseries\footnotesize A picture of D. P. Story, as depicted circa~2001.
    He does not look as good now, as then. Even then, his looks were somewhat
    questionable.}%
}
\end{center}

One can use PSTricks, with pst-blur, to create a blurred shadow effect. Very nice.

\end{document}

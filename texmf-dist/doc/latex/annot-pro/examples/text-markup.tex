\documentclass{article}
\usepackage[designiv,extended]{web}
\usepackage[richtext,dblevel=1]{annot_pro}

\title{text markup annotations}
\author{D. P. Story}
\email{dpstory@acrotex.net}
\subject{text markup annotations}
\keywords{highlight, underline, squiggly, strikeout, pdf annotations, AcroTeX}
\university  {Acro\TeX.NeT}
\version{1.0}



%\usepackage{lucidbry}

% we increase the right margin width to accommodate a margin note
\addtoWebWidth{.5in}
\resetmargins{}{\incby+.5in}{}{}
\setlength{\marginparsep}{4pt}

% try uncommenting the first or second command below
%\mlMarksOn
%\turnSyllbCntOn

\begin{textboxpara}
\rtpara{mypara}{\it{This is italic}, whereas \bf{this is bold}, but wait, we can do
\it{\bf{bold and italic}}. We can \sup{raise} or \sub{lower} text.
    \br\spc\spc
    This is a second paragraph. At your earliest convenience, go to the AcroTeX.Net web site:
    \br\br\span{ulstyle=ul,color=0000FF}{http://www.acrotex.net} (URLs are not supported.)
}
\end{textboxpara}
\setRVVContent{mypara}{mypara}

\setAnnotOptions{subject={An AcroTeX Communiqu\'e},title={D. P. Story}}

\parindent0pt
\parskip4pt

\begin{document}

\makeinlinetitle % defined in web

%\begin{center}
%    Extending \textsf{annot\_pro} to include\\
%    text markup annotations
%\end{center}

Syntax:\\
\verb!\annotpro[type=<type>,<options>]{<content>}{<text>}!\\[3pt]
where, \verb~type=highlight|underline|squiggly|strikeout~

\textbf{Highlight:}
  \annotpro[type=highlight]{Try to remember this, it's important}{A
  \emph{highlight} text markup annotation: Let's extend this text to cross to
  the next line.}

\textbf{Underline:}
  \annotpro[type=underline]{Memorize this passage}{An \emph{underline} text
  \emph{markup} annotation: Let's extend this text to cross to the next
  line.}

\textbf{Squiggly:}
  \annotpro[type=squiggly]{This needs revision}{An \emph{squiggly underline}
  text markup annotation:  Let's extend this text to cross to the next line.}

\textbf{Strike-Out:}
  \annotpro[type=strikeout]{Nonsense!}{An \emph{strikeout} text markup
  annotation:  Let's extend this text to cross to the next line.}

% we are not breaking at an hyphenated word, too bad
\annotpro[type=underline,crackat=37,color=blue]{Memorize this passage}{An
\emph{underline} text \emph{markup} annotation: Let's extend this text to
cross to the next line. We are not near a page boundary, so we must simulate
cracking up the text markup annotation.}

Here's the same example as above, but this time, we'll simulate a page break
and use the \texttt{copycontent} option.

\begingroup % make the next definition local
\def\pb{\penalty-10 \rule{\linewidth}{.4pt}\makebox[0pt][l]{\hspace{\marginparsep}\small page break}\par\smallskip}%\turnSyllbCntOn
\annotpro[type=underline,crackat=37,copycontent,crackinsat=\pb]{Memorize this
    passage}{An \emph{underline} text \emph{markup} annotation: Let's extend this
    text to cross to the next line. We are not near a page boundary, so we must simulate
    cracking up the text markup annotation.}
\endgroup

\smallskip
\textbf{Rich text:} We use \texttt{richtext=mypara}. \annotpro[type=squiggly]{richtext=mypara}{An \emph{squiggly underline}
  text markup annotation:  Let's extend this text to cross to the next line.}


\smallskip
This next example uses the \texttt{hidden} option; this will hide the comment completely
as well as the underlining appearance. When the user presses the push button labeled `C',
the comment becomes visible.

%\textbf{Underline:}
  \annotpro[type=underline,hidden]{Memorize this passage}{An \emph{underline}
  text \emph{markup} annotation: Let's extend this text to cross to the next
  line.} \pushButton[\S{S}\textSize{0}\CA{C}\TU{Press to toggle visibility of this comment}\A{\JS{% var
  annot=this.getAnnot({nPage:this.pageNum,cName:"\currentAnnotName"});\r
  annot.hidden=!annot.hidden;}}]{btn}{}{9bp} This button will work in
  \textsf{Acrobat Reader} and \textsf{PDF-Exchange Viewer}.

\end{document}

%
% AcroTeX.Net : http://www.acrotex.net
% Copyright 2000-2007 D. P. Story
% dpstory@owc.edu
%
% Try this file with the various design options: jeopardy,florida,iceland,hornet,qatar,
% Norway,Germany,bananas,Spain
%
\documentclass{jj_game} % try the twoplayer option as well
\usepackage{amsmath}
%
% By bringing in the exerquiz package, you can also pose
% math fill-in and text-fill-in questions.
%

\titleBanner{The \TeX\ Game!}

\marginsize{.25in}{.25in}{.25in}{.25in} % the default

% This is a little messy because I have parameter values for the
% foils option included here.  I wouldn't think that this
% would be the normal practice.

\DeclareColors{linkColor: red}

\GameDesign
{
    Cat: \TeX,
    Cat: \LaTeX,
    Cat: Classes \&~Packages,
    NumQuestions: 3,
    CellWidth: 1in,
    CellHeight: .5in,
%    Goal: 1,500,              % specify absolute goal
    GoalPercentage: 90,        % specify relative goal
    ExtraHeight: 0pt,
}
\APHidden
{
    Champion: You are TeXerrific!,
    Size: 20,
}
\APDollar
{
    Size: 20,
}

\APRight
{
   Size: 20,
}
\APWrong
{
   Size: 20,
}
\APScore
{
    Font: Arial,
%    Font: TiRo,
    Size: 20,
    CellWidth: 2in,
    align: c,
%    Score: "Score: ",
%    Score: "",
%    Currency: "\string\\u20ac",
    Currency: "$",
}

\newcommand\cmd[1]{\texttt{\string#1}}

\begin{document}

\begin{instructions}
%
% Insert the Instruction page here
%
\textcolor{red}{\textbf{Method of Scoring.}} If you answer a
question correctly, the dollar value of that question is added to
your total.  If you miss a question, the dollar value is
\textit{subtracted} from your total.  So think carefully before
you answer!

\textcolor{red}{\textbf{Instructions.}} Solve the problems in
any order you wish. If your total at the end is more than \$\Goal,
you will be declared  \textbf{\TeX errific}.


\textcolor{red}{\textbf{Important:}} Acrobat Reader 5.0 or later required

\textcolor{red}{\textbf{To Begin:}} Go to the next page.

\end{instructions}

\begin{Questions}

\begin{Category}{TeX}

\begin{Question}[2]
The person who created \TeX. Who is \dots \vspace{2ex}

\Ans0 Sabastian Rahtz &
\Ans1 Donald Knuth \\[2ex]
\Ans0 Leslie Lamport &
\Ans0 David Carisle \\[2ex]
\Ans0 Michel Goossens &
\Ans0 Frank Mittelbach \\[2ex]
\Ans0 Alexander Samarin

\end{Question}

\begin{Question}
The number of scaled points in a point. What is\dots\dots
\Ans0 $72$
\Ans0 $1{,}157$
\Ans1 $65{,}536$
\Ans0 $120{,}745$
\end{Question}

\begin{Question}
When the Main Vertical List accumulates more than enough material
to construct a page, \TeX{} cuts off a chunk of it and places it
in a certain \cmd{\vbox}. What is \dots
\Ans0 \cmd{\box0}
\Ans0 \cmd{\box2}
\Ans1 \cmd{\box255}
\Ans0 \cmd{\box256}
\end{Question}

\end{Category}


\begin{Category}{LaTeX}

\begin{Question}
The command used to typeset the title of an \texttt{article} class
document, what is \dots
\Ans0 \cmd{\title}
\Ans1 \cmd{\maketitle}
\Ans0 \verb+\begin{title}...\end{title}+
\Ans0 \cmd{\typesettitle}
\end{Question}

\begin{Question}
The declaration version of the font command \cmd{\textbf}, what is
\dots
\Ans0 \cmd{\bffamily}
\Ans0 \cmd{\boldface}
\Ans0 \cmd{\bfshape}
\Ans1 \cmd{\bfseries}
\end{Question}

\begin{Question}[2]
The command necessary to typeset the math expression
$x_{a_1}^{2b}$, what is \dots
\Ans0 \verb+$x_a_1^{2b}$+    &
\Ans1 \verb+$x_{a_1}^{2b}$+  \\
\Ans0 \verb+$x_{a_1}^2b$+    &
\Ans0 \verb+$x_a_1^2^b$+     \\
\Ans0 \verb+$x_{a_1}^2^b$+   &
\Ans0 \verb+$x_a_1^2b$+
\end{Question}

\end{Category}

\begin{Category}{Classes \& Packages}

\begin{Question}
The standard \LaTeX{} command used to introduce a \LaTeX{} package into the
document, what is \dots
\Ans0 \cmd{\input}
\Ans0 \cmd{\include}
\Ans1 \cmd{\usepackage}
\Ans0 \cmd{\inputpackage}
\end{Question}

\begin{Question}
The package used to introduce language support into a \LaTeX{}
document, what is \dots
\Ans0 \texttt{multilingual}
\Ans0 \texttt{german}
\Ans0 \texttt{language}
\Ans1 \texttt{babel}
\end{Question}

\begin{Question}
The package used for introducing cross-reference links that become
active when the document is converted to PDF, what is \dots
\Ans1 \texttt{hyperref}
\Ans0 \texttt{x-links}
\Ans0 \cmd{\ref} and \cmd{\pageref}
\Ans0 \texttt{xr-hyper}
\end{Question}

\end{Category}

\end{Questions}

\end{document}

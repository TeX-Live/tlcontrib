\documentclass[12pt]{article}
\usepackage{eforms}
\usepackage{docassembly}
\usepackage{fancyvrb}

\hypersetup{pdfpagemode=UseNone}

\newcommand{\cs}[1]{\texttt{\char`\\#1}}

\pagestyle{empty}
\parskip\medskipamount
\parindent0pt

% Do not compile until you have a digital ID and password
% Actually, you can compile, but do not bring the PDF into Acrobat until you have a digital ID and password.
% You can view the PDF in other PDF viewers such as SumatraPDF.
\begin{docassembly}
\sigInfo{
    cert: "<name>.pfx",
    password: "<password>",
    oInfo: {
      location: "AcroTeX Central, FL",
      reason: "I am certifying this document",
      mdp: "defaultAndComments",
      contactInfo: "dpspeaker@talking@edu"
    }
};
\certifyInvisibleSign
\end{docassembly}

\begin{document}


%\maketitle

\hfill\smash{\raisebox{-\baselineskip}{March 23, 2009}}

\begin{tabular}{@{}ll}
To:         &Honorable Barrister Maxwell Frimpong\\
From:       &D. P. Speaker\\
Subject:    &On Business Proposal\\
\end{tabular}

\vspace{2\baselineskip}

Dear Mr.\ Frimpong;

Thank you for thinking of me concerning an ``important business proposal'' in
your recent and brief email to me on March 23, 2009. Recovering \$12,000,000
(twelve million  Us dollars) in claims sounds intriguing and exciting to me.
Such a large amount of money would certainly come in handy in these tough
times. Yet, regrettably, I must decline your kind offer; though I am in
retirement, I am, none-the-less, quite busy lately sorting my button
collection and don't really have the time to pick up all this easy money.

Thank you again, Barrister Frimpong, for your offer. Please keep me
in mind should future opportunities arise.


\vspace{2\baselineskip}

Best regards,\\
\sigField{sigOfDPS}{2.5in}{4\baselineskip}\\[3pt]
Dr.\ D. P. Speaker\\
Department of Rhetoric\\
Talking University\\
Talkville, FL 12345\\
\texttt{dpspeaker@talking.edu}

\newpage

\section{Creating and Signing a Certified Invisible Signature}

The \textsf{docassembly} package can create a \emph{certified invisible
signature}. You create an certified invisible signature either through the
user-interface of \textsf{Acrobat} or programmatically using JavaScript. With
JavaScript (and \textsf{docassembly}, a trusted version of the
\texttt{\textsl{Doc}.certifyInvisibleSign()} method is used to certify the
document. The certification can be seen by opening the Signature panel
(part of the left-hand panel system of the user-interface).

Also included is an approval signature, as usual, at the end of the letter. Open
this PDF and sign using your own digital ID. Signing should not invalidate
the certification.

The certify invisible sign is done within the \texttt{docassembly}
environment, the script follows:
\begin{Verbatim}[xleftmargin=20pt,fontsize=\small]
\begin{docassembly}
\sigInfo{
    cert: "<name>.pfx", password: "<password>",
    oInfo: {
      location: "AcroTeX Central, FL",
      reason: "I am certifying this document",
      mdp: "defaultAndComments",
      contactInfo: "dpspeaker@talking.edu"
    }
};
\certifyInvisibleSign
\end{docassembly}
\end{Verbatim}
The setup is similar to certify sign, but without the key
\texttt{cSigFieldName}, used to specify a particular field to be signed.
An invisible signature field \emph{is created and signed} using the method
\texttt{\textsl{Doc}.certifyInvisibleSign()}.


The \cs{sigInfo} contains the usual property list, excluding
\texttt{cSigFieldName}. The command \cs{certifyInvisibleSign}
uses the information in this object and calls the trusted version
of \texttt{\textsl{Doc}.certifyInvisibleSign()}, which is defined in
\texttt{aeb\_pro.js}.

Additional information on signatures is found at the
\textbf{Acrobat Developer Center}.\footnote{\url{http://www.adobe.com/go/acrobat_developer}}
Refer to the \emph{JavaScript for Acrobat API Reference}, also found at the \textbf{Acrobat Developer Center},
for details on these methods and their parameters. Adobe is
notorious for moving its reference documents and renaming them, year after
year. Good luck searching the Adobe web site for the references you need.

\section{Compiling this file}

In the preamble of this document, the \texttt{docassembly} environment is found:
\begin{Verbatim}[xleftmargin=20pt,fontsize=\small]
\begin{docassembly}
\sigInfo{
    cert: "<name>.pfx", password: "<password>",
    oInfo: {
      location: "AcroTeX Central, FL",
      reason: "I am certifying this document",
      mdp: "defaultAndComments",
      contactInfo: "dpspeaker@talking.edu"
    }
};
\certifyInvisibleSign
\end{docassembly}
\end{Verbatim}
To compile this document yourself, you need to create a digital ID using
\textsf{Acrobat}. Replace \texttt{<name>} with the file name of your digital
ID. and of course replace \texttt{<password>} with the password you selected
when you created your digital ID. Modify the \texttt{oInfo} property as
designed.


Now, back to my retirement.

\end{document}

References: Try
Digital Signatures Workflow Guide
http://www.adobe.com/devnet-docs/acrobatetk/tools/DigSig/index.html

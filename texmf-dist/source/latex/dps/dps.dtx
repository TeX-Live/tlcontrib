% \iffalse
%<*copyright>
%%%%%%%%%%%%%%%%%%%%%%%%%%%%%%%%%%%%%%%%%%%%%%%%%%%%%%%%%%
%% dps.sty package,             2006-10-06              %%
%% Copyright (C) 2006  D. P. Story                      %%
%%   dpstory@acrotex.net                                %%
%%   storyd@owc.edu                                     %%
%%                                                      %%
%% This program can redistributed and/or modified under %%
%% the terms of the LaTeX Project Public License        %%
%% Distributed from CTAN archives in directory          %%
%% macros/latex/base/lppl.txt; either version 1 of the  %%
%% License, or (at your option) any later version.      %%
%%%%%%%%%%%%%%%%%%%%%%%%%%%%%%%%%%%%%%%%%%%%%%%%%%%%%%%%%%
%</copyright>
%<package>\NeedsTeXFormat{LaTeX2e}
%<package>\ProvidesPackage{dps}
%<package> [2006/10/31 v1.0 Das Puzzle Spiel (dps)]
%<*driver>
\documentclass{ltxdoc}
\usepackage[colorlinks,hyperindex]{hyperref}
\pdfstringdefDisableCommands{\let\\\textbackslash}%
\EnableCrossrefs \CodelineIndex
%\OnlyDescription  % comment out for implementation details
\begin{document}
  \GetFileInfo{dps.sty}
  \title{The \texttt{dps} Package}
  \author{D. P. Story\\
    Email: \texttt{storyd@owc.edu}}
  \date{processed \today}
  \maketitle
  \tableofcontents
  \let\Email\texttt
  \DocInput{dps.dtx}
  \PrintIndex
\end{document}
%</driver>
% \fi
%    \begin{macrocode}
%<*package>
%    \end{macrocode}
% \section{Introduction}
% The \textsf{dps} Package (Das Puzzle Spiel) provides the commands to create a matching
% game and associated puzzle.  As the user answers each question, another part of the puzzle
% is filled in.  After the user has correctly answered all the questions, the message contained
% in the puzzle is fully visible.  The user answers the question by first clicking the checkbox
% of that question, reading and solving the question, then by finding the correct answer listed
% amongst the answer columns.  This game was inspired by one of the handout work sheets of my
% son's eighth grade pre-algebra class.
%
% \section{Options}
% We bring in the \textsf{xkeyval} Package so we can gather our options using
% it's commands, rather than the default \textsf{keyval} commands.
%    \begin{macrocode}
\RequirePackage{xkeyval}
%    \end{macrocode}
%    \begin{macro}{nonrandomize}
% The default behavior is to randomize the questions and answers. With this option, the
% questions and answers are listed in the order they were written in the source file; good
% for debugging, and testing the document.
%    \begin{macrocode}
\DeclareOptionX{nonrandomized}{\werandomizefalse}
%    \end{macrocode}
%    \end{macro}
%    \begin{macro}{viewmode}
% Used for developing the puzzle. When \texttt{viewmode} is optioned, the puzzle and letters in
% the puzzle are seen.  By adjusting the argument of \cs{insertPuzzle} you can get the distribution
% of the puzzle that you want. See also the comments preceding the definition of \cs{makeTextField} below.
%    \begin{macrocode}
\DeclareOptionX{viewmode}{\viewModetrue\previewtrue}
%    \end{macrocode}
%    \end{macro}
%    \begin{macro}{showletters}
% When this option is taken, in the answer columns, the letters that the answers correspond to appear;
% and in the puzzle, the question number that corresponds to that letter.  Generally, this is used
% when web is put in \texttt{forpaper} mode, but can be used in ``screen'' mode.
%    \begin{macrocode}
\DeclareOptionX{showletters}{\showletterstrue}
%    \end{macrocode}
%    \end{macro}
%    \begin{macro}{showanswerlabels}
% We give the user the option of showing the label for the answer, in the case
% of for screen presentation.
%\changes{v1.0}{2006/10/31}
%{
%    Added the \texttt{showanswerlabels} option
%}
%    \begin{macrocode}
\DeclareOptionX{showanswerlabels}
{%
    \ifeqforpaper\else\def\dpsAitemOptArg{}\fi
}
\def\dpsAitemOptArg{[]}
%    \end{macrocode}
%    \end{macro}
%    \begin{macro}{showanswerkey}
% When this option is taken, the solution key appears in the footer. If the
% graphicx package is loaded, the answer key is rotated 180 degrees.  The
% answer key is always computed and saved in the macro \cs{AnswerKey},
%    \begin{macrocode}
\DeclareOptionX{showanswerkey}{\showsolutiontrue}
%    \end{macrocode}
%    \end{macro}
%    \begin{macro}{savedata}
% When this option is taken, the solution key appears in the footer. If the
% graphicx package is loaded, the answer key is rotated 180 degrees.  The
% answer key is always computed and saved in the macro \cs{AnswerKey},
%    \begin{macrocode}
\DeclareOptionX{savedata}{\savepuzzledatatrue}
%    \end{macrocode}
%    \end{macro}
%    \begin{macrocode}
\newif\ifwerandomize \werandomizetrue
\newif\ifviewMode \viewModefalse
\newif\ifshowletters\showlettersfalse
\newif\ifshowsolution\showsolutionfalse
\newif\ifsavepuzzledata\savepuzzledatafalse
\ifeqforpaper\showletterstrue\fi
%    \end{macrocode}
% The only language localizations are the clever commands that appear in the message box.
%    \begin{macro}{lang}
% We offer two language options, \texttt{english} (the default) and \texttt{german}. There
% is a custom option for the author to provide his/her own language localizations.
%    \begin{macrocode}
\define@choicekey*+{dps.sty}{lang}[\val\nr]{english,german,custom}
{%
    \ifcase\nr\relax
        \def\dps@lang@type{%%
%% This is file `dps_str_us.def',
%% generated with the docstrip utility.
%%
%% The original source files were:
%%
%% dps.dtx  (with options: `copyright,english')
%% 
%%%%%%%%%%%%%%%%%%%%%%%%%%%%%%%%%%%%%%%%%%%%%%%%%%%%%%%%%%
%% dps.sty package,             2006-10-06              %%
%% Copyright (C) 2006  D. P. Story                      %%
%%   dpstory@acrotex.net                                %%
%%   storyd@owc.edu                                     %%
%%                                                      %%
%% This program can redistributed and/or modified under %%
%% the terms of the LaTeX Project Public License        %%
%% Distributed from CTAN archives in directory          %%
%% macros/latex/base/lppl.txt; either version 1 of the  %%
%% License, or (at your option) any later version.      %%
%%%%%%%%%%%%%%%%%%%%%%%%%%%%%%%%%%%%%%%%%%%%%%%%%%%%%%%%%%
\def\chooseQ{"You must choose a question to answer before you answer!"}
\def\triedTooMuch{"You have tried this problem too many times,
    I'm adding "
    + \dspenaltypoints
    + " points, and resetting the penalty counter. Bad boy/girl!"}
\def\congratFinished{"Congratulations! You finished the puzzle"
        +((nMissed==0) ? " without missing a single problem, amazing!"
        : ", but you missed " + nMissed + " questions in the process!")}
\def\regretPleased{(( nMissed > n )
        ? "I regret to report that you did not pass the test because
            you missed too many questions."
        : "I am pleased to report that you passed the test!")}
\def\reportPenaltyPoints{"The number of penalty points is "
    + nPenaltyPoints +"."}
\def\finalPenaltyScore{"Final penalty score is " + nTotalPenaltyPoints + ". "}
\def\aPenaltyScale{[-1,0], [0,4], [4, 10], [10,25], [25,5000]}
\def\aPenaltyMsgs
{%
    "Perfect!",
    "Very nice performance!",
    "This is not looking good. Perhaps a review is in order!",
    "Are you trying? No one could do so badly, you only need a seventh grade education!",
    "You're hopeless!"
}
\endinput
%%
%% End of file `dps_str_us.def'.
}
    \or
        \def\dps@lang@type{%%
%% This is file `dps_str_de.def',
%% generated with the docstrip utility.
%%
%% The original source files were:
%%
%% dps.dtx  (with options: `copyright,german')
%% 
%%%%%%%%%%%%%%%%%%%%%%%%%%%%%%%%%%%%%%%%%%%%%%%%%%%%%%%%%%
%% dps.sty package,             2006-10-06              %%
%% Copyright (C) 2006  D. P. Story                      %%
%%   dpstory@acrotex.net                                %%
%%   storyd@owc.edu                                     %%
%%                                                      %%
%% This program can redistributed and/or modified under %%
%% the terms of the LaTeX Project Public License        %%
%% Distributed from CTAN archives in directory          %%
%% macros/latex/base/lppl.txt; either version 1 of the  %%
%% License, or (at your option) any later version.      %%
%%%%%%%%%%%%%%%%%%%%%%%%%%%%%%%%%%%%%%%%%%%%%%%%%%%%%%%%%%
\def\chooseQ{"Du musst erst eine Frage ausw\string\344hlen bevor Du antwortest!"}
\def\triedTooMuch{"Du hast es leider zu oft versucht, ich z\string\344hle  "
    + \dspenaltypoints
    + " Punkte dazu und setze den Z\string\344hler dann zur\string\374ck. Bitte streng Dich an!"}
\def\congratFinished{"Herzlichen Gl\string\374ckwunsch! Du hast das Puzzle beendet"
        +((nMissed==0) ? "  ohne auch nur einen Fehler zu machen, wunderbar!"
        : ", aber leider " + nMissed + " Mal falsch geantwortet!")}
\def\regretPleased{(( nMissed > n )
        ? "Ich bedauere Dir mitteilen zu m\string\374ssen, dass Du den Test leider nicht bestanden hast, weil Du zu viele Fragen falsch beantwortet hast."
        : "Ich bin \string\344usserst erfreut Dir mitteilen zu d\string\374rfen, dass Du den Test bestanden hast!")}
\def\reportPenaltyPoints{"Die Anzahl der Strafpunkte ist  " + nPenaltyPoints +"."}
\def\finalPenaltyScore{"Die Gesamtanzahl der Strafpunkte ist damit " + nTotalPenaltyPoints + ". "}
\def\aPenaltyScale{[-1,0], [0,4], [4, 10], [10,25], [25,5000]}
\def\aPenaltyMsgs
{%
    "Perfekt!",
    "Sehr gute Vorstellung!",
    "Es sieht nicht sonderlich gut aus. Vielleicht w\string\344re eine Wiederholung gut!",
    "R\string\344tst Du eigentlich nur? Niemand kann wirklich so schlecht sein. Das ist Stoff aus der siebten Klasse!",
    "Du bist ein hoffnungsloser Fall!"
}
\endinput
%%
%% End of file `dps_str_de.def'.
}
    \or
        \def\dps@lang@type{%%
%% This is file `dps_str_us.def',
%% generated with the docstrip utility.
%%
%% The original source files were:
%%
%% dps.dtx  (with options: `copyright,english')
%%
%%%%%%%%%%%%%%%%%%%%%%%%%%%%%%%%%%%%%%%%%%%%%%%%%%%%%%%%%%
%% dps.sty package,             2006-10-06              %%
%% Copyright (C) 2006  D. P. Story                      %%
%%   storyd@owc.edu                                     %%
%%                                                      %%
%% This program can redistributed and/or modified under %%
%% the terms of the LaTeX Project Public License        %%
%% Distributed from CTAN archives in directory          %%
%% macros/latex/base/lppl.txt; either version 1 of the  %%
%% License, or (at your option) any later version.      %%
%%%%%%%%%%%%%%%%%%%%%%%%%%%%%%%%%%%%%%%%%%%%%%%%%%%%%%%%%%
\def\chooseQ{"You must choose a question to answer before you answer!"}
\def\triedTooMuch{"You have tried this problem too many times, I'm adding "
    + \dspenaltypoints
    + " points, and resetting the penalty counter. Bad boy/girl!"}
\def\congratFinished{"Student: " + (this.getField("dpsSignInName").value) + ". "
        + "Time stamp: " + (util.printd("mm-dd-yy, H:MM:ss.", new Date())) +"\n"
        +"Congratulations! You finished the puzzle"
        +((nMissed==0) ? " without missing a single problem, amazing!"
        : ", but you missed " + nMissed + " questions in the process!")}
\def\regretPleased{(( nMissed > nPassing )
        ? "I regret to report that you did not pass the test because you missed too many questions."
        : "I am pleased to report that you passed the test!")}
\def\reportPenaltyPoints{"The number of penalty points is " + nPenaltyPoints +"."}
\def\finalPenaltyScore{"Final penalty score is " + nTotalPenaltyPoints + ". "}
\def\aPenaltyScale{[-1,0], [0,4], [4, 10], [10,25], [25,5000]}
\def\aPenaltyMsgs
{%
    "Perfect! You are a statistics wizard!",
    "Very nice performance!",
    "This is not looking good. Perhaps a review is in order!",
    "Are you trying? Do you have the prerequisites for this course?",
    "You're not taking this seriously!"
}
\dlJSStr[noquotes]{\signInMsg}{%
  You must enter your name in the field at the top of
  the page to get credit for this assignment.}
\endinput
%%
%% End of file `dps_str_us.def'.
}
    \else
        \def\dps@lang@type{%%
%% This is file `dps_str_us.def',
%% generated with the docstrip utility.
%%
%% The original source files were:
%%
%% dps.dtx  (with options: `copyright,english')
%% 
%%%%%%%%%%%%%%%%%%%%%%%%%%%%%%%%%%%%%%%%%%%%%%%%%%%%%%%%%%
%% dps.sty package,             2006-10-06              %%
%% Copyright (C) 2006  D. P. Story                      %%
%%   dpstory@acrotex.net                                %%
%%   storyd@owc.edu                                     %%
%%                                                      %%
%% This program can redistributed and/or modified under %%
%% the terms of the LaTeX Project Public License        %%
%% Distributed from CTAN archives in directory          %%
%% macros/latex/base/lppl.txt; either version 1 of the  %%
%% License, or (at your option) any later version.      %%
%%%%%%%%%%%%%%%%%%%%%%%%%%%%%%%%%%%%%%%%%%%%%%%%%%%%%%%%%%
\def\chooseQ{"You must choose a question to answer before you answer!"}
\def\triedTooMuch{"You have tried this problem too many times,
    I'm adding "
    + \dspenaltypoints
    + " points, and resetting the penalty counter. Bad boy/girl!"}
\def\congratFinished{"Congratulations! You finished the puzzle"
        +((nMissed==0) ? " without missing a single problem, amazing!"
        : ", but you missed " + nMissed + " questions in the process!")}
\def\regretPleased{(( nMissed > n )
        ? "I regret to report that you did not pass the test because
            you missed too many questions."
        : "I am pleased to report that you passed the test!")}
\def\reportPenaltyPoints{"The number of penalty points is "
    + nPenaltyPoints +"."}
\def\finalPenaltyScore{"Final penalty score is " + nTotalPenaltyPoints + ". "}
\def\aPenaltyScale{[-1,0], [0,4], [4, 10], [10,25], [25,5000]}
\def\aPenaltyMsgs
{%
    "Perfect!",
    "Very nice performance!",
    "This is not looking good. Perhaps a review is in order!",
    "Are you trying? No one could do so badly, you only need a seventh grade education!",
    "You're hopeless!"
}
\endinput
%%
%% End of file `dps_str_us.def'.
}
    \fi
}{\PackageWarning{dps}{Bad choice for lang, permissible values
are english, german and custom. Try again}}
\def\dps@lang@type{%%
%% This is file `dps_str_us.def',
%% generated with the docstrip utility.
%%
%% The original source files were:
%%
%% dps.dtx  (with options: `copyright,english')
%% 
%%%%%%%%%%%%%%%%%%%%%%%%%%%%%%%%%%%%%%%%%%%%%%%%%%%%%%%%%%
%% dps.sty package,             2006-10-06              %%
%% Copyright (C) 2006  D. P. Story                      %%
%%   dpstory@acrotex.net                                %%
%%   storyd@owc.edu                                     %%
%%                                                      %%
%% This program can redistributed and/or modified under %%
%% the terms of the LaTeX Project Public License        %%
%% Distributed from CTAN archives in directory          %%
%% macros/latex/base/lppl.txt; either version 1 of the  %%
%% License, or (at your option) any later version.      %%
%%%%%%%%%%%%%%%%%%%%%%%%%%%%%%%%%%%%%%%%%%%%%%%%%%%%%%%%%%
\def\chooseQ{"You must choose a question to answer before you answer!"}
\def\triedTooMuch{"You have tried this problem too many times,
    I'm adding "
    + \dspenaltypoints
    + " points, and resetting the penalty counter. Bad boy/girl!"}
\def\congratFinished{"Congratulations! You finished the puzzle"
        +((nMissed==0) ? " without missing a single problem, amazing!"
        : ", but you missed " + nMissed + " questions in the process!")}
\def\regretPleased{(( nMissed > n )
        ? "I regret to report that you did not pass the test because
            you missed too many questions."
        : "I am pleased to report that you passed the test!")}
\def\reportPenaltyPoints{"The number of penalty points is "
    + nPenaltyPoints +"."}
\def\finalPenaltyScore{"Final penalty score is " + nTotalPenaltyPoints + ". "}
\def\aPenaltyScale{[-1,0], [0,4], [4, 10], [10,25], [25,5000]}
\def\aPenaltyMsgs
{%
    "Perfect!",
    "Very nice performance!",
    "This is not looking good. Perhaps a review is in order!",
    "Are you trying? No one could do so badly, you only need a seventh grade education!",
    "You're hopeless!"
}
\endinput
%%
%% End of file `dps_str_us.def'.
}
%    \end{macrocode}
%    \end{macro}
%    \begin{macrocode}
\AtEndOfPackage{\dps@lang@type}
\ProcessOptionsX
%    \end{macrocode}
% \section{Required Packages}
% In addition to the \textsf{web} and \textsf{eforms} packages, the following are
% used in the macro package.
%    \begin{macrocode}
\RequirePackage{verbatim}
\RequirePackage{calc}
\RequirePackage{multicol}
\@ifundefined{nextrandom}{\input{random.tex}}{}
%    \end{macrocode}
%    \begin{macrocode}
\def\nextrandom{\begingroup
 \ifnum\randomi<\@ne % then initialize with time
    \global\randomi\time
    \global\multiply\randomi388 \global\advance\randomi\year
    \global\multiply\randomi31 \global\advance\randomi\day
    \global\multiply\randomi97 \global\advance\randomi\month
    \message{Randomizer initialized to \the\randomi.}%
    \nextrandom \nextrandom \nextrandom
    \xdef\ds@saveRandomSeed{\the\randomi}%
 \fi
 \count@ii\randomi
 \divide\count@ii 127773 % modulus = multiplier * 127773 + 2836
 \count@\count@ii
 \multiply\count@ii 127773
 \global\advance\randomi-\count@ii % random mod 127773
 \global\multiply\randomi 16807
 \multiply\count@ 2836
 \global\advance\randomi-\count@
 \ifnum\randomi<\z@ \global\advance\randomi 2147483647\relax\fi
 \endgroup
}
%    \end{macrocode}
%
% \section{Switches and things}
%
%    \begin{macrocode}
\newif\ifforquestions \forquestionstrue
\newcount\ds@nCnt
\newcount\ds@nMax
\newcount\ds@qNumber\ds@qNumber=0
\newcount\ds@aNumber\ds@aNumber=0
\newcount\ds@probCnt \ds@probCnt=0
\newcount\ds@nCntCols \ds@nCntCols=0
\newcount\ds@getRanNum
\newtoks\ds@listIn \ds@listIn={}
\newtoks\ds@newListIn \ds@newListIn={}
\newtoks\ds@listOut \ds@listOut={}
\newtoks\ds@tmpToks \ds@tmpToks={}
\newtoks\ds@qlistOut \ds@qlistOut={}
\newtoks\ds@alistOut \ds@alistOut={}
\newtoks\ds@PuzzleAppearancetoks \ds@PuzzleAppearancetoks={}
\newtoks\ds@QuesAppearancetoks \ds@QuesAppearancetoks={}
\newtoks\ds@AnsAppearancetoks \ds@AnsAppearancetoks={}
\newwrite \ds@question@write
\ifsavepuzzledata
\newwrite \ds@savedata
\immediate\openout \ds@savedata \jobname_data.sav
\fi
\setlength{\multicolsep}{\topsep}
%    \end{macrocode}
% \section{Main Code}
%    \begin{macro}{\makeTextField}
%    \begin{macro}{\PuzzleAppearance}
% When the user chooses the \texttt{viewmode} option, the underline appear in the puzzle
% with the letter corresponding to that position; otherwise, a Acrobat text field is created.
% Note that for letters what have accents, use the hyperref command \cs{texorpdfstring}
% to give to letters, one for view mode (without accent) and one with accent. See the file
% \texttt{pmg.tex} for an example of this, look for it in the \cs{DeclarePuzzle} command.
%
% The command \cs{PuzzleAppearance} can be used to change the appearance of the text fields for
% the puzzle.
%\changes{v1.0}{2006/10/31}
%{
%   Added a token list to hold any user changes in the appearance of the puzzle, as
%   as suggested by Robert Marik.
%}
%    \begin{macrocode}
%\def\PuzzleAppearance#1{\def\pAppearance{#1}}
\def\PuzzleAppearance#1{\ds@PuzzleAppearancetoks={#1}}
\ifeqforpaper
    \ifviewMode
        \def\makeTextField#1#2{\def\argii{#2}\ifx\argii
            \ds@myspace\let\ds@ul\relax\else\let\ds@ul\underbar\fi
            \ifshowletters\raisebox{-8pt}{%
            \makebox[0pt][l]{\makebox[16bp][c]{%
            \footnotesize\ds@getProbNumber{#1}{#2}}}}\fi
            \ds@ul{\makebox[16bp][c]{\Hy@pdfstringfalse#1}}}
    \else
        \def\makeTextField#1#2{\def\argii{#2}\ifx\argii\ds@myspace
            \let\ds@ul\relax\else\let\ds@ul\underbar\fi
            \ifshowletters\raisebox{-8pt}{%
            \makebox[0pt][l]{\makebox[16bp][c]{%
            \footnotesize\ds@getProbNumber{#1}{#2}}}}\fi
            \ds@ul{\makebox[16bp][c]{\hfil}}}
    \fi
\else
    \ifviewMode
        \def\makeTextField#1#2{\ifshowletters\raisebox{-8pt}{%
            \makebox[0pt][l]{\makebox[16bp][c]{%
            \footnotesize\ds@getProbNumber{#1}{#2}}}}\fi
            \underbar{\makebox[16bp][c]{\Hy@pdfstringfalse#1}}}
    \else
        \def\makeTextField#1#2{\ifshowletters
            \raisebox{-8pt}{\makebox[0pt][l]{\makebox[16bp][c]{%
            \footnotesize\ds@getProbNumber{#1}{#2}}}}\fi
            \edef\textfieldTmp{\noexpand\textField[\noexpand\Q{1}%
            \noexpand\S{U}\noexpand\Ff\noexpand\FfReadOnly
            \the\ds@PuzzleAppearancetoks\noexpand\DV{#1}]%
            {puzzle.#2}{16bp}{11bp}}\Hy@pdfstringtrue\textfieldTmp}
    \fi
\fi
%    \end{macrocode}
%    \end{macro}
%    \end{macro}
%    \begin{macro}{\DeclarePuzzle}
%    \begin{macro}{\insertPuzzle}
% The puzzle is declare using the \cs{DeclarePuzzle} command. It's argument is a finite
% number of two argument pairs. For example,
%\begin{verbatim}
%\DeclarePuzzle{
%   {H}{H}
%   {e}{e}
%   {l}{l}
%   {l}{l}
%   {o}{o}
%   {,}{punc}
%   {}{space}
%   {J}{J}
%   {\texorpdfstring{\"{u}}{\string\374}}{uml}
%   {g}{g}
%   {e}{e}
%   {n}{n}
%   {!}{punc}
%}
%\end{verbatim}
%The first term of the pair is the letter that appears in the puzzle, in this case, the puzzle is
%``Hello, J\"{u}rgen!''.  Spaces between words are left as empty arguments.  A letter that has
%a special accent can be entered as well; however, for preview mode, use the hyperref command
%\cs{texorpdfstring}, as illustrated in the example above. (The first argument of \cs{texorpdfstring}
%will be typeset when the document is in \texttt{viewmode}, the second argument will be inserted
%into the text field and can be an accented character (it's octal representation).
%
%The second term of each pair will form the base name of the form fields created for this document.
%The second argument must be a JavaScript identifier (basically consist of letters and numbers).
%Puzzle entries with the same second argument will appear simultaneously when the corresponding
%question is answered correctly. For example, the puzzle above has two of the letter `e' and
%to of `l'.  As a result, there are only $8$ distinct form names, so $8$ or more questions are
%needed.
%
%Second terms with the name of \texttt{space} and \texttt{punc} are special cases. A \texttt{space} field
%will be type set as a text field with an empty value (maintaining the spacing of the puzzle); the \texttt{punc}
%name is recognized as a signal for punctuation, and a text field is not generated for it. (Note, you can
%give punctuation a field name other than \texttt{punc}, and a text field will be created, and you'll need
%a question to make that field appear.)
%
% You can have more answers than questions, but you cannot have more questions than answers;
% that is, not questions that have no listed answer and does not correspond to a letter
% in the puzzle.
%    \begin{macrocode}
\def\DeclarePuzzle#1{\gdef\puzzleParameters{#1}}
%    \end{macrocode}
% Use \cs{insertPuzzle} to insert the puzzle at the location desired. The puzzle is placed
% in a tabular environment. The only argument of this command is the number of columns
% you want for this tabular environment. For example, \verb!\insertPuzzle{18}! distributes
% the puzzle so that there are $18$ columns per row, one letter per column.
%    \begin{macrocode}
\def\insertPuzzle#1{\gdef\nCols{#1}%
    \ifviewMode\Hy@pdfstringfalse\else\Hy@pdfstringtrue\fi
    \expandafter\aeb@Puzzle\expandafter{\puzzleParameters}}%
%    \end{macrocode}
%    \end{macro}
%    \end{macro}
%    \begin{macro}{\rowsep}
% This command is used to adjust the space between row of the tabular environment
% for the puzzle. The default is \verb!\rowsep{1ex}!.
%    \begin{macrocode}
\def\rowsep#1{\gdef\@rowsep{[#1]}}\rowsep{\rowsep@default}
\def\rowsep@default{1ex}
%    \end{macrocode}
%    \end{macro}
% This code create the tabular environment, creating a new row when necessary,
% and inserts the text field or typesets the puzzle (in the case of \texttt{viewmode}).
%    \begin{macrocode}
\def\eq@tabSep{&}\def\ds@punc{punc}
\def\aeb@Puzzle#1{\edef\eq@tabEnd{\noexpand\\\@rowsep}%
    \aeb@@Puzzle#1\relax\relax}%
\def\aeb@@Puzzle{\begin{tabular*}{\linewidth}{*{\nCols}l}\@aebPuzzlei}%
\def\@aebPuzzlei#1#2{\ifx#1\relax\gdef\nextPuzzleChar{\@aebPuzzleDone}%
    \else\gdef\nextPuzzleChar{\@aebPuzzleii{#1}{#2}}\fi\nextPuzzleChar}%
\def\@aebPuzzleii#1#2{\makeTextField{#1}{#2}\@takeaPeek}
\def\@takeaPeek#1#2{\def\argii{#2}\ifx\argii\ds@punc
    \def\@puzzNext{#1\@setSep@aebPuzzleii}%
    \else\def\@puzzNext{\@setSep@aebPuzzleii{#1}{#2}}\fi\@puzzNext}
\def\@setSep@aebPuzzleii{\global\advance\ds@nCntCols1\relax
    \ifnum\ds@nCntCols=\nCols\expandafter\eq@tabEnd \global\ds@nCntCols=0
    \else\expandafter\eq@tabSep\fi\@aebPuzzlei}%
\def\@aebPuzzleDone{\end{tabular*}\ds@buildAnswerKey}%
%    \end{macrocode}
%    \begin{macro}{\AnswerKey}
% The command \cs{AnswerKey} is defined at the end of the \cs{insertPuzzle} command
% and is available thereafter for insertion into the document. If the \texttt{showanswerkey}
% option is taken, it is displayed at the bottom of the page. If the \texttt{savedata}
% option is taken, the answer key is save to a file along with the random seed that generated
% this answer key. The file name containing the puzzle data is \cs{jobname\_data.sav}.
%    \begin{macrocode}
\def\ds@buildAnswerKey
{%
    {% local
        \count0=0 \toks0={}%
        \loop
            \advance\count0by1
            \xdef\AnswerKey{\the\toks0
            \the\count0--\noexpand
            \csname dps@probLetter\the\count0\noexpand\endcsname; }%
            \toks0 = \expandafter{\AnswerKey}%
        \ifnum\count0 < \ds@qNumber\repeat
    }%
    \ifshowsolution
        \cfooter{\footnotesize\@ifundefined{rotatebox}{\AnswerKey}%
        {\rotatebox{180}{\AnswerKey}}}%
    \fi
    \ifsavepuzzledata
        \AtEndDocument{\ds@writePuzzleData}%
    \fi
}
\def\ds@writePuzzleData{%
    \let\verbatim@out\ds@savedata
    \dps@IWVO{Initial seed: \string\randomi=\ds@saveRandomSeed}%
    \dps@IWVO{Solution Key: \AnswerKey}%
}
%    \end{macrocode}
%    \end{macro}
% \section{Questions and Answers}
%    \begin{environment}{Composing}
% This is the environments in which the composing of questions and answers are made.
% Use the \texttt{cQ} and \texttt{cA} for this purpose.
%    \begin{macrocode}
\newenvironment{Composing}
{\global\ds@qNumber=0 \global\ds@aNumber=0}
{\aftergroup\ds@publishRandomLists}
%    \end{macrocode}
%    \end{environment}
%    \begin{environment}{cQ}
%    \begin{environment}{cA}
% Here are the two environment for composing questions and answers. The argument
% of each environment corresponds to a form field in the puzzle. Each of these
% environments are verbatim write environments; they write each question and solution
% to a separate \texttt{.cut} file. These individual files are ultimately input
% in a random order.
%    \begin{macrocode}
\newenvironment{cQ}[1]
{%
   \global\advance\ds@qNumber1
   \immediate\openout \ds@question@write \jobname_q\the\ds@qNumber.cut
    \let\verbatim@out\ds@question@write
    \dps@IWVO{\noexpand\dpsQ{#1}}%
    \verbatimwrite}
{%
    \endverbatimwrite
    \immediate\closeout \ds@question@write
}
%    \end{macrocode}
% The cA environment has an optional argument. This optional argument is only used
% when the document is compiled with the showletters option.  The value of the argument
% is a letter to appear in the answers column.  Normally, one of the first entries in the
% \cs{DeclarePuzzle} command is used. Cases where you would want to include this optional
% argument are (1) when giving an answer that does not correspond to a question; (2) the
% letter is capitalized, suggesting a proper name or the beginning of a sentence, use the
% optional argument to list the letter in lower case.
%    \begin{macrocode}
\newenvironment{cA}[2][]
{%
   \def\argi{#1}\global\advance\ds@aNumber1
   \immediate\openout \ds@question@write \jobname_a\the\ds@aNumber.cut
    \let\verbatim@out\ds@question@write
    \let\protect\noexpand
    \dps@IWVO{\noexpand\dpsA\ifshowletters
        \ifx\argi\@empty\else[#1]\fi\fi{#2}}%
    \verbatimwrite}
{%
    \endverbatimwrite
    \immediate\closeout \ds@question@write
}
%    \end{macrocode}
%    \end{environment}
%    \end{environment}
%    \begin{macro}{\QuesAppearance}
%    \begin{macro}{\AnsAppearance}
% These two commands can be used to change the appearance of the checkboxes for the questions and answers.
% When the \texttt{forpaper} \textsf{web} option is taken, these have no effect.
%    \begin{macrocode}
\def\QuesAppearance#1{\ds@QuesAppearancetoks={#1}}
\def\AnsAppearance#1{\ds@AnsAppearancetoks={#1}}
%    \end{macrocode}
%    \end{macro}
%    \end{macro}
%    \begin{macro}{\dpsQ}
%    \begin{macro}{\dpsA}
%    \begin{macro}{\afterQhookA}
%    \begin{macro}{\OnFocusQhookAA}
% The next two commands place the check boxes, the argument is the name of the form
% field that corresponds to one of the letters in the puzzle.
%
% The commands \cs{dpsQ} and \cs{dpsA} are the ones that produce the checkboxes, and define the
% JavaScript actions.  These two can be redefined as needed, but be sure to preserve the JS
% functionality.
%
% The two commands \cs{afterQhookA} and \cs{OnFocusQhookAA} are commands that can be redefined
% for additional JS action, the first is a  hook into the mouse up action, and the second is
% a hook to the on focus action.
%    \begin{macrocode}
\let\afterQhookA\@gobble
\let\OnFocusQhookAA\@gobble
%    \end{macrocode}
% Set the checkboxes and JS action for questions.  Allow also, changes to appearance
% through the \cs{QuesAppearance} command.
%    \begin{macrocode}
\newcommand{\dpsQ}[1]{\item\relax\ifeqforpaper\else
    \edef\checkboxTmp{\noexpand\checkBox[\the\ds@QuesAppearancetoks
    \noexpand\A{\noexpand\JS{activeQuestion = event.target.name;
    clearRedCrosses(); \noexpand\afterQhookA{#1}}}%
    \noexpand\AA{\noexpand\AAOnFocus{\noexpand\JS{%
    this.resetForm(["ckbxQ"]); \noexpand\OnFocusQhookAA{#1}}}}]%
    {ckbxQ.#1}{14bp}{11bp}{Yes}}\makebox[0pt][r]{\checkboxTmp\enspace}%
    \fi\ignorespaces}
%    \end{macrocode}
% Set the checkboxes and JS action for answers.  Allow also, changes to appearance
% through the \cs{AnsAppearance} command. This command obeys the \texttt{showanswerlabels}
% option by re-defining the command \cs{dpsAitemOptArg}. The default definition
% is \verb!\def\dpsAitemOptArg{[]}!, which cancels the display of the item labels.
%    \begin{macrocode}
\newcommand{\dpsA}[2][]{\expandafter\item\dpsAitemOptArg\relax
    \ifeqforpaper\else
        \edef\checkboxTmp{\noexpand\checkBox[\the\ds@AnsAppearancetoks
        \noexpand\A{\noexpand\JS{processChoice("#2")}}]%
        {ckbxA.#2}{14bp}{11bp}{Yes}}%
        \makebox[0pt][r]{\checkboxTmp\enspace}%
    \fi
    \def\argi{#1}%
        \ifshowletters
            \ifx\argi\@empty
                \let\getLetterNext\relax
                \let\ds@foundLetter=0\def\ds@currFN{#2}%
                \def\getLetterNext{\ds@typesetPuzzleLetter}%
                \expandafter\getLetterNext
            \else
                \makebox[0pt][r]{\Hy@pdfstringfalse#1\enspace\kern4bp}%
            \fi
        \fi
    \ignorespaces
}
%    \end{macrocode}
%    \end{macro}
%    \end{macro}
%    \end{macro}
%    \end{macro}
%    \begin{macrocode}
\def\ds@typesetPuzzleLetter{%
    \expandafter\typeset@PuzzleLetter\expandafter{\puzzleParameters}}
\def\typeset@PuzzleLetter#1{\typeset@@PuzzleLetter#1\relax\relax}%
\def\typeset@@PuzzleLetter#1#2{%
    \ifx#1\relax
        \gdef\nextPuzzleLetter{\relax}%
    \else
        \gdef\nextPuzzleLetter{\typeset@PuzzleLetteri{#1}{#2}}%
    \fi
    \nextPuzzleLetter
}
\def\typeset@PuzzleLetteri#1#2{\def\argii{#2}\ifx\argii\ds@currFN
    \if\ds@foundLetter0\makebox[0pt][r]{%
        \Hy@pdfstringfalse#1\enspace\kern4bp}\let\ds@foundLetter=1\fi
        \expandafter\typeset@@PuzzleLetter
        \else\expandafter\typeset@@PuzzleLetter\fi}
%    \end{macrocode}
% This command is called at \verb!\end{Composing}!, which, in turn, calls
% the macros \cs{ds@randomizeQuestionList} and \cs{ds@randomizeAnswerList} which randomly
% permutes the \verb!1, 2, ... \the\ds@qNumber! and \verb!1, 2, ... \the\ds@aNumber!,
% where \cs{ds@qNumber} and \cs{ds@aNumber} are the number of questions and answers,
% respectively.
%    \begin{macrocode}
\def\ds@publishRandomLists{%
    \ds@randomizeQuestionList{\the\ds@qNumber}%
    \ds@randomizeAnswerList{\the\ds@aNumber}%
}
\def\ds@myspace{space}
\def\ds@getProbNumber#1#2{%
    \gdef\ds@currentArgi{#1}\gdef\ds@currentArgii{#2}%
    \ifx\ds@currentArgii\ds@myspace\def\ds@probNumNext{\relax}\else
        \def\ds@probNumNext{%
            \@ifundefined{dps@probNum#2}{\global\advance\ds@probCnt1
% \typeout{\ds@probCnt = \the\ds@probCnt}%
                \ds@@getProbNumber{\the\ds@probCnt}%
            }{\csname dps@probNum#2\endcsname}%
    }%
    \fi
    \ds@probNumNext
}
\def\ds@@getProbNumber#1{%
    {\count0=0 \let\\=\ds@getNthOne\the\ds@qlistOut}%
    \csname dps@probNum\ds@currentArgii\endcsname
}
\def\ds@getNthOne#1{\advance\count0by1
    \ifnum\ds@probCnt=#1
        \expandafter
            \xdef\csname dps@probNum\ds@currentArgii\endcsname
                {\the\count0 }%
%\typeout{\the\count0, and \ds@currentArgi}%
        \expandafter
            \xdef\csname dps@probLetter\the\count0\endcsname
                {\expandafter\noexpand\ds@currentArgi}%
    \fi
}
%    \end{macrocode}
%    \begin{macro}{\writeComposingEnv}
% This is a helper macro.  After you declare your puzzle, \cs{DeclarePuzzle},
% you can place this command just after, if needed, like so
%\begin{verbatim}
%\writeComposingEnv
%\begin{document}
%\end{document}
%\end{verbatim}
%In the case where \cs{DeclarePuzzle} is in the preamble; above, we begin the doc
% and end the doc; assuming you have not developed your questions yet. The command
% writes to the file \cs{jobname\_comp.def}. This file will be a skeleton of your
% Composing environment, with correct labeling. Copy and paste it into your document
% in some appropriate location, and begin writing your questions.
%    \begin{macrocode}
\def\writeComposingEnv{%
    \newwrite \ds@composing@write
    \immediate\openout \ds@composing@write \jobname_comp.def
    \let\verbatim@out\ds@composing@write
    \dps@IWVO{\string\begin{Composing}}%
    \dps@IWVO{}%
    \expandafter\write@ComposingEnv\expandafter{\puzzleParameters}%
}
\def\write@ComposingEnv#1{\write@@ComposingEnv#1\relax\relax}%
\def\write@@ComposingEnv#1#2{\ifx#1\relax
    \gdef\nextPuzzlePair{%
    \dps@IWVO{\string\end{Composing}}%
    \immediate\closeout \ds@composing@write}%
    \else\gdef\nextPuzzlePair{\write@@@ComposingEnv{#1}{#2}}\fi
    \nextPuzzlePair}%
\def\write@@@ComposingEnv#1#2{%
    \gdef\ds@currentArgi{#1}\gdef\ds@currentArgii{#2}%
    \ifx\ds@currentArgii\ds@myspace
        \def\ds@probNumNext{\write@@ComposingEnv}%
    \else
        \ifx\ds@currentArgii\ds@punc
            \def\ds@probNumNext{\write@@ComposingEnv}\else
            \def\ds@probNumNext{%
                \@ifundefined{dps@compQ#2}{% write to file
                \expandafter\gdef\csname dps@compQ#2\endcsname{found}%
                \dps@IWVO{\string\begin{cQ}{\noexpand#1}}%
                \dps@IWVO{\string\end{cQ}}%
                \dps@IWVO{\string\begin{cA}{\noexpand#1}}%
                \dps@IWVO{\string\end{cA}}%
                \dps@IWVO{}%
                }{}%
                \write@@ComposingEnv
            }%
    \fi\fi
    \ds@probNumNext
}
%    \end{macrocode}
%    \end{macro}
% A standard \cs{verbatim} write used in exerquiz and other package in the AeB family.
%    \begin{macrocode}
\def\verbatimwrite{\@bsphack
    \let\do\@makeother\dospecials
    \catcode`\^^M\active \catcode`\^^I=12
    \def\verbatim@processline{%
        \immediate\write\verbatim@out
        {\the\verbatim@line}}%
    \verbatim@start}
\def\endverbatimwrite{\@esphack}
\def\dps@IWVO{\immediate\write\verbatim@out}
%    \end{macrocode}
% \cs{ds@populateList} is a utility command, its argument is a positive integer, \texttt{n},
% and it generates a list of the form \verb!\\{1}\\{2}...\\{n}!. This listing is later
% randomly permuted by \cs{ds@randomizeQuestionList} and \cs{ds@randomizeAnswerList}.
%    \begin{macrocode}
\def\ds@populateList#1{%
    \ds@listIn = {}%
    \ds@nCnt=0
    \@whilenum \ds@nCnt < #1\do {%
        \advance\ds@nCnt1
        \edef\ds@listInHold{\the\ds@listIn\noexpand\\{\the\ds@nCnt}}%
        \ds@listIn = \expandafter{\ds@listInHold}%
    }%
}
%    \end{macrocode}
% Used in \cs{ds@randomizeList} to build the permuted list of numbers.
%    \begin{macrocode}
\def\ds@processi#1{\advance\ds@nCnt1
    \ifnum\ds@nCnt=\ds@getRanNum\edef\ds@listOutHold{\the\ds@listOut}%
        \global\ds@listOut=\expandafter{\ds@listOutHold\\{#1}}%
    \else
        \edef\ds@listInHold{\the\ds@newListIn}%
        \ds@tmpToks = \expandafter{\ds@listInHold\\{#1}}%
        \ds@newListIn = \expandafter{\the\ds@tmpToks}%
    \fi
}
%    \end{macrocode}
% Used in \cs{displayRandomizedQuestions} to input the questions.
%    \begin{macrocode}
\def\ds@processii#1{\input{\jobname_q#1.cut}}
%    \end{macrocode}
% Used in \cs{displayRandomizedAnswers} to input the answers.
%    \begin{macrocode}
\def\ds@processiii#1{\input{\jobname_a#1.cut}}
%    \end{macrocode}
% Used in \cs{displayRandomizedAnswersLeftPanel} to input the top half
% of the permuted list.
%    \begin{macrocode}
\def\ds@processL#1{\advance\count0by1 %\divide\ds@aNumber2
    \ifnum\count0>\ds@aNumber\relax\else\input{\jobname_a#1.cut}\fi}
%    \end{macrocode}
% Used in \cs{displayRandomizedAnswersRightPanel} to input the bottom half
% of the permuted list.
%    \begin{macrocode}
\def\ds@processR#1{\advance\count0by1 %\divide\ds@aNumber2
    \ifnum\count0>\ds@aNumber\relax\input{\jobname_a#1.cut}\fi}
%    \end{macrocode}
%    \begin{macro}{\displayRandomizedQuestions}
%    \begin{macro}{\displayRandomizedAnswers}
%    \begin{macro}{\displayRandomizedAnswersLeftPanel}
%    \begin{macro}{\displayRandomizedAnswersRightPanel}
% These are user commands that actually display the randomized questions
% and answers.
%    \begin{macrocode}
\def\displayRandomizedQuestions{\let\\=\ds@processii\the\ds@qlistOut}
\def\displayRandomizedAnswers{\let\\=\ds@processiii\the\ds@alistOut}
\def\displayRandomizedAnswersLeftPanel{{\let\\=\ds@processL\count0=0
    \divide\ds@aNumber2 \the\ds@alistOut}}
\def\displayRandomizedAnswersRightPanel{{\let\\=\ds@processR\count0=0
    \divide\ds@aNumber2 \the\ds@alistOut}}
%    \end{macrocode}
%    \end{macro}
%    \end{macro}
%    \end{macro}
%    \end{macro}
% Develop a random permuted list for the questions.
%    \begin{macrocode}
\def\ds@randomizeQuestionList#1{%
    \global\ds@listIn={}\global\ds@newListIn={}\global\ds@listOut={}%
    \global\ds@tmpToks={}%\global\ds@qlistOut={}\global\ds@alistOut={}%
    \ds@nMax=#1\relax\ds@populateList{\the\ds@nMax}%
    \global\forquestionstrue
    \ifwerandomize
        \expandafter\ds@randomizeList
    \else
%        \global\let\ds@qlistOut=\ds@listIn
        \global\ds@qlistOut=\expandafter{\the\ds@listIn}
    \fi
}
%    \end{macrocode}
% Develop a random permuted list for the answers. Note that \cs{ifwerandomize}
% we randomize, else, the output list is the same as the input list.
%    \begin{macrocode}
\def\ds@randomizeAnswerList#1{%
    \global\ds@listIn={}\global\ds@newListIn={}\global\ds@listOut={}%
    \global\ds@tmpToks={}%\global\ds@qlistOut={}\global\ds@alistOut={}%
    \ds@nMax=#1\relax\ds@populateList{\the\ds@nMax}%
    \global\forquestionsfalse
    \ifwerandomize
%       \nextrandom \nextrandom \nextrandom
        \expandafter\ds@randomizeList
    \else
%        \global\let\ds@alistOut=\ds@listIn
        \global\ds@alistOut=\expandafter{\the\ds@listIn}%
    \fi
}
%    \end{macrocode}
% The loop that does all the work for randomizing.
%    \begin{macrocode}
\def\ds@randomizeList{%
    \let\\=\ds@processi
    \setrannum{\ds@getRanNum}{1}{\ds@nMax}%
    \ds@nCnt=0\relax
%    \typeout{LISTING: \the\ds@listIn}%
    \the\ds@listIn
    \ds@loopTest
}
%    \end{macrocode}
% The loop that does all the work for randomizing.
%    \begin{macrocode}
\def\ds@loopTest{\advance\ds@nMax-1\relax
    \ifnum\ds@nMax>0\relax
        \def\@next{%
            \ds@listIn=\expandafter{\the\ds@newListIn}%
            \ds@newListIn={}\ds@randomizeList}%
    \else
        \let\@next\relax
        \ifforquestions
            \global\ds@qlistOut=\expandafter{\the\ds@listOut}%
%            \typeout{\ds@qlistOut = \the\ds@qlistOut}%
        \else
%            \global\ds@alistOut=\expandafter{\@tmp}
            \global\ds@alistOut=\expandafter{\the\ds@listOut}%
%            \typeout{\ds@alistOut = \the\ds@alistOut}%
        \fi
    \fi
    \@next
}
%    \end{macrocode}
%    \begin{macro}{\placeMessageField}
% The command inserts the required message field. The optional
% first parameters enables the author to change the appearance
% of the field, the second two required arguments are the width
% and the height of the text field.
%    \begin{macrocode}
\newcommand{\placeMessageField}[3][]{%
    \ifeqforpaper\else\textField[#1\Ff\FfMultiline]{report}{#2}{#3}\fi
}
%    \end{macrocode}
%    \end{macro}
% \section{Miscellaneous Settings}
%    \begin{macro}{\threshold}
%    \begin{macro}{\penaltypoints}
%    \begin{macro}{\passing}
% Threshold for number of incorrect answers for trying to answer one question.
% If the threshold is exceeded, \cs{dspenaltypoints} are added to the final
% points. Passing is missing no more than \cs{dspassing}.
%    \begin{macrocode}
\newcommand{\threshold}[1]{\def\dsthreshold{#1}}
\threshold{3}
\newcommand{\penaltypoints}[1]{\def\dspenaltypoints{#1}}
\penaltypoints{3}
\newcommand{\passing}[1]{\def\dspassing{#1}}
\passing{4}
%    \end{macrocode}
%    \end{macro}
%    \end{macro}
%    \end{macro}
%    \begin{macrocode}
%</package>
%    \end{macrocode}
% \section{Language Cutomizations}
%    \begin{macrocode}
%<*english>
\def\chooseQ{"You must choose a question to answer before you answer!"}
\def\triedTooMuch{"You have tried this problem too many times,
    I'm adding "
    + \dspenaltypoints
    + " points, and resetting the penalty counter. Bad boy/girl!"}
\def\congratFinished{"Congratulations! You finished the puzzle"
        +((nMissed==0) ? " without missing a single problem, amazing!"
        : ", but you missed " + nMissed + " questions in the process!")}
\def\regretPleased{(( nMissed > n )
        ? "I regret to report that you did not pass the test because
            you missed too many questions."
        : "I am pleased to report that you passed the test!")}
\def\reportPenaltyPoints{"The number of penalty points is "
    + nPenaltyPoints +"."}
\def\finalPenaltyScore{"Final penalty score is " + nTotalPenaltyPoints + ". "}
\def\aPenaltyScale{[-1,0], [0,4], [4, 10], [10,25], [25,5000]}
\def\aPenaltyMsgs
{%
    "Perfect!",
    "Very nice performance!",
    "This is not looking good. Perhaps a review is in order!",
    "Are you trying? No one could do so badly, you only need a seventh grade education!",
    "You're hopeless!"
}
%</english>
%<*german>
\def\chooseQ{"Du musst erst eine Frage ausw\string\344hlen bevor Du antwortest!"}
\def\triedTooMuch{"Du hast es leider zu oft versucht, ich z\string\344hle  "
    + \dspenaltypoints
    + " Punkte dazu und setze den Z\string\344hler dann zur\string\374ck. Bitte streng Dich an!"}
\def\congratFinished{"Herzlichen Gl\string\374ckwunsch! Du hast das Puzzle beendet"
        +((nMissed==0) ? "  ohne auch nur einen Fehler zu machen, wunderbar!"
        : ", aber leider " + nMissed + " Mal falsch geantwortet!")}
\def\regretPleased{(( nMissed > n )
        ? "Ich bedauere Dir mitteilen zu m\string\374ssen, dass Du den Test leider nicht bestanden hast, weil Du zu viele Fragen falsch beantwortet hast."
        : "Ich bin \string\344usserst erfreut Dir mitteilen zu d\string\374rfen, dass Du den Test bestanden hast!")}
\def\reportPenaltyPoints{"Die Anzahl der Strafpunkte ist  " + nPenaltyPoints +"."}
\def\finalPenaltyScore{"Die Gesamtanzahl der Strafpunkte ist damit " + nTotalPenaltyPoints + ". "}
\def\aPenaltyScale{[-1,0], [0,4], [4, 10], [10,25], [25,5000]}
\def\aPenaltyMsgs
{%
    "Perfekt!",
    "Sehr gute Vorstellung!",
    "Es sieht nicht sonderlich gut aus. Vielleicht w\string\344re eine Wiederholung gut!",
    "R\string\344tst Du eigentlich nur? Niemand kann wirklich so schlecht sein. Das ist Stoff aus der siebten Klasse!",
    "Du bist ein hoffnungsloser Fall!"
}
%</german>
%    \end{macrocode}
% \section{JavaScript Support}
%    \begin{macrocode}
%<*package>
\begin{insDLJS}[processChoice]{match}{JS Update of Matching Puzzle}
var missesByQuestion = new Array();
var nMissed = 0;
var nPenaltyPoints = 0;
var activeQuestion = "";
function processChoice(fieldname)
{
    // Get the question field that corresponds to this question,
    // see if checked.
    var f = this.getField("ckbxQ."+fieldname);
    if ( (f != null) && (f.isBoxChecked(0)) ) { // right
        clearRedCrosses ();
        this.resetForm(["puzzle."+fieldname]);
        event.target.textColor = ["RGB", 0, 0.6, 0];
        f.strokeColor = ["RGB", 0, 0.6, 0];
        f.readonly = true;
        event.target.readonly=true;
        try { afterCorrectChoiceHook() } catch(e) {};
        checkForFinished(\dspassing);
    } else { // wrong
        if ( activeQuestion != "" ) var h = this.getField(activeQuestion);
        if ( (activeQuestion=="") || (h.readonly) ) { // active question already answered
            event.target.value = "Off";
            var g = this.getField("report");
            str = \chooseQ;
            g.value = str;
            var to = app.setTimeOut("clearMessages()", 2000);
        } else {
            event.target.style = style.cr;
            event.target.textColor = color.red;
            ++nMissed;
            if ( typeof missesByQuestion[activeQuestion] !="number" )
                missesByQuestion[activeQuestion] = 1;
            else
                missesByQuestion[activeQuestion] += 1;
            if (missesByQuestion[activeQuestion] > \dsthreshold) {
                var f = this.getField("report");
                str = \triedTooMuch;
                f.value = str;
                nPenaltyPoints += \dspenaltypoints;
                missesByQuestion[activeQuestion] = 0;
                clearRedCrosses ();
                var to = app.setTimeOut("clearMessages()", 2000)
            }
        }
    }
}
function clearMessages()
{
    var f = this.getField("report");
    f.value = "";
}
function clearRedCrosses ()
{
    var g = this.getField("ckbxA");
    h = g.getArray();
    for ( i=0; i < h.length; i++) {
        if ( h[i].style == style.cr) h[i].checkThisBox(0,false);
        h[i].style = style.ch;
    }
 }
function checkForFinished(n)
{
    var f = this.getField("puzzle");
    var g = f.getArray();
    var anyEmpty = false;
    for ( var i=0; i < g.length; i++) {
        if ( (g[i].name != "puzzle.space") && (g[i].value.replace(/\s/g,"") == "") ) {
            anyEmpty = true;
            break;
        }
    }
    var nTotalPenaltyPoints=nMissed + nPenaltyPoints;
    if ( !anyEmpty ) {
        try { dpsFinishedHook() } catch (e) {};
        var f = this.getField("report");
        str = \congratFinished
        + "\n" + \regretPleased
        + "\n" + \reportPenaltyPoints
        + "\n" + \finalPenaltyScore
        + " " + finalRating(nTotalPenaltyPoints);
        f.value = str;
    }
}
function finalRating(n) {
    var aPenaltyScale = new Array( \aPenaltyScale\space);
    var aPenaltyMsgs = new Array( \aPenaltyMsgs\space);
    for ( var i=0; i<aPenaltyScale.length; i++)
        if ( (n > aPenaltyScale[i][0]) && (n <= aPenaltyScale[i][1]) )
            return aPenaltyMsgs[i];
}
\end{insDLJS}
%</package>
%    \end{macrocode}
%  \Finale
\endinput

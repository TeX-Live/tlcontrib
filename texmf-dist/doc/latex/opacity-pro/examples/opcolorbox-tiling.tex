\documentclass{article}
\usepackage{graphicxsp}
\usepackage[tight,designiv,usetemplates,usesf]{web}
\usepackage{aeb_tilebg}
\usepackage{digicap-pro}

\title{The \textsf{Opacity Pro} Package
    \texorpdfstring{\\\cs{opcolorbox}, Transparency, Tiling}
        {: \textbackslash{opcolorbox}, Transparency, Tiling}}
\author{D. P. Story}
\subject{Test file for digicap-ro}
\keywords{LaTeX, Web package, tiled backgrounds, Adobe Acrobat, opacity-pro}
\university{Acro\negthinspace\TeX.Net}
\email{dpstory@acrotex.net}
\def\webversion{\textcolor{webbrown}{www.acrotex.net}}
\revisionLabel{Prepared:}
\versionLabel{}

\graphicspath{{./digis}}

\embedEPS[transparencyGroup]{cle_ind_back}{bg_cle_tile}

\newcommand{\cs}[1]{\texttt{\char`\\#1}}

\parindent0pt\parskip\medskipamount


\begin{document}

\maketitle

\newpage

\setTileBgGraphic[hiresbb,scale=.4,name=cle_ind_back]{\null}

\null\vskip-\baselineskip\vfil

\begin{center}
    \opcolorbox[%
      borderwidth=4pt,
      fboxsep=10pt,
      width=0.75\linewidth,
      bordercolor=blue,      % bordercolor=named color | nocolor, default blue
      bgcolor=white,         % bgcolor=named color | nocolor, default white
      borderop=.5,
      bgop=.4,
    ]{%
        \parskip6pt\bfseries Someone asked me if the border could be
        made transparent. On first blush, I said, ``No! Not at this
        time.'' The latter phrase I threw in to cover myself in case
        the answer is ``Yes!''
     }
\end{center}

\end{document}

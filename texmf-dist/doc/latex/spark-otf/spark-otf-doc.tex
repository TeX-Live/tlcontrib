%% $Id: lucida-otf-doc.tex 556 2017-09-18 06:22:29Z herbert $
% Copyright 2012-2015 TeX Users Group.
% 
% Copying and distribution of this file, with or without modification,
% are permitted in any medium, without royalty.

\listfiles
\documentclass[11pt]{article}
\usepackage{geometry}

\usepackage%[usefilenames,]
  {spark-otf}  % support opentype math fonts

\usepackage{biblatex}
\addbibresource{\jobname.bib}
\usepackage{array}
\usepackage{metalogo} % for \XeTeX logo
\usepackage{booktabs} % for examples
\usepackage{listings} % for code
\usepackage{dtk-logos} % for Wikipedia W

\pagestyle{headings}

\usepackage[colorlinks,hyperfootnotes=false]{hyperref}
% define \code for url-like breaking of typewriter fragments.
\ifx\nolinkurl\undefined \let\code\url \else \let\code\nolinkurl \fi

% Define \cs to prepend a backslash, and be unbreakable:
\DeclareRobustCommand\cs[1]{\mbox{\texttt{\char`\\#1}}}

% An environment like quote, but less space above and more below:
\newenvironment{demoquote}
   {\tabularx{\dimexpr\linewidth+\marginparwidth}{@{} X >{\ttfamily}l @{}}}
   {\endtabularx}


\title{Support for the Spark OpenType fonts}
\author{Herbert Voß}
\begin{document}
\maketitle
\tableofcontents

\section{Introduction}

A sparkline is a very small line chart, typically drawn without axes or coordinates. 
It presents the general shape of the variation (typically over time) in some measurement, 
such as temperature or stock market price, in a simple and highly condensed way. 
Sparklines are small enough to be embedded in text, or several sparklines may be 
grouped together as elements of a small multiple. Whereas the typical chart is 
designed to show as much data as possible, and is set off from the flow of text, 
sparklines are intended to be succinct, memorable, and located where they are discussed.


\section{The fonts}

The fonts are available from \url{https://github.com/aftertheflood/spark} and should be saved
either in \path{Library/fonts/} (MAC OSX), \path{c:\Windows\Fonts} (Windows) or
\path{/usr/local/share/fonts} (Linux)  or any other location where
the fonts will be found by the system.

\begin{verbatim}
-rw-r--r-- 1 voss voss 24708 Sep 15 11:20 Spark - Bar - Medium.otf
-rw-r--r-- 1 voss voss 24696 Sep 15 11:20 Spark - Bar - Narrow.otf
-rw-r--r-- 1 voss voss 24680 Sep 15 11:20 Spark - Bar - Thin.otf
-rw-r--r-- 1 voss voss 22140 Sep 15 11:20 Spark - Dot-line - Medium.otf
-rw-r--r-- 1 voss voss 24616 Sep 15 11:20 Spark - Dot - Medium.otf
-rw-r--r-- 1 voss voss 24580 Sep 15 11:20 Spark - Dot - Small.otf
\end{verbatim}


The package defines the following families:

\small
\begin{verbatim}
\newfontfamily\sparkBarMedium{SparkBar-Medium}[RawFeature=+calt,\spark@DefaultFeatures]
\newfontfamily\sparkBarNarrow{SparkBar-Narrow}[RawFeature=+calt,\spark@DefaultFeatures]
\newfontfamily\sparkBarThin{SparkBar-Thin}[RawFeature=+calt,\spark@DefaultFeatures]
%
\newfontfamily\sparkDotLine{Spark-Dot-lineMedium}[RawFeature=+calt,\spark@DefaultFeatures]
%
\newfontfamily\sparkDotMedium{Spark-DotMedium}[RawFeature=+calt,\spark@DefaultFeatures]
\newfontfamily\sparkDotSmall{Spark-DotSmall}[RawFeature=+calt,\spark@DefaultFeatures]
\end{verbatim}

\normalsize

\section{The macros}

\begin{verbatim}
\sparkBar[<Type>][<No>]{values}[<No>]
\sparkDot[<Type>][<No>]{values}[<No>]
\end{verbatim}

If \texttt{[<Type>]} is missing, \texttt{Medium} is assumed. The type is mandatory if you use
the first \texttt{[<No>]} argument!



\section{Text examples}

\footnotesize
\begin{verbatim}
\begin{description}
\item[Bar-Medium] Text \sparkBar{14,95,68,9,19,41,91,1,81,97,79,45,96,76,17,65,8,92} Text
\item[Bar-Medium] Text \sparkBar[Medium]{14,95,68,9,19,41,91,1,81,97,79,45,96,76,17,65,8,92} Text
\item[Bar-Narrow] Text \sparkBar[Narrow]{19,32,93,4,95,46,13,23,50,86,94,68,58,41,89,57,74,8} Text
\item[Bar-Thin] Text \sparkBar[Thin]{13,15,59,73,42,1,41,51,4,97,35,55,37,24,89,21,30,22} Text
\item[Bar-Medium] Text \sparkBar[Medium][14]{14,95,68,9,19,41,91,1,81,97,79,45,96,76,17,65,8,92}[92] Text
\item[Bar-Narrow] Text \sparkBar[Narrow][19]{19,32,93,4,95,46,13,23,50,86,94,68,58,41,89,57,74,8}[8] Text
\item[Bar-Thin] Text \sparkBar[Thin][13]{13,15,59,73,42,1,41,51,4,97,35,55,37,24,89,21,30,22}[22] Text
\end{description}
\end{verbatim}

\normalsize
\begin{description}
\item[Bar-Medium] Text \sparkBar{14,95,68,9,19,41,91,1,81,97,79,45,96,76,17,65,8,92} Text
\item[Bar-Medium] Text \sparkBar[Medium]{14,95,68,9,19,41,91,1,81,97,79,45,96,76,17,65,8,92} Text
\item[Bar-Narrow] Text \sparkBar[Narrow]{19,32,93,4,95,46,13,23,50,86,94,68,58,41,89,57,74,8} Text
\item[Bar-Thin] Text \sparkBar[Thin]{13,15,59,73,42,1,41,51,4,97,35,55,37,24,89,21,30,22} Text
\item[Bar-Medium] Text \sparkBar[Medium][14]{14,95,68,9,19,41,91,1,81,97,79,45,96,76,17,65,8,92}[92] Text
\item[Bar-Narrow] Text \sparkBar[Narrow][19]{19,32,93,4,95,46,13,23,50,86,94,68,58,41,89,57,74,8}[8] Text
\item[Bar-Thin] Text \sparkBar[Thin][13]{13,15,59,73,42,1,41,51,4,97,35,55,37,24,89,21,30,22}[22] Text
\end{description}


\footnotesize
\begin{verbatim}
\begin{description}
\item[Dot-Medium] Text \sparkDot{54,39,26,65,29,58,36,99,16,56,76,69,71,77,7,40,79,1} Text
\item[Dot-Medium] Text \sparkDot[Medium]{54,39,26,65,29,58,36,99,16,56,76,69,71,77,7,40,79,1} Text
\item[Dot-Small] Text \sparkDot[Small]{1,79,88,46,54,77,91,24,70,22,27,29,40,33,31,95,26,76} Text
\item[Dot-Line] Text \sparkDot[Line]{9,4,2,1,6,7,3,8,3,7,1,4,9,2,8,5,1,8} Text
\item[Dot-Medium] Text \sparkDot[Medium][54]{54,39,26,65,29,58,36,99,16,56,76,69,71,77,7,40,79,1}[1] Text
\item[Dot-Small] Text \sparkDot[Small][1]{1,79,88,46,54,77,91,24,70,22,27,29,40,33,31,95,26,76}[76] Text
\item[Dot-Line] Text \sparkDot[Line][9]{9,4,2,1,6,7,3,8,3,7,1,4,9,2,8,5,1,8}[8] Text
\end{description}
\end{verbatim}

\normalsize
\begin{description}
\item[Dot-Medium] Text \sparkDot{54,39,26,65,29,58,36,99,16,56,76,69,71,77,7,40,79,1} Text
\item[Dot-Medium] Text \sparkDot[Medium]{54,39,26,65,29,58,36,99,16,56,76,69,71,77,7,40,79,1} Text
\item[Dot-Small] Text \sparkDot[Small]{1,79,88,46,54,77,91,24,70,22,27,29,40,33,31,95,26,76} Text
\item[Dot-Line] Text \sparkDot[Line]{9,4,2,1,6,7,3,8,3,7,1,4,9,2,8,5,1,8} Text
\item[Dot-Medium] Text \sparkDot[Medium][54]{54,39,26,65,29,58,36,99,16,56,76,69,71,77,7,40,79,1}[1] Text
\item[Dot-Small] Text \sparkDot[Small][1]{1,79,88,46,54,77,91,24,70,22,27,29,40,33,31,95,26,76}[76] Text
\item[Dot-Line] Text \sparkDot[Line][9]{9,4,2,1,6,7,3,8,3,7,1,4,9,2,8,5,1,8}[8] Text
\end{description}


\nocite{*}
\printbibliography


\end{document}


<p class=spark-bar-medium>{14,95,68,9,19,41,91,1,81,97,79,45,96,76,17,65,8,92}</p>

<p class=spark-bar-narrow>{19,32,93,4,95,46,13,23,50,86,94,68,58,41,89,57,74,8}</p>

<p class=spark-bar-thin>{13,15,59,73,42,1,41,51,4,97,35,55,37,24,89,21,30,22}</p>

<p class=spark-dot-medium>{54,39,26,65,29,58,36,99,16,56,76,69,71,77,7,40,79,1}</p>

<p class=spark-dot-small>{1,79,88,46,54,77,91,24,70,22,27,29,40,33,31,95,26,76}</p>

<p class=spark-line-medium>{9,4,2,1,6,7,3,8,3,7,1,4,9,2,8,5,1,8}</p>
\NewDocumentCommand\spark{omo}{{\sparklinesbarmedium
		\IfValueT{#1}{#1}\string{#2\string}\IfValueT{#3}{#3}}}

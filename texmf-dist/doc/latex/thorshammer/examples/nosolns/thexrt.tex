\documentclass{article}
\usepackage{amstext}
\usepackage{web}
\usepackage[usesumrytbls,allowrandomize]{exerquiz}
\usepackage{ran_toks}
\usepackage[usebatch]{thorshammer}

\DeclareQuiz{q1}

% It is important to freeze the seed so that (1) you can reproduce the exact
% same quiz at a later time; (2) allow content written to the AUX file to
% come up to date. This is important when using summary tables.
\useRandomSeed{1344524586} 

\setInitMag{fitwidth}
\hypersetup{pdfpagemode=UseNone} % don't need to see bookmarks
\hypersetup{pdfpagelayout=OneColumn}
\reversemarginpar

\showCreditMarkup

% \previewOn\pmpvOn

\useBeginQuizButton[\CA{Begin}]
\useEndQuizButton[\CA{End}]

\PTsHook{($\eqPTs^{\text{pts}}$)}

\useMCCircles

% reset the paths for \instrPath and \classPath for your system
\instrPath{/C/Users/dpstory/Desktop/Test Folder/target/_Thor}
\classPath{/C/Users/dpstory/Desktop/Test Folder/target/myClass}
\classMember{Peter}{Pan}{A/_Thor}
\classMember{J\oct374rgen}{Gilg}{B/_Thor}
\classMember{Thors}{Hammer}{C/_Thor}

\autoCopyOn

%\distrToInstrOff
%\distrToStudentsOff

\begin{makeClassFiles}
\sadQuizzes
\end{makeClassFiles}

\begin{document}

\noindent In this file, we attempt to \emph{duplicate} and
\emph{extend} the work of \texttt{theexuc.tex} by developing a
scheme to randomize choices and randomize the questions. To
randomize the choices, we modify the preamble to include the
option \texttt{allowrandomize} of \textsf{exerquiz}, to
randomize the order of the questions, we include the
\textsf{ran\_toks} package. The `quiz body' concept is used
to introduced multiple equivalent quizzes.


\newpage


\declareQuizBody{qzbody1}
\declareQuizBody{qzbody2}

% The qzbody env encloses the entire body of the quiz
\begin{qzbody1}

\bRTVToks{\currQuiz}

\thQuizHeader

\noindent\textbf{Instructions:} (For the student) Press
`\textsf{Begin}' to begin the quiz; after completing the quiz,
press `\textsf{End}'. Use the `\textsf{Save}' button to save the
document.

\begin{quiz*}{\currQuiz}
Solve each of these problems, passing is 100\%.
\begin{questions}

\begin{rtVW}
  \item\PTs{3} Which of these are true ?
\begin{answers}{4}
\Ans1 True & \Ans0 False
\end{answers}
\end{rtVW}

\begin{rtVW}
  \item \PTs{4} Select which of the following is true.
\begin{answers}{4}
\Ans1 True & \Ans0 False & \Ans0 Maybe & \Ans0 Sometimes
\end{answers}
\end{rtVW}

\begin{rtVW}
  \item\PTs{2} $9+8=\RespBoxMath{17}{1}{.0001}{[0,1]}$
\end{rtVW}

\begin{rtVW}
\essayQ{5} % num points assigned
\item\PTs{5} Write a short history of Acro\negthinspace\TeX.\par
\RespBoxEssay{\linewidth}{1in}
\end{rtVW}

\begin{rtVW}
\item\PTs{3} Which of the following are numbers?
\begin{manswers}{6}
\bChoices[random=true]
  \Ans[-1]{0}d\eAns
  \Ans[1]{1}17\eAns
  \Ans[-1]{0}p\eAns
  \Ans[1]{1}88\eAns
  \Ans[-1]{0}s\eAns
  \Ans[1]{1}105\eAns
\eChoices
\end{manswers}
\end{rtVW}


\begin{rtVW}
\multipartquestion
    \item\PTs{20} Answer each of the following multiple selection problems. Each correct answer
    is worth $3$ points, and each incorrect answer is worth $-2$ points.
    \begin{questions}

\rowsep{3pt}

        \item\PTs{9} Select which people who served as a President
                     of the United States. (Select all correct choices.)

        \begin{manswers}*{2}%
            \bChoices[random=true]
                \Ans[-2]{0} Henry Clay\eAns
                \Ans[-2]{0} Ben Franklin\eAns
                \Ans[3]{1}  Andrew Jackson\eAns
                \Ans[3]{1}  Ronald Reagan\eAns
                \Ans[-2]{0} George Meade\eAns
                \Ans[3]{1}  Grover Cleveland\eAns
                \Ans[-2]{0} John Jay\eAns
                \Ans[-2]{0} Paul Revere\eAns
            \eChoices
        \end{manswers}

\rowsep{3pt}

        \item\PTs{9} Select which people who served as a Chancellor of the
            German Republic. (Select all correct choices.)
        \begin{manswers}*{2}%
            \bChoices[nCols=2,random=true]
                \Ans[-2]{0} Gustav Heinemann\eAns
                \Ans[-2]{0} Theodor Heu{\ss}\eAns
                \Ans[3]{1}  Konrad Adenauer\eAns
                \Ans[-2]{0} Richard von Weizs\"{a}cker\eAns
                \Ans[3]{1}  Willy Brandt\eAns
                \Ans[-2]{0} Heinrich L\"{u}bke\eAns
                \Ans[-2]{0} Roman Herzog\eAns
                \Ans[3]{1} Ludwig Erhard\eAns
            \eChoices
        \end{manswers}

\rowsep{3pt}

      \item\PTs{2} If you select all choices in part~(a), you will
          receive -1 points as a penalty for bad guessing. \textbf{Question:}
          Determine the \emph{number of correct choices} in part~(a)?
          \begin{answers}{4}
          \bChoices[random=true]
            \Ans0 1\eAns
            \Ans0 2\eAns
            \Ans1 3\eAns
            \Ans0 4\eAns
            \Ans0 5\eAns
            \Ans0 6\eAns
            \Ans0 7\eAns
            \Ans0 8\eAns
          \eChoices
          \end{answers}
    \end{questions}
\end{rtVW}

\eRTVToks

% Ok, now display this questions in a random order.
\displayListRandomly{\thisQuiz}

\end{questions}
\writeProListAux
\end{quiz*}\quad\thQuizTrailer

\end{qzbody1}

\begin{qzbody2}

\thQuizHeader

\begin{quiz*}{\currQuiz}
Solve each of these problems, passing is 100\%.
\begin{questions}

\essayQ{10} % num points assigned
\item\PTs{10} Comment on the experience of taking a quiz the `Thorsten way.'\par
\RespBoxEssay{\linewidth}{1in}

\end{questions}
\writeProListAux
\end{quiz*}\quad\thQuizTrailer

\end{qzbody2}

% Now we input the qzbody back in two times, though it can be more than that. The quiz name
% modified in each instance.
%
% Each instance of the quiz has a randomized order
%
\InputQuizBody{qzbody1}

\InputQuizBody{qzbody1}

\InputQuizBody{qzbody2}


\end{document}

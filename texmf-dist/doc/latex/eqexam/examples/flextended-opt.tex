\documentclass{article}
\usepackage[fleqn]{amsmath}
\usepackage[pdf,forcolorpaper,nopoints,useforms,
% Try compiling this file under one of these options:
    nosolutions,
%   answerkey,
%   vspacewithsolns,
  flextended
]{eqexam}[2018/01/12]

\subject[Essay]{Extending Fill Lines}
\title[HW]{Essay HW}
\author{Dr.\ D. P. Story}
\date{Spring 2018}
\duedate{03/07/2018}
\keywords{Homework due \theduedate}

\forceNoColor
\DoNotFitItIn
\useFillerLines
\vspacewithkeyOn
\turnflnosolnsOn
\turnflanskeyOn
\turnContAnnotOn
\setFillLinesFmt{fltype=line,numbers=left}

\solAtEndFormatting{\eqequesitemsep{3pt}}

\let\opt\texttt
\let\env\texttt

% these commands pass their arguments to the priorworkarea and solution
% environments, respectively.
\priorworkareaCmds{\baselineskip\wlVspace\parindent\wlVspace}
\solutionafterExCmds{\baselineskip\wlVspace
  \parindent\wlVspace}

\begin{document}

\maketitle

\begin{exam}{fl1}

\begin{instructions}[]
Respond to each problem, use your best effort. Turn in your essays by the end of the day.
\end{instructions}

\flPageBreakMsg{\textbf{Problem~{\eqeCurrProb} continues on next page\strut}}%

\begin{problem}
Expound on all you know on the subject.
\begin{priorworkarea}
When the \opt{flextended} option is not in force, these three lines are blank. However, in this example,
\opt{flextended} is active, and I can write to these lines. 
\end{priorworkarea}
\begin{solution}[nLines=3]
Your guess is as good as mine. 
\end{solution}
\end{problem}

\setFillLinesFmt{fltype=grid,gridtype=line,numbers=right,topline,color=lightgray}

\begin{problem}
Expound on all you know on the subject.
\begin{priorworkarea}
Essay area.
\end{priorworkarea}
\begin{solution}[42pt]
Your guess is as good as mine.
\end{solution}
\end{problem}

\vspace{2.1in}
\smash{\makebox[\linewidth][c]{\parbox{.5\linewidth}{\bfseries\slshape
  This space left blank so the next problem can break across the page.}}}
\vspace{2.1in}

\begin{problem}
Expound on all you know on the subject.
\begin{priorworkarea}
When the \opt{flextended} option is not in force, these three lines are
blank. However, in this example, \opt{flextended} is active, and I can write
to these lines.

The contents of the \env{priorworkarea} environment can also break across a
page boundary. We'll try to make it so.\vspace{2\baselineskip}

Are we on the next page? I think yes.
\end{priorworkarea}
\begin{solution}[nLines=10]
Your guess is as good as mine. The content has the capability of breaking across pages.

I've added more lines because this problem has a chance breaking across a page boundary.
Let's prattle on until we go to the next page.\vspace{\baselineskip}

We'll jump down a couple of lines cause I don't have much to say. Gotta keep
going to get to the next page. If all works as it should, I'll see you one
the other side! Perfect! \vspace{3\baselineskip}

\noindent
All in all, this is very cool.\enspace dps
\end{solution}
\end{problem}
\end{exam}

\end{document}

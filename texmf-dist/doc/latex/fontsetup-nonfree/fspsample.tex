%% after xelatex run
%% pdfcrop fspsample.pdf output.pdf
%% to crop it
\documentclass{article}
\pagestyle{empty}
\usepackage{xgreek,graphicx}
\pagestyle{empty}
%\usepackage[greek]{babel}
%\usepackage[utf8x]{inputenc}
%\usepackage{amsfonts}

\usepackage[default]{fontsetup}
%\usepackage[gfsartemisia]{fontsetup}
%\usepackage[gfsdidot]{fontsetup}
%\usepackage[gfsdidotclassic]{fontsetup}
%\usepackage[gfsneohellenic]{fontsetup}
%\usepackage[cambria]{fontsetup}
%\usepackage[lucida]{fontsetup}
%\usepackage[kerkis]{fontsetup}
%\usepackage[fira]{fontsetup}
%\usepackage[times]{fontsetup}
%\usepackage[palatino]{fontsetup}
%\usepackage[stixtwo]{fontsetup}
%\usepackage[neokadmus]{fontsetup}
%\usepackage[msgaramond]{fontsetup}
%\usepackage[ebgaramond]{fontsetup}
%\usepackage[minion]{fontsetup}
%\usepackage[neoeuler]{fontsetup}
%\usepackage[libertinus]{fontsetup}
%\usepackage[olddefault]{fontsetup}



\newtheorem{theorem}{Theorem}
\newtheorem{theoremg}[theorem]{Θεώρημα}


\begin{document}

\begin{theorem}[Dominated convergence of Lebesgue]
%Let $g$ be an
Assume that $g$ is an
in\-te\-grable func\-tion defined on the measurable set $E$ and that
  $(\,f_n)_{n\in\mathbb N}$ is a sequence of mea\-sur\-able functions so that
  $|\,f_n|\leq g$. If $f$ is a function so that $f_n\to f$ almost everywhere
  then $$\lim_{n\to\infty}\int f_n=\int f.$$
\end{theorem}
\textit{Proof}: The function $g-f_n$ is non-negative and thus from Fatou lemma
we have that $\int(g-f\,)\leq\liminf\int(g-f_n)$. Since $|\,f\,|\leq g$ and
$|\,f_n|\leq g$ the  functions $f$ and $f_n$ are integrable and we have
$$\int g-\int f\,\leq \int g-\limsup\int f_n,$$ so
$$\int f\,\geq \limsup \int f_n.$$
\par
\begin{theoremg}[Κυριαρχημένης σύγκλισης του Lebesgue]
  Έστω ότι
η $g$ είναι μια ολοκληρώσιμη συνάρτηση ορισμένη στο μετρήσιμο σύνολο
$E$ και η $(\,f_n)_{n\in\mathbb N}$ είναι μια ακολουθία μετρήσιμων συναρτήσεων ώστε
$|\,f_n| ≤ g$. Υποθέτουμε ότι υπάρχει μια συνάρτηση $f$
ώστε η  $(f_n)_{n\in\mathbb N}$ να
τείνει στην $f$ σχεδόν παντού. Τότε
$$\lim \int f_n =\int f.$$
\end{theoremg}
\textit{Απόδειξη}: Η συνάρτηση $g − f_n$ είναι μη αρνητική και άρα από
το Λήμμα του Fatou ισχύει
$\int (f-g) ≤ \liminf \int (g-f_n)$. Επειδή
$|\,f\,| ≤ g$ και $|\,f_n| ≤g$ οι $f$ και $f_n$ είναι ολοκληρώσιμες, έχουμε
$$\int g −\int f\, ≤ \int g − \limsup\int f_n,$$
άρα
$$\int f\,\geq \limsup \int f_n.$$





\end{document}

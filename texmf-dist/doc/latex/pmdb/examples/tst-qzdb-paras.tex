\documentclass{book}
\usepackage{amstext}
\usepackage[forcolorpaper]{web}
\usepackage{exerquiz}[2019/12/17]
\usepackage[dbmode,!tight]{pmdb}

\reversemarginpar

\def\cs#1{\texttt{\eqbs#1}}

\useBeginQuizButton[\CA{Begin}]
\useEndQuizButton[\CA{End}]

%\previewOn\pmpvOn

\rfooter{\dirTOCItem} % \dirTOCItem from web

% Make quiz solutions into a chapter* event
\InputQzSolnsLevel*[ch:Qz]{chapter} % exerquiz 2019/12/17


\title{Poor Man's Database}
\author{D. P. Story}
\email{dpstory@acrotex.net}

\optionalPageMatter{\begin{center}
  \fbox{\begin{minipage}{.67\linewidth}
    This file demonstrates how to input both quiz items and paragraph content.
    Two commands are defined \cs{InputQuizItems} and \cs{InputParas} to make it
    convenient to switch between the two modes of input.
  \end{minipage}}
\end{center}}

\begin{document}

\maketitle

\tableofcontents

\chapter{The quizzes}

\section{Section 1}

Expand the \cs{InputQuizItems} here.

\InputQuizItems

\begin{quiz*}{myquiz1}
Solve each of these problems, passing is 100\%.
\begin{questions}

\pmInput{probs/prob1.tex}

\pmInput{probs/prob2.tex}

\pmInput{probs/prob3.tex}

\end{questions}
\end{quiz*}

\section{Section 2}

\begin{quiz*}{myquiz2}
Solve each of these problems, passing is 100\%.
\begin{questions}

\pmInput{probs/prob4.tex}

\pmInput{probs/prob6.tex}

\pmInput{probs/prob5.tex}

\end{questions}
\end{quiz*} %\ScoreField\currQuiz\CorrButton\currQuiz

\medskip\noindent
\displayChoices{}{11bp}


\medskip\noindent
Expand the \cs{InputParas} here.
\InputParas

\pmInput{chapters/doc1.tex}

%\input{"my doc2.tex"}
\pmInput{"C:/Users/Public/Documents/My TeX Files/tex/latex/aeb/pmdb/examples/chapters/doc2.tex"}

\medskip\noindent
\displayChoices{}{11bp}\cgBdry[1em]\clrChoices{}{11bp}

\end{document} 
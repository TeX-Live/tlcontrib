% \iffalse
%<*copyright>
%%%%%%%%%%%%%%%%%%%%%%%%%%%%%%%%%%%%%%%%%%%%%%%%%%%%%%%%%%
%% annot_pro package,                                   %%
%% Copyright (C) 2002-2017  D. P. Story                 %%
%%   dpstory@uakron.edu                                 %%
%%                                                      %%
%% This program can redistributed and/or modified under %%
%% the terms of the LaTeX Project Public License        %%
%% Distributed from CTAN archives in directory          %%
%% macros/latex/base/lppl.txt; either version 1 of the  %%
%% License, or (at your option) any later version.      %%
%%%%%%%%%%%%%%%%%%%%%%%%%%%%%%%%%%%%%%%%%%%%%%%%%%%%%%%%%%
%</copyright>
%<package>\NeedsTeXFormat{LaTeX2e}
%<package>\ProvidesPackage{annot_pro}
%<package> [2018/04/26 v1.2 Provides selected PDF annotation support (dps)]
%<*driver>
\documentclass{ltxdoc}
\usepackage[fleqn]{amsmath}
\usepackage[colorlinks,hyperindex=false]{hyperref}
\let\uif\textsf\let\app\textsf
\let\pkg\textsf\let\env\texttt
\def\psf#1{\textsf{\textbf{#1}}}
\let\opt\texttt
%\def\texorpdfstring#1#2{#1}
%\pdfstringdefDisableCommands{\let\\\textbackslash}
\OnlyDescription  % comment out for implementation details
\EnableCrossrefs \CodelineIndex \RecordChanges
\InputIfFileExists{aebdocfmt.def}{\PackageInfo{annot_pro}{Inputting aebdocfmt.def}}
    {\def\IndexOpt{\DescribeMacro}\def\IndexKey{\DescribeMacro}\let\setupFullwidth\relax
     \PackageInfo{annot_pro}{aebdocfmt.def cannot be found}}
\begin{document}
  \GetFileInfo{annot_pro.sty}
  \title{annot\_pro: Selected PDF Annotations}
  \author{D. P. Story\\
    Email: \texttt{dpstory@uakron.edu}}
  \date{processed \today}
  \maketitle
  \tableofcontents
  \let\Email\texttt
  \DocInput{annot_pro.dtx}
\IfFileExists{\jobname.ind}{\newpage\setupFullwidth\par\PrintIndex}{\paragraph*{Index}
The index goes here. Execute
\begin{quote}
    \texttt{makeindex -s gind.ist -o annot\_pro.ind annot\_pro.idx}
\end{quote}
    on the command line and recompile \texttt{annot\_pro.dtx}.}
\IfFileExists{\jobname.gls}{\PrintChanges}{\paragraph*{Change History} The list
of changes goes here. Execute
\begin{quote}
    \texttt{makeindex -s gglo.ist -o annot\_pro.gls annot\_pro.glo}
\end{quote}
    on the command line and recompile \texttt{annot\_pro.dtx}.}
\end{document}
%</driver>
% \fi
%
% \MakeShortVerb{|}
% \InputIfFileExists{aebdonotindex.def}{\PackageInfo{web}{Inputting aebdonotindex.def}}
%    {\PackageInfo{web}{cannot find aebdonotindex.def}}
%
% \section{Alternate package name: \texorpdfstring{\protect\pkg{annot-pro}}{annot-pro}}
% CTAN lists this package (\pkg{annot\_pro}) as \pkg{annot-pro}, so we'll create
% a dummy package by that name.
% \changes{v1.2}{2018/04/26}{Added dummy package \string\pkg{annot-pro}}
%    \begin{macrocode}
%<*altpkgname>
\NeedsTeXFormat{LaTeX2e}
\ProvidesPackage{annot-pro}
 [2018/04/26 v1.0 annot-pro Alt-name (dps)]
\DeclareOption*{\PassOptionsToPackage{\CurrentOption}{annot_pro}}
\ProcessOptions
\RequirePackage{annot_pro}
%    \end{macrocode}
%    \begin{macrocode}
%</altpkgname>
%<*package>
\RequirePackage{xkeyval}
%    \end{macrocode}
%    \changes{v1.1}{2016/09/29}{Introduced Free Text annotation}
%    \changes{v1.0c}{2016/09/28}{Reformatted manual, reformatted DTX file}
% \section{Background Information}
%
% The following annotation types will be supported (taken from table 8.20)
%\begin{itemize}
%   \item \textbf{Text}: Subtype \textbf{Text}. See Table 8.23, page 621. In particular, the values
%         of the \textbf{Name} key are \textbf{Comment}, \textbf{Help}, \textbf{Insert}, \textbf{Key}, \textbf{NewParagraph},
%         \textbf{Note} (the default), and \textbf{Paragraph}. Other supported names are \textbf{Check}, \textbf{Circle}, \textbf{Cross},
%         \textbf{Star}, \textbf{RightArrow}, \textbf{RightPointer}, \textbf{UpArrow}, \textbf{UpLeftArrow}
%   \item \textbf{Stamp}: Subtype \textbf{Stamp}. See Table 8.32, page 635. The values of the \textbf{Name} key are
%         \textbf{Approved}, \textbf{AsIs}, \textbf{Confidential},
%         \textbf{Departmental}, \textbf{Draft} (the default),
%         \textbf{Experimental}, \textbf{Expired}, \textbf{Final},
%         \textbf{ForComment}, \textbf{ForPublicRelease}, \textbf{NotApproved},
%         \textbf{NotForPublicRelease}, \textbf{Sold}, \textbf{TopSecret}.
%   \item \textbf{FileAttachment}: Subtype \textbf{FileAttachment}. Table 8.35, page 638. The Name key has values
%          \textbf{Graph}, \textbf{Paperclip}, \textbf{PushPin}, and \textbf{Tag}.
%\end{itemize}
% The options available for any supported annotation
%\begin{itemize}
%   \item \verb!type=text | stamp | fileattachment!  (text, the default)
%   \item \texttt{Contents} will be an argument of the command \cs{annotpro}
%   \item \texttt{NM} unique name for the page, optional, if not provided,
%         one will be generated.
%   \item \texttt{color}, to color the annotation
%   \item \verb!margin! to place the annotation in the margin
%   \item \texttt{nohspace}, leave no horizontal space
%   \item \texttt{novspace}, leave no vertical space
%   \item \texttt{nospace}, leave neither vertical or horizontal space.
%\end{itemize}
% \section{Annotation types supported}
% \subsection{Text Annotations}
% The options for the text annotation (\texttt{type=text}) are
%\begin{itemize}
%   \item \texttt{name}, one of the standard icons listed above, default Note
%   \item \texttt{open}, default \texttt{false}
%\end{itemize}
% \subsection{Rubber Stamp Annotations}
% The options for the text annotation (\texttt{type=stamp}) are
%\begin{itemize}
%   \item \texttt{name}, the default is draft. If a name begins with \texttt{\#}, that signals
%   a custom appearance.
%   \item \texttt{ap}, when the name is a custom name, the \texttt{ap} is used to hold the indirect reference
%         to the appearance. If a custom name is specified, \texttt{ap} is required, and \textsf{graphicxsp} is required.
%   \item \texttt{open}, default \texttt{false}
%\end{itemize}
% \subsection{File Attachment Annotations}
% The options for the text annotation (\texttt{type=fileattachement}) are
%\begin{itemize}
%   \item \texttt{name}, the default is \texttt{PushPin}.
%   \item \texttt{file}, the file specification \textbf{FS} key
%\end{itemize}
% \subsection{Text Box (Free Text) Annotations}
%   Beginning with Free Text
%    \begin{macrocode}
\DeclareOptionX{preview}{\previewtrue}
\@ifundefined{ifpreview}{\newif\ifpreview\previewfalse}{}
\providecommand{\previewOn}{\previewtrue}
\providecommand{\previewOff}{\previewfalse}
%    \end{macrocode}
%    \changes{v1.1}{2016/09/29}{Added \string\texttt{useA10Icons} and \string\texttt{useTextBox} options}
%    (2016/09/29) Added \texttt{useA10Icons} and \texttt{useTextBox} options
%    \begin{macrocode}
\newif\ifuseAAXdim\useAAXdimtrue
\DeclareOptionX{useA10Icons}{\useAAXdimtrue}
\DeclareOptionX{!useA10Icons}{\useAAXdimfalse}
%    \end{macrocode}
%    Require \pkg{richtext}, dated 2016/10/03 or later.
%    \changes{v1.1a}{2016/10/18}{Require \string\pkg{richtext}, dated 2016/10/03 or later.}
%    \changes{v1.1b}{2016/10/18}{Corrected date specification}
%    \begin{macrocode}
\DeclareOptionX{useTextBox}{\def\ap@requireRichTextAsNeeded{%
    \RequirePackage{richtext}[2016/10/03]}}
\let\ap@requireRichTextAsNeeded\relax
%    \end{macrocode}
%    The \IndexOpt{scandoc}\opt{scandoc} option calls some doc assembly JavaScript to scan the document
%    for stamp annotations. For stamps shipped by Adobe this is not necessary, but for any custom stamps,
%    this is needed.
%    \begin{macrocode}
\DeclareOptionX{scandoc}{\let\anp@InputScanDoc\anp@scanDoc
    \PassOptionsToPackage{execJS}{insdljs}}
%    \end{macrocode}
%    During document development, you don't what to scan the newly created PDF after you build it to,
%    perhaps, edit the text. In this case, use the \IndexOpt{!scandoc}\opt{!scandoc} option. You want to scan the document when you build the document for the last time
%    before publishing it. Be sure to save the document using the SaveAs menu item. This saves the
%    stamp appearances (symbology in this case) as part of the document.
%    \begin{macrocode}
\DeclareOptionX{!scandoc}{\let\anp@InputScanDoc\relax}
%    \end{macrocode}
%    The default is not to scan the document. Remember, the final document must be scanned; that is,
%    it must be build with the \opt{scandoc} option in effect.
%    \begin{macrocode}
\let\anp@InputScanDoc\relax
\def\anp@scanDoc{\InputIfFileExists{scandoc-anp.def}
    {\PackageInfo{annot_pro}{Inputting the scandoc-anp.def file}}
    {\PackageWarning{annot_pro}{Cannot find the file scandoc-anp.def}}}
%    \end{macrocode}
%    \begin{macrocode}
\ProcessOptionsX\relax
\RequirePackage{trig}
\RequirePackage{hyperref}
\RequirePackage{calc}
\RequirePackage{insdljs}
\ifx\anp@InputScanDoc\relax\else\let\execjs=y\fi
\RequirePackage[structure]{taborder}
\ap@requireRichTextAsNeeded
%    \end{macrocode}
% We see if \textsf{graphicxsp} is loaded, and set a simple switch one way or the other.
% This switch is later used when creating a custom stamp.
%    \begin{macrocode}
\@ifpackageloaded{graphicxsp}{%
    \let\ap@graphicxsp=1}{\let\ap@graphicxsp=0}
%    \end{macrocode}
% For now, we'll support setting the tab order only if the \pkg{eforms} package is loaded.
%    \begin{macrocode}
\@ifpackageloaded{eforms}{}
    {\newif\ifto@structOrder\to@structOrderfalse}
%    \end{macrocode}
%    \cs{csarg} from \pkg{comments} package, handy.
%    \begin{macrocode}
\def\ap@csarg#1#2{\expandafter#1\csname#2\endcsname}
%    \end{macrocode}
% Definitions for creating a rectangle used to create a building box for the annot.
%    \begin{macrocode}
\def\ap@Bbox#1#2{%
    \edef\bbox@width{#1}\edef\bbox@height{#2}%
    \hbox{\ifpreview\setlength\fboxrule{0.4pt}\setlength\fboxsep{0pt}%
    \@tempdima=\bbox@width\advance\@tempdima by-\fboxrule
    \@tempdimb=\bbox@height\advance\@tempdimb by-\fboxrule
    \fbox{\parbox[c][\@tempdimb][c]{\@tempdima}{\hfill\vfill}}\else
    \parbox[c][\bbox@height][c]{\bbox@width}{\hfill\vfill}\fi}%
}
%    \end{macrocode}
% The postscript operators \texttt{pdf@llx}, etc., are defined in \textsf{hyperref}.
%    \begin{macrocode}
\def\ap@Rect{/Rect [pdf@llx pdf@ury pdf@urx pdf@lly]}
%    \end{macrocode}
%    \begin{macro}{\defineAPath}
% A simple macro to define path.
%    \begin{macrocode}
\newcommand{\defineAPath}[1]{\def\ap@ctrl@Name{#1}%
    \hyper@normalise\ap@defineAPath}
\def\ap@defineAPath#1{\expandafter\xdef\ap@ctrl@Name{#1}}
%    \end{macrocode}
%    \end{macro}
% \section{Key-Value Pairs for \texorpdfstring{\protect\cs{annotpro}}{\textbackslash annotpro}}
%    \begin{macro}{type}
% We support three types of annotation, text, stamp, and fileattachment.
%    \begin{macrocode}
\define@choicekey+{annotpro}{type}[\val\nr]%
    {text,stamp,fileattachment,textbox}[text]{%
    \edef\ap@save@type{\nr}%
    \ifcase\nr\relax
        \def\ap@type{Text}\or
        \def\ap@type{Stamp}\or
        \def\ap@type{FileAttachment}\or
        \def\ap@type{FreeText}\fi
}{\PackageWarning{annot_pro}{Bad choice for type, permissible values
   are text, stamp, fileattachment, textbox. Try again}}
\def\ap@freetext@type{3}
\let\ap@type\@empty
%    \end{macrocode}
%    \end{macro}
%    \begin{macro}{name}
% The name key can have many different values. The permitted values are
% a function of the \texttt{type}.
%    \begin{macrocode}
\define@key{annotpro}{name}[]{\def\annotpro@name{#1}}
\let\annotpro@name\@empty
%    \end{macrocode}
%    The \opt{internalID} key allows you to specify the \psf{NM} entry. Must be
%    unique throughout the page. The default is \texttt{annotpro\string\the\string\ap@annot@cnt}.
%    The key is used only for the most particular of applications.
%    \changes{v1.1c}{2017/06/06}{Added \string\texttt{internalID}}
%    \begin{macrocode}
\define@key{annotpro}{internalID}[]{\def\annotpro@internalID{#1}}
\let\annotpro@internalID\@empty
%    \end{macrocode}
%    \end{macro}
% For a text annotation, the following are permitted values for \texttt{name}:
%    \begin{macrocode}
\define@choicekey+{annnotprotext}{name}{%
Comment,Key,Note,Help,NewParagraph,Paragraph,Insert,Check,%
Circle,Cross,Star,RightArrow,RightPointer,UpArrow,UpLeftArrow,%
CrossHairs}[Note]{\def\annotpro@@name{#1}}%
{\PackageWarning{annot_pro}{Bad choice for name, permissible values
are Comment, Key, Note, Help, NewParagraph, Paragraph, Insert, Check,
Circle, Cross, Star, RightArrow, RightPointer, UpArrow,
UpLeftArrow, CrossHairs. Try again}}
\let\annotpro@@name\@empty
%    \end{macrocode}
% For a stamp annotation, the following are permitted values for \texttt{name}.
%    \begin{macrocode}
\define@choicekey+{annnotprostamp}{name}{%
Approved,AsIs,Confidential,Departmental,Draft,%
Experimental,Expired,Final,ForComment,%
ForPublicRelease,NotApproved,NotForPublicRelease,Sold,%
TopSecret}[Draft]{\def\annotpro@@name{#1}}%
{\def\ap@custom@name{#1}}
\let\ap@custom@name\@empty
%    \end{macrocode}
% If the user enters a value other than the one listed, that value is recorded
% in the text macro \cs{ap@custom@name}. Usually, the user would enter the
% names of stamps that come on his system, but are not part of the PDF Specs.
%    \begin{macrocode}
%    \end{macrocode}
%    \begin{macro}{customStamp}
% There is a third case. The user can create a custom stamp using \pkg{graphicxsp}
% package.
%    \begin{macrocode}
\define@key{annotpro}{customStamp}[]{\def\ap@customstamp@name{#1}}
\let\ap@customstamp@name\@empty
%    \end{macrocode}
%    \end{macro}
%    \begin{macro}{ap}
% When the user specifies \texttt{customStamp}, he must supply a custom appearence
% for the stamp with the \texttt{ap} key. The \textsf{graphicxsp} package is required
% for this option.
%    \begin{macrocode}
\define@key{annotpro}{ap}[]{\def\annotpro@ap{#1}}
\let\annotpro@ap\@empty
%    \end{macrocode}
%    \end{macro}
% There is a third case. The user can create a custom stamp using graphicxsp
% package.
%    \begin{macro}{width}
%    \begin{macro}{height}
%    \begin{macro}{rotate}
%    \begin{macro}{scale}
%    \begin{macro}{widthTo}
%    \begin{macro}{heightTo}
% As of this writing, these are used when the user sets the name key equal
% to one of the many stamp names that come with Acrobat (other than ones
% that are documented in the PDF Specs. Use the width and height to set the
% bounding box. Normally, distiller, will place the largest stamp the box
% will hold, and center it in that box. (I think.)
%    \begin{macrocode}
\define@key{annotpro}{width}[]{\def\annotpro@width{#1}}
\let\annotpro@width\@empty
\define@key{annotpro}{height}[]{\def\annotpro@height{#1}}
\let\annotpro@height\@empty
\define@key{annotpro}{rotate}[]{\def\annotpro@rotate{#1}}
\let\annotpro@rotate\@empty
\define@key{annotpro}{scale}[]{\def\annotpro@scale{#1}}
\let\annotpro@scale\@empty
\define@key{annotpro}{widthTo}[]{\def\annotpro@widthTo{#1}}
\let\annotpro@widthTo\@empty
\define@key{annotpro}{heightTo}[]{\def\annotpro@heightTo{#1}}
\let\annotpro@heightTo\@empty
%    \end{macrocode}
%    \end{macro}
%    \end{macro}
%    \end{macro}
%    \end{macro}
%    \end{macro}
%    \end{macro}
% For a file attachment annotation, the following are permitted values for \texttt{name}.
%    \begin{macrocode}
\define@choicekey+{annnotprofileattach}{name}{%
Graph,Paperclip,PushPin,Tag}[PushPin]{\def\annotpro@@name{#1}}
{\PackageWarning{annot_pro}{Bad choice for name, permissible values
are Graph,Paperclip,PushPin,Tag. Try again}}
%    \end{macrocode}
%    \begin{macro}{file}
% When there is a file attachment, the absolute path to the file must be
% specified using the \texttt{file} key. A path to the file can be defined
% using \cs{defineAPath}.
%    \begin{macrocode}
\define@key{annotpro}{file}[]{%
    \edef\annotpro@file{#1}%
    \ifx\annotpro@file\@empty\else
    \edef\ap@exp@tmp{\noexpand\filename@parse{#1}}\ap@exp@tmp
    \edef\ap@thisFileName{\filename@base.\filename@ext}\fi
}
\let\annotpro@file\@empty
\define@key{annotprofreetext}{name}[FreeText]{\def\annotpro@@name{#1}}
\define@key{annotprofreetext}{defstyle}[]{\def\annotpro@@ds{#1}}
\let\annotpro@@ds\@empty
\define@key{annotprofreetext}{richtext}[]{\def\annotpro@@rt{#1}}
\let\annotpro@@rt\@empty
%    \end{macrocode}
%    \end{macro}
%    \begin{macro}{margin}
%    \begin{macro}{margintext}
%    \begin{macro}{\margintextformat}
% When margin is used, the command \cs{marginpar} is used to place the annotation
% in the margins, this is nice for sticky notes (text annotations), small stamps,
% and file attachments.
%
% You can optionally place (small amount of) text under the icon in the margins. The
% format for the text can be redefined using the command \cs{margintextformat}.
%    \begin{macrocode}
\define@key{annotpro}{margin}[0]{%
    \def\annotpro@margin{0}}
\let\annotpro@margin\@empty
\define@key{annotpro}{margintext}[]{%
    \def\apmargintext{\centering\break
    \parbox{\linewidth}{\apmargintextformat\strut#1}}}
\let\apmargintext\@empty
\newcommand{\margintextformat}[1]{\def\apmargintextformat{#1}}
\margintextformat{\bfseries\tiny\color{blue}}
%    \end{macrocode}
%    \end{macro}
%    \end{macro}
%    \end{macro}
%    \begin{macro}{margprior}
%    Again, when the \texttt{margin} option is taken, we can insert content just inside
%    the left brace of the \cs{marginpar} command.
%    \changes{v1.1c}{2017/06/06}{Added the key \string\texttt{margprior}}
%    \begin{macrocode}
\define@key{annotpro}{margprior}[]{\def\anp@margprior{#1}}
\let\anp@margprior\@empty
%    \end{macrocode}
%    \end{macro}
%    \begin{macro}{presets}
% A convenience key to allow the introduction of pre-defined options, for example,
% you might like all your comment fields to be red, so you can define
% \verb!\def\myComments{type=text,name=Comment,color=red}!, then say
%\begin{verbatim}
%   \annotpro[presets=\myComment]{Way to go!}
%\end{verbatim}
%    \begin{macrocode}
\define@key{annotpro}{presets}[]{\def\annotpro@presets{#1}}
\let\annotpro@presets\@empty
%    \end{macrocode}
% Added 07/08/09 to support structure tab order.
%    \end{macro}
%    \begin{macro}{objdef}
% This is a pdfmark feature that inserts a references to this COS object,
% used with setting the tab order using the structure. Tabbing using structure
% is supported by \textsf{eforms}.
%    \begin{macrocode}
\define@key{annotpro}{objdef}[]{\def\eq@objdef{/_objdef {#1}}%
    \def\eq@objdefName{#1}}
\let\eq@objdef\@empty
%    \end{macrocode}
%    \end{macro}
%    \begin{macro}{taborder}
%    \begin{macrocode}
\define@key{annotpro}{taborder}[]{\def\eq@taborder{#1}}
\let\eq@taborder\@empty
%    \end{macrocode}
%    \end{macro}
%    \begin{macro}{open}
% A key to set whether the popup annotation is open or not.
%    \begin{macrocode}
\define@choicekey+{annotpro}{open}[\val\nr]{true,false}[true]{%
    \def\annotpro@open{#1}}{\PackageWarning{annot_pro}{Bad
    choice for open, permissible values are true and false. Try again}}
\def\annotpro@open{false}
%    \end{macrocode}
%    \end{macro}
%    \begin{macro}{subject}
%    \begin{macro}{title}
% The subject and title (usually the author's name) of the annotation
%    \begin{macrocode}
\define@key{annotpro}{subject}[]{\def\annotpro@subject{#1}}
\define@key{annotpro}{title}[]{\def\annotpro@title{#1}}
%    \end{macrocode}
%    \end{macro}
%    \end{macro}
%    \begin{macro}{color}
% The color of the annotation.
%    \begin{macrocode}
\define@key{annotpro}{color}[1 0 0]{\def\annotpro@color{#1}}
%    \end{macrocode}
%    For the FreeText annotation, we can specify background color \IndexOpt{bgcolor}\texttt{bgcolor},
%    (default white) and border color \IndexOpt{bcolor}\texttt{bcolor} (default black).
%    \begin{macrocode}
\define@key{annotpro}{bgcolor}[]{\def\annotpro@bgcolor{#1}}
\def\annotpro@bgcolor{white}
\define@key{annotpro}{bcolor}[]{\def\annotpro@bcolor{#1}}
\let\annotpro@bcolor\@empty
\def\annotpro@def@bcolor{black}
\def\ap@BE@i{4.08768}
\def\ap@BE@i@calc{\ap@BE@i\space\ap@bwidthNum\space add}
\def\ap@BE@ii{8.26857}
\def\ap@BE@ii@calc{\ap@BE@ii\space add \ap@bwidthNum}
\define@choicekey+{annotpro}{borderstyle}[\val\nr]{none,solid,%
    dash1,dash2,dash3,dash4,dash5,dash6,cloud1,cloud2}[solid]{%
    \ifcase\nr\relax
        \def\ap@border@code{}\or
        \def\ap@border@code{%
            /BS <</S/S\ap@bwidth>>}\or
        \def\ap@border@code{%
            /BS <</D[2.0 2.0]/S/D\ap@bwidth>>}\or
        \def\ap@border@code{%
            /BS <</D[3.0 3.0]/S/D\ap@bwidth>>}\or
        \def\ap@border@code{%
            /BS <</D[4.0 4.0]/S/D\ap@bwidth>>}\or
        \def\ap@border@code{%
            /BS <</D[4.0 3.0 2.0 3.0]/S/D\ap@bwidth>>}\or
        \def\ap@border@code{%
            /BS <</D[4.0 3.0 16.0 3.0]/S/D\ap@bwidth>>}\or
        \def\ap@border@code{%
            /BS <</D[8.0 4.0 4.0 4.0]/S/D\ap@bwidth>>}\or
        \def\ap@border@code{%
            /BE<</I 1.0/S/C>>/RD[\ap@BE@i@calc\space\ap@BE@i@calc
            \space\ap@BE@i@calc\space\ap@BE@i@calc]%
        \ifx\ap@bwidth\@empty\else/BS <<\ap@bwidth>>\fi}\or
        \def\ap@border@code{%
            /BE<</I 2.0/S/C>>/RD[\ap@BE@ii@calc\space\ap@BE@ii@calc
            \space\ap@BE@ii@calc\space\ap@BE@ii@calc]%
        \ifx\ap@bwidth\@empty\else/BS <<\ap@bwidth>>\fi}\fi
    \def\annotpro@border{#1}}{\PackageWarning{annot_pro}{Bad
    choice for borderstyle,\MessageBreak permissible values are none,
    solid, dash1, dash2, dash3,\MessageBreak dash4, dash5, dash6,
    cloud1, cloud2. Try again}}
\def\ap@border@code{/BS <</S/S\ap@bwidth>>}
\define@choicekey+{annotpro}
    {borderwidth}[\val\nr]{.5,1,2,3,4,6,8,10}[1]{%
    \edef\ap@bwidthNum{\val}%
    \def\ap@bwidth{/W #1}}{\PackageWarning{annot_pro}{Bad
    choice for borderwidth,\MessageBreak permissible values are
    .5, 1, 2,\MessageBreak 3, 4, 6, 8, 10. Try again}}
\def\ap@bwidthNum{1}
\let\ap@bwidth\@empty
%    \end{macrocode}
%    \end{macro}
%    \begin{macro}{readonly}
% Set the annotation to readonly.
%    \begin{macrocode}
%\define@key{annotpro}{readonly}[ 64 add]{\def\ap@readonly{#1}}
\define@boolkey{annotpro}{readonly}[true]{\ifKV@annotpro@readonly
  \def\ap@readonly{ 65 add}\else\let\ap@readonly\@empty\fi}
\let\ap@readonly\@empty
%    \end{macrocode}
%    Added the \texttt{hidden} option.
%    \changes{v1.1c}{2017/06/06}{Added \string\texttt{hidden} property}
%    \begin{macrocode}
%\define@key{annotpro}{hidden}[ 2 add]{\def\ap@hidden{#1}}
\define@boolkey{annotpro}{hidden}[true]{\ifKV@annotpro@hidden
  \def\ap@hidden{ 2 add}\else\let\ap@hidden\@empty\fi}
\let\ap@hidden\@empty
%    \end{macrocode}
%    \end{macro}
%    \begin{macro}{nohspace}
%    \begin{macro}{nohspace}
%    \begin{macro}{nospace}
% These are used only with text annotations. The position of the
% annotation depends only on the upper-right corner. The bounding
% box can be made to have no width or not height, or both. The upper-left
% corner is always defined. The text annotation will take no horizontal,
% vertical, or no space at all in {\TeX} space. The same effect can be accomplished
% using 0pt width boxes or by smashing the bounding box.
%    \begin{macrocode}
\define@key{annotpro}{nohspace}[]{\def\annotpro@hspace{0pt}}
\define@key{annotpro}{novspace}[]{\def\annotpro@vspace{0pt}}
\define@key{annotpro}{nospace}[]{\def\annotpro@hspace{0pt}%
    \def\annotpro@vspace{0pt}}
\let\annotpro@hspace\@empty\let\annotpro@vspace\@empty
%    \end{macrocode}
%    \end{macro}
%    \end{macro}
%    \end{macro}
%    \begin{macro}{opacity}
% Set the \texttt{opacity} of the annot, if creating a custom stamp using
% graphicxsp, then the file needs to be distilled using \texttt{standard\_transparency}
% job settings for distiller.
%    \begin{macrocode}
\define@key{annotpro}{opacity}[]{%
    \def\ap@add@ops{[ /ca #1 /SetTransparency pdfmark}%
    \def\annotpro@opacity{#1}}
\let\annotpro@opacity\@empty
%    \end{macrocode}
%    \end{macro}
% \subsection{Dimensions of Standard Text Icons}
% Using the user interface, and the dimensions of each of the standard text annotations
% were obtained and recorded here.
%    \begin{macrocode}
% Note in A10, text annots are not 18bp x 18bp and A10 adds CrossHairs
\def\getargsiii#1#2#3{\def\ap@argi{#1}\def\ap@argii{#2}%
    \def\ap@argiii{#3}}
\ifuseAAXdim
\@tfor\ap@text@dimen:={{Comment}{18bp}{18bp}}%
{{Key}{13bp}{18bp}}{{Note}{18bp}{20bp}}{{Help}{20bp}{20bp}}%
{{NewParagraph}{13bp}{20bp}}{{Paragraph}{20bp}{20bp}}%
{{Insert}{17bp}{20bp}}{{Check}{20bp}{19bp}}%
{{Circle}{20bp}{20bp}}{{Cross}{20bp}{19bp}}%
{{Star}{20bp}{19bp}}{{RightArrow}{20bp}{20bp}}%
{{RightPointer}{20bp}{17bp}}{{UpArrow}{17bp}{20bp}}%
{{UpLeftArrow}{17bp}{17bp}}{{CrossHairs}{20bp}{20bp}}\do{%
    \expandafter\getargsiii\ap@text@dimen
    \ap@csarg\edef{ap@\ap@argi Width}{\ap@argii}%
    \ap@csarg\edef{ap@\ap@argi Height}{\ap@argiii}%
} \else
\@tfor\ap@text@dimen:={{Comment}{20bp}{18bp}}%
{{Key}{18bp}{17bp}}{{Note}{18bp}{20bp}}{{Help}{20bp}{20bp}}%
{{NewParagraph}{13bp}{20bp}}{{Paragraph}{11bp}{20bp}}%
{{Insert}{20bp}{17bp}}{{Check}{20bp}{19bp}}%
{{Circle}{20bp}{20bp}}{{Cross}{20bp}{20bp}}%
{{Star}{20bp}{19bp}}{{RightArrow}{20bp}{14bp}}%
{{RightPointer}{20bp}{18bp}}{{UpArrow}{19bp}{20bp}}%
{{UpLeftArrow}{17bp}{17bp}}{{CrossHairs}{24bp}{24bp}}\do{%
    \expandafter\getargsiii\ap@text@dimen
    \ap@csarg\edef{ap@\ap@argi Width}{\ap@argii}%
    \ap@csarg\edef{ap@\ap@argi Height}{\ap@argiii}%
} \fi
%    \end{macrocode}
% \subsection{Dimensions of Standard File Attachment Icons}
% Similarly, the dimensions of each of the standard file attachment annotations
% were obtained and recorded. All these dimensions have been reduced by 2pt, so that there final
% dimensions match what is displayed through the ui.
%    \begin{macrocode}
\ifuseAAXdim
\@tfor\ap@fa@dimen:={{Graph}{20bp}{20bp}}
 {{Paperclip}{7bp}{17bp}}{{PushPin}{14bp}{20bp}}
 {{Tag}{20bp}{16bp}}\do{%
    \expandafter\getargsiii\ap@fa@dimen
    \ap@csarg\edef{ap@\ap@argi Width}{\ap@argii}%
    \ap@csarg\edef{ap@\ap@argi Height}{\ap@argiii}%
}
\else
\@tfor\ap@fa@dimen:={{Graph}{18bp}{18bp}}
 {{Paperclip}{5bp}{15bp}}{{PushPin}{12bp}{18bp}}
 {{Tag}{18bp}{14bp}}\do{%
    \expandafter\getargsiii\ap@fa@dimen
    \ap@csarg\edef{ap@\ap@argi Width}{\ap@argii}%
    \ap@csarg\edef{ap@\ap@argi Height}{\ap@argiii}%
} \fi
%    \end{macrocode}
% \subsection{Dimensions of Standard Stamp Appearances}
%
% Some default stamp dimensions, when the dimensions are not known or provided. May be
% redefined.
%    \begin{macrocode}
\newcommand{\defaultStampWidth}{50bp}
\newcommand{\defaultStampHeight}{50bp}
%    \end{macrocode}
% The dimensions of the stamp appearances in the file \texttt{Standard.pdf}
%    \begin{macrocode}
\newcommand{\standardStampWidth}{245.38bp}
\newcommand{\standardStampHeight}{64.53bp}
%    \end{macrocode}
% The dimensions of the stamp appearances in the file \texttt{StandardBusiness.pdf}
%    \begin{macrocode}
\@tfor\ap@dimen:={{SBApproved}{163.50bp}{49.50bp}}
{{SBCompleted}{175.87bp}{49.50bp}}
{{SBConfidential}{209.62bp}{49.50bp}}
{{SBDraft}{113.14bp}{49.75bp}}
{{SBFinal}{97.80bp}{49.75bp}}
{{SBForComment}{211.87bp}{49.50bp}}
{{SBForPublicRelease}{290.04bp}{49.55bp}}
{{SBInformationOnly}{275.00bp}{49.55bp}}
{{SBNotApproved}{218.62bp}{49.50bp}}
{{SBNotForPublicRelease}{346.04bp}{49.55bp}}
{{SBPreliminaryResults}{304.04bp}{49.55bp}}
{{SBVoid}{91.04bp}{49.75bp}}\do{%
    \expandafter\getargsiii\ap@dimen
    \ap@csarg\edef{ap@\ap@argi Width}{\ap@argii}%
    \ap@csarg\edef{ap@\ap@argi Height}{\ap@argiii}%
}
%    \end{macrocode}
% The dimensions of the stamp appearances in the file \texttt{SignHere.pdf}
%    \begin{macrocode}
\@tfor\ap@dimen:={{SBRejected}{22.68bp}{22.66bp}}
{{SHAccepted}{23.29bp}{20.45bp}}
{{SHInitialHere}{99.69bp}{30.34bp}}
{{SHSignHere}{99.69bp}{30.34bp}}
{{SHWitness}{99.69bp}{30.34bp}}\do{%
    \expandafter\getargsiii\ap@dimen
    \ap@csarg\edef{ap@\ap@argi Width}{\ap@argii}%
    \ap@csarg\edef{ap@\ap@argi Height}{\ap@argiii}%
}
%    \end{macrocode}
% The dimensions of the stamp appearances in the file \texttt{\texttt{Dynamic.pdf}}
%    \begin{macrocode}
\catcode`\#=11\relax
\@tfor\ap@dimen:={{#DApproved}{255.64bp}{53.83bp}}
{{#DConfidential}{163.28bp}{57.10bp}}
{{#DReceived}{255.64bp}{55.11bp}}
{{#DReviewed}{255.64bp}{55.11bp}}
{{#DRevised}{127.47bp}{55.85bp}}\do{%
    \expandafter\getargsiii\ap@dimen
    \ap@csarg\edef{ap@\ap@argi Width}{\ap@argii}%
    \ap@csarg\edef{ap@\ap@argi Height}{\ap@argiii}%
}
\catcode`\#=6\relax
\newcommand{\stampWidth}[1]{\csname ap@#1Width\endcsname}
\newcommand{\stampHeight}[1]{\csname ap@#1Height\endcsname}
%    \end{macrocode}
% The dimensions of the stamp appearances in the file Words.pdf
%    \begin{macrocode}
\@tfor\ap@dimen:={{WordsAGenuineFind}{120.68bp}{190.83bp}}
{{WordsBigSavings}{282.84bp}{117.86bp}}
{{WordsDeals}{345.40bp}{132.55bp}}
{{WordsHappyBirthday}{326.50bp}{166.32bp}}
{{WordsHot}{249.64bp}{176.75bp}}
{{WordsLookWhatsNew}{316.59bp}{229.81bp}}
{{WordsNeato}{171.12bp}{86.88bp}}
{{WordsTheBestJustGotBetter}{153.53bp}{153.53bp}}\do{%
    \expandafter\getargsiii\ap@dimen
    \ap@csarg\edef{ap@\ap@argi Width}{\ap@argii}%
    \ap@csarg\edef{ap@\ap@argi Height}{\ap@argiii}%
}
%    \end{macrocode}
% The dimensions of the stamp appearances in the file \texttt{Faces.pdf}
%    \begin{macrocode}
\@tfor\ap@dimen:={{FacesEcstatic}{223.20bp}{202.80bp}}
{{FacesElectric}{295.80bp}{280.50bp}}
{{FacesGrumpy}{289.92bp}{194.64bp}}
{{FacesHappy}{161.65bp}{161.65bp}}
{{FacesHurt}{289.50bp}{285.75bp}}
{{FacesNeutral}{291.04bp}{279.32bp}}
{{FacesSmirk}{276.75bp}{292.50bp}}
{{FacesSurprised}{220.08bp}{215.04bp}}
{{FacesWorried}{298.11bp}{235.12bp}}
{{FacesWow}{242.32bp}{289.17bp}}
{{FacesZippy}{290.59bp}{263.81bp}}\do{%
    \expandafter\getargsiii\ap@dimen
    \ap@csarg\edef{ap@\ap@argi Width}{\ap@argii}%
    \ap@csarg\edef{ap@\ap@argi Height}{\ap@argiii}%
}
%    \end{macrocode}
% The dimensions of the stamp appearances in the file \texttt{Pointers.pdf}
%    \begin{macrocode}
\@tfor\ap@dimen:={{PointersBogus}{128.32bp}{53.82bp}}
{{PointersBug}{128.32bp}{51.94bp}}
{{PointersDOA}{136.99bp}{51.48bp}}
{{PointersFYEO}{128.32bp}{48.60bp}}
{{PointersFix}{145.60bp}{58.11bp}}
{{PointersNuke}{128.32bp}{52.30bp}}
{{PointersOptimize}{128.32bp}{48.60bp}}
{{PointersRedFlag}{128.32bp}{57.12bp}}
{{PointersValidate}{128.32bp}{48.60bp}}
{{PointersWarning}{128.32bp}{57.23bp}}
{{PointersWhen}{128.32bp}{63.79bp}}\do{%
    \expandafter\getargsiii\ap@dimen
    \ap@csarg\edef{ap@\ap@argi Width}{\ap@argii}%
    \ap@csarg\edef{ap@\ap@argi Height}{\ap@argiii}%
}
%    \end{macrocode}
%    \subsection{Default dimensions of Free Text annotation}
%    \begin{macrocode}
\def\ap@FreeTextWidth{144bp}
\def\ap@FreeTextHeight{72bp}
%    \end{macrocode}
%    \begin{macro}{\makeStamp}
% A command to help the creation of custom stamps using graphicxsp. You've got to see it
% to believe it.
%    \begin{macrocode}
\newcommand{\makeStamp}[3]{%
\ap@csarg\def{stamp@#2}{%
    \begin{sp@createImage}{#1}{#2\the\ap@cusstamp@cnt}
        \ap@add@ops#3
    \end{sp@createImage}}}
\let\ap@add@ops\@empty
%    \end{macrocode}
%    \end{macro}
% This command is used internal in \cs{annotpro} to generate an appearance for
% the stamp.
%    \begin{macrocode}
\newcommand{\ap@useStamp}[1]{%
    \global\advance\ap@cusstamp@cnt1\relax
    \@nameuse{stamp@#1}}
%    \end{macrocode}
%    \begin{macro}{\setAnnotOptions}
% A command for setting global options.
%    \begin{macrocode}
\newcommand{\setAnnotOptions}[1]{%
    \setkeys{annotpro}{#1}%
}
%    \end{macrocode}
%    \end{macro}
% The following code is taken from the \textsf{datetime} package (and modified) as maintained by
% Nicola Talbot. There is no need for the whole package, we just need
% \cs{pdfdate}, named \cs{ap@pdfdate} in \textsf{annot\_pro}. We make all calculations in a group.
%\changes{v1.0a}{2011/07/14}{Added \cs{ap@pdfdate}}
%    \begin{macrocode}
\begingroup
\def\withoptz@#1{\ifnum#1<10 0\fi\the#1}
\edef\ap@pdfdate{\the\year\withoptz@\month\withoptz@\day}
%    \end{macrocode}
% Calculate the minutes after the hour time by dividing by 60.
%    \begin{macrocode}
\count0=\time \count2=\count0
%    \end{macrocode}
% Calculate the number of hours after midnight by dividing by 60
%    \begin{macrocode}
\divide\count2by60
\edef\ap@pdfdate{\ap@pdfdate\withoptz@{\count2}}
\multiply\count2by60 \advance\count0by-\count2
\xdef\ap@pdfdate{\ap@pdfdate\withoptz@{\count0}00}
\endgroup
%    \end{macrocode}
% \section{The main code}
%    \changes{v1.1c}{2017/06/06}{Define convenience command \string\cs{currentAnnotName}}
%    \begin{macrocode}
\newcount\ap@annot@cnt
\def\currentAnnotName{annotpro\the\ap@annot@cnt}
\newcount\ap@cusstamp@cnt
{\catcode`\#=12\gdef\ap@signalCustomAP{#}}
\def\pStamp#1{\ap@signalCustomAP#1}
%    \end{macrocode}
%    \begin{macrocode}
\def\ap@RV@Body{<?xml version="1.0"?><body %
    xfa:APIVersion="Acrobat:9.5.5" %
    xfa:contentType="text/html" %
    xfa:spec="2.0.2" xmlns="http://www.w3.org/1999/xhtml" %
    xmlns:xfa="http://www.xfa.org/schema/xfa-data/1.0/">}
\def\ap@RV@endBody{<body>}
%    \end{macrocode}
%    The implementation of rich text in a FreeText annotation is a little different from
%    that of a rich text form field. We made some adjustments here for an annotation.
%    \begin{macrocode}
\def\ap@sub#1{%
    \rvorvstring{<span style="vertical-align:-0.0pt;">#1</span>}{#1}}%
\def\ap@sup#1{%
    \rvorvstring{<span style="vertical-align:+0.0pt;">#1</span>}{#1}}%
\def\ap@uDec#1{\rt@amp\rt@hashtag#1;}
\def\ap@uHex#1{\rt@amp\rt@hashtag x#1;}
\def\ap@afterParaFont{%
%    \end{macrocode}
%    Cancel out certain attributes that are not used in annotation.
%    \begin{macrocode}
    \rt@formfieldfalse
}
\def\ap@afterFont{%
    \rt@formfieldfalse
    \let\rt@url\@empty
}
%    \end{macrocode}
%    \begin{environment}{textboxpara}
%    Rich text for the Text Box annotation is enclosed within this environment, then use \cs{rtpara} to
%    format the text.
%    \begin{macrocode}
\newenvironment{textboxpara}{%
    \let\rt@sub\ap@sub\let\rt@sup\ap@sup
    \let\rt@afterParaFont\ap@afterParaFont
    \let\rt@afterFont\ap@afterFont
    \def\rt@br{\rvorvstring{\rt@amp\rt@hashtag13;}{\string\r}}%
    \def\rt@spc{\space}\let\uDec\ap@uDec\let\uHex\ap@uHex
}{}
%    \end{macrocode}
%    \end{environment}
%    \begin{macro}{\annotpro}
% Here is the main command of this package, the first optional parameter
% are key-value pairs, as defined above; the second parameter is the content
% of the message. File attachments have not popup message, so the content
% becomes the description of the attachment in the attachments panel; hence,
% in this case, the length of the message should be short.
%    \begin{macrocode}
\newcommand{\annotpro}{%
    \begingroup
%    \end{macrocode}
% Some of the stamp names begin with \texttt{\#}, the user must use
% \verb!\#!
%    \begin{macrocode}
    \let\#\ap@signalCustomAP
    \@nnotpro}
%    \end{macrocode}
%    Added a star option to \cs{annotpro}. When specified (\DescribeMacro{\annotpro*}\cs{annotpro*}),
%    the options are expanded with \cs{setkeys*}. Also new is the
%    options are expanded first before \cs{setkeys} or \cs{setkeys*} is applied.
%    \changes{v1.1c}{2017/06/06}{Added a star option to \string\cs{annotpro}}
%    \begin{macrocode}
\def\@nnotpro{\@ifstar{\def\qr@isStar{*}\@nnotproi}
    {\let\qr@isStar\@empty\@nnotproi}}
\newcommand{\@nnotproi}[1][]{%
%    \end{macrocode}
% Process options after expanding them
%    \begin{macrocode}
    \expandafter\setkeys\qr@isStar{annotpro}{#1}%
%    \edef\temp@exp{\noexpand\setkeys\qr@isStar{annotpro}{#1}}\temp@exp
\ifx\annotpro@internalID\@empty
%    \end{macrocode}
% Increment a global counter, to count the annots as we create them.
%    \begin{macrocode}
    \global\advance\ap@annot@cnt1\relax
    \xdef\currentAnnotName{annotpro\the\ap@annot@cnt}\else
    \xdef\currentAnnotName{\annotpro@internalID}\fi
    \@nn@tpro}
\newcommand{\@nn@tpro}[1]
{%
%    \end{macrocode}
% See if there were any presets, if so, expand them, and apply
% \cs{setkeys} again.
%    \begin{macrocode}
  \ifx\annotpro@presets\@empty\else
    \edef\ap@exp@tmp{\noexpand\setkeys{annotpro}{\annotpro@presets}}%
    \ap@exp@tmp\fi
%    \end{macrocode}
% If no type was specified, assume \texttt{Text}.
% \cs{setkeys} again.
%    \begin{macrocode}
  \ifx\ap@type\@empty\def\ap@type{Text}\def\ap@save@type{0}\fi
%    \end{macrocode}
% Now get the value of the name parameter, permissible values depend
% on the stated \texttt{type}.
%    \begin{macrocode}
  \ifcase\ap@save@type\relax
    \def\annot@subtype{annnotprotext}%
    \ifx\annotpro@subject\@empty\def\annotpro@subject{Sticky Note}\fi
    \ifx\annotpro@name\@empty\def\annotpro@name{Note}\fi
    \def\ap@def@name{Note}\or
    \ifx\annotpro@subject\@empty\def\annotpro@subject{Rubber Stamp}\fi
    \def\annot@subtype{annnotprostamp}%
    \def\ap@def@name{Draft}\or
    \ifx\annotpro@subject\@empty
        \def\annotpro@subject{File Attachment}\fi
    \def\annot@subtype{annnotprofileattach}%
    \def\ap@def@name{PushPin}\or
    \ifx\annotpro@subject\@empty
        \def\annotpro@subject{Text Box}\fi
    \def\annot@subtype{annotprofreetext}%
    \def\ap@def@name{FreeText}%
  \fi
%    \end{macrocode}
% Get the name!
%    \begin{macrocode}
    \edef\ap@exp@tmp{%
        \noexpand\setkeys{\annot@subtype}{name=\annotpro@name}}%
    \ap@exp@tmp\xdef\ap@def@type{\ap@type}%
%    \end{macrocode}
% Encode text for contents, title, subject, and color; here \textsf{hyperref}
% commands are used.
%    \begin{macrocode}
  \let\annotpro@enc@contents\@empty
%    \end{macrocode}
% \changes{v1.0b}{2012/11/10}{Included \cs{makespecialJS}}
% We'll use \cs{makespecialJS} here.
%    \begin{macrocode}
  \makespecialJS
%    \end{macrocode}
%    If this is a FreeText annot, we get the rich text and default style from \texttt{\#1}.
%    Supported keys are richtext and defstyle. The values of these keys are names defined
%    by \cs{setRVVContents} and \cs{setDefaultStyle}. If \texttt{defstyle} is not specified, we
%    use \cs{useDefaultDS}.
%    \begin{macrocode}
\if\ap@save@type\ap@freetext@type
     \let\eq@RV@Body\ap@RV@Body\let\eq@RV@endBody\ap@RV@endBody
     \setkeys{annotprofreetext}{#1}%
     \ifx\annotpro@@ds\@empty
        \def\ap@defaultstyle{\useDefaultDS}\else
        \edef\ap@defaultstyle{\noexpand\useDS{\annotpro@@ds}}\fi
     \ifx\annotpro@@rt\@empty
        \let\annotpro@RC\@empty\let\annotpro@C\@empty\else
        \edef\annotpro@RC{\noexpand\useRVContent{\annotpro@@rt}}%
        \edef\annotpro@C{\noexpand\useVContent{\annotpro@@rt}}\fi
%    \end{macrocode}
%    If this is a FreeText annot, we define its rich and plain values; requires \pkg{richtext} package
%    \begin{macrocode}
\else
  \pdfstringdef\annotpro@enc@contents{#1}%
\fi
  \let\annotpro@enc@title\@empty
  \pdfstringdef\annotpro@enc@title{\annotpro@title}%
  \let\annotpro@enc@subject\@empty
  \pdfstringdef\annotpro@enc@subject{\annotpro@subject}%
%    \end{macrocode}
%    Color handling for \texttt{color}, \texttt{bgcolor}, and \texttt{bcolor} via \pkg{hycolor} package
%    \begin{macrocode}
  \HyColor@XZeroOneThreeFour{\annotpro@color}%
    {\annotpro@enc@color}{annotpro}{color}%
\ifx\annotpro@bgcolor\@empty\else
  \HyColor@XZeroOneThreeFour{\annotpro@bgcolor}
    {\annotpro@enc@bgcolor}{annotpro}{bgcolor}\fi
\ifx\annotpro@bcolor\@empty
    \edef\annotpro@bcolor{\annotpro@def@bcolor}\fi
  \HyColor@FieldColor{\annotpro@bcolor}
    {\annotpro@enc@bcolor}{annotpro}{bcolor}%
%    \end{macrocode}
% Initialize some commands to be used later
%    \begin{macrocode}
  \let\ap@this@AP\@empty\let\ap@this@FS\@empty\let\ap@this@CA\@empty
  \let\ap@this@Rotate\@empty
%    \end{macrocode}
% See if opacity is requested, and set appropriate key.
%    \begin{macrocode}
  \ifx\annotpro@opacity\@empty\else
        \def\ap@this@CA{/CA \annotpro@opacity}\fi
%    \end{macrocode}
% Now, perform annotation specific tasks such as setting the
% bounding box.
%    \begin{macrocode}
  \ifcase\ap@save@type
%    \end{macrocode}
% \textbf{Text annotation.} We use the pre-defined dimensions for
% these text icons, unless the user has specified \texttt{nohspace},
% \texttt{novspace}, and \texttt{nospace}. If the name has not been
% specified, we use the default (\texttt{Note}).
%    \begin{macrocode}
     \ifx\annotpro@@name\@empty\edef\annotpro@@name{\ap@def@name}\fi
      \edef\ap@thisWidth{\ifx\annotpro@hspace\@empty
        \@nameuse{ap@\annotpro@@name Width}\else
        \annotpro@hspace\fi}%
      \edef\ap@thisHeight{\ifx\annotpro@vspace\@empty
        \@nameuse{ap@\annotpro@@name Height}\else
      \annotpro@vspace\fi}%
%    \end{macrocode}
% \textbf{Stamp annotation.} This is a bit trickier, there are three
% cases: (1) the user references a standard stamp; (2) the user
% references a non-standard stamp, one that ships with Acrobat; (3)
% the user wants to create his own custom stamp using
% \textsf{graphicxsp} techniques.
%    \begin{macrocode}
  \or
    \ifx\annotpro@@name\@empty
        \ifx\ap@custom@name\@empty
            \ifx\ap@customstamp@name\@empty
%    \end{macrocode}
% Neither \texttt{name} nor \texttt{customStamp} was specified, so we assume the default.
%    \begin{macrocode}
                \ifx\annotpro@@name\@empty
                    \edef\annotpro@@name{\ap@def@name}\fi
                \edef\ap@thisWidth{\standardStampWidth}%
                \edef\ap@thisHeight{\standardStampHeight}%
            \else
%    \end{macrocode}
% Here the user wants a stamp using \textsf{graphicxsp} techniques.
%    \begin{macrocode}
                \if\ap@graphicxsp0\PackageError{annot_pro}%
                {Specifying a custom name for a stamp requires
                    the graphicxsp package.}\fi
                \ifx\annotpro@ap\@empty
                \PackageError{annot_pro}{You must specify a value
                    for the ap key}\fi
                \def\ap@ImageRef{\annotpro@ap\the\ap@cusstamp@cnt}%
                \expandafter\ap@useStamp\expandafter{\annotpro@ap}%
%    \end{macrocode}
%    We do not allow custom stamps to be initially rotated.
%    \changes{v1.1c}{2017/06/06}{For custom stamps, do not allow the rotate
%    key to be nonzero.}
%    \begin{macrocode}
                \ifx\annotpro@rotate\@empty\else
                    \PackageWarning{annot_pro}{The rotate
                    key for a custom stamp is not\MessageBreak
                     supported. Setting rotate=0}%
                    \let\annotpro@rotate\@empty
                \fi
%    \end{macrocode}
% After making many experiments, there seems to be a need to
% reduce dimensions by \texttt{2bp} so that the desired dimensions
% are displayed through the user interface. There seems to
% be an 1bp border around all stamps.
%    \begin{macrocode}
                \edef\ap@thisWidth{\widthOf{\ap@ImageRef}bp}%
                \@tempdima=\ap@thisWidth\advance\@tempdima-2bp
                \edef\ap@thisWidth{\the\@tempdima}%
                \edef\ap@thisHeight{\heightOf{\ap@ImageRef}bp}%
                \@tempdima=\ap@thisHeight\advance\@tempdima-2bp
                \edef\ap@thisHeight{\the\@tempdima}%
                \def\ap@this@AP{/AP << /N {\ap@ImageRef} >>}%
                \edef\annotpro@@name{%
                    \ap@signalCustomAP\ap@customstamp@name}%
            \fi
        \else
%    \end{macrocode}
% The command \cs{ap@custom@name} is non-empty, so the user it trying
% to access a stamp on his own system. The user has indicated one of
% the stamps documented in the PDF Reference, these all have the same
% dimension
%
% For the bounding box, we use  \cs{defaultStampWidth} by
% \cs{defaultStampHeight} as the dimensions, unless the values of
% \texttt{width} and/or \texttt{height} are specified.
%    \begin{macrocode}
            \edef\annotpro@@name{\ap@custom@name}%
            \@ifundefined{ap@\ap@custom@name Width}{%
                \edef\ap@thisWidth{\ifx\annotpro@width\@empty
                    \defaultStampWidth\else\annotpro@width\fi}%
                \edef\ap@thisHeight{\ifx\annotpro@height\@empty
                    \defaultStampWidth\else\annotpro@height\fi}%
            }{%
                \edef\ap@thisWidth{\ifx\annotpro@width\@empty
                    \@nameuse{ap@\annotpro@@name Width}\else
                    \annotpro@width\fi}%
                \edef\ap@thisHeight{\ifx\annotpro@height\@empty
                    \@nameuse{ap@\annotpro@@name Height}\else
                    \annotpro@height\fi}%
            }%
        \fi
    \else
%    \end{macrocode}
% This is the case of specifying one of the document stamps, as
% listed in the PDF Reference.
%    \begin{macrocode}
        \edef\ap@thisWidth{\standardStampWidth}%
        \edef\ap@thisHeight{\standardStampHeight}%
    \fi
%    \end{macrocode}
% Let's use calc to allow for calculations on dimensions,
% with this, we can conveniently resize the box.
%    \begin{macrocode}
     \ifx\annotpro@scale\@empty
         \ifx\annotpro@widthTo\@empty
             \ifx\annotpro@heightTo\@empty\else % heightTo
                \edef\ap@thisWidth{%
                    (\ap@thisWidth)*\noexpand
                     \ratio{\annotpro@heightTo}{\ap@thisHeight}}%
                \edef\ap@thisHeight{\annotpro@heightTo}%
             \fi
         \else % widthTo
            \edef\ap@thisHeight{%
                (\ap@thisHeight)*\noexpand
                 \ratio{\annotpro@widthTo}{\ap@thisWidth}}%
            \edef\ap@thisWidth{\annotpro@widthTo}%
         \fi
     \else %scale
        \edef\ap@thisWidth{(\ap@thisWidth)*\noexpand
            \real{\annotpro@scale}}%
        \edef\ap@thisHeight{(\ap@thisHeight)*\noexpand
            \real{\annotpro@scale}}%
     \fi
     \setlength{\@tempdima}{\ap@thisWidth}%
     \edef\ap@thisWidth{\the\@tempdima}%
     \setlength{\@tempdima}{\ap@thisHeight}%
     \edef\ap@thisHeight{\the\@tempdima}%
      \ifx\annotpro@rotate\empty\else
          \def\ap@this@Rotate{/Rotate \annotpro@rotate}%
          \ap@set@Rotate@BBox
      \fi
  \or
%    \end{macrocode}
% \textbf{File Attachment annotation.} There are only a few icons for
% file attachments, and I've determined their dimensions using the user
% interface. So, we use those.
%
% If a name has not been specified, we use the default (\texttt{PushPin}).
%    \begin{macrocode}
    \ifx\annotpro@@name\@empty\edef\annotpro@@name{\ap@def@name}\fi
    \ifx\annotpro@file\@empty\PackageError{annot_pro}%
        {You must specify a path the file to be
         attached using the file key.}\fi
%    \end{macrocode}
% Set the \textbf{FS} key to specify the file.
%    \begin{macrocode}
    \def\ap@this@FS{/FS {apEmbedFile\the\ap@annot@cnt}}%
    \edef\ap@thisWidth{\@nameuse{ap@\annotpro@@name Width}}%
    \edef\ap@thisHeight{\@nameuse{ap@\annotpro@@name Height}}%
    \ap@EmbedThisFile
  \or
%    \end{macrocode}
% \textbf{Free text annotation.}
%    \begin{macrocode}
     \ifx\annotpro@@name\@empty\edef\annotpro@@name{\ap@def@name}\fi
     \ifx\annotpro@width\@empty\else
        \setlength\@tempdima{\annotpro@width}%
            \def\annotpro@width{\the\@tempdima}\fi
    \edef\ap@thisWidth{\ifx\annotpro@width\@empty
        \@nameuse{ap@\annotpro@@name Width}\else
        \annotpro@width\fi}%
     \ifx\annotpro@height\@empty\else
        \setlength\@tempdima{\annotpro@height}%
            \def\annotpro@height{\the\@tempdima}\fi
    \edef\ap@thisHeight{\ifx\annotpro@height\@empty
        \@nameuse{ap@\annotpro@@name Height}\else
        \annotpro@height\fi}%
  \fi
%    \end{macrocode}
% if \texttt{margin} is specified, we insert a \cs{marginpar} here.
%    \begin{macrocode}
  \ifx\annotpro@margin\@empty\let\apmargintext\@empty
    \expandafter\relax\else\expandafter\marginpar\fi
%    \end{macrocode}
% Set the bounding box, and layout the annotation using the \textbf{pdfmark} operator.
%    \changes{v1.1c}{2017/06/06}{added \string\cs{anp@margprior} followed by braces, so
%    \string\cs{anp@margprior} could operate on the annot as a whole.}
%    \begin{macrocode}
  {\anp@margprior{%
   \pdf@rect{\ap@Bbox{\ap@thisWidth}{\ap@thisHeight}}\literalps@out{%
  [ \eq@objdef\ap@Rect
    /F 4\ap@hidden\ap@readonly
    /Open \annotpro@open
    /Title (\annotpro@enc@title)
    /Subj (\annotpro@enc@subject)
%    \end{macrocode}
%    We have different set of keys for FreeText when for the other annotation types.
%    \begin{macrocode}
\if\ap@save@type\ap@freetext@type
%    \end{macrocode}
%    The definitions of \cs{eq@RV@Body} and \cs{eq@RV@endBody} come from \pkg{richtext}, originally from
%    \pkg{eforms}. The \psf{RC} and \psf{Contents} keys get their entries from \cs{annotpro@RC} and
%    \cs{annotpro@C}, which are acquired from the \pkg{richtext} package.
%    \begin{macrocode}
    /RC(\eq@RV@Body\annotpro@RC\eq@RV@endBody)
    /Contents (\annotpro@C)
    /DS (\ap@defaultstyle)
    \ifx\annotpro@bgcolor\@empty\else
    /C [\annotpro@enc@bgcolor]\fi
    /DA (\annotpro@enc@bcolor\space/Helv 12 Tf)
    \ap@border@code
\else
    /Open \annotpro@open
    /Contents (\annotpro@enc@contents)
    /Color [\annotpro@enc@color]
\fi
%    \end{macrocode}
% \changes{v1.0a}{2011/07/14}{Changed the value of the \texttt{/Name} key.}
% Changed the value of \texttt{/Name} form a name to a string
%    \begin{macrocode}
    /Name (\annotpro@@name)
    /Subtype/\ap@type
    /NM (\currentAnnotName)
%   /NM (annotpro\the\ap@annot@cnt)
    /M (D:\ap@pdfdate)
    \ap@this@AP\ap@this@FS\ap@this@CA\ap@this@Rotate
    /ANN pdfmark}\apmargintext}}%end grouping, end marginpar
    \to@insertStrucTabOrder{Annot}\endgroup
}
%    \end{macrocode}
%    \end{macro}
% When the user specifies the rotate key, we need to enlarge the
% bounding box so the stamp can be rotated within that box.
%    \begin{macrocode}
\def\ap@set@Rotate@BBox{%
    \CalculateSin\annotpro@rotate\CalculateCos\annotpro@rotate
    \edef\c@cos{\UseCos\annotpro@rotate}%
    \edef\c@sin{\UseSin\annotpro@rotate}%
%    \end{macrocode}
% Determine the quadrant i or iii versus ii or iv
%    \begin{macrocode}
    \@tempdimb=\c@cos pt
    \@tempdimb=\c@sin\@tempdimb
    \let\ap@sign\@empty
    \ifdim\@tempdimb<\z@\def\ap@sign{-}\fi
%    \end{macrocode}
%What we are doing is calculating the \emph{smallest rectangle} whose sides are parallel
%to the horizontal and vertical, that contained the rotated stamp. We use this computed
%rectangle as the bounding rectangle of the rectangle.
%
%The basic algorithm is as follows:
% If the angle is in the first or third quadrant, we use
%\begin{align*}
%    \text{width}  &=\text{width}'\cos(\theta)+\text{height}'\sin(\theta)\\
%    \text{height} &=\text{width}'\sin(\theta)+\text{height}'\cos(\theta)
%\end{align*}
%else, we use
%\begin{align*}
%    \text{width} &=-\text{width}'\cos(\theta)+\text{height}'\sin(\theta)\\
%    \text{height}&=\text{width}'\sin(\theta)-\text{height}'\cos(\theta)
%\end{align*}
% where, $\text{width}'$ and $\text{height}'$ are the original width and height of the bounding rectangle,
% and $\text{width}$ and $\text{height}$ are the new dimensions of the rectangle, after rotation through
% an angle of $\theta$.
% \paragraph*{Width calculation.} Width held in \cs{ap@temp@thisWidth}.
%    \begin{macrocode}
    \@tempdima=\ap@sign\ap@thisWidth
    \@tempdima=\c@cos\@tempdima
    \@tempdimb=\ap@thisHeight
    \@tempdimb=\c@sin\@tempdimb
    \advance\@tempdima by\@tempdimb
    \edef\ap@temp@thisWidth{\the\@tempdima}%
%    \end{macrocode}
% \paragraph*{Height calculation.} Final height held is \cs{@tempdima}.
%    \begin{macrocode}
    \@tempdima=\ap@thisWidth
    \@tempdima=\c@sin\@tempdima
    \@tempdimb=\ap@thisHeight
    \@tempdimb=\c@cos\@tempdimb
    \advance\@tempdima by\ap@sign\@tempdimb
%    \end{macrocode}
% Any negative dimensions are changed to positive.
%    \begin{macrocode}
    \@tempdimb=\ap@temp@thisWidth\relax
    \ifdim\@tempdimb<\z@\@tempdimb=-\the\@tempdimb\fi
    \ifdim\@tempdima<\z@\@tempdima=-\the\@tempdima\fi
%    \end{macrocode}
% Here are the final calculations, \cs{ap@thisWidth} and \cs{ap@thisHeight}.
%    \begin{macrocode}
    \edef\ap@thisWidth{\the\@tempdimb}%
    \edef\ap@thisHeight{\the\@tempdima}%
}
%    \end{macrocode}
% Some code for embedding a file in the PDF document.
%    \begin{macrocode}
\def\ap@EmbedThisFile{%
    \literalps@out{%
    [ /_objdef {apEmbedFile\the\ap@annot@cnt}/type/dict/OBJ pdfmark
    [ {apEmbedFile\the\ap@annot@cnt} <<
        /F (\ap@thisFileName)
        /UF (\ap@thisFileName)
        /EF << /F {apFileStream\the\ap@annot@cnt} >>
        /Type/Filespec
    >> /PUT pdfmark}%
    \literalps@out{%
    [ /_objdef {apFileStream\the\ap@annot@cnt}/type/stream/OBJ pdfmark
    [ {apFileStream\the\ap@annot@cnt} (\annotpro@file)
        (r) file /PUT pdfmark
    [ {apFileStream\the\ap@annot@cnt} <<
        /Type/EmbeddedFile
    >> /PUT pdfmark
    [ {apFileStream\the\ap@annot@cnt} /CLOSE pdfmark}%
}
%</package>
%<*scandoc>
%    \end{macrocode}
%    \subsection{Document assembly}
%    The document assembly code is executed when the \texttt{scandoc} option is used. It is executed
%    then the document is first opened.
%    \begin{macrocode}
\begin{execJS}{scan4anp}
var aANPStamps=new Array();
if (typeof scancomplete=="undefined") {
    var scancomplete=false;
    this.syncAnnotScan();
    for (var p=0; p<this.numPages; p++) {
         a=this.getAnnots(p);
         if (a!=null) {
            for (var i=0; i<a.length; i++) {
%    \end{macrocode}
%    At this time, we search only for annots of type \texttt{"Stamp"} and with AP beginning
%    with `\texttt{\#}', for these are user installed stamps or dynamic stamps.
%    \begin{macrocode}
                if ( a[i].type=="Stamp" && a[i].AP.indexOf("\#")==0 ) {
                    aANPStamps.push(p);
                    break;
                }
            }
        }
    }
}
ANPscrollPage.index=0;
function ANPscrollPage() {
%    \end{macrocode}
%    We go to the page where each stamp of the type we are interested in. Opening the page
%    causes \app{Acrobat} to populate the dynamic stamps.
%    \begin{macrocode}
    if (ANPscrollPage.index<aANPStamps.length) {
        this.pageNum=aANPStamps[ANPscrollPage.index];
        ANPscrollPage.index+=1;
    } else {
        app.clearInterval(anpTO);
        scancomplete=true;
        this.pageNum=0;
    }
}
if (aANPStamps.length>0)
    var anpTO=app.setInterval("ANPscrollPage()", 5);
\end{execJS}
%</scandoc>
%<*package>
%    \end{macrocode}
% Set global options for this package.
%    \begin{macrocode}
\setAnnotOptions{title,subject,open=false,color}
%    \end{macrocode}
% Input \texttt{scandoc-anp.def}, if required.
%    \begin{macrocode}
\anp@InputScanDoc
%</package>
%    \end{macrocode}
%  \Finale
\endinput

%%%%%%%%%%%%%%%%%%%%%%%%%CUT HERE%%%%%%%%%%%%%%%%%%%%%%%%%%%%%%%%%%

% This is ptexproc.tex, an example file for use with the SIAM Plain TeX
% Proceedings Series macros. Comments are placed at the beginning and 
% throughout this file. Please take the time to read them as they document
% how to use these macros. This file can be composed and printed out for
% use as sample output. Please ignore the underfull \vbox on page 3.

% Any comments or questions regarding these macros should be directed to:
%
%                 Corey Gray
%                 SIAM
%                 3600 University City Science Center
%                 Philadelphia, PA 19104-2688
%                 USA
%                 Telephone: (215) 382-9800
%                 Fax: (215) 386-7999
%                 e-mail: gray@siam.org


% This file is to be used as an example for style only. It should not be read
% for content.

%%%%%%%%%%%%%%% PLEASE NOTE THE FOLLOWING STYLE RESTRICTIONS %%%%%%%%%%%%%%%

%%  1. You must use the numbered reference style([1],[2]), listing the 
%%     references at the end of the chapter either by order of citation
%%     or alphabetically.
%% 
%%  2. This macro is set up for three levels of headings. Use the commands
%%     \headone, \headtwo, and \headthree. The macro will automatically
%%     number the headings.
%%
%%  3. Theorems, Lemmas, Definitions, etc. are to be double-numbered, 
%%     indicating the section and the occurrence of that element
%%     within that section. (For example, the first theorem in the second
%%     section would be numbered 2.1.) This numbering must
%%     be done manually. 
%%
%%  4. Proofs are handled by \prf\endprf. If you want to use an end-of-proof
%%     box, insert \qed right before the \endprf command.
%%
%%  5. Figures and equations must be manually single-numbered. Use \leqno 
%%     for equation numbering. The macro provides the \fig for including 
%%     figures. This command consists of three fields. The first field is
%%     used for inserting the appropriate space for the figure. The second 
%%     field is the figure number. The third field is the caption. See the 
%%     example included in this file. SIAM supports the use of psfig for 
%%     including Postscript figures. All Postscript figures should be sent 
%%     as separate files. A hardcopy version of all Postscript figures is 
%%     also required. See note regarding this under How to Submit Your Paper.
%%
%%  6. Use of \title\endtitle and \lasttitle\endlastitle.
%%     This macro package provides two possible commands for handling the 
%%     title of your paper. The commands \title\endtitle should be used for
%%     all lines except the last line of multiple line titles. The commands 
%%     \lasttitle\endlasttitle should be used for the last line of multiple 
%%     line titles. In the case of a single line title, \lasttitle\endlasttitle
%%     should be used.
%%
%%  7. Use of \author\endauthor and \lastauthor\endlastauthor.
%%     As in the title macro, two possible commands are provided for the 
%%     author. The commands \author\endauthor should be used for the first
%%     line of authors if there are more than one line. The commands
%%     \lastauthor\endlastauthor should be used for the last line
%%     of multiple lines of authors. In the case of a single line of authors,
%%     \lastauthor\endlastauthor should be used. A maximum of four authors 
%%     should be placed on any one line. The appropriate space must also be
%%     hard coded between authors on the same line. The spacing is as follows:
%%       
%%       If 2 authors; \hskip4pc between
%%       If 3 authors; \hskip3pc between
%%       If 4 authors; \hskip2pc between
%%
%% 8. Grant information and author affiliations.
%%     This information is included by using the \footnote command and the 
%%     appropriate footnote symbol. SIAM uses footnote symbols in a 
%%     particular order. Below is a list of these symbols:
%%
%%      asterisk                
%%      single-dagger           
%%      double-dagger           
%%      section sign            
%%      paragraph               
%%      parallel                
%%      double asterisk         
%%      double single-dagger    
%%      double double-dagger    
%%
%%     For illustrative purposes, all footnote symbols have been used in the
%%     example file.
%%
%%     A note regarding \footnote. This command seems to leave extra white
%%     space between footnotes. This is quite evident in the example file. If
%%     any user of these macros has a solution to this problem, I would 
%%     appreciate hearing from you. Send your comments to: gray@siam.org.
%%
%%    The following general rules apply for including grants and affiliations:
%%      a) If there is a single grant for the paper, then the grant 
%%         information should be footnoted to the title.
%%      b) If there is more than one grant, included the grant information
%%         with each authors affiliation.
%%      c) If there are different grants for the paper but the authors share
%%         the same affiliation, footnote the grant information to the title.
%%         For example, The work of the first author was supported by xyz.
%%         The work of the second author was supported by abc. And so on.
%%
%%
%% 9. Special fonts.
%%     SIAM supports the use of AMS-TeX fonts version 2.0 and later. As 
%%     described in the manual for these fonts, they can be included by
%%     \input{amssym.def} and \input{amssym.tex}. 
%%
%% 

%%%%%%%%%%%%%%%%%%%%%%%%%%%%%%%%%%%%%%%%%%%%%%%%%%%%%%%%%%%%%%%%%%%%%%%%%%%%%%%

\input ptexproc.sty

\def\leftrh{}
\def\rightrh{}

\startchapter %Place this command at the beginning of the file immediately
              %after the \input command.



\title SIAM Proceedings Series Macros\endtitle
\lasttitle for Use with Plain TeX\footnote*{Any information 
regarding grants should be placed here.}\endlasttitle

\author J. Corey Gray\footnote{$^{\dag}$}{Production Manager, Society for
Industrial and Applied Mathematics, Philadelphia, PA.}
\hskip2truepc Tricia Manning\footnote{$^{\ddag}$}{Publications 
Specialist, Society for Industrial and Applied Mathematics, Philadelphia, 
PA.}
\hskip2truepc Vickie Kearn\footnote{$^{\S}$}{Publisher, Society for Industrial
and Applied Mathematics, Philadelphia, PA.}
\hskip2truepc Nancy Abbott\footnote{$^{\P}$}{Design Supervisor, Society for
Industrial and Applied Mathematics, Philadelphia, PA.}
\endauthor
\lastauthor Sue Ciambrano\footnote{$^{\parallel}$}{Acquisition Editor, Society for
Industrial and Applied Mathematics, Philadelphia, PA.}
\hskip2truepc Paul Duggan\footnote{$^{**}$}{Composition 
Specialist, Society for Industrial and Applied Mathematics, Philadelphia, 
PA.}
\hskip2truepc Robbi Anne Albert\footnote{$^{\dagger\dagger}$}{Production, 
Assistant, Society for Industrial
and Applied Mathematics, Philadelphia, PA.}
\hskip2truepc Jean Anderson\footnote{$^{\ddag\ddag}$}{Composition Coordinator, Society for
Industrial and Applied Mathematics, Philadelphia, PA.}
\endlastauthor

% Use \headone for the first level headings. The macro will automatically
% number the headings.

\centerline{\bf Abstract}
\abstract An equivalence is shown between realizability of input/output (i/o) operators by
rational control systems and high-order algebraic differential equations for
i/o pairs.  This generalizes, to nonlinear systems, the equivalence
between autoregressive representations and finite dimensional linear
realizability.\endabstract


\headone{Problem Specification}
In this paper, we consider the solution of the $N \times N$ linear
system
$$A x = b\leqno (1)$$
where $A$ is large, sparse, symmetric, and positive definite.  We consider
the direct solution of by means of general sparse Gaussian
elimination.  In such a procedure, we find a permutation matrix $P$, and
compute the decomposition
$$
P A P^{t} = L D L^{t}
\leqno (2)$$

\noindent where $L$ is unit lower triangular and $D$ is diagonal.  


\headone{Design Considerations}
Several good ordering algorithms (nested dissection and minimum degree)
are available for computing $P$  [1], [2].
Since our interest here does not
focus directly on the ordering, we assume for convenience that $P=I$,
or that $A$ has been preordered to reflect an appropriate choice of $P$.

% Use \thm and \endthm for theorems. They must be numbered manually.
% Lemmas (\lem \endlem), corollaries (\cor \endcor), and 
% propositions (\prop \endprop) are coded the same as theorems and must
% also be numbered manually.

\thm{Theorem 2.1.} The method  was extended to three
dimensions. For the standard multigrid
coarsening
(in which, for a given grid, the next coarser grid has $1/8$
as many points), anisotropic problems require plane
relaxation to
obtain a good smoothing factor.\endthm 

Several good ordering algorithms (nested dissection and minimum degree)
are available for computing $P$  [1], [2].
Since our interest here does not
focus directly on the ordering, we assume for convenience that $P=I$,
or that $A$ has been preordered to reflect an appropriate choice of $P$.
Several good ordering algorithms (nested dissection and minimum degree)
are available for computing $P$  [1], [2].
Since our interest here does not
focus directly on the ordering, we assume for convenience that $P=I$,
or that $A$ has been preordered to reflect an appropriate choice of $P$.

% Use \prf and \endprf to begin and end a proof.
% The use of \qed will produce an end-of-proof box.

\prf{Proof} In this paper we consider two methods. The first method
is
basically the method considered with two differences:
first, we perform plane relaxation by a two-dimensional
multigrid method, and second, we use a slightly different
choice of
interpolation operator, which improves performance
for nearly singular problems. In the second method coarsening
is done by successively coarsening each.\qed\endprf

% Use \dfn and \enddfn to begin and end definitions.

\dfn{Definition 2.1.}We describe the two methods in \S\ 1.2. This is a 
definition in the plain tex macro.\enddfn 

This is accomplished by exploiting the m-tree,
a particular spanning tree for the graph of the filled-in matrix.
Our purpose here is to examine the nonnumerical complexity of the
sparse elimination algorithm given in  [3].
As was shown there, a general sparse elimination scheme based on the
bordering algorithm requires less storage for pointers and
row/column indices than more traditional implementations of general
sparse elimination. This is accomplished by exploiting the m-tree,
a particular spanning tree for the graph of the filled-in matrix.


Our purpose here is to examine the nonnumerical complexity of the
sparse elimination algorithm given in  [3].
As was shown there, a general sparse elimination scheme based on the
bordering algorithm requires less storage for pointers and
row/column indices than more traditional implementations of general
sparse elimination. This is accomplished by exploiting the m-tree,
a particular spanning tree for the graph of the filled-in matrix.
Since our interest here does not
focus directly on the ordering, we assume for convenience that $P=I$,
or that $A$ has been preordered to reflect an appropriate choice of $P$.

% Use \lem and \endlem to begin and end lemmas.

\lem{Lemma 2.1.}We discuss first the choice for $I_{k-1}^k$
which is a generalization. We assume that $G^{k-1}$ is
obtained
from $G^k$
by standard coarsening; that is, if $G^k$ is a tensor product
grid $G_{x}^k \times G_{y}^k \times G_{z}^k$,
$G^{k-1}=G_{x}^{k-1} \times G_{y}^{k-1} \times G_{z}^{k-1}$,
where $G_{x}^{k-1}$ is obtained by deleting every other grid
point of $G_x^k$ and similarly for $G_{y}^k$ and $G_{z}^k$.
\endlem

% Use \fig to insert space for figures .

\fig{10pc}{Fig. 1}{This is the caption for figure one.}

To our knowledge, the m-tree previously has not been applied in this
fashion to the numerical factorization, but it has been used,
directly or indirectly, in several optimal order algorithms for
computing the fill-in during the symbolic factorization phase
[4] - [10], [5], [6]. In \S 1.3., we analyze the complexity of the old and new
approaches to the intersection problem for the special case of
an $n \times n$ grid ordered by nested dissection. The special
structure of this problem allows us to make exact estimates of
the complexity. To our knowledge, the m-tree previously has not been applied in this
fashion to the numerical factorization, but it has been used,
directly or indirectly, in several optimal order algorithms for
computing the fill-in during the symbolic factorization phase
[4] - [10], [5], [6].

% Use \headtwo for second level headings. They will be numbered automatically.

\headtwo{Robustness}In \S 1.2, we review the bordering algorithm, and introduce
the sorting and intersection problems that arise in the
sparse formulation of the algorithm.  

\headtwo{Versatility} In \S 1.3., we analyze the complexity of the old and new
approaches to the intersection problem for the special case of
an $n \times n$ grid ordered by nested dissection. The special
structure of this problem allows us to make exact estimates of
the complexity. To our knowledge, the m-tree previously has not been applied in this
fashion to the numerical factorization, but it has been used,
directly or indirectly, in several optimal order algorithms for
computing the fill-in during the symbolic factorization phase
[4] - [10], [5], [6].


% Use \headthree for third level headings.

\headthree{Complexity.}For the old approach, we show that the
complexity of the intersection problem is $O(n^{3})$, the same
as the complexity of the numerical computations.  For the
new approach, the complexity of the second part is reduced to
$O(n^{2} (\log n)^{2})$.  

% The command \Refs sets the word Reference as a heading and allows the proper
% amount of space before the start of the references. Each reference must
% begin with \ref\\. The article or title of the reference should be in 
% italic. Use the \it command within brackets. End each reference with
% \endref and allow two returns between references. Use the command
% \sameauthor (see reference 8)  when the same author or group of authors
% is listed consecutively.

\Refs

\ref 1\\R.~E. Bank, {\it PLTMG  users' guide, edition 5.0}, tech. report,
  Department of Mathematics, University of California, San Diego, CA, 1988.\endref

\ref 2\\R.~E. Bank, T.~F. Dupont, and H.~Yserentant, {\it The hierarchical basis
  multigrid method}, Numer. Math., 52 (1988), pp.~427--458.\endref

\ref 3\\R.~E. Bank and R.~K. Smith, {\it General sparse elimination requires no
  permanent integer storage}, SIAM J. Sci. Stat. Comput., 8 (1987),
  pp.~574--584.\endref

\ref 4\\S.~C. Eisenstat, M.~C. Gursky, M.~Schultz, and A.~Sherman, {\it
  Algorithms and data structures for sparse symmetric gaussian elimination},
  SIAM J. Sci. Stat. Comput., 2 (1982), pp.~225--237.\endref

\ref 5\\A.~George and J.~Liu, {\it Computer Solution of Large Sparse Positive
  Definite Systems}, Prentice Hall, Englewood Cliffs, NJ, 1981.\endref

\ref 6\\K.~H. Law and S.~J. Fenves, {\it A node addition model for symbolic
  factorization}, ACM TOMS, 12 (1986), pp.~37--50.\endref

\ref 7\\J.~W.~H. Liu, {\it A compact row storage scheme for cholesky factors
  using elimination trees}, ACM TOMS, 12 (1986), pp.~127--148.\endref

\ref 8\\\sameauthor , {\it The role of
  elimination trees in sparse factorization}, Tech. Report CS-87-12,Department
  of Computer Science, York University, Ontario, Canada, 1987.\endref

\ref 9\\D.~J. Rose, {\it A graph theoretic study of the numeric solution of
  sparse positive definite systems}, in Graph Theory and Computing, 
  Academic Press, New York, 1972.\endref

\ref 10\\D.~J. Rose, R.~E. Tarjan, and G.~S. Lueker, {\it Algorithmic aspects of
  vertex elimination on graphs}, SIAM J. Comput., 5 (1976), pp.~226--283.\endref


\bye

%end of example file

\documentclass{article}
\usepackage[fleqn]{amsmath}
\usepackage[
    web={centertitlepage,designv,forcolorpaper,tight*,latextoc,extended},
    eforms={usealtadobe,setcorder},aebxmp
]{aeb_pro}
\usepackage[dynamic]{bargraph-js}
%\usepackage[ImplMulti]{dljslib}
\usepackage{graphicx,array,fancyvrb}
\usepackage{aeb_mlink}
%\usepackage{myriadpro}
%\usepackage{calibri}
\usepackage[altbullet]{lucidbry}

\hfuzz2pt
\makePDasXOn
 %\previewOn\pmpvOn

\def\hardspace{{\fontfamily{cmtt}\selectfont\symbol{32}}}
\let\uif\textsf

\advance\marginparwidth12pt

\usepackage{acroman}
\usepackage[active]{srcltx}

\edef\amtIndent{\the\parindent}

\addtolength{\marginparwidth}{2pt}

\urlstyle{tt}

\def\STRUT{\rule{0pt}{14pt}}

\makeatletter
\newcount\hesheCnt \hesheCnt=-1
\def\heshe{\@ifstar{\heshei}{\global\advance\hesheCnt1\relax\heshei}}
\def\heshei{\ifodd\hesheCnt she\else he\fi}
\def\HeShe{\@ifstar{\HeShei}{\global\advance\hesheCnt1\relax\HeShei}}
\def\HeShei{\ifodd\hesheCnt She\else He\fi}
\def\hisher{\@ifstar{\hisheri}{\global\advance\hesheCnt1\relax\hisheri}}
\def\hisheri{\ifodd\hesheCnt her\else his\fi}
\def\himher{\@ifstar{\himheri}{\global\advance\hesheCnt1\relax\himheri}}
\def\himheri{\ifodd\hesheCnt her\else him\fi}
\makeatother

\DeclareDocInfo
{
    university={\AcroTeX.Net},
    title={The \textsf{bargraph-js} Package},
    author={D. P. Story},
    email={dpstory@acrotex.net},
    subject=Documentation for the bargraph-js package,
    talksite={\url{www.acrotex.net}},
    version={0.7,2019/04/07},
    Keywords={LaTeX, form fields, bargraphs using JavaScript, AcroTeX},
    copyrightStatus=True,
    copyrightNotice={Copyright (C) \the\year, D. P. Story},
    copyrightInfoURL={http://www.acrotex.net}
}

\universityLayout{fontsize=Large}
\titleLayout{fontsize=LARGE}
\authorLayout{fontsize=Large}
\tocLayout{fontsize=Large,color=aeb}
\sectionLayout{indent=-62.5pt,fontsize=large,color=aeb}
\subsectionLayout{indent=-31.25pt,color=aeb}
\subsubsectionLayout{indent=0pt,color=aeb}
\subsubDefaultDing{\texorpdfstring{$\bullet$}{\textrm\textbullet}}

\begin{insDLJS*}{lbl}
\begin{newsegment}{Labeling function}
function customLabelsForBars(fld,v){
var pos=fld.indexOf(".");
var bargraph=fld.substring(0,pos);
var bar=fld.substring(pos+1);
  switch(bargraph) {
    case "vehiclesH@hBar":
      switch(bar) {
        case "auto":
          return "Automobiles: "+v;
        case "truck":
         return "Pickup trucks: "+v;
        case "suv":
          return "SUV types: "+v+", costing big \$\$s";
        case "van":
          return "Family vans: "+v
            +", these cost some serious \u20AC\u20ACs";
        default:
          return simpleBarLabels(fld,v);
      }
// other cases can be included
    default:
      return simpleBarLabels(fld,v);
  }
}
\end{newsegment}
\begin{newsegment}{Validate input data}
function validateArrayNonNegNums(aValue){
  var bOk=true;
  for (var i=0; i<aValue.length; i++) {
    var x=1*aValue[i];
    if(isNaN(x)) {bOk=false; break;}
    else if (x<0) {bOk=false; break;}
  }
  return bOk
}
\end{newsegment}
\begin{newsegment}{Display Table of Probabilities}
function displayTable(fld,aDistr) {
  var tableHead=util.printf("   \%s","k ")
    +util.printf("     \%s     ","pmf")
    +util.printf("     \%s     ","cdf")+"\r";
  var str=tableHead;
  for (var i=0; i<aDistr.length;i++) {
    str +=(util.printf("\%4d",aDistr[i][0])
      + util.printf("\%10.6f",aDistr[i][1])
      + util.printf("\%10.6f",aDistr[i][2])
      + "\r"
    );
  }
  var f=this.getField(fld);
  if (f!=null) f.value=str;
}
\end{newsegment}
\end{insDLJS*}
\barLabelsTU{customBarLabels} % applies to all bars with a \TU key
%\barLabelsNoTU{customLabelsForBars} % applies to all other bars


\chngDocObjectTo{\newDO}{doc}
\begin{docassembly}
var titleOfManual="The bargrah-js Package";
var manualfilename="Manual_BG_Print_bgraph-js.pdf";
var manualtemplate="Manual_BG_Brown.pdf"; // Blue, Green, Brown
var _pathToBlank="C:/Users/Public/Documents/ManualBGs/"+manualtemplate;
var doc;
var buildIt=false;
if ( buildIt ) {
    console.println("Creating new " + manualfilename + " file.");
    doc = \appopenDoc({cPath: _pathToBlank, bHidden: true});
    var _path=this.path;
    var pos=_path.lastIndexOf("/");
    _path=_path.substring(0,pos)+"/"+manualfilename;
    \docSaveAs\newDO ({ cPath: _path });
    doc.closeDoc();
    doc = \appopenDoc({cPath: manualfilename, oDoc:this, bHidden: true});
    f=doc.getField("ManualTitle");
    f.value=titleOfManual;
    doc.flattenPages();
    \docSaveAs\newDO({ cPath: manualfilename });
    doc.closeDoc();
} else {
    console.println("Using the current "+manualfilename+" file.");
}
var _path=this.path;
var pos=_path.lastIndexOf("/");
_path=_path.substring(0,pos)+"/"+manualfilename;
\addWatermarkFromFile({
    bOnTop:false,
    bOnPrint:false,
    cDIPath:_path
});
\executeSave();
\end{docassembly}


\begin{document}

\maketitle

\selectColors{linkColor=black}
\tableofcontents
\selectColors{linkColor=webgreen}


\section{Introduction}\label{intro}

This package can create bar graphs that represent categorical or quantitative
data; special techniques can be use to create bargraph representations of a
discrete probability distribution.

Bar graphs created by this package are \emph{dynamic}; that is, the user
can enter data and the bar graph changes to reflect this new data. This magic
is performed using \app{Acrobat} forms and \app{Acrobat} JavaScript. The
individual ``bars'' of a bar graph are form fields, JavaScript is used to
change their bounding rectangles.

The document requires, for the dynamic bar graph features to work, the user to
be viewing the document in \app{AR} (or the \app{Acrobat} application
(\app{AA}), of course). JavaScript even within \app{AR} is not fully
supported on all devices, as a result, \app{AR} or \app{AA} on a desktop or
laptop is required.

Before we start describing this package, we present a simple example. Enter nonnegative numbers
in the text fields to the right.\vcgBdry[6bp]

\fbox{\begin{bargraphenv}[width=(23bp*4),height=2in,o=vert]{vehiclesV}
\presetsbarfor{vBar}{auto}{\BG{red}}
\presetsbarfor{vBar}{truck}{\BG{green}}
\presetsbarfor{vBar}{suv}{\BG{yellow}}
\presetsbarfor{vBar}{van}{\BG{magenta}}
\begin{bargraph}[nbars=4,gap=3]{vBar}
\barfor{auto}\barfor{truck}\barfor{suv}\barfor{van}
\end{bargraph}
\end{bargraphenv}}\hfill
\begin{minipage}[b][2in][c]
  {\linewidth-2\fboxsep-2\fboxrule-(23bp*4)-10pt}\kern0pt\parskip3pt
\makebox[\widthof{Truck:}][l]{Auto:}
  \inputFor{vehiclesV}{vBar}{auto}{.5in}{11bp}\vcgBdry[3bp]
\makebox[\widthof{Truck:}][l]{Truck:}
  \inputFor{vehiclesV}{vBar}{truck}{.5in}{11bp}\vcgBdry[3bp]
\makebox[\widthof{Truck:}][l]{SUV:}
  \inputFor{vehiclesV}{vBar}{suv}{.5in}{11bp}\vcgBdry[3bp]
\makebox[\widthof{Truck:}][l]{Van:}
  \inputFor{vehiclesV}{vBar}{van}{.5in}{11bp}\vcgBdry[4bp]
\pushButton[\TU{This button re-scales the bar graph so that the
tallest bar takes the entire height of the region.}\CA{Optimize}
  \A{\JS{optimizeScaling("vehiclesV");}
}]{optimize2}{}{13bp}
\end{minipage}\vcgBdry[6bp]

Reset vertical bar graph:\marginpar{\small\itshape
\raisebox{2in+6bp}{\parbox[t]{\marginparwidth}{\raggedleft
  This example appears in \texttt{examples/\allowbreak bgjs-basic1.tex}}}}
  \pushButton[\CA{Reset}\A{\JS{resetBargraphs("vehiclesV");}}
]{reset}{}{13bp}\vcgBdry[6bp]

\textbf{Motivation.} The motivation for the package comes from the request of
an {Acro\negthinspace\TeX} user. He wanted a bar graph that summarizes the
results of a questionnaire. Each of some eighty question is classified into
one of eight categories. A bar graph is then generated based on the
responses.

\section{Requirements and options of the package}

The \pkg{bargraph-js} package uses the \pkg{eforms} package
dated 2019/03/16 or later. Use the package in the usual way,
\begin{Verbatim}[xleftmargin=\amtIndent,commandchars=!()]
\documentclass{article}
\usepackage{bargraph-js}
\end{Verbatim}
There are no package options. If you want to use options for the \pkg{eforms} package, load
it earlier than \pkg{bargraph-js}:
\begin{Verbatim}[xleftmargin=\amtIndent,commandchars=!()]
\documentclass{article}
\usepackage[usealtadobe,setcorder]{eforms}
\usepackage{bargraph-js}
\end{Verbatim}

\part{Bar graph with a fixed number of bars}

In this part of the manual, the construction of bar graphs is discussed with a known number
of bars.

\section{The bar graph environments}

The package defines two environments, these are \env{bargraphenv} and
\env{bargraph}; the latter environment is placed within the former one.
\bVerb\takeMeasure{\string\begin\darg{bargraphenv}[\ameta{bgenv-options}]\darg{\ameta{bgenv-name}}}%
\def\1{\hskip\amtIndent\relax}%
\begin{dCmd}[commandchars=!()]{\bxSize}
\begin{bargraphenv}[!ameta(bgenv-options)]{!ameta(bgenv-name)}
!1!meta(\presetsbarfor)!sffamily!textsl( commands, optional)
!1\begin{bargraph}[!ameta(bg-options)]{!ameta(bg-name)}
!1!1\barfor{!ameta(bar-name!SUB1)}...\barfor{!ameta(bar-name!SUB(n))}
!1\end{bargraph}
!1!sffamily!textsl(additional bargraph environments, optional)
\end{bargraphenv}
\end{dCmd}
\eVerb The \ameta{bgenv-name} argument must be unique throughout the
document, the other names (\ameta{bg-name} and \ameta{bar-name}) may be
reused in other \env{bargraphenv} environments. Within a \env{bargraphenv}
environment, however, the \ameta{bg-name} must be unique, that is, not repeated. The
\cs{presetsbarfor} commands are a way of changing the appearances of the
individual bars. Within the \env{bargraph} environment, only \cs{barfor}\marginpar{\small\itshape\raggedleft Only \cs{barfor}
and \cs{cmd} are permitted} or
\cs{cmd} commands are recognized (the latter is not shown). The \cs{barfor}
command gives a name to the bar so it can be referenced and manipulated. More
than one \env{bargraph} environment within a \env{bargraphenv}
environment is supported.

Before describing the individual components, we present the verbatim listing of
the example given above.\def\1{\hskip\amtIndent\relax}
\begin{Verbatim}[fontsize=\small,commandchars={!~@}]
\fbox{%
!1\begin{bargraphenv}[width=(23bp*4),height=2in,o=vert]{vehiclesV}
!1!1\presetsbarfor{vBar}{auto}{\BG{red}}
!1!1\presetsbarfor{vBar}{truck}{\BG{green}}
!1!1\presetsbarfor{vBar}{suv}{\BG{yellow}}
!1!1\presetsbarfor{vBar}{van}{\BG{magenta}}
!1!1\begin{bargraph}[nbars=4,gap=3]{vBar}
!1!1!1\barfor{auto}\barfor{truck}\barfor{suv}\barfor{van}
!1!1\end{bargraph}
!1\end{bargraphenv}
}
\end{Verbatim}
The key-values of \env{bargraphenv} set the dimensions of the `display
window' (here, we use \cs{fbox} to enclose it). The option \texttt{o=vert}
states these bar graphs will be vertically oriented. The next four lines are
\cs{presetsbarfor} commands that pass appearances on to the bars themselves;
this accounts for the coloring of the individual bars demonstrated in the
interactive example above. The optional key-values for the \env{bargraph}
environment are \texttt{nbars=4} (the number of bars in this bar graph,
required); and \texttt{gap=3} (distance between bars in big points)

%; and \texttt{order=fl} (the order this bars are placed in the display
%window, here `first-to-last).

\paragraph*{Description of \protect\env{bargraphenv} arguments.} \env{bargraphenv} is
a \env{minipage} environment, in which the bars are drawn and
displayed. The following are the key-values recognized within the optional
argument of \env{bargraphenv}.
\begin{aebDescript}
  \item[\texttt{width=\ameta{length}}] The width of the environment.
  \item[\texttt{height=\ameta{length}}] The height of the environment.
  \item[\texttt{o=\ameta{\upshape{horiz|vert}}}] Declares the orientation of
      the bar graph contained within the environment: \texttt{o=horiz}
      specifies the bar graphs are to be horizontal; \texttt{o=vert} sets the
      bar graphs to be vertical. The default is vertical.
  \item[\texttt{origin=\ameta{\upshape{0|.5}}}] The position of
      the horizontal axis (when \texttt{o=vert}) or the vertical
      axis (when \texttt{o=horiz}). A value of \texttt{origin=0}
      is the default.
  \item[\texttt{showaxis\ameta{\upshape{true|false}}}] If
      \texttt{showaxis=true} (or just \texttt{showaxis}, the horizontal rule
      (if \texttt{o=vert}) or a vertical rule (if \texttt{o=horiz}) is drawn.
      The default is no axis is drawn.

\end{aebDescript}

\paragraph*{Description of \protect\env{bargraph} arguments.} The \env{bargraph} environment
is placed within the \env{bargraphenv} environment; it has a name
(\ameta{bg-name}) and must contain, within its body, only the \cs{barfor} and
\cs{cmd} commands. The optional arguments (\ameta{bg-options}) are used to
describe the bars that is contained in the environment.

\begin{aebDescript}
   \item[\texttt{nbars=\ameta{num}}] (Required) The number of bars to appear in this
       \env{bargraph} environment. The value \ameta{num} must be a positive
       integer. There is no enforcement of this restriction; just disaster.
       The value specified by
       \texttt{nbars}\marginpar{\small\raggedleft\slshape\cs{nbars}} is saved in the
       command \cs{nbars}; \cs{nbars} is globally defined and is overwritten
       then when the next \env{bargraph} environment is expanded. The default
       for \texttt{nbars} is zero (0); consequently, this is a required key.
   \item[\texttt{gap=\ameta{num}}] (Optional) The amount space between bars; \ameta{num}
       is a number of reasonable size; the number is dimensionless, but is
       interpreted as a big point. No enforcement, its your document. The gap
       can be access through the command \cs{bargap}\marginpar{\small\raggedleft\slshape\cs{bargap}};
       if \texttt{gap=3}, then \cs{bargap} expands to \texttt{3bp}. The definition of \cs{bargap}
       is global, it is overwritten when the next \env{bargraph} environment is expanded. The default
       is \texttt{gap=0}.
   \item[\texttt{bardimen=\ameta{num}}] (Optional) A dimensionless positive number
       interpreted as the width (in the case of \texttt{o=vert}) and height
       (when \texttt{o=horiz}) in big points. The default is 20. The value of the
       \texttt{bardimen} key is saved in the \cs{bardimen}\marginpar{\small\raggedleft\slshape\cs{bardimen}}
       command as \texttt{\ameta{num}bp}; this is  global definition and is overwritten when
       the next \env{bargraph} environment is expanded.
%   \item[\texttt{order=\ameta{\upshape{fl|lf}}}] The order the bars are
%       placed in the \env{bargraphenv} environment; \texttt{o=fl} is the
%       default, they are built in the same order they are listed; for
%       \texttt{o=lf}, the bars are built in reverse order to what they are
%       listed. This option may not really be needed, if you want them displayed in
%       reverse order, just list them in the order you wish and recompile.
\end{aebDescript}
In the example given earlier, the \env{bargraph} environment was,
\begin{Verbatim}[xleftmargin=\amtIndent,fontsize=\small]
\begin{bargraph}[nbars=4,gap=3]{vBar}
\barfor{auto}\barfor{truck}\barfor{suv}\barfor{van}
\end{bargraph}
\end{Verbatim}
We declare this environment to have a name of \texttt{vBar}
(\ameta{bg-name}). It contains four bars, each with a gap of 3 (3bp). The
environment contains four \cs{barfor} commands named \texttt{auto},
\texttt{truck}, \texttt{suv}, and \texttt{van} (these are the
\ameta{bar-name}s).


\paragraph*{Description of \texorpdfstring{\protect\cs{presetsbarfor}}{\textbackslash{presetsbarfor}}.}
The \cs{barfor} command creates a (rectangular) push button used to represent
a single `bar' in a bar graph. The \cs{presetsbarfor} commands are the chosen
way to pass (form field) options to the push buttons created by \cs{barfor}.
\bVerb\takeMeasure{\string\presetsbarfor\darg{\ameta{bg-name}}\darg{\ameta{bar-name}}\darg{\ameta{KV-pairs}}}%
\def\1{\hskip\amtIndent\relax}%
\begin{dCmd}[commandchars=!()]{\bxSize}
\presetsbarfor{!ameta(bg-name)}{!ameta(bar-name)}{!ameta(KV-pairs)}
\end{dCmd}
\eVerb To uniquely identify which form field these key-value pairs
(\ameta{KV-pairs}) apply, we supply the name of the \env{bargraph}
environment (\ameta{bg-name}) the bar is located in, and the name of the
particular bar (\ameta{bar-name}). The \ameta{KV-pairs} argument consists of
any key-value pairs supported by the \pkg{eforms} package for form fields.
\bVerb\def\1{\hskip\amtIndent\relax}
\begin{Verbatim}[xleftmargin=\amtIndent,fontsize=\small,commandchars={!()}]
\presetsbarfor{vBar}{auto}{\BG{red}}
\presetsbarfor{vBar}{truck}{\BG{green}}
\presetsbarfor{vBar}{suv}{\BG{yellow}}
\presetsbarfor{vBar}{van}{\BG{magenta}}
\begin{bargraph}[nbars=4,gap=3]{vBar}
!1\barfor{auto}\barfor{truck}\barfor{suv}\barfor{van}
\end{bargraph}
\end{bargraphenv}}
\end{Verbatim}
\eVerb
These commands reference the bars within the \texttt{vBar} \env{bargraph}
environment. For example, for the bar named \texttt{auto}, we assign a background
(or fill) color of red, and so on.

\paragraph*{The \texorpdfstring{\protect\cs{barfor} and \protect\cs{cmd}}{\textbackslash{barfor} and \textbackslash{cmd}} commands.}
The body of \env{bargraph} consists solely of \cs{barfor} and \cs{cmd}
commands (\cs{cmd} is a local command, defined for the \env{bargraph}
environment). The syntax for these are \cs{barfor\darg{\ameta{bar-name}}} and
\cs{cmd\darg{\ameta{markup}}}, where \ameta{markup} refers to any {\LaTeX}
markup. The author created \cs{cmd} to add typeset labeling to the bar graph,
but there are other applications. Illustrations of \cs{cmd} are found in
\hyperref[multiBars]{Section~\ref{multiBars}} on drawing multiple bar graphs
and \hyperref[labelBars]{Section~\ref*{labelBars}} on labeling.


\section{Inputting data for a bar graph}

Now, given you have set up a \env{bargraphenv} environment with its various
innards, how do you input data so the bar graph responds? There are several schemes.

\subsection{Individual input via \texorpdfstring{\protect\cs{inputFor}}{\textbackslash{inputFor}}}

The method illustrated in the example on page~\pageref{intro} is for the user to input individual data
using the package provided command \cs{inputFor}.
\bVerb\takeMeasure{\small\string
\inputFor[\ameta{KV-pairs}]\darg{\ameta{bgenv-name}}\darg{\ameta{bg-name}}\darg{\ameta{bar-name}}\darg{\ameta{width}}\darg{\ameta{height}}}%
\def\1{\hskip\amtIndent\relax}%
\begin{dCmd}[commandchars=!(),fontsize=\small]{\bxSize}
\renewcommand\presetinputfor[2]{!ameta(eforms-actions)}
\inputFor[!ameta(KV-pairs)]{!ameta(bgenv-name)}{!ameta(bg-name)}{!ameta(bar-name)}{!ameta(width)}{!ameta(height)}
\end{dCmd}
\eVerb The command \cs{inputFor} doesn't have to reside within a
\env{bargraphenv} environment, so a full path is needed to characterize which
of the bars this text field references. Use \ameta{KV-pairs} to change the
appearance of the field. One such example from page~\pageref{intro} is,
\begin{Verbatim}[xleftmargin=\amtIndent,fontsize=\small,commandchars={!()}]
\inputFor{vehiclesV}{vBar}{auto}{.5in}{13bp}
\end{Verbatim}
You need one \cs{inputFor} command for each of the bars in the targeted \env{bargraphenv} environment.
%This is the simplest scenario.

Use \cs{presetinputfor} to set any actions you require of your \cs{inputFor}
commands; any re-definition of \cs{presetinputfor} should occur prior to the
targeted \cs{inputFor} commands. The package initial definition is,
\begin{Verbatim}[xleftmargin=\amtIndent,fontsize=\small,commandchars={!~@}]
\newcommand{\presetinputfor}[2]{%
  \AAkeystroke{AFNumber_Keystroke(0, 0, 0, 0, "", true);}
  \AAformat{AFNumber_Format(0, 0, 0, 0, "", true);}
  \AAvalidate{AFRange_Validate(true, 0, false, 0);}%!textsf~ remove to allow all numbers@
}
\end{Verbatim}
This requires the entries in the \cs{inputFor} commands to be nonnegative
numbers. The two arguments of \cs{presetinputfor} are normally not used, but
for the record \texttt{\#1} is \ameta{bgenv-name} and \texttt{\#2} is
\texttt{\ameta{bg-name}.\ameta{bar-name}}.

\subsection{Providing comma-delimited data}

Another way to enter data is through a comma-delimited list of numbers.

\medskip\noindent\bgroup\setlength{\fboxrule}{1bp}\fbox
{\begin{bargraphenv}[width=(14bp*20),height=2in,o=vert]{statdemo}%(\linewidth-2\fboxsep-2\fboxrule)
\begin{bargraph}[nbars=20,gap=0,bardimen=14]{histogram}
\barfor{bar1}\barfor{bar2}\barfor{bar3}\barfor{bar4}\barfor{bar5}
\barfor{bar6}\barfor{bar7}\barfor{bar8}\barfor{bar9}\barfor{bar10}
\barfor{bar11}\barfor{bar12}\barfor{bar13}\barfor{bar14}\barfor{bar15}
\barfor{bar16}\barfor{bar17}\barfor{bar18}\barfor{bar19}\barfor{bar20}
\end{bargraph}%
\end{bargraphenv}}\egroup\vcgBdry[6pt]

\noindent\hglue1bp\marginpar{\small\itshape
\raisebox{2in+6bp}{\parbox[t]{\marginparwidth}{\raggedleft
  This example appears in \texttt{examples/\allowbreak bgjs-comma1.tex}}}}
\textField[\TU{Enter up to twenty nonnegative numbers separated by commas}
  \AAvalidate{resetBargraphs("statdemo");\r
  \populateCommaData("statdemo","histogram",event.value,validateArrayNonNegNums)}
]{commaed}{(\bardimen*\nbars+2\fboxsep+2bp)}{13bp}\vcgBdry[6pt]

\hglue1bp\pushButton[\CA{Symmetrical}\AAmouseup{%
var str="1,2,3,4,5,6,7,8,9,10,10,9,8,7,6,5,4,3,2,1";\r
\populateCommaData("statdemo","histogram",str);
}]{symmetrical}{}{13bp}\cgBdry[.5em]
\pushButton[\CA{Skew left}\AAmouseup{%
var str="1,2,2,3,3,4,5,6,8,10,12,14,16,19,20,19,17,15,13,11";\r
\populateCommaData("statdemo","histogram",str);}]{skewleft}{}{13bp}\cgBdry[.5em]
\pushButton[\CA{Skew right}\AAmouseup{%
var str="17,18,19,20,19,18,16,14,12,10,8,7,7,6,6,4,4,3,2,1";\r
\populateCommaData("statdemo","histogram",str);}]{skewright}{}{13bp}\vcgBdry[6pt]

\hglue1bp\pushButton[\TU{This button re-scales the bar graph so that the longest bar
takes the entire width of the region.}\CA{Optimize}\A{\JS{optimizeScaling("statdemo");}}]{optimize1}{}{13bp}\cgBdry[.5em]
\displaysfFor{statdemo}{.5in}{13bp}\olBdry
  \manualsfFor[\CA{Rescale}\TU{Enter a new scale factor in the text field, then press this button}
  ]{statdemo}{}{13bp} \texttt{\%}\textsf{ Can perform simple arithmetic operations in text field}
  \vcgBdry[6bp]

\hglue1bp\pushButton[\CA{Reset}\AAmouseup{resetBargraphs("statdemo","commaed","rescale.statdemo")}]{reset}{}{13bp}\vcgBdry[6pt]

Enter comma-delimited nonnegative numbers into the text field above, or press
any of the three buttons \uif{Symmetrical}, \uif{Skew left}, or
\uif{Skew right} to populate the bar graph with comma-delimited data. Press
the \uif{Optimize} button to automatically re-scale the graph; manually
re-scale the graph by entering a scale factor in the text field, then press
the \uif{Rescale} button. Clear all fields by pressing the \uif{Reset}
button. These buttons are discussed in detail in a later section.

\section{Multiple bar graphs within an \texorpdfstring{\protect\env{bargraphenv}}{bargraphenv} environment}\label{multiBars}

More than one \env{bargraph} environment can appear within a \env{bargraphenv}, as illustrated in the next two examples.\medskip

\noindent
\fbox{\begin{bargraphenv}[width=23bp*10,height=1.4in,o=vert]{math1}
\presetsbarfor{class1}{A}{\BG{red}\TU{Class1: @v@ students received an 'A'}}
\presetsbarfor{class1}{B}{\BG{red}\TU{Class1: @v@ students received an 'B'}}
\presetsbarfor{class1}{C}{\BG{red}\TU{Class1: @v@ students received an 'C'}}
\presetsbarfor{class1}{D}{\BG{red}\TU{Class1: @v@ students received an 'D'}}
\presetsbarfor{class1}{F}{\BG{red}\TU{Class1: @v@ students received an 'F'}}
\presetsbarfor{class2}{A}{\BG{blue}\TU{Class2: @v@ students received an 'A'}}
\presetsbarfor{class2}{B}{\BG{blue}\TU{Class2: @v@ students received an 'B'}}
\presetsbarfor{class2}{C}{\BG{blue}\TU{Class2: @v@ students received an 'C'}}
\presetsbarfor{class2}{D}{\BG{blue}\TU{Class2: @v@ students received an 'D'}}
\presetsbarfor{class2}{F}{\BG{blue}\TU{Class2: @v@ students received an 'F'}}
\begin{bargraph}[nbars=5,gap=3]{class1}
\barfor{A}\barfor{B}\barfor{C}\barfor{D}\barfor{F}
\end{bargraph}
\begin{bargraph}[nbars=5,gap=3]{class2}\cmd{\hs{\bargap}}
\barfor{A}\barfor{B}\barfor{C}\barfor{D}\barfor{F}
\end{bargraph}
\end{bargraphenv}}\vcgBdry[6pt]

\noindent\marginpar{\small\itshape
\raisebox{1.4in+6pt}{\parbox[t]{\marginparwidth}{\raggedleft
  This example appears in \texttt{examples/\allowbreak bgjs-comma2.tex}}}}%
\begin{minipage}[c]{222pt}
class1:
  \textField[\TU{Enter five natural numbers separated by commas}
  \AAvalidate{\populateCommaData("math1","class1",event.value,validateArrayNonNegNums)}]{txtclass3}{2in}{13bp}\cgBdry[.5em]
  \pushButton[\CA{Class 1}\AAmouseup{%
    var str="12,15,23,10,15";\r
    \populateCommaData("math1","class1",str);
}]{math1class1}{}{13bp}\vcgBdry[4bp]
class2: \textField[\TU{Enter five natural numbers separated by commas}
  \AAvalidate{\populateCommaData("math1","class2",event.value,validateArrayNonNegNums)}]{txtclass4}{2in}{13bp}\cgBdry[.5em]
  \pushButton[\CA{Class 2}\AAmouseup{%
    var str="10,17,29,10,20";\r
    \populateCommaData("math1","class2",str);
}]{math1class1}{}{13bp}
\end{minipage}\cgBdry[.5em]
\pushButton[\TU{This button re-scales the bar graph so that the tallest bar
takes the entire height of the region.}\CA{Optimize}\A{\JS{optimizeScaling("math1");}}]{optimize3}{}{13bp}\cgBdry[.5em]
\pushButton[\CA{Reset}\A{\JS{resetBargraphs("math1","txtclass3","txtclass4");}}]{reset}{}{13bp}\vcgBdry[6pt]
A partial verbatim listing is presented below:
\begin{Verbatim}[xleftmargin=\amtIndent,fontsize=\small,commandchars={!~@}]
\fbox{\begin{bargraphenv}[width=23bp*10,height=1.4in,o=vert]{math1}
...
\begin{bargraph}[nbars=5,gap=3]{class1}
\barfor{A}\barfor{B}\barfor{C}\barfor{D}\barfor{F}
\end{bargraph}
\begin{bargraph}[nbars=5,gap=3]{class2}!textbf~\cmd{\hs{\bargap}}@
\barfor{A}\barfor{B}\barfor{C}\barfor{D}\barfor{F}
\end{bargraph}
\end{bargraphenv}}
...
class1: \textField[\TU{Enter five natural numbers separated by commas}
  \AAvalidate{%
  var str=event.value;\r
  \populateCommaData("math1","class1",str,validateArrayNonNegNums)}
  ]{txtclass3}{2in}{13bp}\cgBdry[.5em]
  \pushButton[\CA{Class 1}\AAmouseup{%
    var str="12,15,23,10,15";\r
    \populateCommaData("math1","class1",str);
  }]{math1class1}{}{13bp}
\end{Verbatim}
To populate a bar graph with comma-delimited data, use the \cs{populateCommData} command, a {\LaTeX} helper
command for an underlying JavaScript function. Its arguments are
\begin{Verbatim}[xleftmargin=\amtIndent,fontsize=\small,commandchars={!~@}]
\populateCommaData("!ameta~bgenv-name@","!ameta~bg-name@",!ameta~str@[,!ameta~validate@])
\end{Verbatim}
Where, \ameta{str} is a comma-delimited string of numbers; the optional \ameta{validate} argument
is a name of a validation function. In the example above, the package defined
\texttt{validateArrayNonNegNums} is used.

One\marginpar{\small\raggedleft\textbf{Important}} more comment on the
listing above: Observe the markup in bold font; the gap is not applied to the
last bar for a \env{bargraph} environment, so to get same gap as we move from
the \texttt{class1} bar graph to the \texttt{class2} bar graph, we insert
\verb~\cmd{\hs{\bargap}}~ the desired gap. Read the comments following the next example
for more detailed explanation of the (local) commands \cs{cmd} and \cs{hs}.\medskip

You can adjust the positions of the bar graph to have a more side-by-side comparison.\vcgBdry[4bp]

\fbox{\begin{bargraphenv}[width=33bp*5,height=2in,o=vert]{math2} %+13bp+10bp +13bp
\presetsbarfor{class1}{A}{\BG{red}}
\presetsbarfor{class1}{B}{\BG{red}}
\presetsbarfor{class1}{C}{\BG{red}}
\presetsbarfor{class1}{D}{\BG{red}}
\presetsbarfor{class1}{F}{\BG{red}}
\presetsbarfor{class2}{A}{\BG{blue}}
\presetsbarfor{class2}{B}{\BG{blue}}
\presetsbarfor{class2}{C}{\BG{blue}}
\presetsbarfor{class2}{D}{\BG{blue}}
\presetsbarfor{class2}{F}{\BG{blue}}
\begin{bargraph}[nbars=5,gap=13]{class1}
\barfor{A}\barfor{B}\barfor{C}\barfor{D}\barfor{F}
\end{bargraph}%
\begin{bargraph}[nbars=5,gap=13]{class2}\cmd{\hs{13bp}}
\cmd{\color{red}\hs{-33bp*5+10bp}}\barfor{A}\barfor{B}\barfor{C}\barfor{D}\barfor{F}\cmd{\hs{-13bp}}
\end{bargraph}
\end{bargraphenv}}\vcgBdry[6pt]

%\hs{-33bp*5} \hs{-23bp*5}\kern5bp}

\begin{defineJS}{\pbaction}
var f=this.getField("txtclass5");
for (var value="",i=0; i<5; i++) value+=(""+(Math.round(Math.random()*200))+",");
f.value=value.substring(0,value.length-1);
for (var value="",i=0; i<5; i++) value+=(""+(Math.round(Math.random()*200))+",");
var g=this.getField("txtclass6");
g.value=value.substring(0,value.length-1);
\end{defineJS}

\noindent\marginpar{\small\itshape
\raisebox{2in+6pt}{\parbox[t]{\marginparwidth}{\raggedleft
  This example appears in \texttt{examples/\allowbreak bgjs-comma2.tex}}}}%
\begin{minipage}[c]{222pt}
class1: \textField[\TU{Enter five natural numbers separated by commas}
  \AAvalidate{\populateCommaData("math2","class1",event.value,validateArrayNonNegNums)}]{txtclass5}{2in}{13bp}\cgBdry[.5em]
  \pushButton[\CA{Class 1}\AAmouseup{%
    var str="12,15,23,10,15";\r
    \populateCommaData("math2","class1",str);
}]{math1class1}{}{13bp}\vcgBdry[4bp]
class2: \textField[\TU{Enter five natural numbers separated by commas}
  \AAvalidate{\populateCommaData("math2","class2",event.value,validateArrayNonNegNums)}]{txtclass6}{2in}{13bp}\cgBdry[.5em]
  \pushButton[\CA{Class 2}\AAmouseup{%
    var str="10,17,29,10,20";\r
    \populateCommaData("math2","class2",str);
}]{math1class1}{}{13bp}
\end{minipage}\cgBdry[.5em]
\pushButton[\CA{Test}\TU{Press to automatically enter data into the two fields above}\A{\JS{\pbaction}}]{tstdata}{}{13bp}\cgBdry[.5em]
\pushButton[\TU{This button re-scales the bar graph so that the tallest bar
takes the entire height of the region.}\CA{Optimize}\A{\JS{optimizeScaling("math2");}}]{optimize3}{}{13bp}\cgBdry[.5em]
\pushButton[\CA{Reset}\A{\JS{resetBargraphs("math2","txtclass5","txtclass6");}}]{reset}{}{13bp}\vcgBdry[6pt]

A partial listing is given next:
\begin{Verbatim}[xleftmargin=\amtIndent,fontsize=\small,commandchars={!~@}]
\begin{bargraph}[nbars=5,gap=13]{class1}
\barfor{A}\barfor{B}\barfor{C}\barfor{D}\barfor{F}
\end{bargraph}
\begin{bargraph}[nbars=5,gap=13]{class2}
!textbf~\cmd{\hs{-33bp*5+10bp}}@
\barfor{A}\barfor{B}\barfor{C}\barfor{D}\barfor{F}
\end{bargraph}
\end{bargraphenv}
\end{Verbatim}
This listing introduces two new commands, shown in bold font:
\cs{cmd}\marginpar{\small\raggedleft\slshape\cs{cmd}} is a generic command for placing
{\LaTeX} content into the bar graph; \cs{hs} (and \cs{vs}) are short-hand
commands for entering a horizontal skip
(\cs{hs}\marginpar{\small\raggedleft\slshape\cs{hs}}) or a vertical skip
(\cs{vs}\marginpar{\small\raggedleft\slshape\cs{vs}}) into the page. These two should be
used only within the \cs{cmd} command argument. In the above listing, we shift
the contents of the second \env{bargraph} environment to the left to appear as shifted
over 10bp from the first \env{bargraph}.\medskip

The markup shown in bold font horizontally shifts the second bar graph
\texttt{33bp*5} to the left ($\texttt{33bp}=\cs{bardimen}+\cs{bargap}$) and
$5 = \cs{nbars} $. Since $\cs{bardimen}=\texttt{20bp}$ (the default value),
we shift \texttt{10bp} ($\cs{bardimen}/2$) to the right to get the two first
bars offset by half their width.

\section{Labeling the bars}\label{labelBars}

Now we come to the difficult topic of labeling (or captioning) the bars.
Because of the dynamic nature of these bar graphs, there can be no scaling on
the axis; however, if you rollover any of the bars, you will see labeling
that provides some information. Throughout this document, so far, the
``default'' labeling is applied, as exhibited by the red bar graph below;
there is a richer scheme that is possible when the bars have a \cs{TU} (tool
tip) markup. When you provide an appropriately coded tool tip you can build a
much more meaningful labeling of the bars, as illustrated by the blue bar
graph below.

\noindent
\fbox{\begin{bargraphenv}[width=23bp*10,height=1.4in,o=vert]{math3}
\presetsbarfor{class1}{A}{\BG{red}}
\presetsbarfor{class1}{B}{\BG{red}}
\presetsbarfor{class1}{C}{\BG{red}}
\presetsbarfor{class1}{D}{\BG{red}}
\presetsbarfor{class1}{F}{\BG{red}}
\presetsbarfor{class2}{A}{\BG{blue}\TU{Class2: @v@ students received an 'A'}}
\presetsbarfor{class2}{B}{\BG{blue}\TU{Class2: @v@ students received an 'B'}}
\presetsbarfor{class2}{C}{\BG{blue}\TU{Class2: @v@ students received an 'C'}}
\presetsbarfor{class2}{D}{\BG{blue}\TU{Class2: @v@ students received an 'D'}}
\presetsbarfor{class2}{F}{\BG{blue}\TU{Class2: @v@ students received an 'F'}}
\begin{bargraph}[nbars=5,gap=3]{class1}
\barfor{A}\barfor{B}\barfor{C}\barfor{D}\barfor{F}
\end{bargraph}
\begin{bargraph}[nbars=5,gap=3]{class2}\cmd{\hs{\bargap}}
\barfor{A}\barfor{B}\barfor{C}\barfor{D}\barfor{F}
\end{bargraph}
\end{bargraphenv}}\vcgBdry[6pt]

\begin{minipage}[c]{222pt}
class1: \textField[\TU{Enter five natural numbers separated by commas}
  \AAvalidate{\populateCommaData("math1","class1",event.value,validateArrayNonNegNums)}]{txtclass7}{2in}{13bp}\cgBdry[.5em]
  \pushButton[\CA{Class 1}\AAmouseup{%
    var str="12,15,23,10,15";\r
    \populateCommaData("math3","class1",str);
}]{math1class1}{}{13bp}\vcgBdry[4bp]
class2: \textField[\TU{Enter five natural numbers separated by commas}
  \AAvalidate{\populateCommaData("math1","class2",event.value,validateArrayNonNegNums)}]{txtclass8}{2in}{13bp}\cgBdry[.5em]
  \pushButton[\CA{Class 2}\AAmouseup{%
    var str="10,17,29,10,20";\r
    \populateCommaData("math3","class2",str);
}]{math1class1}{}{13bp}
\end{minipage}\cgBdry[.5em]
\pushButton[\TU{This button re-scales the bar graph so that the tallest bar
takes the entire height of the region.}\CA{Optimize}\A{\JS{optimizeScaling("math3");}}]{optimize3}{}{13bp}\cgBdry[.5em]
\pushButton[\CA{Reset}\A{\JS{resetBargraphs("math3","txtclass7","txtclass8");}}]{reset}{}{13bp}\vcgBdry[6pt]
A partial verbatim listing is presented below:
\begin{Verbatim}[xleftmargin=\amtIndent,fontsize=\small,commandchars={!()}]
\fbox{\begin{bargraphenv}[width=23bp*10,height=1.4in,%
  o=vert]{math3}
\presetsbarfor{class1}{A}{\BG{red}}          %!textsf( The !cs(TU) key is not provided)
...
\presetsbarfor{class2}{A}{\BG{blue}
  !textbf(\TU{Class2: @v@ students received an 'A'})} %!textsf( The !cs(TU) key is provided)
...
\begin{bargraph}[nbars=5,gap=3]{class1}
\barfor{A}\barfor{B}\barfor{C}\barfor{D}\barfor{F}
\end{bargraph}
\begin{bargraph}[nbars=5,gap=3]{class2}\cmd{\hs{\bargap}}
\barfor{A}\barfor{B}\barfor{C}\barfor{D}\barfor{F}
\end{bargraph}
\end{bargraphenv}}
\end{Verbatim}
Shown in bold font above is the \cs{TU} key for the first bar of
\texttt{class2}. The value of the \cs{TU} key contains the string
`\texttt{@v@}'; this three-character string is replaced by the actual value
of the bar. Concepts and methods are outlined in the next section.

\subsection{Methods for labeling bars}

Fundamentally, there are two methods for labeling bars: (1) using the tooltip
feature of {\PDF} forms; and (2) non-tooltip methods. Within each of these
two methods, several techniques have been developed, which are explained in
the next two sections.

\subsubsection{Tooltip techniques}

There are two commands, \cs{barLabelsTUs} and \cs{barLabelsNoTU}, to declare
how the {\PDF} tooltips are formed. They handle the cases of \cs{TU} key is
provided and the \cs{TU} key is not provided. We look at each of these in
turn.

\paragraph*{The \cs{TU} key is provided.} The \cs{barLabelsTU} takes either a
JavaScript function name or a JavaScript string argument.
\bVerb\takeMeasure{\string\barLabelsTU\darg{\ameta{JS-function}\string|\ameta{JS-string}}}%
\begin{dCmd}[commandchars=!()]{\bxSize}
\barLabelsTU{!ameta(JS-function)|!ameta(JS-string)}
\end{dCmd}
\eVerb \cs{barLabelsTU} \emph{applies to all bars with a \cs{TU} key}. The
\cs{barLabelsTU} \emph{can be used within the body of the document} to change the
method of assigning labels to bars. The tooltip can comprise of the variables
\texttt{@env@}, \texttt{@barname@}, \texttt{@bar@}, and \texttt{@v@} to
compose the string; these variables are replaces with their respective values
\ameta{bgenv-name}, \ameta{bg-name}, \ameta{bar-name} and \ameta{value} when
the bar is populated with \ameta{value}.

\paragraph*{Description of argument types}
\subparagraph*{\ameta{JS-function}} Such a JavaScript function takes two
arguments \texttt{fld} and \texttt{v}; the latter is the current value of the
bar, the former is the field name of the bar. Such a function should return a
string that will be displayed as the tooltip of the bar. The default is
\texttt{customBarLabels},
\begin{Verbatim}[xleftmargin=\amtIndent,fontsize=\small,commandchars=!()]
\barLabelsTU{customBarLabels} % !textsf(applies to all bars with a !cs(TU) key)
\end{Verbatim}
which is part of the \pkg{bargraph-js} package. Note that this declaration
is equivalent to \verb|\barLabelsTU{}| and \verb|\barLabelsTU{""}|.

\subparagraph*{\ameta{JS-string}} When a string is provided, \emph{the
\cs{TU} key is ignored}. Instead a formatted string is used.
Use the variables \texttt{@env@}, \texttt{@barname@}, \texttt{@bar@}, and
\texttt{@v@} to compose the string, as seen below.
\begin{Verbatim}[xleftmargin=\amtIndent,fontsize=\small]
 \barLabelsTU{"Within the \\"@env@\\" environment, within the
   \\"@barname@\\" environment, the bar \\"@bar@\\"
   has a value of @v@"}
\end{Verbatim}
When the empty string (\verb|\barLabelsTU{""}|) or the empty argument
(\verb|\barLabelsTU{}|) is passed, the result is to default to
\texttt{customBarLabels}.

\paragraph*{No \cs{TU} key is provided.}
For the case that a bar does not have a \cs{TU} key (no tooltip specified),
the argument of \cs{barLabelsNoTU}, which is JavaScript function or a
JavaScript string, provides tooltips for the bar.
\bVerb\takeMeasure{\string\barLabelsNoTU\darg{\ameta{JS-function}\string|\ameta{JS-string}}}%
\begin{dCmd}[commandchars=!()]{\bxSize}
\barLabelsNoTU{!ameta(JS-function)|!ameta(JS-string)}
\end{dCmd}
\eVerb \cs{barLabelsNoTU} is specified in the preamble\marginpar{\small\itshape
    \raisebox{2\baselineskip}{\parbox{\marginparwidth}{\raggedleft Declare
    \cs{barLabelsNoTU} in preamble}}} and cannot be changed in the body of the
    document. This labeling system applies to all bars with no \cs{TU} key.
\paragraph*{Description of argument types}
\subparagraph*{\ameta{JS-function}} When the argument is a JavaScript
     function, the referenced function must be written. In the sample file
     \texttt{bgjs-basic1.tex}, the JavaScript function \texttt{customLabelsForBars(fld,v)} is defined;
     use this function as a template for writing your own custom handlers.
\begin{Verbatim}[xleftmargin=\amtIndent,fontsize=\small,commandchars=!()]
\barLabelsNoTU{customLabelsForBars} %!textsf( applies to all bars without a !cs(TU) key)
\end{Verbatim}

\subparagraph*{\ameta{JS-string}} When the argument is a JavaScript string,
    use the variables \texttt{o.env}, \texttt{o.barname}, \texttt{o.bar}, and
    \texttt{o.value} to compose the string. The default is,
\begin{Verbatim}[xleftmargin=\amtIndent,fontsize=\small]
\barLabelsNoTU{o.barname+": "+o.bar+", Value: "+o.value}
\end{Verbatim}
This is equivalent to \verb|\barLabelsNoTU{}|. Refer to the demo file
\texttt{bgjs-basic1.tex} for more information.

\subsubsection{Non-tooltip techniques}

In addition to using the {\PDF} form feature of tool tips, the bars can be visually labeled by
\begin{itemize}
  \item using typeset text between the bars, refer to \texttt{examples/bgjs-basic2.tex};
  \item using form fields between the bars, refer to \texttt{examples/bgjs-basic3.tex}, also illustrated below;
  \item using layers (OCG) between the bars, refer to \texttt{examples/bgjs-pro1.tex}, this example
    requires a \app{dvips}/\allowbreak\app{Distiller} workflow.
\end{itemize}\par\vcgBdry

\textbf{Class instructions.} Go to a road of your choice and count the number
of vehicles of each type during rush hour (for one hour), fill out the form
below and turn it in for credit.\footnote{It is strongly recommended you read the documentation
\texttt{bgjs-examples.pdf} on \texttt{bgjs-basic3.tex} as this example uses the
command \cs{labelFld} not formally documented in this manual.}\marginpar{\small\itshape
\raisebox{-6bp-\baselineskip}{\parbox[t]{\marginparwidth}{\raggedleft
  This example appears in \texttt{examples/\allowbreak bgjs-basic3.tex}}}}\vcgBdry[6bp]

\fbox{\begin{bargraphenv}[width=.67\linewidth,height=2in,
  o=horiz]{vehiclesH}\def\WD{2in}
\presetsbarfor{hBar}{auto}{\BG{red}}
\presetsbarfor{hBar}{truck}{\BG{green}}
\presetsbarfor{hBar}{suv}{\BG{yellow}}
\presetsbarfor{hBar}{van}{\BG{magenta}}
\begin{bargraph}[nbars=4,gap=3]{hBar}
\barfor{auto}
\cmd{\vs{-3bp}\labelFld[\textSize{10}]
  {Automobiles (two or four door)}{hBar.auto}{\WD}{13bp}\vs{3bp}}
\barfor{truck}
\cmd{\vs{-3bp}\labelFld[\textSize{10}]
  {Pickup trucks}{hBar.truck}{\WD}{13bp}\vs{3bp}}
\barfor{suv}
\cmd{\vs{-3bp}\labelFld[\textSize{10}]
  {Sport utility vehicle (SUV)}{hBar.suv}{\WD}{13bp}\vs{3bp}}
\barfor{van}
\cmd{\vs{0bp}\labelFld[\textSize{10}]
  {Passenger van}{hBar.van}{\WD}{13bp}}
\end{bargraph}
\end{bargraphenv}}\hfill
\begin{minipage}[b][2in][c]
  {.33\linewidth-2\fboxsep-2\fboxrule-10pt}\kern0pt\parskip3pt
\def\presetinputfor#1#2{%\Ff{\FfReadOnly}
  \AAkeystroke{EFNumber_Keystroke(0, 0, 0, 0, "", true);}
  \AAformat{try{EFNumber_Format(0, 0, 0, 0, "", true);\r
  if(event.rc)toggleFldVisibility("#2@vehiclesH",%
(event.value!=0));}catch(e){}}
  \AAvalidate{EFRange_Validate(true, 0, false, 0);}
}
\makebox[\widthof{Truck:}][l]{Auto:}
  \inputFor{vehiclesH}{hBar}{auto}{.5in}{11bp}\vcgBdry[3bp]
\makebox[\widthof{Truck:}][l]{Truck:}
  \inputFor{vehiclesH}{hBar}{truck}{.5in}{11bp}\vcgBdry[3bp]
\makebox[\widthof{Truck:}][l]{SUV:}
  \inputFor{vehiclesH}{hBar}{suv}{.5in}{11bp}\vcgBdry[3bp]
\makebox[\widthof{Truck:}][l]{Van:}
  \inputFor{vehiclesH}{hBar}{van}{.5in}{11bp}\vcgBdry[3bp]
\pushButton[\TU{This button re-scales the bar graph so that the
longest bar takes the entire width of the region.}\CA{Optimize}
\A{\JS{optimizeScaling("vehiclesH");}}]{optimize1}{}{13bp}
\end{minipage}\vcgBdry[6bp]

Reset horizontal bar graph: \pushButton[\CA{Reset}
  \A{\JS{resetBargraphs("hBar","vehiclesH");\r
}}]{reset}{}{13bp}\vcgBdry[6bp]

One advantage of using forms over the other two methods is that the labeling can be dynamically
revised using JavaScript; for example, by inserting the value of a bar (perhaps as a percentage).

\part{Bar graph with an unspecified number of bars}

This package also supports the \env{bargraph} environment that contain no
\cs{barfor} or \cs{cmd} tokens. These kinds of bar graphs are primarily
designed to support the creation of probability mass functions (pmf) and
cumulative distribution functions (cdf).

The creation\marginpar{\small\itshape\raggedleft\opt{dynamic} option} of bar graph with an unspecified number of bars require the
\opt{dynamic} option:
\begin{Verbatim}[xleftmargin=\amtIndent]
\usepackage[dynamic]{bargraph-js}
\end{Verbatim}
This brings in a JavaScript function \texttt{displayDyBargraph()}.

When used to describe the bar graphs of a discrete probability distribution,
it is assumed the weights are evenly distributed and contiguous on the number
line.\footnote{This restriction of contiguous can be bypassed with a little
trickery.}


\section{Dynamic bar graphs}

As in \textbf{Part I}, the package defines two environments,
\env{bargraphenv} and \env{bargraph}; the latter environment is placed within
the former one.
%\bVerb\takeMeasure{\string\begin\darg{bargraphenv}[\ameta{bgenv-options}]\darg{\ameta{bgenv-name}}}%
\bVerb\def\1{\hskip\amtIndent\relax}\takeMeasure{\1\string\begin\darg{bargraph}\darg{\ameta{bg-name}}%
\textbf{\string\isdynamic}\string\end\darg{bargraph}}%
\begin{dCmd}[commandchars=!()]{\bxSize}
\begin{bargraphenv}[!ameta(bgenv-options)]{!ameta(bgenv-name)}
!1\begin{bargraph}{!ameta(bg-name)}!textbf(\isdynamic)\end{bargraph}
\end{bargraphenv}
\end{dCmd}
\eVerb The first thing to note is that the \env{bargraph} environment has no
options; the options \opt{nbars}, \opt{gap}, and \opt{bardimen} are
automatically assigned. Multiple \env{bargraph} environments are not
supported within a \env{bargraphenv} environment. As a signal to the package
that the current \env{bargraph} environment is one that is dynamic, place the
\cs{isdynamic} command within the \env{bargraph} content, and nothing else.

Multiple \env{bargraph} environments are not supported within the same
\env{bargrapenv}, but you can have two \env{bargraphenv} environments, one to hold
the \uif{pmf} and another to hold the \uif{cdf} bar graphs.

The function that does all the work is \texttt{displayDyBargraph}, it has
four required arguments and one optional one.
\bVerb\takeMeasure{displayDyBargraph(\ameta{bgenv-name},%
\ameta{aPr},\ameta{bPmf},\ameta{bOptimize}[,\ameta{object}])}%
\begin{dCmd}[commandchars={!~@}]{\bxSize}
displayDyBargraph(!ameta~bgenv-name@,!ameta~aPr@,!ameta~bPmf@,!ameta~bOptimize@[,!ameta~object@])
\end{dCmd}
\eVerb
The arguments are described next.
\begin{description}
 \item[\ameta{bgenv-name}] is the name of the target environmet, a JavaScript
     string that is enclosed in double (or single) quotes
     (\texttt{"\ameta{bgenv-name}"}).
 \item[\ameta{aPr}] is an array of probability mass information; each entry
     in this array has the form \texttt{[k,pmfPr,cdfPr]},  where \texttt{k}
     is a value of the distribution, \texttt{pmfPr} is the probability mass
     at that value; and \texttt{cdfPr} is the cumulative mass up to and
     including the value \texttt{k}.
 \item[\ameta{bPmf}] is true if we are to draw bar graph for a probability
     mass function (\uif{pmf}), and false if we are to draw bar graph for a cumulative
     distribution function (\uif{cdf}).
 \item[\ameta{bOptimize}] is true if we optimize the graph (applies to \uif{pmf}
     only), otherwise, we use the current scale factor.
 \item[\ameta{object}] is a JavaScript object having the following properties:
 \begin{description}
      \item[\texttt{bc:\ameta{color}}] is the border color for each of the bars,
      \ameta{color} is a JavaScript color: \texttt{["RBG",1,0,1]} or \texttt{color.red}.
      The default is \texttt{color.red}.
      \item[\texttt{fc:\ameta{color}}] is the fill color for each of the bars,
      \ameta{color} is a JavaScript color: \texttt{["RBG",1,0,1]} or \texttt{color.blue}.
      The default is \texttt{color.blue}.
      \item[\texttt{lbl:\ameta{JS-func}}] is a JavaScript function that provides the tool tips
      for each of the bars. When this property is not present, the built-in JavaScript
      function \texttt{\_labelDyBars(pr,v,bPmf)} is used.

    A custom labeling function can be written and specified as the value of
    the \texttt{lbl} property. The arguments for a labeling function are
    \texttt{pr} (the current value of the random variable); \texttt{v} (the
    probability associated with \texttt{pr}, \texttt{v} is a \uif{pmf} or
    \uif{cdf} value depending on the switch \texttt{bPmf}); and \texttt{bPmf}
    is \texttt{true} when this is for a \textsf{pmf} and \texttt{false} when
    this is for a \textsf{cdf}.
 \end{description}
\end{description}

\def\displayTable{\textField[\autoCenter{n}\BC{}\BG{}\textSize{6}
  \Ff{\FfMultiline}]{displayTable}{1in}{2in+6pt+3\baselineskip}}

\begin{center}\bfseries
   The bar graph for the probability mass function (pmf)
\end{center}
\noindent\makebox[0pt][r]{\smash
  {\raisebox{2\fboxsep+2\fboxrule-6pt-3\baselineskip}{\displayTable}%
    \hspace{\marginparsep}}}
\fbox{\begin{bargraphenv}[width=\linewidth-3\fboxsep-3\fboxrule,%
  height=2in,o=vert]{Pmf}
\begin{bargraph}{pmfBar}\isdynamic\end{bargraph}
\end{bargraphenv}}\vcgBdry[\medskipamount]
\pushButton[\TU{This button re-scales the bar graph so that the
  tallest bar takes the entire height of the region; shift-click
  reverts bar graph to its original scaling.}\CA{Optimize}\AAmouseup{%
  try{displayDyBargraph("Pmf",aPmfCdf,true,!event.shift)}
    catch(e){};}
]{optimize}{}{13bp}\cgBdry[1em]
Under normal scaling, the height of this region is 1 unit, when the
bar graph is optimized, the height is the height of the tallest bar.
\begin{flushleft}
\pushButton[\TU{Randomly generate a probability distribution}
  \CA{Random}\AAmouseup{%
  var maxN=40;\r
  var aPmfCdf=new Array();\r
  var n=0;
  while (n==0)\r\t
    var n=Math.round(Math.random()*maxN);\r
  var aValues=[],apmfs=[],acdfs=[];\r
  var total=0;\r
  for (var i=0; i<n; i++) {\r\t
    aValues[i]=i;\r\t
    apmfs[i]=Math.round(Math.random()*maxN);\r\t
    total+=(apmfs[i]);\r
  }\r
  for (var i=0; i<n; i++) {\r\t
    apmfs[i]=apmfs[i]/total;\r\t
    acdfs[i]=apmfs[i]+((i==0)?0:acdfs[i-1]);\r\t
    aPmfCdf[i]=[aValues[i],apmfs[i],acdfs[i]];\r
  }\r
  displayTable("displayTable",aPmfCdf);\r
  displayDyBargraph("Pmf",aPmfCdf,true,true);\r
}]{Distr3}{}{13bp}\cgBdry[.5em]
 \pushButton[\CA{Reset}
    \TU{Press to clear the fields of this page, and shift-press to
      clear all fields.}
    \AAmouseup{%
    this.calculate=true;\r
    if (event.shift)\r\t
      this.resetForm();\r
    else {\r\t
      this.removeField("Pmf@pmfBar");\r\t
      this.removeField("Cdf@cdfBar");\r\t
      this.resetForm("displayTable");\r
    }
}]{reset}{}{13bp}\cgBdry[2em]
\pushButton[\CA{Toggle Border}\AAmouseup{%
  var f=this.getField("Pmf@pmfBar");\r
  if ( f!=null ) {\r\t
    var a=f.getArray();\r\t\t
    if (f.saveStrokeColor==undefined) f.saveStrokeColor=color.red;\r\t
    if (!color.equal(a[0].strokeColor,color.transparent))\r\t\t
      f.saveStrokeColor=a[0].strokeColor;\r\t
    var g=getField("Cdf@cdfBar");\r\t
    if (color.equal(a[0].strokeColor,color.transparent)) {\r\t\t
      f.strokeColor=f.saveStrokeColor;\r\t\t
      g.strokeColor=f.saveStrokeColor;\r\t
    } else {\r\t\t
      f.strokeColor=color.transparent;\r\t\t
      g.strokeColor=color.transparent;\r\t
    }\r
  }
}]{toggleBdry}{}{13bp}
\end{flushleft}
Notice\marginpar{\small\itshape
\parbox[t]{\marginparwidth}{\raggedleft
  This example appears in \texttt{examples/\allowbreak bgjs-dyn1.tex}}} that the widths of the bars
  are automatically adjusted to allow all the bars to be displayed in the bar graph window. The minimum
  width is \opt{bardimen}, which is set to 20(bp). In this example, maximum weight is randomly chosen less
  than or equal to 40. In this demo we do not display the \uif{cdf}. Populating a table, as seen in the margin,
  is optional. Each row is one entry in the \ameta{aPr} array. In the sample file \texttt{bgjs-dyn1.tex}, several
  techniques are used to create the \ameta{aPr} array.

\part{Postscript}

We finish this documentation with additional miscellaneous commands, comments
on demonstration files, and notes on my retirement.

\section{Miscellaneous commands}

\subsection{The bar graph data structure}

The \pkg{bargraph-js} package associates with each \env{bargraphenv}
environment a hidden text field named \texttt{internalData.\ameta{bgenv-name}}.
This field uses the JavaScript object \texttt{dataForEnv} to store information
about the environment. The data stored is,
\begin{Verbatim}[commandchars=!()]
if (typeof dataForEnv=="undefined")
  var dataForEnv=new Object;
dataForEnv["!ameta(bgenv-name)"]=new Object;
dataForEnv["!ameta(bgenv-name)"].width=!ameta(width);  //!textsf( width in big points)
dataForEnv["!ameta(bgenv-name)"].height=!ameta(heigh); //!textsf( height in big points)
dataForEnv["!ameta(bgenv-name)"].horiz=!ameta(!upshape(true)|(false));
dataForEnv["!ameta(bgenv-name)"].sf=!ameta(scale-factor);
dataForEnv["!ameta(bgenv-name)"].bgs=[!ameta(list-of-bg-names)];
dataForEnv["!ameta(bgenv-name)"].origin=!ameta(!upshape0|.5);
\end{Verbatim}

\subsection{Setting the scale} The initial scale of a \env{bargraphenv} environment is set
by the command
\bVerb\takeMeasure{\string\scaleFactorDef\darg{\ameta{pos-num}}}%
\begin{dCmd}[commandchars=!()]{\bxSize}
\scaleFactorDef{!ameta(pos-num)}
\end{dCmd}
\eVerb When placed in the preamble, \cs{scaleFactorDef} sets the default
scale factor; the \pkg{bargraph-js} declares \cs{scaleFactorDef\darg{2}}.
When a bar graph is reset (cleared), the value \ameta{pos-num} is used to
reset the scale value

When executed in
the body, the initial scale factor of all \env{bargraphenv} environments that
follow are affected, unless the declaration occurs in a group. For example,
\cs{scaleFactorDef\darg{3}} sets initial scale factor to 3; a more
provocative example is found in sample file \texttt{bgjs-adv1.tex} where we
declare,
\begin{Verbatim}[xleftmargin=\amtIndent]
\fbox{\scaleFactorDef{dataForEnv["math"].height/100}%
\begin{bargraphenv}[width=2in,height=2in,o=vert]{math}
...
\end{bargraphenv}}
\end{Verbatim}
Here we set the scale factor to be the height of the environment (in bp
points) divided by~100; that way, data are re-scaled as a proportion of the
height of the bar graph. We make this declaration inside the \cs{fbox} to
make the declaration local.

\subsection{Optimizing a bar graph}

To optimize all the bar graphs within a \env{bargraphenv} environment, use the built-in
JavaScript function \texttt{optimizeScaling("\ameta{bgenv-name}")}. Below is a ``generic
example'' for optimizing an environment.
\begin{Verbatim}[xleftmargin=\amtIndent,commandchars={!~@}]
\pushButton[\TU{!ameta~tool tip text@}\CA{Optimize}
  \AAmouseup{optimizeScaling("!ameta~bgenv-name@");}]{optimize}{}{13bp}
\end{Verbatim}
The \texttt{optimizeScaling()} function is used in numerous examples of this package.

\subsection{Manually scaling a bar graph}

The user can be allowed to manually re-scale the bar graph. For this purpose,
\pkg{bargraph-js} defines two commands \cs{displaysfFor} and \cs{manualsfFor},
\begin{Verbatim}[xleftmargin=\amtIndent,commandchars={!()}]
\displaysfFor[!ameta(eforms-opts)]{!ameta(bgenv-name)}{.5in}{13bp}
\manualsfFor[!ameta(eforms-opts)]{!ameta(bgenv-name)}{}{13bp}
\end{Verbatim}
where, \ameta{bgenv-name} is the name of the \env{bargraphenv} environment
that these controls are to re-scale. This pair of commands appear in the
sample files \texttt{bgjs-basic4.tex} and \texttt{bgjs-comma1.tex}.

Within the text field generated by \cs{displaysfFor}, simple arithmetic
operations can be made; eg, if current scale factor is \texttt{3.3} (the
optimized scale value, perhaps) then entering \texttt{3.3\,/\,2} and pressing
the re-scale button, re-scales the bar graphs to half their height or width,
depenting on whether \texttt{o=vert} or \texttt{o=horiz}.


\subsection{Resetting a bar graph}

Reset one or more bar graphs, as well as other fields using the JavaScript function
\texttt{resetBargraphs(\ameta{various})}
\begin{Verbatim}[xleftmargin=\amtIndent,commandchars={!~@}]
\pushButton[\CA{Reset}!ameta~other-options@
!quad\AAmouseup{resetBargraphs(!ameta~various@);}
]{reset}{}{13bp}
\end{Verbatim}
The \ameta{various} argument consists of a list of field names and
\env{bargraphenv} names to be reset. The name of the \env{bargraphenv}
environment on page~\pageref{intro} is \texttt{vehiclesV}. The code for
the \uif{Reset} button in that example is,
\begin{Verbatim}[xleftmargin=\amtIndent]
\pushButton[\CA{Reset}\A{\JS{resetBargraphs("vehiclesV");}}
]{reset}{}{13bp}\end{Verbatim}
Notice, the argument of \texttt{resetBargraphs} is \texttt{"vehiclesV"}, this will
clear not only all bar graphs in the \env{bargraphenv} environment, but also the
input fields (created by \cs{inputFor}); in the latter case, this is because the
root name of the \cs{inputFor} fields is \texttt{"vehiclesV"}.

The function \texttt{resetBargraphs} is used extensively in the sample file
of this distribution.

\section{Demonstration files}

This distribution comes with numerous sample files. The files  are packed
into the DTX file \texttt{bgjs-examples.dtx} (and can be unpacked by latexing
\texttt{bgjs-examples.ins}, if not done already). Read the documentation file
\texttt{bgjs-examples.pdf} for detailed descriptions of each sample file.

\section{My retirement}

Now, I simply must get back to it. \dps

\end{document}

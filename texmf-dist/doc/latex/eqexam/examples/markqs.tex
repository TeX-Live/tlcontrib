\documentclass{article}
\usepackage[forpaper,pointsonleft,totalsonright,
    nosolutions
]{eqexam}[2012/05/16]


\university
{%
      NORTHWEST FLORIDA STATE COLLEGE\\
          Department of Mathematics%
}
\email{storyd@nwfsc.edu}

\examNum{1}\numVersions{1}\forVersion{a}
\subject[DM]{Demoing Marks}
\longTitleText
    {Test~\nExam}
\endlongTitleText
\shortTitleText
    {T\nExam}
\endshortTitleText
\title[\sExam]{\Exam}
\author{Dr.\ D. P. Story}
\date{Spring, 2012}
\duedate{2012/04/24}
\keywords{Demonstrating various markers}

\begin{document}

\maketitle

\begin{exam}{Test\nExam}

\begin{instructions}\relax\parindent0pt\parskip6pt
Solve each without error. Passing is 100\%.

We summarize the information in the test. (Compile three times.)

Section I: \markStartFor{SectionI}--\markEndFor{SectionI}, this section
has \markNumQsFor{SectionI} problems.

Section II: \markStartFor{SectionII}--\markEndFor{SectionII}, this section
has \markNumQsFor{SectionII} problems.

Section III: \markStartFor{SectionIII}--\markEndFor{SectionIII}, this section
has \markNumQsFor{SectionIII} problems.

Section IV: \markStartFor{SectionIV}--\markEndFor{SectionIV}, this section
has \markNumQsFor{SectionIV} problems.
\end{instructions}

\calcQsBtwnMarkers[SectionII]{SectionI}

\begin{problem}[5]
Problem from Section I
\end{problem}

\begin{problem}[5]
Problem from Section I
\end{problem}

\begin{problem}[5]
Problem from Section I
\end{problem}

\begin{problem}[5]
Problem from Section I
\end{problem}

\begin{problem}[5]
Problem from Section I
\end{problem}

\begin{problem}[5]
Problem from Section I
\end{problem}

\begin{problem}[5]
Problem from Section I
\end{problem}

\calcQsBtwnMarkers[SectionIII]{SectionII}

\begin{problem}[5]
Problem from Section II
\end{problem}

\begin{problem}[5]
Problem from Section II
\end{problem}

\begin{problem}[5]
Problem from Section II
\end{problem}

\begin{problem}[5]
Problem from Section II
\end{problem}

\calcQsBtwnMarkers[SectionIV]{SectionIII}

\begin{eqComments}[Section III.]
You can separate the sections of the test with an header, like this one.
\end{eqComments}

\begin{problem}[5]
Problem from Section III
\end{problem}

\begin{problem}[5]
Problem from Section III
\end{problem}

\begin{problem}[5]
Problem from Section III
\end{problem}

\begin{problem}[5]
Problem from Section III
\end{problem}

\begin{problem}[5]
Problem from Section III
\end{problem}

\begin{problem}[5]
Problem from Section III
\end{problem}

\calcQsBtwnMarkers[EndExam]{SectionIV}

\promoteNewPage

\begin{eqComments}[Section IV.]
You can separate the sections of the test with an header, like this one.
\end{eqComments}
\begin{problem}[5]
Problem from Section IV
\end{problem}

\begin{problem}[5]
Problem from Section IV
\end{problem}

\begin{problem}[5]
Problem from Section IV
\end{problem}

\begin{problem}[5]
Problem from Section IV
\end{problem}

\begin{problem}[5]
Problem from Section IV
\end{problem}

\calcQsBtwnMarkers{EndExam}

\end{exam}

\parindent0pt \parskip6pt

We summarize the information in the test. (Compile three times.)

Section I: \markStartFor{SectionI}--\markEndFor{SectionI}, this section
has \markNumQsFor{SectionI} problems.

Section II: \markStartFor{SectionII}--\markEndFor{SectionII}, this section
has \markNumQsFor{SectionII} problems.

Section III: \markStartFor{SectionIII}--\markEndFor{SectionIII}, this section
has \markNumQsFor{SectionIII} problems.

Section IV: \markStartFor{SectionIV}--\markEndFor{SectionIV}, this section
has \markNumQsFor{SectionIV} problems.

\end{document}

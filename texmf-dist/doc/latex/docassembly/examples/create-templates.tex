\documentclass{article}
\usepackage{xcolor}
\usepackage{docassembly}
\usepackage{eforms}
\usepackage{lipsum}

\hypersetup{pdfpagemode=UseNone,colorlinks}

\begin{defineJS}{\ADActn}
if (typeof tmplObj != "object")
  tmplObj=this.getTemplate("AdobeDon");
tmplObj.spawn({
  nPage: 0,
  bOverlay:false,
  oXObject: tmplObj
});
\end{defineJS}
\begin{defineJS}{\contActn}
if (typeof tmplObjSpawn != "object") {
  var tmplObjSpawn=this.getTemplate("myStory");
  tmplObjSpawn.spawn({
    nPage: this.pageNum+1,
    bOverlay:false,
    oXObject: tmplObjSpawn
  });
  this.pageNum++;
}
\end{defineJS}

% Look in the Adobe Acrobat JavaScript manual under the Doc.createTemplate to see all the possible parameters.
\begin{docassembly}
\insertPages({
  nPage: -1,
  cPath: "graphics/AdobeDon-btn.pdf"
});
var tmplObj=\createTemplate({cName: "AdobeDon", nPage: 0});
tmplObj.hidden=true;
\insertPages({
  nPage: -1,
  cPath: "resources/myStory.pdf"
});
var tmplObj1=\createTemplate({cName: "myStory", nPage: 0});
tmplObj1.hidden=true;
\executeSave();
\end{docassembly}

\begin{document}

\noindent
\pushButton[\CA{View author}\AAmouseup{\ADActn}]{vTmpl}{}{11bp}\medskip

\noindent
\lipsum[1-4]\medskip

\noindent
Write a short history of your life. Press the \textsf{Continue} link if you need more space for your story, and then continue on the
next page.\medskip

\noindent
\textField[\Ff\FfMultiline]{myStory}{\linewidth}{3\baselineskip}\\[6bp]\null
\hfill\setLink[\A{\JS{\contActn}}\linktxtcolor{red}]{Continue on next page as needed}

\newpage
\noindent\lipsum[5-10]

\end{document}
